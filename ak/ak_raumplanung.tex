% !TEX TS-program = pdflatex
% !TEX encoding = UTF-8 Unicode
% !TEX ROOT = main.tex

\section{AK Bibliotheks- und Raumplanung}

  \textbf{Protokoll vom:} 02.06.2018, %???
  Beginn: 09:15 Uhr,
  Ende: 11:00 Uhr \\
  \textbf{Redeleitung:} Stefan (Uni Köln) \\ %???
  \textbf{Protokoll:} %??? \\
  \textbf{anwesende Fachschaften:} TU Gratz, Uni Dresden, Uni Bonn, Uni Bielefeld, Uni Würzburg, Uni Chemnitz, TU Berlin, Uni Wien, TU Wien

  \subsection*{Informationen zum AK}
    \begin{itemize}
      \item \textbf{Ziel des AKs}: Positionspapier
      \item \textbf{Folge-AK}: nein
      \item \textbf{Zielgruppe}: Leute, die an der menschenfreundlichen und kommunikativen Weiterentwicklung dezentraler Raumstrukturen interessiert sind
      \item \textbf{Voraussetzungen}: keine
    \end{itemize}

  \subsection*{Protokoll}
      \begin{itemize}
        \item TU Gratz: Ein gemeinsames Physikgebäude soll entstehen. Hoffnung: Was ist wichtig für Studierende in Gebäuden

        \item Uni Dresden: Beobachter

        \item Würzburg: Haben ein sehr offenes Ohr bei Fakultätsleitung. Gestaltung mit Hilfe der Studierenden entspannt

        \item Bonn: Es gibt keine Zentralisierung der Bibliotheken. Es wurden sogar geschlossene Bibliotheken wieder geöffnet. Sehr altes Gebäude.
        Es existieren fast gar keine Gruppenarbeitsräume. (Eigentlich nur von der Mathe). Es soll nicht nur Lernen sondern auch Erholung beachtet werden.

        \item TU Chemnitz: Es wird eine neue (zentrale) Bibliothek gebaut. Es gibt ein paar Lernräume, aber könnte besser werden.

        \item Konstanz: Beobachterin

        \item Bielefeld: Hat eine zentrale Bibliothek, für alle Fakultäten. Mit sehr vielen Arbeitsräumen. Findet das Konzept gut

        \item Wien: Haben einen Anbau genehmigt bekommen (Physik). Es sollte für alle Statusgruppen mehr Platz geschaffen werden.

        \item TU Berlin: Neue Mathe- und Physik-Gebäude. Da soll auch die Fachschaft beteiligt werden.

        \item Bochum: Es wird alles renoviert. Deshalb mehrere Umzüge. Angst vor unerwarteten Gefahren, wie Raumverlust.

        \item TU Wien: Neue Bibliotheksleitung geht auf die FSen zu und fragt nach Mitgestaltung. FS hat zu wenig Input.
      \end{itemize}

    \paragraph{Einleitung}
      Es gibt sehr viele dezentrale Bibliotheken (136), ist geschichtlich gewachsen.
      Zentrale Bibliothek wird sehr viel benutzt.
      Neuer Bibliotheksleiter möchte die desolate Zentralbibliothek erneuern. Hierbei sollen die dezentralen Bibliotheken verschwinden. Anschuldigungen der ineffektiven Arbeit von dezentralen Bibliotheken.
      Dezentrale Bibliotheken sind privater, mit mehr Austausch innerhalb des Fachs, sowohl zwischen Professoren und Studierenden, als auch Studierenden an sich.
      Eine Zentralbibliothek bietet andere Dienste, die eine kleine Bibliothek nicht leisten kann. Allerdings werden diese Dinge vor Ort informell gemacht.

    \begin{itemize}
      \item Würzburg: Es gibt zentrale Bibliotheken und auch Teilbibliotheken der Fachbereiche, die deutlich spezialisierter sind und eigene Bibliotheken pro Lehrstuhl. Zentralbibliothek vor allem Lehrbuchsammlung. Alle Bücher sind in einem Katalog verfügbar.

      \item Köln: Die Verträge sind deutlich schwieriger als in Würzburg, welche aber vereinfacht werden.

      \item Würzburg: Kurse wurden extra zentralisiert, damit bessere Ausbildung gewährleistet werden kann, was das gegenseitige Ansehen verbessere.

      \item Köln: Humanwissenschaften und Wirtschaftswissenschaften haben sich nicht mit dem Thema beschäftigt.
    \end{itemize}

      Mittlerweile gibt es fast alle Bücher im Katalog.

    \begin{itemize}
      \item Gratz: Fast alle Bücher werden in niedrigen Semestern digital genutzt. Erst höhere Semetster nutzen die Bibliotheken

      \item Wien: Es gibt große Unterschiede bei Lehre und Abschlussarbeiten. Sobald diversere Anforderungen existieren, sind Papierbücher unablässlich

      \item Köln: Viele digitale Bücher werden dann ausgedruckt. Wenn sie analog verfügbar sind, passiert das weniger. \\
      Erkentnisse aus Exkursionen:
      \begin{itemize}
        \item Papierbücher werden nur benutzt, wenn sie unmittelbar verfügbar sind, also auch ohne über die Straße laufen.
        \item Wenn Bücher ohne Scheine ausleihbar sind, bringen die Studierenden sie zurück. Dadurch wurden Studierende in den Arbeitsraum gelockt.
      \end{itemize}

      \item Würzburg: Mit einem Zähler, bei der Ausleihe wurden Bestände erneuert und an die Studierendenschaft angepasst.

      \item Köln: Delft: Es wird nicht kontrolliert, was ausgeliehen wird und der Schwund ist geringer als gedacht.

      \item Wien: Fachbereichsbibliotheken: Die Bücher gibt es kaum analog, werden aufgrund des digitalen Angebots auch nicht vermisst.

      \item Köln: Zettel hinten in den Büchern fördern die Kommunikation

      \item Bonn: Es gibt beide Arten Bibliotheken. In der Zentralbibliothek gibt es große anzahlen von Büchern. In der Eval der Vorlesung wird auch Bücherbenutzung abgefragt und danach wird nachbestellt.

          Hier Thesen einfügen. (Handout)

      \item Wien: Die Gestaltungsmöglichkeiten sollen auf nicht so hoher Ebene gegeben werden.

      \item Köln: Die Architektur verhindert viel, wenn sie

      \item Bonn: Die dezentralen Strukturen geben nicht die Möglichkeit, von experten gestaltet zu werden. Die Profs machen eher ihr eigenes Ding

      \item TU Wien: man muss ich nicht an so viele Leute anpassen, wenn die Räume weniger Fachgruppen einbinden.

      \item Köln: Zentrale Bibliotheken sehr organisch gewachsen, was verschiedene anpassungen erlaubt

      \item Bochum: Alle struktueren (FS-Raum, Bibliotheken, Computerpool, etc.) sind eine Einheit, was viele medientypen verknüft und soziale interaktion fördert. Raum ist selbstverwaltet, was aber keine Probleme bereitet.

      \item Bielefeld: Bielefeld hat genau ein Gebäude, wo eine Etage nur Bibliotheken ist. Die Bibliotheken ist quasi verbindung zwischen den "Zähnen" des Gebäudes. Verschiedene Arten von Arbeitsräumen grenzen daran an, wo verschiedene Arten des Lernens gefördert werden.

      \item Dresden: Wenn man lediglich ausleihen will, bieten dezentrale Strukturen schwierigkeiten.

      \item Würzburg: Raumplanung sollten eher in den Gremien koordiniert werden. Damit die Bibliothekenliothekar*innen nicht überlastet werden.

      \item Gratz: Grade bei Neuplanung werden die prioritätetn anders gesetzt und hier müssen direkt klare Aufträge formuliert werden. Hier sollten vorallem gute Vorbilder gefunden werden, wo man Ideen übernehmen kann.

      \item Köln: Die verscheidenen Arbeitsräume müssen in der nähe Liegen um verschiedene Arbeitsweisen zu verknüpfen.

      \item Köln: Um Aufmerksamkeit zu erhalten, muss man seine Argumente begründen.

      \item Chemnitz: Es gibt beide Arten Bibliotheken, und arbeitsräume sind um fachbibs angesiedelt.

      \item Köln: Man kann sich selbst den Raum gestalten. Dafür muss es Leute geben, die Verantworung übernehmen

      \item Würzburg: Es gab eine neuorganisierung der Raumsitutation und der Senat hat aktiv die Studierendenschaft daran beteiligt. Damit sowas passiert muss man die Strukturen freundlich nerven. Damit man auch aufmerksamkeit auf eigene Anliegen lenken kann.

      \item Gratz: Es wird das Center of Physics geplant. Die Profs haben das auf "geheimen" Treffen geplant. Dies wurde über umwege an die FS gebracht, welche durch eigeninitative sich in die Planung eingebracht. Hier zahlt sicht vor allem Hartnäckigkeit aus.

      \item Wien: Es existiert akuter Platzmangel. Lehrräume sind mehr als erwartet. Beim Anbau gab es erst mal feste Konzepte, wo die Studierenden nicht beachtet wurden. Es wurde mit Genehmigungsgrenzen Argumentiert.

      \item Köln: Man kann das Dekanat auch umerziehen mit der freundlichen Keule.

      \item Gratz: So früh wie möglich mitreden.

      \item Köln: Auch bei alten Sachen kann man sehr viel erreichen, durch Umorganisierung. Manchmal können billige Veränderungen große Wirkung zeigen. Auch bei Neubauten gibt es fehlplanungen die man kreativ korrigieren kann.

      \item Bonn: Pluralismus und verschiedene Raumkonzepte sind größtenteils konsens

      \item Köln: Ist nicht gegen Zentralbibs sondern nur für den dezentralen Ausbau

      \item Wien: Die Räume sind das was man draus macht.

      \item Würzburg: Man sollte versuchen von anfang an dabei sein. Gerüchte müssen aufgegriffen werden. Informationen über Maßnahmen könnten zentraler gesammelt werden, damit auch neueinsteiger*innen Informationen finden können.
    \end{itemize}

      Zwei wichtige Themen:
      \begin{itemize}
        \item Welche Bibliothekenformen will man fördern?
        \item Wie geht man früh in die Planung?
      \end{itemize}

    \begin{itemize}
      \item Bielefeld: Alles zentral organisiert: keine selbstorganisierten Räume. Die bibverwaltung schafft auch Räume, sodass es funktioniert.

      \item Dresden: Es gibt die eine Zentrale Bibliothek, wo alle hingehen.

      \item Würzburg: Diese AK-Form sollte auch auf der nächsten ZaPF wieder auftauchen und die Informationen gesammelt werden.
      In diesem AK soll die Raumgestaltung im Fokus stehen. \\

      Es soll ein Handout erarbeitet werden.

      \item Bielefeld: Ein Positionspapier ist kein Mehrwert gegenüber der Reso SoSe17 "Resolution zur studentischen Beteiligung bei Bauvorhaben"

    \end{itemize}

   Handouts von Köln müssen noch zur verfügung gestellt werden
                                                                                                                               1,1        Anfang
