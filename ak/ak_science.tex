% !TEX TS-program = pdflatex
% !TEX encoding = UTF-8 Unicode
% !TEX ROOT = main.tex

\section{AK Open-Science}

  \textbf{Protokoll vom:} 02.06.2018,
  Beginn: 11:30 Uhr,
  Ende: 13:30 Uhr \\
  \textbf{Redeleitung:} Marius (Göttingen) \\
  \textbf{Protokoll:} Merten (Göttingen) \\
  \textbf{anwesende Fachschaften:} Uni Augsburg, Uni Köln, Uni Wien, TU Dresden, TU Darmstadt, Uni Potsdam, TU Ilmenau, Freiberg, HU Berlin, Uni Tübingen, Uni Oldenburg, TU München, TU Chemnitz

  \subsection*{Informationen zum AK}
    \begin{itemize}
      \item \textbf{Ziel des AKs}: Bewusstsein für das Thema Open-Science, Best Practices für die Vermittlung des Themas an Studierende. Auf einem Folge-AK könnte es auch eine Resolution geben.
      \item \textbf{Folge-AK}: nein
      \item \textbf{Zielgruppe}: Masterstudierende, Promovierende, und ganz allgemein Menschen, die Interesse an Wissenschaftspolitik haben
      \item \textbf{Ablauf}: Etwas Input (je nach vorhander Kompetenz im AK), Diskussion über Vor- und Nachteile, Diskussion über Best Practices
      \item \textbf{Materialien}: Open-Access-Diskussion auf der ZaPF (\url{https://zapf.wiki/Open_Access}), Wikipedia gibt einen guten Überblick (\url{https://en.wikipedia.org/wiki/Open_science}), die EU will da ganz vorne mit dabei sein (\url{https://ec.europa.eu/research/openscience/})
    \end{itemize}

  \subsection*{Einleitung}
    Open-Science ist ein sehr weites Feld. Das geht von Open-Access (zu dem Thema gibt es bereits eine ausführliche Wiki-Seite) über Open-Peer-Review und Open-Data, bis hin zu Open-Source. \\
    In diesem AK soll es einen Überblick und AUstausch zu dem Thema geben. Außerdem soll sich darüber unterhalten werden, wie das Thema an Studierende vermittelt werden kann.
    Auf einem Folge-AK könnte außerdem über eine Resolution zum Thema diskutiert werden.

  \subsection*{Protokoll}
    \paragraph{Erfahrungen}
      \begin{outline}
        \1 Chemnitz: Vortrag von Brian Nosek; Plattform für Open-Science (Center of Open-Science, cos.io). Hauptsächlich Psychologie, aber Ausweitung auf alle Fachbereiche; Vortrag über Peer Reviews; Publication bias, nur gelungene Experimente, insbesondere Positivbefunde werden angenommen/publiziert. Zeigte Möglichkeiten auf, das über Qualitätssicherung zu umgehen. Die Macht der Publikation liegt bei den Verlagen.
        \1 Uni Wien: Zu der Verlagsbindung: CCC in Leipzig, Vortrag dort über "Science is broken".
        \1 Merten: Impactfactors und ähnliches, aber auch Ansätze mit anderen Metriken. Wissenschaft involoviert oft den Computer. Hier ist Open-Source extrem wichtig, da der Auswertungsprozess das wichtige ist.
        \1 Köln: Open-Data - mal abgesehen von der Reproduzierbarkeit kann es auch in anderen Projekten oder Fachbereichen verwendet werden.
        \1 Darmstadt: Wenn man mit Open-Access an Verlage herantritt, muss man leider mehr zahlen. Aber dazu gibt es eine gute Alternative: \url{https://sci-hub.tw/}. Was macht man mit Unmengen an Daten?
        \1 Merten: Gold-Open-Access: Open-Access Journals, Green-Open-Access: preprint ist auch öffentlich.
        \1 Köln: Es müssen sinnvolle Tools für Datahandling etc erstellt werden. In UK wird das sehr befürwortet.
        \1 Merten: Die EU soll ein Leuchtturmprojekt in Bezug auf Open-Science sein. Horizon 2020. ZaPF hat eine Resolution 2009 im Bezug auf Open-Access veröffentlicht. Können wir uns zu den anderen Themen positionieren? Idee wäre, einen Gast einzuladen, mit dem man dann darüber reden kann. Man darf den Aufwand für Open-Access reduzieren, deswegen benötigen wir Tools. Das könnte ein mögliches Thema sein.
          \2 Open-Science Policy Plattform
          \2 Brian Nosek
          \2 Pawel Richter (Open Knowledge International)
        \1 Chemnitz: Schwierig alles unter einen Hut zu bringen.
        \1 Merten: Themen für Folge-AKs?
        \1 München: Open-Science Tools ist das wichtigste, da aus diesem viel abgeleitet wird. Außerdem wird die Arbeit dadurch erleichtert. Desweiteren ist der Zugang zu Papern wichitg.
        \1 Open-Eductaion ist ein wichtiger Aspekt. Kann aber in E-Learning behandelt werden.
      \end{outline}

    \paragraph{Folge Aks}
      \begin{itemize}
        \item Open-Access
        \item Open-Science Tools
        \item Open-Source
      \end{itemize}
      Merten: Einladungen zu allen AKs in Würzburg wäre schön. Allerdings wird das technisch wahrscheinlich nicht möglich sein.

    \paragraph{Roadmap}
      Im Sinne der Stationen einer wissenschaftlichen Arbeit.\\
        \begin{itemize}
		\item W18: Open-Science Tools
		\item S19: Open Source und/oder Open Access
		\item W19: Überbleibsel von S19
	\end{itemize}

    \paragraph{Formulierung einer Selbstverpflichtung}
      Die ZaPF möchte sich in den folgenden Jahren mit Open-Science (OSc) beschäftigen. Folgende Roadmap wird hierzu vorgeschlagen:
      \begin{itemize}
        \item WiSe 18/19: Open-Science Tools (OST)
        \item SoSe 19: Open-Source, Open-Access (OS, OA)
      \end{itemize}
      Zu diesen Themen sollen auch Experten eingeladen werden. \\ Desweitern wird eine Anregung an den E-Learning AK gegeben, sich mit Open-Education zu befassen.

    \paragraph{Fragen an die Referenten}
      \begin{itemize}
        \item Open-Science Überblick
        \item Beispiel oder aktuelle Projekte
        \item Vor- und Nachteile
        \item Wie bringe ich diese Tools zu den Publizierenden?
        \item Rechtslage (geistiges Eigentum, Veröffentlichungen)
      \end{itemize}
      Fachschaften sollen das bei der Bachelor-Master-Umfrage propagieren (wenn es rechtlich möglich ist).

    \paragraph{Sonstiges}
      Ausgehend vom Thema Peer-Review entspinnt sich eine ausführliche Diskussion über Interdisziplinarität. In Wien gibt es einen interdisziplinären Journal Club. \\
      $\rightarrow$ Hierzu sollte es in Würzburg einen AK geben.
      
  \subsection*{Zusammenfassung}
    \begin{outline}
      \1 Es wurde zusammengefaasst, was Open-Science beinhaltet.
      \1 Es wurde eine Roadmap erstellt, auf welcher Zeitskala die ZaPF sich mit dem Thema Open-Science beschäftigen soll.
      \1 Auf die kommenden ZaPFen soll zu den einzelnen Themen jeweils ein geeigneter Referent oder eine geeignete Referentin eingeladen werden.
      \1 Es soll ein entsprechender Arbeitsauftrag an den StAPF formuliert werden. \\
        Nachtrag: Nach Rücksprache wird kein Arbeitsauftrag formuliert, sondern lediglich im Abschlussplenum verantwortliche Personen gesucht, welche in Absprache mit dem StAPF entsprechende Personen einladen.
    \end{outline}
