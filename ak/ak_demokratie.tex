% !TEX TS-program = pdflatex
% !TEX encoding = UTF-8 Unicode
% !TEX ROOT = main.tex

\section{AK Hochschuldemokratie}

  \textbf{Protokoll vom:} 31.05.2018, % ???
  Beginn: 14:00 Uhr, % ???
  Ende: 16:15 Uhr \\ % ???
  \textbf{Redeleitung:} Sven (Uni Köln) \\
  \textbf{Protokoll:} Michel (Uni Köln) \\
  \textbf{anwesende Fachschaften:} Frankfurt, Oldenburg, Dresden, FU Berlin, Saarland, Darmstadt, Marburg, TU Wien, Münster, Mainz, Konstanz, Bielefeld, Göttingen, LMU München, Würzburg, Jena, Uni Wien, Bochum, Bonn, Siegen, Köln, TU Berlin

  \subsection*{Informationen zum AK}
    \begin{itemize}
      \item \textbf{Ziel des AKs}: Information, Austausch, Resolution
      \item \textbf{Folge-AK}: nein
      \item \textbf{Zielgruppe}: Alle, besonders VertreterInnen von Mündigkeit im Studium
      \item \textbf{Materialien}: \url{http://uni-aktionsbuendnis.uni-koeln.de/index.php/hochschulgesetz/}
      \item \textbf{Ablauf}: Kleiner Input zum geplanten Hochschulgesetz in NRW und in anderen Bundesländern. Anschließende Diskussion und Resolutionsfindung.
    \end{itemize}

  \subsection*{Einleitung}
    In vielen Bundesländern wird gerade das Hochschulgesetz geändert. Hierbei werden vor allem die Statusgruppen der Studierendenschaft und des Mittelbaus benachteiligt. Hierbei sollen viele studentische Gremien abgeschafft werden oder Paritäten aufgehoben. \\ \\

    Am Beispiel NRW wollen wir in das Thema einführen, aber auch von anderen Bundesländern die aktuellen Entwicklungen aufgreifen. Unter dem Namen der Freiheit, werden hier viele Rückschritte eingeführt, die das selbstbestimmte Studium einschränken und das politische Agieren innerhalb der Hochschule erschwehren. Gerade im Entwurf in NRW stehen Sätze, mit denen keine Vertretung der Studierenden einverstanden sein kann, wie z.B. Es „soll die Verpflichtung der Hochschule gestrichen werden, die Interessen der Mitglieder der nichtprofessoralen Gruppen (…) angemessen sicherstellen zu müssen“. Gleichzeitig hält es die Landesregierung nicht mehr für notwendig, dass die Hochschulen zu Frieden, Demokratie und Nachhaltigkeit beitragen. Wir wollen mit euch diskutieren wie wir diese Entwicklung einordnen, was die nächsten Schritte sind, was wir gegen diese Entwicklung unternehmen wollen und vor allem warum. Das Ergebnis wird in einer Resolution festgehalten.

  \subsection*{Protokoll}

    \paragraph{Das Gesetz}
      \begin{itemize}
        \item Unter dem Namen ``Freiheitsgesetz''
        \item Das rot/grüne Gesetz soll rückgängig gemacht und auf den Status des Schwarz/Gelben Wissenschaftsfreiheitsgesetz rückgeführt werden.
        \item Hochschulrat (selbst-reproduzierendes Gremium, das vermittelnd zwischen Ministerium und Senat agiert): Soll über dem Senat stehen und ist weisungsbefugt.
        \item Senat (höchstes Gremium der Universität): Müssen nicht mehr paritätisch besetzt werden.
        \item Studienbeiräte (beschließt Studienordnung und ist paritätisch aus Studierenden und Dozierenden besetzt. Sollen Studiengänge evaluieren und weiterentwickeln. Basiert auf Konsensprinzip): Sollen optional werden. Können bei Missfallen abgeschafft werden.
        \item Gegenstromverfahren (Finanzierung der Hochschule): Die Unis sollen sich ab sofort wieder um die Geldtöpfe streiten. Fördern von Diskussionen.
        \item Anwesenheitspflichten (bisher größtenteils verboten) dürfen wieder eingeführt werden.
        \item Studienverlaufsvereinbarungen (Vertrag über den Studienverlauf zwischen Uni und Studierenden): Könnte eine Exmatrikulation nach sich ziehen.
        \item Kodex für gute Arbeitsbediungen soll kündbar werden.
        \item SHK Räte sollen abgeschafft werden mit der Begründung, sie seien Fremdkörper im System.
        \item Die Zivilklausel (Unis sollen zu Nachhaltigkeit, Frieden und Demokratie beitragen) soll abgeschafft werden.
      \end{itemize}

    \paragraph{Dinge die in Köln passiert sind}
      \begin{itemize}
        \item Verschiedene Statusgruppen, unter anderem die Fachschaftenkonferenz, haben sich gegen das Gesetz ausgesprochen.
        \item Bochum: Ministerin hat dazu gesagt: ``Zu ihrer Zeit wäre man dagegen noch auf die Strasse gegangen.'' \\
          $\rightarrow$ Konsequenzen aus den Gesprächen soll ein Positionspapier ergeben.
        \item Siegen: Wurde in der Senatskommission angesprochen, es war niemand begeistert, aber es passiert auch noch nicht viel.
      \end{itemize}

    \paragraph{Andere Bundesländer}
      \begin{itemize}
        \item Marburg: Hessen: wird in einem Jahr das HSG renoviert und braucht deshalb vorlagen und anregungen aus anderen Bundesländern
        \item Würzburg: Bayern: Alle Unis in Bayern laufen nach demselben System, das vom Gestz vorgeschrieben ist. 
        \item Diese positiven Enwicklungen werden von FU Berlin nicht gesehen. Es gibt immernoch keine paritätische Vertretung in den hohen Gremien. Die Gremien sind nicht öffentlich, was zu viel internen Absprachen führt, wo die sehr unterrepresentierten Statusgruppen wie Studierende nicht viel zu melden haben.
        \item Berlin: Das letze HSG ist von Rot/Rot und beinhaltet die Erprobungsklausel, wo sich Hochschulen sehr viele Freiheiten einräumen können. Die neue Regierung hat eine Hochschuldemokratiekommission gegründet, welche sich zum Stillstand zerstritten hat. Das führt zu einem HSG, das wahrscheinlich zu einen sehr drastischen Schnitt bedeuted. Es ist nicht aufgabe der Kommisssion ein weiter so zu beschließen.
        \item Thüringen: Der Senat soll paritätisch besetzt und gestärkt werden. Es gibt eine allgemeine positive Grundstimmung: Regierung Rot/Rot.
      \end{itemize}

    \paragraph{ZÜ}
      \begin{outline}
        \1 Köln: Der Freiheitsbegriff suggeriert Autonomie, aber es wird eine Hirachie etabliert. \\
        Senat sagt:" Wir sind auch gegen das Gesetz, ihr habt nichts zu befürchten, wir ändern das nicht." Dies gilt aber nur so lange, wie das aktuelle Gremium aktiv ist.
        \1 Auch an anderen Unis könnten diese Gesetzesänderungen eingeführt werden.
        \1 Bochum: Es ist nicht alles schlecht, weil an einigen Punkten auch gute Autonomie eingeführt wird (z.B. Bauentscheidungen).
        \1 Köln: Die Ministerien sollen nicht mehr die Hochschulen erspressen können. Dies ist begrüßenswert, aber wird an anderer Stelle mit Studienverläufen genauso eingeführt. \\
        \1 Zitat: "Es ist nicht mehr notwendig, dass die nichtprofessoralen Gruppen angemessen vertreten sind".
        \2 Wie sollen wir dies bewerten?: \\
        $\rightarrow$ Dies bedeutet einen Rechtsruck
        \1 Es soll den Konservativen an der Uni die Möglichkeit gegeben werden, zu fortschrittliche Gedanken zu unterdrücken.
        \1 So lange es klappt, muss man keine Änderungen einführen, aber wenn es unangenehm wird, kann man das entsprechende Gremium abschaffen.
        \1 Marburg: Es soll direkt an der Hochschule die studentische Partizipation eingeschränkt werden. Man kann sich also nicht auf die eigene Uni verlassen. Grade wenn das HSG der Hochschule Rechte der Einschränkung einräumen.
        \1 Köln: Wie wollen wir in den Gremien arbeiten. Können wir dort noch frei reden, oder muss man sich unter Androhung der Abschaffung zensieren?
        \1 Bonn: Telekom möchte mit Fachhochschulen eigene Lehrstellen schaffen und fusionieren. Das hört sich danach an, als würde die Regierung diesen Weg mit ebnen wollen.
        \2 Köln: Das sieht man auch daran, dass der Hochschulrat mehr Macht kriegen soll.
        \2 RWTH AAchen hat mit viel Presse Drittmittel gekündigt, aufgrund der Zivilklausel - Mit der Begründung: Freiheit.
        \1 Es werden mehr Optionen geöffnet, um wirtschaftsnahen Unternehmungen den Rücken zu stärken.
        \1 Das alte Gesetz hat eher den demokratischen Kräften den Rücken gestärkt.
        \1 Bochum: Mit dieser Aufweichung, können gezielt an Unis Probleme durch Abschaffung dieser gelöst werden.
        \1 Köln: Wir haben nicht viel Diskussionsbedarf und sollten uns als Physikfachschaften die Unis mehr zur Zusammenarbeit führen.
        \1 Bonn: Viele wussten gar nicht, was in diesem Gesetz steht. Fachschaften sollten ihre Studierenden mehr über das Gesetz informieren und Initiative ergreifen.
        \1 Köln: Was für Konsequenzen sollen wir ziehen? Das Ministerium macht es für die Rektoren, die aber gar nichts von ihrem Glück wissen wollen. Die Unis müssen sich positionieren. Aber wenn man auf die Studiengebühren schaut, wurden alle Anträge angenommen, aber es gibt nicht genug Unis die sich engagieren. Auch Kleinigkeiten können schon helfen. Einfach mal im Senat nachfragen oder Flyer austeilen.
        \1 Bochum: Wie leicht ist es, die Studierenden zu mobilisieren und zu erreichen? Wie schafft man Aufmerksamkeit?
        \2 Köln: Darüber sollten wir uns alle immer Gedanken machen.
        \2 Frankfurt: Geziehlt ansprechen
        \2 Wenn die Initative aus mehreren Richtungen kommt und aktuellen Medien nutzt, kann man die Informationsverbreitung zu einem Lauffeuer anfachen.
        \1 Köln: Wir haben eine Sonderfachschaftskonferenz einberufen und eine Stellungname beschlossen. Diese wurde in der Mensa verteilt und an einem sonnigen Tag auf dem Hauptplatz der Uni eine Infoveranstaltung gegeben, wo die Leute Schlange standen.
        \2 Bochum: Die Leute haben nicht genug Zeit solche Veranstaltungen in ihrem Alltag unterzubringen.
        \1 Marburg: Was ist das Ziel dieses AK?
        \1 Gibt es \underline{Flyer}? Ja auf der im Wiki verlinkten Seite \url{http://uni-aktionsbuendnis.uni-koeln.de}
        \1 Köln: Es gibt schon eine Resolution von letztem Jahr, die aber sehr kurz ist und wenig begründet. Wir sollten eine besser begrüdete Resolution beschließen.
        \1 Es wurde ein Sketch zu den Studienverlaufsplänen aufgeführt, der Aufmerksamkeit geschaffen hat.
        \1 Lernfabriken Meutern: Das Bündnis war schon in der Stellungnahme der FSK Köln mit einbezogen.
        \1 Frankfurt: Es wäre cool, wenn man diese gesammten Bewegungen, die immer wieder passieren, in einem Wiki oder ähnlichem sammelt.
        \1 Marburg: Wiki ja! Es sollte auch eine Resolution geben.
        \1 Köln: Es sollte auf jedenfall eine Begründung geben, warum man eine Maßnahme will oder nicht.
      \end{outline}

    \paragraph{Resolution}
      \textbf{Wie stellen wir uns die Traum-Uni vor?} \\
      \begin{outline}
        \1 Bochum: Wir befinden uns auf einem guten Weg und wenn einem die Mündigkeit genommen wird, wird man auch weniger ernstgenommen.
        \1 Köln: Wie wird Parität umgesetzt? Meistens führt es zur Qualität von Entscheidungen, wenn nicht über eine Gruppe hinweg entschieden werden kann. Parität erfordert, dass man der anderen Statusgruppe zuhört.
        \1 Marburg: Gegenseitige Wertschätzung sollte hervorgehoben werden.
        \1 Es wurde bereits ein Positionspapier zu dem Thema in Jena beschlossen. \\
          $\rightarrow$ Positionspapier zur demokratischen Mitbestimmung in Hochschulgremien SoSe13 % TODO: Link raussuchen
        \1 Köln: Vertreten wir das Papier immer noch oder wollen wir etwas hinzufügen?
        \1 Nordrhein-Westfalen sollte nicht Hauptbestandteil eines Papiers werden.
      \end{outline}
      Positionspapier wird verlesen.
      \begin{outline}
        \1 Köln sieht Potential, dass Papier zu erweitern.
        \1 Marburg: Es steht einer Resolution nicht im Weg, sondern unterstützt sie.
        \1 Köln: Das Positionspapier sollte in eine Resolution überführt werden. Von der aktuellen Lage aus sollten die Schwerpunkte verändert werden $\rightarrow$ Weniger Details, mehr Begründungen
        \1 Man sollte niemanden übergehen können.
        \1 Bochum: Sollte man Forderungen stellen, wenn wir einen Gesetzesentwurf verhindern möchte?
          \2 Köln: Wir sollten sagen, was wir richtig finden und nicht ihnen entgegenkommen. Man kann auch hoch in die Verhandlung reingehen.
          \2 Köln: Wir müssen das aktuelle Gesetz nicht verteidigen, sondern unsere wichtigen Punkte anbringen.
        \1 Köln: Sollen wir auch die Zivilklausel ansprechen?
          \2 Wir haben uns zur Zivilklausel auch schon separat sehr viel beschäftigt.
          \2 Köln: Gibt es Abschnitte, die wir dann wiederverwenden können?
          \2 Es gibt bereits ein Positionspapier.
        \1 Köln: Statt die Lehre zu verbessern, wird Anwesenheit und Studienverlauf eingeführt.
        \1 Köln: Wir sollten kein Nordrhein-Westfalen-spezifisches Problem als Aufhänger wählen.
        \1 Marburg: Als Aufhänger schwierig, aber als Beispiel durchaus sinnvoll.
        \1 Bonn: Wie funktionieren Studienverlaufspläne?
        \1 Köln: \textit{Beispiel:} Wenn man sich nicht an die vorgegebene Dauer des Studiums hält, wird man exmatrikuliert.
        \1 Bonn: Neue Argumentationsstratgie: Über die Wirtschaftlichkeit der MINT-Fächer.
        \1 Marburg: Es geht nicht nur um MINT-Fächer. Nur weil Leute länger brauchen, heißt das nicht, dass sie schlechter abschließen.
        \1 Köln: Zurück zur Zivilklausel
        \1 Bochum: Es müssen nicht nur vier Punkte sein. Man sollte erreichen, dass viele Leute das Ding lesen.
          \2 Bielefeld: Wenn wir die Studierendenschaft erreichen wollen, sollten wir nicht über die Zivilklausel gehen, sondern die Studierendenverlaufsvereinbarungen.
          \2 Wenn wir die Dozierenden erreichen wollen, sollten die Gremien den Hauptschwerpunkt bilden.
        \1 Die Resolution sollte nicht zu breit werden.
        \1 Köln: Wenn man was aufschreibt, von dem man selbst überzeugt ist, wirkt es überzeugender.
        \1 Bochum: Die Ziviliklausel ist nicht unumstritten und sollte nicht der Hauptaufhänger werden.
        \1 Daniela: Hauptpunkte unserer Resolution festlegen. Der Text sollte dann in einem BackUp-AK ausgelagert werden.
          \2 Köln: Wegen Nordrhein-Westfalen auch an die Hochschulleitungen in Nordrhein-Westfalen. Die sollte man begründen, weil die das Lesen.
          \2 Bielefeld: Die SHK-Räte sehen sich nicht als Fremdkörper. SHKs werden auch gerne vertreten.
        \1 Daniela: Eine Resolution mit Gremien und Studienverlauf und ein Positionspapier mit Zivilklausel für Nordrhein-Westfalen.
          \2 FU Berlin: Zivilklausel existiert schon, das müssen wir nur noch mal verschicken. Die Resolution ist noch nicht fertig und sollte davon unabhängig geschaffen werden.
          \2 Bonn: Der Einfachheit halber sollten wir zwei machen.
        \1 Bochum: Am Ende werden die ersten zwei Punkte mehr Beachtung finden, weshalb man darauf den Fokus legen sollte. Die Zivilklausel wird übersehen.
          \2 Köln: Wir sollten so das Positionspapier verwenden. Trifft aber keine Aussage über den Ort der Verankerung. Ob im Gesetz oder an der Hochschule geht nicht hervor. Wenn es aus dem Gesetz gestrichen wird, steht die Landesregierung nicht mehr in der Pflicht, Geld dafür zu geben. Dies muss auch abgedeckt im Gesetz werden \textbf{\#Yoda}.
          \2 Köln: Es wird nicht verkompliziert, wenn man der Resolution einen Punkt hinzufügt. Es sollte auch der Zusammenhang klar herausgestellt werden.
          \2 Daniela: Die Resolution soll weiter gefasst werden, die alle Hochschulpolitik betrifft.
          \2 Die einzelnen Punkte sollten explizit den einzelnen Leuten auf den Tisch geknallt werden.
          \2 Die Wirkung wird verstärkt, wenn man die Punkte auf die Adressaten abstimmt.
        \1 Bochum: Wenn wir die Punkte trennen, schaffen wir weniger Aufmerksamkeit. Die Resolution soll auch an die Studierenden weitergegeben werden. Deshalb sollen alle Themen zusammen stehen.
          \2 Daniela: Die Studierendenschaft wäre ein extra Adressat, für die alle Punkte gesammelt werden. Die aktuellen Handlungsaufforderungen sollten getrennt herausgegeben werden.
          \2 Köln: Alle Adressaten sind in alle Themen involviert. Welche Korrelation gibt es zwischen Adressaten und Punkten?
          \2 Wien: Es soll getrennt werden, weil man die Studierenden besser erreicht.
          \2 Bonn: Es soll pragmatisch reingeschrieben werden, was große Massen erreicht.
        \1 Bielefeld: SHK Räte arbeiten schon aktiv gegen die Abschaffung. Diese sollten unterstützt werden.
        \1 Bitte an den AK: Wenn eine Resolution angestrebt ist, dann sollte dies heute Abend geschehen, damit sie in der Postersession bearbeitet werden kann.
        \1 Daniela: Die Leute bekommen vier Emails von uns, wenn wir das aufteilen, das unterstreicht den Arbeitsauftrag. Wollen wir einzelne Punkte oder einen Gesammten?
        \2 Köln: Ist es hilfreich viele Mails zu schicken oder den Zusammenhang klarzustellen? Die Entscheidung sollte nach der Postersession getroffen werden.
        \2 Berlin: Gremienarbeit baut auf Jena auf, und hilft auch in Zukunft eine einzelne Meinung zu haben. Modularität fördert die Nutzung in Zukunft.
      \end{outline}
        Leute, die mitmachen wollen, treffen sich heute Abend um acht vorm Tagungsbüro. Treffpunkt ins Wiki und über das Tagungsbüro verteilen. \\
        \underline{Stimmungsbild:} Soll die Resolution als Gesammtbild oder modular gestaltet werden? \\
        \begin{center}
          Alles zusammen 3 - Modularität 10
        \end{center}

      AK Wissenschaftliche Arbeitsbedinungen sollte sich mit uns vernetzen.
