% !TEX TS-program = pdflatex
% !TEX encoding = UTF-8 Unicode
% !TEX ROOT = ../main.tex

\section[AK Bachelor-Börse \& Bacheloranden-Recruiting]{AK Bachelor-Börse und Bacheloranden-Recruiting in der Physik}

  \textbf{Protokoll vom:} 02.06.2018, %???
  Beginn: 09:15 Uhr,
  Ende: 11:00 Uhr \\
  \textbf{Redeleitung:} Kathrin Rieken (Uni Augsburg)\\
  \textbf{Protokoll:} Chantal Beck (Uni Würzburg), Benedikt Bieringer (Uni Münster) \\
  \textbf{anwesende Fachschaften:} TU Graz, Uni Innsbruck, Uni Münster, Uni Graz, Uni Augsburg, FU Berlin, Uni Jena, Uni Dortmund, Uni Wuppertal, Uni Karlsruhe, Uni Potsdam, Uni Bonn, Uni Würzburg, Uni Darmstadt, Uni Osnabrück, Uni Cottbus, Uni Dresden, Uni Erlangen, Uni Rostock, Uni Düsseldorf,

  \subsection*{Informationen zum AK}
    \begin{itemize}
      \item \textbf{Ziel des AKs}: Austausch von Erfahrungen, Positionspapier
      \item \textbf{Folge-AK}: nein
      \item \textbf{Zielgruppe}: Leute, die an der menschenfreundlichen und kommunikativen Weiterentwicklung dezentraler Raumstrukturen interessiert sind
      \item \textbf{Voraussetzungen}: keine
    \end{itemize}

  \subsection*{Protokoll}
    \paragraph{generelle Gliederung}

    \begin{enumerate}
      \item Vorstellung des AKs, Klärung von Fragen, Vorstellung des Konzepts in Augsburg (und Marburg)
      \item Austausch - wie ist es an den anderen Unis?
      \item Diskussion - Sollten Bacheloranden selbst auf die Suche nach Arbeitsthemen gehen oder ist eine Bachelorbörse notwendig (an jeder Uni)? (Falls ja: schriftliche Anleitung für Fachschaften zur Organisation einer Bachelor-Börse)
    \end{enumerate}

    Die AK-Leitung stellt das Augsburger Modell vor: In einer Vorstellung präsentieren sich alle Institute, anschließend kann sich in einer Poster-Session über mögliche Bachelorarbeiten informiert weden. \\
    \begin{itemize}
      \item Graz hat vor einem Jahr versucht, das einzuführen, und versucht es dieses Jahr wieder - wollte ursprünglich nur Vorstellung der Arbeitsgruppen haben.
      \item Karlsruhe hat nur eine Poster-Session (inkl. Arbeitsgruppenvorstellung und Abschlussarbeitsthemen) von 2-3 Stunden Dauer an einem Abend, wird von der jDPG mit der Fachschaft zusammen organisiert.
      \item Würzburg hatte die letzten Jahre einen Bachelor-Infoabend, der wurde letztes Jahr Richtung Messe umstrukturiert (wie Augsburg). Auch organisiert von der Fachschaft. Nächstes Jahr mit Prospekt im Vorfeld mit ersten Infos zu den einzelnen Lehrstühlen.
      \item Dortmund hat keine Poster-Session, sondern Mitte/Ende des fünften Semesters einen Aushang mit Stundenplan, in dem jeder Prof. sich einträgt, um eine halbe Stunde lang seinen Lehrstuhl vorzustellen. Ist relativ offen, Professoren reden auch nachher noch im Büro mit Leuten und bieten teilweise Laborführungen an. Bachelorkolloquium dient der Vorstellung von Studenten für Studenten.
      \item Darmstadt hat im fünften Semester begleitende Vortragsreihe \flqq attraktive Physik\frqq, in der sich alle Professoren für ihre Themen vorstellen. Vorher wird in einer allgemeinen Vorstellung Allgemeines geklärt. Jedes Mal so 2-3 Professoren (1.5 Stunden).
      \item Münster hat Bachelor-Master-Tage im Winter-Semester. Einführender Vortrag, anschließend Poster-Session mit Bachelor-Master vereint. Von Fachschaft organisiert. Arbeitsgruppen stellen sich auf Website vor. jDPG: \flqq Doctors Diaries\frqq: Einblick in Forschung direkt durch Doktoranden.
    \end{itemize}

    Unis, bei denen es keine solche Vorstellung gibt: \\
    \begin{itemize}
      \item Innsbruck: Institute schreiben Arbeiten mit kurzem thematischen Abstract und Zielformulierung selber aus. Es gibt keine Präsentationsveranstaltung.
      \item Düsseldorf: Direkt zu Professoren.
      \item Bonn: Vortragsreihe mit fünf Vorträgen im Semester (30 Professoren, also werden nicht alle Arbeitsstühle ausreichend repräsentiert).
      \item Osnabrück: Berichte von Arbeitsgruppen, auch in Form von Vorträgen
      \item Graz: keine Angebote $\rightarrow$ Studierende haben Fragen, wie sie anfangen können mit der Suche nach einer Bachelorarbeit
    \end{itemize}

    \textit{Augsburg:} Es hat mehr den Anschein eines \flqq Anwerbens\frqq. \\
    Karlsruhe fragt nach wegen der Formulierung \flqq richtet sich an fünftes Semester\frqq. Alle sind sich einig, dass natürlich alle angesprochen werden sollen, auch wenn Fünftsemester Hauptzielgruppe ist. Außerdem Nachfrage, ob ausschließlich für Bacheloranden oder auch für Masteranden.
    Ausgburg: Masteranden können sich so etwas selber suchen.

    \begin{itemize}
      \item Potsdam: pro Arbeitsgruppe circa 10-minütiger Vortrag für Studierende am Ende des vierten Semesters mit Fokus auf mögliche Bachelorarbeit, damit man sich während des fünften Semesters Gedanken machen kann. Zeitpunkt zwar durchaus sinnvoll, aber erneute Veranstaltung im/am Ende des fünften Semester wäre nochmal wichtig. Hinzu kommt eine Ringvorlesung, eingebettet in ein \flqq Schlüsselkompetenzenmodul\frqq: jede Arbeitsgruppe/jeder Prof hat einen Vorlesungsblock Zeit, um die Forschungsarbeit der Gruppe vorzustellen.
      \item Erlangen: 20-minütige Vorträge (arbeitsgruppenspezifisch) mit anschließender Poster-Session. Jede Arbeitsgruppe stellt sich oder Bachelorthemen vor, teilweise geben Arbeitsgruppen oder Institute Flyer mit Themenvorschlägen und Beschreibungen aus.
      Mehr als 40 Vorträge über 4 Tage, aufgeteilt nach Themengebieten. Die bis zu fünf Stunden Vortrag am Tag sind sehr anstrengend, deshalb sind die Poster-Sessions recht kurz. Die Fachschaft kümmert sich um Verpflegung bei der Poster-Session.
    \end{itemize}

    \textit{Nachfrage:} Gibt es Credit-Points, wenn es sich über einen längeren Zeitraum erstreckt/eine Ringvorlesung ist? Nicht der Fall. Augsburg denkt über so etwas nach.

    \begin{itemize}
      \item Wuppertal hat sowas gar nicht. Man geht zu Professoren hin und fragt. Schade ist, dass bei mehreren Lehrstühlen nur ein geringer Teil direkt angesprochen wird. Das Gesamtkonzept funktioniert dort aber auch nicht, weil die anderen Arbeitsgruppen viel zu wenige Bacheloranden abbekommen.
      \item Jena: Einzelne Lehrstühle stellen sich vor. Lehrstühle mit weniger Bacheloranden sind an Fachschaft herangetreten, ob die Fachschaft nicht etwas dagegen machen kann. Liegt meist daran, weil diese Professoren keine Grundvorlesungen halten.
      \item FU Berlin: eine nachmittägige Veranstaltung, in der sich Arbeitsgruppen vorstellen $\rightarrow$ denken darüber nach, wie man das verbessern kann.
      \item Rostock: Es kümmern sich die Fünftsemester selbst um eine Info-Veranstaltung, bei der sich die Arbeitsgruppen vorstellen. Diese findet an einem Tag Anfang des Semsters statt.
    \end{itemize}

    Frage: Wer organisiert das? \\
    \begin{itemize}
      \item Augsburg: Institutsassistent für die Fachschaft
      \item Darmstadt: Studienbüro
      \item Rostock: Semester
      \item Allgemein: Fachschaft, Institut/Uni
    \end{itemize}

    Frage: Wie viele Arbeitgruppen/Lehrstühle gibt es? Zunächst erst einmal Definitionsproblem: was ist Arbeitsgruppe bzw Lehrstuhl?
    \begin{itemize}
      \item 20-$\inf$ Professoren: 12
      10-20 Professoren: 3
      5-10 Professoren: 3
      0-5 Professoren: 0
    \end{itemize}

    Potsdam hat allgemein Ringvorlesungen, die gar nicht Bachelorarbeitsorientiert sind, aber für die Vorstellung der Arbeitsgruppen sind.
    Um allgemeinen Konsens zu klären: \\
    Wer findet eine Veranstaltung in Form einer Bachelor-Börse gut? Alle \\

    Cottbus äußert Zweifel: Inwieweit ist das bei kleinen Unis umsetzbar? (Sudierendenzahlen von 5-10 pro Semester)
    Ziel des AKs ist es auf jeden Fall nicht, sich auf eine feste Umsetzung zu einigen. Stattdessen: was \\

    Karlsruhe: Findet das Konzept eines allgemeinen Infovortrags \flqq Wie geht Bachelorarbeit\frqq gut. (Haben z.B. Würzburg und Münster)

    Würzburg: Hatte einen Infovortrag zu allen Instituten/Fachbereichen gemacht, ist zeitlich eskaliert. Deshalb Auslagerung in Poster-Session. Allgemeiner Info-Vortrag von Studienberater und eventuell Bacheloranden.

    Graz: Poster-Session ist sinnvoll; Infoveranstaltung im Stil von \flqq How-To Bachelorarbeit \frqq auch.

    Münster: Bei Poster-Session sind zusammengehörige AGs geclustert $\rightarrow$ räumliche Strukturierung der Poster-Session nach Fachbereichen

    FU Berlin: Was ist Ziel des AKs? Es wäre sinnvoll, konkreter herauszuarbeiten, wo Vor- und Nachteile der einzelnen Modelle liegen, damit andere Universitäten sich daran orientieren können und das für sie passende Model.

    Würzburg: zu Münster: genaues Gegenteil, Theorie und Experimentalphysik durchgemischt, weil nur sehr wenige zu Theorielehrstühlen gehen.

    AK-Leitung: Was sind Nachteile der Modelle?
    \begin{itemize}
      \item FU Berlin: Nachteil bei Ringvorlesung könnte sein, dass man schon Vorwissen braucht, zu welchen man tatsächlich gehen möchte. Und erstreckt sich über längere Zeit: Motivationsfrage
      \item Würzburg: Vorstellung von allen in Frontalvortrag. Vorteil: geringerer Aufwand als bei Poster-Session; Nachteile: Kurzvorträge, also nur wenige Informationen pro Lehrstuhl und trotzdem sehr lang $\rightarrow$ Konzentration der Studierenden leidet; wenn schon Interesse besteht für bestimmte Lehrstühle muss man sich trotzdem alles anhören \\
      Poster-Session: Lehrstühle mit besserem Ruf werden fast überrannt. Experimentalphysik viel stärker frequentiert (lässt sich eventuell durch Raumstrukturierung der Poster-Session vermeiden)
      \item Karlsruhe: Poster-Session in Foyer über zwei Etagen. Möglicher Nachteil: AGs wissen nicht wirklich, dass es tatsächlich darum geht, Bachelorarbeiten zu bewerben $\rightarrow$ können auf konkrete Nachfrage nur begrenzt antworten
      \item Graz: bei kleinen Universitäten ist Vorstellung aller in Frontalvortrag durchaus möglich; Idee: thematische Aufsplittung
      \item Allgemeine Meinung: bei wenigen AGs Vorstellung aller in Vortrag durchaus sinnvoll, bei mehr AGs Poster-Session oder Ringvorlesung
      \item Würzburg: Posterwände müssen geholt und aufgebaut werden, definitiv mehr Aufwand als \flqq nur\frqq Hörsaal buchen und Laptop anschließen

    \end{itemize}

    Bonn: Frage an Poster-Session-Unis: Ist immer Professor da oder eher Doktorand oder Masterand? Und wie viel Ahnung haben diese tatsächlich?
    \begin{itemize}
      \item Münster: Professoren versuchen da zu sein $\rightarrow$ haben Interesse daran, Bacheloranden zu bekommen. Manche nehmen auch noch Doktoranden mit, die die Professoren unterstützen. Leute mit weniger Wissen findet man quasi nicht.
      \item Erlangen: auch Professoren mit Doktoranden
      \item Würzburg: Professoren mit Doktoranden und Masteranden, teilweise auch Bacheloranden. Für nächstes Jahr: Anzahl an Menschen begrenzen, ein Lehrstuhl kam mit 7 Personen, dafür war nicht genug Platz.
      \item Augsburg: auch Professoren und Doktoranden
      \item Graz: auch Doktoranden betreuen mehr oder weniger inoffiziell Bacheloranden $\rightarrow$ deswegen kann es auch sinnvoll sein, dass diese da sind (als direkte Schnittstelle)
      \item Münster: nicht explizite Begrenzung der Anwesenden, sondern Ausdrucken von leeren Namensschildern im Vorfeld $\rightarrow$ für Würzburg notiert.
      \item Bonn bedankt sich für die Antworten und merkt an, dass es bei ihnen an der Uni bei manchen Professoren nicht vorstellbar ist, dass diese tatsächlich bei einer Bachelor-Börse erscheinen würden

    \end{itemize}

    Dortmund: Nachfrage zu Poster-Session,wird das häufiger angeboten oder einmalige Veranstaltung?
    $\rightarrow$ an den meisten nur ein Nachmittag
    \begin{itemize}
      \item Münster: zweieinhalb Stunden reicht sehr gut; ansonsten kann man danach auch noch zu den Professoren gehen. Für einen Überblick passt das.
      \item Augsburg: Zwar durchaus voll, aber für einen Überblick reicht das.
    \end{itemize}

    \begin{itemize}
      \item Münster: zurück zu Pros und Cons. Website, auf der AGs vorgestellt werden $\rightarrow$ relativ zeitaufwändig. Pro: Poster-Session für Bachelor UND Master mit vorherigem Vortrag klappt sehr gut
      \item Erlangen: Heftchen, in denen mögliche Bachelorarbeiten vorgestellt werden $\rightarrow$ kommt sehr gut an, allerdings durchaus mit Aufwand verbunden
      \item Potsdam: Ringvorlesung alternativ zu Vorträgen, aber durchaus zusätzlich zur Poster-Session $\rightarrow$ Möglichkeit für detailliertere Infos zu AGs
      \item Karlsruhe: Anmerkung, dass manche Institute mögliche Bachelorarbeitsthemen nicht veröffentlichen $\rightarrow$ das könnte in den entsprechenden Fällen durchaus an die Zuständigen in den Fakultäten weitergetragen werden $\rightarrow$ man weiß dann, was möglich ist
      \item FU Berlin: wäre zwar denkbar, aber meist erleichtert Bachelorarbeit nur die Arbeit des Doktoranden/Professoren. Das heißt, wenn sich kein Bachelorand findet, muss es die zuständige Person eben selbst machen $\rightarrow$ zu viel Aufwand, um das in einen Text zu formulieren
      \item Würzburg: ist erstaunt, dass andere Unis Bacheloranden wollen
      \item Augsburg: Ziel ist es durch die frühe Anwerbung, dass die Bacheloranden dann auch für Master und auch eventuell für Doktor bleiben
      \item Münster: Hier werden Bacheloranden durchaus auch für nützliche wissenschaftliche Arbeiten genutzt - es scheint also auch möglich, einfachere Themen zu finden, in denen man sich trotzdem an wiss. Arbeiten gewöhnt.
      \item Potsdam: war in einem anderen AK. wenn Themen nicht veröffentlicht werden, sagen manche Professoren vielleicht zum einen Bacheloranden Nein und zum Nächsten dann Ja. Versteht WÜrzburgs Meinung, aber wäre für mehr Transparenz.
      \item Würzburg: wurde in einem Lehrstuhl explizit diskutiert und sich danach dagegen ausgesprochen $\rightarrow$ zu hoher Administrationsaufwand
    \end{itemize}
    AK-Leitung möchte Diskussion zur eigentlichen Frage zurückführen. \\

    FU Berlin: Was ist, wenn man Probleme hat, Professoren anzusprechen? $\Rightarrow$
    Würzburg: Aus Erfahrung: Professoren sprechen durchaus Bacheloranden an, die schüchtern am Stand vorbeilaufen und auf die Plakate schielen. \\

    Bonn: Wenn Heft mit Bachelorarbeitsthemen unsinnig ist, weil sich das zu schnell ändert: Gibt es an anderen Universitäten Möglichkeiten des Schwarzen Brettes oder ähnliches für Bachelorarbeits-Möglichkeiten
    $\rightarrow$ Wird an den meisten Unis nicht so sehr angenommen (etwa hängen an der FU Berlin noch Doktorarbeiten von vor Jahren) \\

    FU Berlin: Master- und Doktorarbeiten sind geeignter für so etwas, weil das strukturellere und langfristigere Planung erfordert. Für Bachelorarbeiten können sich Professoren in 10 Minuten ein Thema überlegen, da ist das in dieser Form nicht nötig: sinnvoller, Plätze aufzuzeigen, an denen Bacheloranden Infos für eine Bachelor-Arbeit bekommen \\

    Sind Bachelorarbeiten besser ausgearbeitet, wenn die für eine Bachelorbörse gebraucht werden?
    $\Rightarrow$ Allgemein ja. \\
    Münster hat z.B. Professoren, die durchaus schreiben, was der Rahmen der Arbeit ist. Nicht alle, es wurde ja vorhin aber schon angesprochen, dass das je nach Forschungsfeld nicht unbedingt funktioniert (wegen der Schnelllebigkeit). \\

    Wie kommen solche Zettel mit Infos zur Bachelorarbeit an? $\Rightarrow$ Gemischte Erfahrungen werden berichtet. \\
    In Konstanz: Tag der Bachelor- (und inzwischen auch ein Tag der Master-) Arbeiten. Wird vom Fachbereich organisiert und Fachschaft wird dazugenommen: Fertige Bacheloranden berichten davon, was sie in ihren Arbeiten gemacht haben $\rightarrow$ danach noch mit Kaffee und Kuchen; wurde als sinnvoll bezeichnet, aber die Anzahl der Interessierten sank in den letzten Jahren. \\

    FU Berlin: Hilft das eher Leuten, die gerade nach einer Bachelorarbeit suchen, oder eher denen, die bereits in einer Bachelorarbeit sind, und erfahren wollen, wie man damit umgeht?

    Konstanz: Für Viert- oder Fünftsemester, um auch zu wissen, ob man in eine Firma, ins Ausland oder wohinauchimmer gehen möchte.

    Potsdam: Findet es gut, die Vorstellung von Bachelorvorträgen selbst mit der Poster-Session zu kombinieren: Einen Bacheloranden dazustellen.

    Karlsruhe: Findet die Idee gut, sieht nur die Gefahr, dass sich keine fertigen Bacheloranden finden lassen.

    Augsburg: Bei dieser Poster-Session war tatsächlich Bachelorandin anwesend. $\rightarrow$ Kam gut an. Vor allem sind dann noch genauere Infos möglich, wie der Professor seine Studierenden \flqq behandelt\frqq.

    Würzburg: Bei Vortrag im Vorfeld nicht nur Vortrag von Studienberatern, sondern auch ein, zwei Bacheloranden, die das Thema präsentieren. Bei Poster-Session ist tatsächlich genau das der Fall, dass Bacheloranden mit am Stand waren.Man kann aber auch Masteranden dazu stellen.

    $\rightarrow$ Konzept mit möglichen Abläufen für Bachelor-Börse wird noch zusammengeschrieben und dann ins Wiki eingebunden.

    \subsection*{Zusammenfassung}
      Zum Schluss wurden die einzelnen Punkte, bei denen Konsens herrschte, noch einmal diskutiert und ein Beispiel-Programm einer Bachelorbörse erarbeitet. Die schriftliche Ausformulierung dessen ist in der Handreichung (unter \url{https://zapf.wiki/images/9/99/Handreichung_Bachelor-Börse.pdf}) zu lesen. Sie soll Fachschaften helfen, eine solche bei sich einzuführen oder ihr Konzept zu verbessern.
