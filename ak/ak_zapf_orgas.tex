% !TEX TS-program = pdflatex
% !TEX encoding = UTF-8 Unicode
% !TEX ROOT = main.tex

\section{AK Vernetzung der ZaPF Orgas}

	\textbf{Protokoll vom:} 31.05.2018,
	Beginn: 16:30Uhr,
	Ende: 18:30 Uhr \\
	\textbf{Redeleitung:}  \\ %???
	\textbf{Protokoll:}  \\ %???
  \textbf{anwesende Fachschaften:} Uni Siegen, Uni Bonn, Uni Würzburg, Uni Heidelberg, Uni Freiburg, Uni Münster, Uni Innsbruck, Uni Rostock, Uni Greifswald

	\subsection*{Informationen zum AK}
		\begin{itemize}
			\item \textbf{Ziel des AKs}: Informationsaustausch zwischen den Orgas der vergangenen und der kommenden ZaPFen
			\item \textbf{Folge-AK}: ja (findet auf jeder ZaPF statt)
      \item \textbf{Materialien}: wenn möglich Orga-Material vergangener ZaPFen mitbringen
			\item \textbf{Zielgruppe}: ehemalige Orgas und Orgas kommender ZaPFen, sowie Fachschaften, die Interesse haben vielleicht eine ZaPF auszurichten (nicht verbindlich!)
			\item \textbf{Ablauf}: Diskussion der Orga-Struktur, Finanzierung, ...
			\item \textbf{Voraussetzungen}: keine
		\end{itemize}

  \subsection*{Protokoll}
    \begin{itemize}
      \item Svenja (Bonn): Fragt BuFaTas an, die in der gleichen Stadt tagen. Darüber kommt man oft an Schlafpätze.
      \item (Würzburg): Haben eine Halle bekommen, die sonst nicht für Schlafplätze zur Verfügung stehen. Es hilft, wenn man Kontakte nutzt.
      \item Lina (Siegen,Innsbruck): Vielleicht könnte die HD Orga mal erzählen.
      \item Jan(Heidelberg): Viele Infos waren sehr Situationsspezifisch. Oft geht es im Orga-AK um spezielle Dinge.
      \item (Berlin): Vernetzung untereinander sorgt dafür, dass der AK in manchen Teilen nicht mehr so wichtig ist.
      \item Andy(Wü): Was die Tagung an sich betrifft, gibt es etwas wo ihr sagt, das hätten wir anders machen müssen?
      \item Jan(HD): Viele Leute im Vorfeld mit auf ZaPFen genommen, die dann aber nicht dabei geblieben sind. Wenn man Leute mit einbindet, die noch nie auf einer ZaPF waren, dann muss man sich darauf einstellen, dass man andere Vorstellungen hat. \\
      In den letzten Wochen die Struktur über Bord geworfen, weil es sich nicht mehr gelohnt hat. \\
      Thomi: Dann muss man sich daran gewöhnen, dass nicht immer jeder weiß was passiert.
      \item (Berlin): Nicht alles zu Tode diskutieren, das spart manchmal Zeit und Nerven.
      \item Benedikt(Münster): Ist das nicht Struktur?
      \item Jan(HD): Es wird alles spontan verteilt.
      \item Lina(Siegen): Hatten eine Struktur und das war sehr hilfreich. Es wurde auch sehr viel ausdiskutiert, aber das hat dazu geführt, dass auf der Tagung nicht mehr alles ausdiskutiert werden musste.
      \item Carina(Rostock): Zur Vernetzung
      \item Lina(Siegen): Kannte aus jeder Orga zwei Personen, die sie dann angesprochen hat und sie weiterleiten konnten.
      \item Svenja(Bonn): Es gab auch noch die Idee Mumble Sitzungen der Orgas zu machen.
      \item Ansonsten auch StAPF Sitzungen.
      \item Carina (Rostoc): Momentan recht große Fachschaft ca. 20, versuchen jetzt schonmal ein bisschen Vorarbeit zu machen.
      \item Jan (HD): Wollte nicht generell sagen, dass Strukturen unsinnvoll sind. Man sollte sich Gedanken machen welche Struktur die sinnvollste ist.
      \item Thomi(HD): Eine gute Wahrnung ist, dass die Leute, die die ZaPF am Anfang planen, sind nicht unbedingt die, die sie am Ende durchführen. Auch wenn es viele Leute gibt, kann es viele Ausfälle geben und darauf muss man dann auch reagieren.
      \item Jan(HD): Wenn man genau Zuständigkeiten hat und dinge nicht erledigt werden muss man auch mal Leute rausschmeißen.
      \item (Berlin): Mann muss deadlines setzten und bei nicht Einhaltung kosequenzen ziehen (Leute aus der Orga schmeissen).
      \item Benedikt(Münster): Regelmäßig absprechen. Soll es eine Standard ZaPF sein oder will man neue Dinge ausprobieren? Man muss aufpassen, dass das was die Bereiche machen zum Gesamtkonzept passt.
      \item Elli (TUB): Nicht sofort Resors verteilen, die erst in einem Jahr wichtig sind. Jetzt könnt ihr Arbeit in Dinge stecken, für die ihr später keine Zeit mehr habt. Grade bei mehr Fachschaften muss darauf achten, dass man Kompetenzen abgibt. Man muss dann nur noch überprüfen, ob die Zeitfristen passen.
      \item (Berlin): Treffen über Mumble können dafür sorgen, dass Orga Leute, die an diesen treffen nicht anwesend waren Informationen nicht mitbekommen und getroffene Entscheidungen hinterfragen.
      \item Lina(Innsbruck): Waren auch in verschiedenen Städte und haben sich daher viel über Mumble geredet aber man soll sich auch persönlich zusammensetzen.
      \item Carina: Wie weit kann und sollen sich Finanzen und Sponsoring unterscheiden?
      \item Jenny: Es ist sinnvoll, wenn Finanzen und Sponsoring zwei verschiedene Personen sind, da eingehende Zuschüsse direkt verechnet werden müssen  (doppelte Buchführung)
      \item Jens(Siegen): Kommt auf die Größe der Orga an.
      \item Andy(Würzburg): Macht die Würzburger Finanzen aber macht nicht das Sponsoring.
      \item Johannes(Bonn): Macht in Bonn Finanzen und Sponsoring. Adressen raussuchen und Briefe falten kann man auf viele Leute verteilen. Man kann auch einzelnen Menschen einzelne Unternehmen zuweisen.
      \item Andy(WÜ): Sponsoring hängt davon ab, wie gut man deligierne kann. Einer sollte den Überblick behalten.
      \item Würzburg: Fachschaftsraum für zwei Tage sperren und zum Call-Center machen.
      \item (Berlin): Beide Aufgaben sehr viel Verantwortung; das möchten die meisten nicht alleine machen. Aber sollten zusammenarbeiten, sodass der eine dem anderen über die Schulter schaut.
      \item Carina(Rostock): Kriegen wir unser eigenes Konto, wie läuft das mit dem ZaPF e.V.? \\
          $\rightarrow$ Andy(WÜ): Ihr bekommt einen Finanzer für die ZaPF in Rostock/Greifswald. Dann bekommt jede ZaPf ein eigenes Unterkonto.
      \item Jens(Siegen): Wenn ihr Sponsoring macht, dann könnt ihr das auch über das FSR Konto machen.
      \item Andy(Wü): Das BMBF setzt eine gewisse Menge an Eigenbeitrag und Drittmittelbeitrag voraus, um gefördert zu werden.
      \item (Bonn): Hat gehört. Fachschaftskassen von ZaPF-Kassen trennen.
      \item Jenny (Berlin), Marcel (Bonn): Stimmen zu
      \item (Berlin): Was ist BMBF-förderbar und was nicht:
         \begin{itemize}
           \item nicht förderbar: Teilnehmika T-Shirts, Tassen
           \item förderbar: Helfer-Shirts
         \end{itemize}
      \item Für Essen beim Studierendenwerk nachfragen. Fragen, ob für externe Studierende auch der Studipreis gilt.
      \item Foodsharing (Brötchen, Brot) ist eine gute Alternative zu alles beim Supermarkt kaufen.
      \item Benedikt: Es ist immer hilfreich, wenn man mit anderen Orgas redet, dass man ihnen bei bestimmten Dingen über die Schulter schauen kann.
      \item Carina():Wie habt ihr zusammengearbeitet?
      \item Elli(TUB):Wir haben Open Project verwendet, es war total unintuitiv.
      \item Benedikt: Das Open Project kann sehr viel aber es ist nicht unintuitiv. Viel über Pads geregelt. Hat halbwegs funktioniert. Kommunikation über Messenger und Mails.
      \item Johannes: Es gibt ein wiki, dort wird alles festgehalten was erarbeitet wurde.
      \item Svenja: TeamGeant, ist ein Zeitplaner und ähnlich zum Open Project. Man kann Dinge abhaken
      \item Jan: Wekan Irgend ist wie ein ToDo. Haben ein Wiki verwendet. Vieles was es gibt, aber man muss dran glauben.
      \item Andy: Würzburg verwendet Easy Note. Damit kann man Aufgaben sortieren und zuweisen. Hat eine Kalenderfunktion, die sie aber wenig verwenden.
      \item Jens: Egal was man verwendet, es müssen alle mitmachen.
      \item Thomi: Es sind immer alle scharf auf solche Software aber eigentlich ist es nur wichtig, dass irgendwo auch analog Leute einen Überblick haben was passieren muss.
      \item Johannes: Verwenden generell Slack, dort kann man auch Untergruppen erstellen.
      \item Chris: Grade um Leuten in den Arsch zu treten sind regelmäßige Treffen wichtig. Auch wenn Treffen lustig sein sollen, manchmal muss man auch Leuten in den Hintern treten.
      \item Jenny: Jedes Treffen muss ein Ziel haben (Treffen zum Aufgaben erledigen, Treffen zum Diskutieren, Treffen zum grillen und chillen)
      \item Carina():Wussten nicht, wie man mit alten Orgas kommunizieren kann. Würden gerne den Kontakt zu den Orgas herstellen.
      \item Lina: Hatten nach Siegen zu einer Mumble Sitzung eingeladen und vorher Feedback von Teilnehmika eingeholt. Ein halbwegs regelmäßiges Treffen der Orgas wäre sehr sinnvoll.
      \item Tobias (Bonn): Wie viel wurde auf der ZaPF getrunken?
      \item HD: Es wruden 25 Kästen Wasser getrunken in den ersten 24 h.
      \item Jenny: Nicht alkoholische Getränke sind idR BMBF gefördert
      \item HD: Große Getränke Händler geben dinge auch auf Kommission raus
      \item Jens: Hatte aus Siegen eine Liste gemacht.
      \item Johannes: Es sollten Konsumgüter ins Wiki eingetragn werden, damit die orgas der nächsten ZaPFrn besser planen kann.
      \item Thomi: Mal bei Aachen anfragen, wie groß die Mengen waren, die getrunken wurden.
      \item Jan: Zwei wichtige Dinge gelernt.1) Wenn es im Orga Team einen unterschwelligen Konflikt gibt, dann klärt ihn gleich. 2) Man muss Personalführung betreiben. Leuten haben Schwächen, man muss darauf achten, was Leute können.
      \item Jenny: man steht besonders kurz vor der zapf total unter stress. Es ist aber alles kein Weltuntergang. Einfach entspannen, dann funktioniert es für alle besser.
      \item Jan: Man sollte vieleicht eine Person haben, die über dem ganzen steht und der orga sagen kann, wenn sich jemand gerade daneben benimmt.
      \item Es ist am wichtigsten die Tagung auszurichten und keine Eventveranstaltung zu bauen.
      \item Jenny: Es kann auch mal stress geben, da muss man dann drübe stehen (?)
      \item Bonn: Wie viele Teilnehmika habt ihr, mit Helfern? Insgesamt ca. 205 Teilnehmika ca.5\item10 Helfika.
      \item Johannes: Was sind typische orga\itemteam größen am anfang und am ende?
      \item Jenny: Es können 5 am anfang und eine person am ende sein, es kann aber auch ganz anders laufen.
      \item HD: So 8 aber zwischendurch sehr abgesackt.
      \item Siegen: Am Anfang so 7 Leute, danach mehr Leute motiviert aber nicht alle dabei geblieben.
      \item Johannes: viele sagen, das ist noch so weit weg.
      \item Lina: Es gibt Aufgaben, die man auslagern kann. Dinge, die nicht regelmäßig gemacht werden müssen. Beschilderung und so ausgelagert. Zur ZaPF hin, wenn es darum geht Schichten zu füllen. Einfach alle anquatschen.
      \item Chris: In HD hat es sich bewährt, dass nich jeder kleine Bereich direkt aufgeteilt wurde, sondern das als Gruppe zu machen und zu verteilen. Wenn man merkt, dass ein Bereich zu groß wird, dann auslagern. Man hat nicht am Anfang gleich 10 Leute, die Helfen, aber das ist ok. Man muss aufpassen, dass Leute nicht abspringen weil sie unterbeschäftigt.
      \item Thomi: Trivialität, haltet alles schriftlich detailliert fest, wenn euch Leute Dinge zusagen!
      \item Jan: Dinge müssen fertig werden!
      \item Andy: Das Program mit dem ihr die AK Verteilung im Plenum gemacht habt, wo bekommt man das her?
      \item Thomi: Damit wäre ich vorsichtig, weil es vor dem Plenum noch sehr schwer war aber es ist wichtig, dass man sich Sachen vorher anguckt und auch restricktiv ist.
      \item Jan(HD): Ist keine Auspacken und fertig Software, weil sie mit der Info FS gemeinsam gemacht haben.
      \item Elli: Schaut euch vorher bei den ZaPFen um, wie die Plenen ablaufen. Open Slides ist nicht nur gut aber hat viel Gutes. Was gut geklappt hat, war Leute vorher zu nerven ihre AKe vorher einzutragen und Bedingungen zu schicken.
      \item Benedikt: HD ist eine FS, die einen eigenen Server hat. Ansonsten kann einem der TOPF sehr gut helfen.
      \item Thomi: Man denkt, dass alles funktioniert, weil es schon auf den letzten ZaPFen funktioniert hat. Das klappt nicht einfach so. Man muss alles für sich einrichten! Dinge sind nicht in Stein gemeißelt. Man muss sich früh genug um Dinge kümmern.
      \item Jan(HD): Gefährliches Wunschdenken, wollte nichts mit der Plenentechnik zu tun haben. Dinge passieren nicht von alleine.
      \item Benedikt: Probiert aus ein Plenum zu simulieren. Das haben sie 3 Monate vorher gemacht, das war einige Arbeit und er hat vor dem Plenum im Hörsaal übernachtet.
      \item Lina: Leute arbeiten unterschiedlich. Man muss ein bisschen Vertrauen in Leute haben.
      \item Thomi: Ihr kennt die Leute, wie sie in der Fachschaft arbeiten. In Berlin wurde die Plenumsleitung zum ersten Mal outgesourct.
      \item Lina: Benni und Benedikt haben sich um die Plenen gekümmert. Adriana hat sich ums Essen gekümmert. Drei andere haben sich ums Orgabüro und Helfika gekümmert. Sie haben das Tagungsbüro abgegeben. Die Zentrale hatte eine ganz andere ZaPf als der Rest.
      \item Carina(Rostock): Wie schnell stand fest, dass ihr Dinge outsourct.
      \item Jens: Redeleitung stand dadurch fest, dass sie zu wenig Leute waren. Tagungsbüro hat sich angeboten, weil sie auf der Klausurtagung zu besuch waren.
      \item Lina: Die erste Orga Sitzung war nur mit Physikern, danach waren dann noch andere dabei. Köln hat angeboten zu helfen. Die haben Marcus gefragt, ob er sich um eine Redeleitung kümmert.
      \item Jan: Haben sich implizit entschieden nichts abzugeben. Das hätte nicht zu ihrer Arbeit gepasst, weil sie sich jeden Tag in der Uni treffen. Wenn wir wissen wie wir zusammenarbeiten, dann funktioniert es besser so.
      \item Jenny: Plenumsleitung ist etwas was man sich nicht zusätzlich antun muss.
      \item Andy: Es gibt einen ganz guten Pool an Personen, die sich mit der GO und Satzung auskennen.
      \item Benedikt: Allgemeine Dinge 1) Plenumsprotokolle wurden unterschätzt. Die Protokollanten müssen genau wissen, wie sie Protokollieren. 2) Tägliches Treffen der Orga auf der Tagung 3) Man muss darauf gefasst sein, dass im Protokoll Dinge nicht
      \item Denkt daran zu schlafen!
      \item Jenny: jeder muss für sich entscheiden, wie protokolliert wird. die protokollanten müssen regelmäßig ausgetauscht werden.
      \item Wenn eine Orga bedarf hat, soll sie über die ZaPFlist einladen. Oder für Kontakte den StAPF fragen.
      \item Themen der Einladung sollen klar sein.
    \end{itemize}
