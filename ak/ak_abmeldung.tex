% !TEX TS-program = pdflatex
% !TEX encoding = UTF-8 Unicode
% !TEX ROOT = main.tex

\section{AK Flexibler Umgang mit Prüfungsan- und abmeldungen}

  \textbf{Protokoll vom:} 31.05.2018,
  Beginn: 14:00 Uhr,
  Ende: 16:00 Uhr \\
  \textbf{Redeleitung:} Simon Schmitt (Uni Marburg) \\
  \textbf{Protokoll:} Sebastian Schmidt (TU Dresden) \\
  \textbf{anwesende Fachschaften:} Freie Universität Berlin, Humboldt-Universität zu Berlin Rheinische Friedrich-Wilhelms-Universität Bonn, Technische Universität Chemnitz, Brandenburgische Technische Universität Cottbus, Technische Universität Darmstadt, Technische Universität Dortmund, Universität Duisburg-Essen, Technische Universität Dresden, Albert-Ludwigs-Universität Freiburg, Friedrich-Schiller-Universität Jena, Universität Konstanz, Technische Universität München, Westfälische Wilhelms-Universität Münster, Philipps-Universität Marburg, Universität Potsdam, Julius-Maximilians-Universität Würzburg

  \subsection*{Informationen zum AK}
    \begin{itemize}
      \item \textbf{Ziel des AKs}: Resolution, Unterstützung/Überarbeitung eines Resolutionsvorschlags der KaWuM
      \item \textbf{Folge-AK}: nein (schon von der KaWuM besprochen)
      \item \textbf{Zielgruppe}: jeder
      \item \textbf{Materialien}: Resolutionsvorschlag der KaWuM (\url{https://fachschaften.rwth-aachen.de/etherpad/p/Reso_flexiblerePrüfungsabmeldung#})
      \item \textbf{Voraussetzungen}: Durcharbeiten des Resolutionsvorschlags der KaWuM, eigene Meinungsbildung und Diskussion, dann Entscheidung, ob Resolution unterstützt wird oder auch selbst von uns aufgegriffen wird.
    \end{itemize}

  \subsection*{Protokoll}
    \paragraph{Zusammenfassung AK-Gegenstand}
      \begin{itemize}
        \item an einigen Unis ist Prüfungsan-/abmeldung sehr unflexibel, feste Zeitpunkte für An-/Abmeldung, oft Monate vor Prüfung; dafür haben andere Universitäten gar keine An-/Abmeldung (sehr flexibel).
        \item KaWuM hat sich dazu beraten, Probleme zusammengetragen und angefangen eine Resolution zu formulieren; möchte höheres Stimmgewicht und verteilt Resolution über MeTaFa, damit weitere Studierendenvertreter sich dieser anschließen können.
        \item Leider ist dies nicht wirklich umgesetzt und eingearbeitet worden und das Thema ist bei der letzten KaWuM verwaist.
        \item Wir könnten dieses Thema nun annehmen und für die KaWuM weiterbearbeiten.
      \end{itemize}

    \paragraph{Diskussion des Resovorschlags der KaWuM}
      \url{https://fachschaften.rwth-aachen.de/etherpad/p/Reso_flexiblerePrüfungsabmeldung#}
      \begin{outline}
        \1 TU München: \flqq Sollten Zulassungsvorraussetzungen für Module bestehen, so sollten diese Zulassungsvorraussetzungen spätestens zum letztmöglichen Anmeldezeitpunkt erbracht und im Prüfungsamt eingetragen sein.\frqq \\
          $\rightarrow$ streichen und auf Prüfungsanmeldung unter Vorbehalt ändern, da nicht wirklich realisierbar
        \1 Uni Essen: \flqq Automatische Wiederanmeldungen/Zwangs(wieder)anmeldungen zu Prüfungen und Nachprüfungen sollen entfallen.\frqq \\
          $\rightarrow$ Warum können automatische Anmeldungen auch gut sein?
          \2 Uni Bonn: Automatische Anmeldung für die Nachklausur. Beide Versuche werden als ein Prüfungsversuch gewertet. Abmeldungen sind bis Mitternacht vor der Klausur möglich.
          \2 Marburg: Es gibt auch Studiengänge, wo man automatisch nach Studienablaufplan angemeldet wird.
          \2 TU München: Erstis werden zu Grundprüfungen (müssen statt eines NC abgelegt werden, wenn durchgefallen wird exmatrikuliert) automatisch angemeldet.
          \2 Marburg: Automatische Anmeldungen sind meistens schlecht. Das wird nur vielleicht dadurch relativiert, wenn man sich dessen bewusst ist und mit einer kurzen Frist vor der Prüfung noch abmelden kann.
          \2 Cottbus: Wozu braucht man automatische Anmeldungen? Studierende sind alt genug.
          \2 TU München: Könnte auch in eine separate Resolution. Dort ist ein Nicht-Erschienen kein Fehlversuch. Diese Dinge werden unterschieden.
          \2 Dortmund: An-/Abmeldungen per Liste/Mail an den/die ProfessorIn, sehr flexibel, außerdem ist Abmeldung bei Krankheit vor Ort möglich.
        \1 Bonn: \flqq Prüfer müssen Klausuren in passender Anzahl drucken oder Termine für mündliche Prüfungen vergeben\frqq \\
          $\rightarrow$ Die Prüfer wissen, wann die Prüfung ist; außerdem können noch kurzfristig mehr Prüfungen gedruckt werden.
        \1 Anmerkung: Wenn wir das auf der ZaPF beschließen wollen, werden wahrscheinlich einige den Punkt anbringen, ob eine Prüfungsanmeldung überhaupt sinnvoll ist.
        \1 Vorschlag Marburg: Wenn eine Resolution verabschiedet werden soll, dann soll diese so formuliert werden, dass sie nur im Falle einer zwingenden Prüfungsan-/abmeldungen gelten solle (\flqq Wenn es sie schon geben muss, dann doch bitte so\frqq). Erstrebenswert sei, dass man keine Anmeldung/Abmeldung braucht (\flqq Anmeldung mit Anwesenheit\frqq).
        \1 Uni Münster: \textit{Zu Bedenken:} Wie läuft es bei mündlichen Prüfungen mit der Terminfindung?
          \2 TU München: Hier könnte man Dummy-Termine vergeben und den eigentlichen Termin mündlich mit dem Prüfer ausmachen.
        \1 Bonn: Auch schriftliche Prüfungen schwer planbar, da Studierendenzahlen stark fluktuieren, Anmeldung durch Anwesenheit eher schlecht für sie.
          \2 FU Berlin: Hat bereits Anmeldung mit Anwesenheit, funktioniert auch, Abschätzung über Teilnahme PVL (Übungen,...) möglich, es gibt auch keine vorherige Modulanmeldung.
          \2 Bonn: Wenn es so geht, dann wäre es auch gut.
        \1 Freiburg: Prüfungsfristen sehr eng, außerdem automatische Anmeldungen zur Nachprüfungen (welche sehr kurz nach der eigentlichen Prüfung ist) ohne Möglichkeit zur Abmeldung. \\
          $\rightarrow$ Großes Interesse an dieser Resolution.
          \2 Allgemeine Ablehnung gegenüber Verfahren in Freiburg.
        \1 Kurzfristige Abmeldung durch Attest:
          \2 Marburg: Sollte nicht der letzte Ausweg sein, warum geht es nicht auch so, sich kurz vor der Prüfung abzumelden, außerdem diskreditiert es Ärzte (Stichwort Symptompflicht) und Studis.
          \2 Cottbus, Berlin: es ist einfach möglich, sich kurzfristig ein Fake-Attest zu holen, dadurch würden Ärzte in die Lage gezwungen, unethisch zu handeln. \\
            $\rightarrow$ Kann man als Gegenargument bringen, außerdem blockiert dies wertvolle Sprechzeit für ernste Fälle.
          \2 Münster: \flqq Die Hausärzte und insbesondere die Notfallambulanzen werden entlastet und können sich um tatsächliche Krankheits- und Notfälle kümmern.\frqq \\
            $\rightarrow$ Nicht so gut, dies unterstellt Studis, dass sie nicht richtig krank sind, außerdem schadet es der Glaubwürdigkeit von Ärzten.
            \3 Formulierung im Konjuktiv möglich, allgemein sollte dieser Punkt sehr vorsichtig formuliert werden.
        \1 Cottbus: Es gibt viele Widersprüche innerhalb von Universitäten bzgl. Prüfungsan-/abmledung, Attesten etc., eine kurzfristige Möglichkeit zur Abmeldung würde bürokratischen Aufwand verringern.
        \1 Münster: Mögliches Gegenargument (möchte entkräftigt werden): Wenn ich unsicher bin/Prüfungsangst habe, werde ich durch Ablauf der Prüfungsabmeldungsfrist in meiner Entscheidung bestätigt und kann mich auf die Prüfung konzentrieren (\flqq jetzt bin ich drin, jetzt muss ich durch\frqq).
        \2 Man hatte auch jetzt schon immer die Möglichkeit sich kurzfristig durch Fake-Atteste abzumelden.
        \2 Außerdem Möglichkeit zur Selbstdiziplinierung, wenn man jederzeit die Wahl hat sich abzumelden.
      \end{outline}

    \subsubsection{Sammlung zur Ausarbeitung einer eigenen Resolution}
      \paragraph{Gegenstand der Kritik}
        \begin{itemize}
          \item Prüfungsanmeldefristen dienen ursprünglich der besseren Planbarkeit der Klausurenphase, vor allem der benötigten Räume. Durch moderne rechnergestütze Planungsabläufe sind die Bearbeitungszeiten stark gesunken. Daher ist eine Prüfungsanmeldung zwar immer noch notwendig, jedoch kann diese zu einem deutlich späteren Zeitpunkt erfolgen, als es zurzeit an vielen Hochschulen üblich ist. \\
          \item An den meisten Hochschulen ist eine Abmeldung von angemeldeten Klausuren unüblich oder die Fristen sind sehr lang. Dies führt zu einer hohen Zahl an Prüfungsunfähigkeitsmeldungen. Hierbei wird die Wertigkeit einer tatsächlichen Prüfungsunfähigkeit kranker Studierender herabgesetzt. \\
          \item Automatische Wiederanmeldungen zu nicht bestandenen Klausuren sind nicht zeitgemäß.
        \end{itemize}

      \paragraph{Grundsätzliche Forderung}
        \begin{itemize}
          \item Die unterzeichnenden BuFaTas fordern die Prüfungsan- und abmeldung flexibler und zeitgemäßer zu handhaben. In unseren Augen gibt es keinen Grund, warum Studierende zum Teil mehrere Monate vor dem Prüfungstermin von einer Prüfungsanmeldung zurücktreten müssen und sehen in dieser Form der Handhabung unnötige Hürden für Studierende. Die BuFaTas fordern eine Anmeldefrist von maximal einem Monat vor Beginn der Prüfung die Abmeldung soll bis mindestens eine Woche vor der Prüfung ohne Grund möglich sein. Die Prüfungsabmeldung danach soll im besten Fall bis einen Tag vor Prüfungsbeginn ohne Angabe eines Grundes möglich sein. Danach sollte unter Angabe eines Grundes (bei Krankheit sollte eine Arbeitsunfähigkeitsbescheinigung als Beleg genügen) die Abmeldung möglich sein.
          \item Die KIF hat dazu in den letzten Jahren zwei bis drei Resolutionen verabschiedet: \url{https://kif.fsinf.de/wiki/KIF445:Resolutionen/An-_und_Abmeldefristen} und \url{https://kif.fsinf.de/wiki/KIF420:Resolutionen/Pr%C3%BCfungsabmeldung} und \url{https://kif.fsinf.de/wiki/KIF420:Resolutionen/Pr%C3%BCfungsunf%C3%A4higkeit}.
        \end{itemize}

      \paragraph{Argumente für die Forderung}
        \begin{itemize}
          \item Die Hausärzte und insbesondere die Notfallambulanzen werden entlastet und können sich um tatsächliche Krankheits- und Notfälle kümmern.
          \item Prüfungsunfähigkeit wird in den Hochschulämtern ernster genommen als zur Zeit.
          \item Die Prüfungsausschüsse, Prüfungsämter etc. werden entlastet und können ihren eigentlichen Plichten besser und umfassender nachkommen.
          \item Insbesondere an großen Hochschulen kommt es zu logistischen Problemen bei der Verwaltung.
          \item Bei der automatischen Wiederanmeldung werden Studierende je nach Hochschulkultur gezwungen, im vorlesungsfreien Semester ein Modul zu absolvieren. Selbst wenn sie diese Klausur \flqq schieben\frqq können, führen Regelungen zur Höchstanzahl an geschobenen Klausuren dazu, dass Studierende zwangsweise Klausuren schreiben müssen, auch wenn es doch gute Gründe zum Schieben gäbe.
          \item Wenn Forderungen stellen, dann bitte mit konkreten Zeitvorstellungen - keine frei interpretierbaren \flqq flexibel und zeitgemäß\frqq - das kann alles bedeuten.
          \item (Nur zur Hilfestellung: An der TU Berlin ist es gang und gebe sich bis eine Woche vor der Klausur anzumelden und bis einen Tag vorher auch wieder abzumelden.)
        \end{itemize}

      \paragraph{Gegenargumente und Bedenken}
        \textit{Hier werden alle Gegenargumente gesammelt, um diese entkräften zu können.} \\
        \begin{outline}
          \1 Hochschulen müssen sowohl die Raum- als auch die Personalkapazität ihrer Prüfungen planen. (Teilweise passiert das jetzt schon auf Daten des Vorjahres und nicht auf tatsächlicher Anmeldezahl, weil Planung vor Anmeldung stehen muss. Quelle: TU Ilmenau)
            \2 Generell für alle Modulangemeldeten zu planen oder informelle Umfragen zu machen wird dadurch nicht ausgeschlossen.
          \1 Prüfer müssen Klausuren in passender Anzahl drucken oder Termine für mündliche Prüfungen vergeben.
          \1 Für ein angemeldetes Fach lernt man verbindlicher. \textit{Kommentar:} Das ist kein Gegenargument zur flexibleren Abmeldung.
          \1 Das Wahren bzw. Verpassen von Fristen und die daraus resultierenden Konsequenzen gehören zur akademischen Erziehung. Aber: bei kurzen Fristen müssen Fristen auch wirklich scharf sein.
            \2 Ist kein Gegenargument: Die Frist, die Prüfung zu schreiben ist immer noch da, nur liegt es dann in der Verantwortung des Studierenden, diese Frist als solche für sich wahrzunehmen (die Konsequenz wäre bspw., dass die Prüfung erst im nächsten Semester/Jahr wieder geschrieben werden könnte). Mit einer Flexibilisierung bereiten wir Studierende also auch noch mehr auf ihre Zukunft und den Umgang mit Verantwortung vor. Darüber hinaus sollen in einer Fachprüfung die Fachkenntnisse geprüft und benotet werden und nicht das persönliche Zeitmanagement.
            \2 In meinem Prüfungsausschuss laufen einige Anträge an, weil Studierende den zentralen Anmeldezeitraum verpasst haben - das muss eigentlich nicht sein. Auch das entgültige Nicht-Bestehen, wenn sich ein Studierender bei der dritten Prüfung nicht angemeldet hat, sollte nicht sein.
          \1 Studierende sollten nicht ewig Klausuren vor sich her schieben. So verzögert sich das Studium unnötig und das Risiko in hohen Semestern Module endgültig nicht zu bestehen oder zu spät festzustellen, für das Studium nicht geeignet zu sein, steigt.
        \end{outline}
