% !TEX TS-program = pdflatex
% !TEX encoding = UTF-8 Unicode
% !TEX ROOT = ../main.tex

\section{AK CHE}

	\textbf{Protokoll vom:} 31.05.2018,
	Beginn: 08:12 Uhr,
	Ende: 10:00 Uhr \\
	\textbf{Redeleitung:} Valentin (HU Berlin) \\
	\textbf{Protokoll:} Sonja Gehring (Uni Bonn) \\
	\textbf{anwesende Fachschaften:} Universität Augsburg, Humboldt-Universität zu Berlin, Rheinische Friedrich-Wilhelms-Universität Bonn, Brandenburgische Technische Universität Cottbus, Heinrich Heine Universität Düsseldorf, Universität Dresden, Albert-Ludwigs-Universität Freiburg, Ernst Moritz Arndt Universität Greifswald, Martin-Luther-Universität Halle-Wittenberg, Technische Universität Ilmenau, Universität Konstanz, Fachhochschule Lübeck, Carl von Ossietzky Universität Oldenburg, Universität Rostock, Eberhard Karls Universität Tübingen, Karlsruher Institut für Technologie, Universität Wien

	\subsection*{Informationen zum AK}
		\begin{itemize}
			\item \textbf{Ziel des AKs}: Zukunft der Kooperation mit dem CHE
			\item \textbf{Folge-AK}: ja
      \item \textbf{Materialien}: AK-Protokolle vergangener ZaPFen
			\item \textbf{Zielgruppe}: alle ZaPFika, die Wert auf eine faire Beurteilung der Studiengänge legen.
			\item \textbf{Ablauf}: Diskussion der aktuellen Ergebnisse
			\item \textbf{Voraussetzungen}: keine
		\end{itemize}

  \subsection*{Einleitung}
    \paragraph{Inhalt}
      \begin{itemize}
        \item Kurzer historischer Abriss: ZaPF, Rankings, CHE
        \item Bericht vom Fachbeiratstreffen
        \item Ergebnisse des aktuellen Rankings, Zugang im Wiki\footnote{\url{https://zapf.wiki/SoSe18_AK_CHE_Ranking\#Protokoll}}
        Spiegeln die Ergebnisse euren Eindruck von eurer Uni wider, sind die "objektiven" Angaben korrekt?
        Vermittelt das Ranking einen sinnvollen Eindruck von eurer Uni?
        \item Wie weiter mit dem CHE?
      \end{itemize}
      Außerdem gibt es folgende Ideen aus den letzten ZaPFen, die wir mal verwirklichen könnten:
      \begin{itemize}
        \item Ein Infotext zum CHE Ranking für Studieninteressierte, den die Fachschaften nutzen können. Gewissermaßen ein Update hierfür\footnote{\url{https://zapf.wiki/images/2/20/Infotext_CHE.pdf}}
        \item Die Haltung der ZaPF zu Rankings diskutieren. In Dresden wurde ein Thesenpapier entworfen, auf dessen Grundlage ein Positionspapier geschrieben werden sollte... dazu kam es bisher jedoch nicht.
        \item Thesenpapier\footnote{\url{https://zapf.wiki/images/a/ab/Thesen_Rankings_WiSe16.pdf}}
        \item Diskussions-AK in Dresden\footnote{\url{https://zapf.wiki/WiSe16_AK_Diskussion_Rankings_und_CHE_allgmein}}
        \item Richtlinien zur zukünftigen Entwicklung und Zusammenarbeit mit dem CHE entwickeln, damit man nicht immer von vorn beginnen muss.
      \end{itemize}

  \subsection*{Protokoll}
    \subsubsection*{Regulärer AK}
      Viele der Anwsesenden kennen das Ranking nicht, deswegen wird es von Null auf vorgestellt.
        \begin{itemize}
          \item Es ist das größte Hochschulranking in Deutschland.
          \item Daten werden alle 3 Jahre erhoben.
          \item Es gab und gibt viel Kritik am Ranking, seit zwei Befragungsrunden gibt es jedoch einen konstruktiven Dialog zwischen ZaPF und CHE. Mehr Informationen zum CHE und der Arbeit dazu auf der ZaPF gibt es auf der Themenseite: \url{https://zapf.wiki/Kategorie:CHE}
        \end{itemize}

      Es gibt für jedes Fach einen Fachbeirat, mit dem das CHE die Entwicklung des Rankings diskutiert. Die KFP und ZaPF haben standardmäßig Mitglieder im Fachbeirat.
      \begin{itemize}
        \item Die Zeit veröffentlicht Ergebnisse des Rankings im Studienführer (Printausgabe) und als Online-Ranking.
        \item In der Printausgabe werden nur eine Auswahl der erhobenen Indikatoren abgedruckt.
        \item Indikatoren kommen einerseits durch Antworten von Studierenden, andererseits von Angaben der Hochschulen zustande.
      \end{itemize}

      Über den Online-Zugang für Hochschulen werden die Methoden und Ergebnisse des letzten Rankings betrachtet. Folgende Punkte/Probleme sind dabei aufgefallen:
      \begin{itemize}
        \item Grundsätzlich: die Befragungen für den Master sind neu. Wie kann man das umstrukturieren? Was sind Unterschiede zur Darstellung zum Bachelor (z.B. Qualität der Forschung)?
        \item Mail vom CHE sehr chaotisch, nicht einladend an der Befragung teilzunehmen.
        \item Tübingen: Kriterien zur Internationalen Ausrichtung nicht repräsentativ (Tübingen) Kriterien sehr einheitlich gewählt (was nutzen die Studenten), Angebote der Universität aufnehmen/besser darstellen
        \item Ähnliche Sachen/Fächer werden jetzt zusammengeworfen, damit auch Bindestrich-Studiengänge gerankt werden können, teilweise eigenartige Zuordnung
        \item ``Der Indikator englischsprachige Arbeitsgruppen ist (wahrscheinlich vor allem für Abiturienten) missverständlich! Es könnten darunter Lerngruppen verstanden werden, gemeint sind aber Forschungsgruppen!''

      \end{itemize}

      Highlights vom Fachbeirat:
      \begin{itemize}
        \item Das Gesamturteil als einzelne Frage, wurde auf Wunsch von ZaPF und KFP in der letzten Ausgabe (2015) nicht erhoben, ist jetzt auf Wunsch von CHE/Zeit aber wieder abgefragt worden und im Online-Ranking zu finden.
        \item Der Onlineauftritt wird im Heft jetzt besser beworben.
        \item Viele Themenblöcke wurden anders strukturiert, es gab viele Änderungen zum Ranking davor. Diese Änderungen könnten mal bis zur / auf der nächsten ZaPF genauer untersucht werden
      \end{itemize}

      Zukünftige Arbeit der ZaPF mit und zum CHE:
      \begin{itemize}
        \item In den letzten Jahren war es schwierig ZaPFika bei der SaCHE zu behalten. Es besteht weiterhin Interesse an einer Diskussion über das und mit dem CHE.
        \item Vertreter der ZaPF sollten auf jeden Fall am Fachbeirat teilnehmen, notfalls nur als Kontrollfunktion. Man könnte eine Infosammlung als Orientierung (Richtlinien zur gewünschten Entwicklung) zusammenstellen, damit sich die Arbeit für die Vertreter in einem angemessenen Rahmen bleibt.
        \item Es gibt einen alten Infotext der ZaPF für Studienanfänger mit Kritikpunkten am Ranking. Dieses sollte überarbeitet werden, da sich vieles am Ranking geändert hat\footnote{\url{https://zapf.wiki/images/2/20/Infotext_CHE.pdf}}.
        \item In Dresden wurden Thesen zu Rankings im allgemeinen zusammengestellt. Daraus sollte ein Positionspapier entstehen, was allerdings noch nicht passiert ist. (Thesenpapier\footnote{\url{https://zapf.wiki/images/a/ab/Thesen_Rankings_WiSe16.pdf}},  Diskussions-AK in Dresden\footnote{\url{https://zapf.wiki/WiSe16_AK_Diskussion_Rankings_und_CHE_allgmein}})
        \item Es gab in den letzten Jahren eine Taskforce zum CHE, die LEUTe zur SaCHE. Diese hat sich zwischen den ZaPFen bei Mumble getroffen und AKs sowie Fachbeiratstreffen vorbereitet. Diese Arbeit soll fortgestetzt werden, es werden weitere Leute auf die Mailingliste che@zapf.in gesetzt.
      \end{itemize}

      Wie geht es während dieser ZaPF weiter?
      \begin{itemize}
        \item Richtlinien und Infosammlung könnten/sollten zusammen bearbeitet werden.
        \item Abstimmung, welcher der Punkte sollte am ehesten bei der ZaPF bearbeitet werden?
          \begin{itemize}
            \item Richtlinien 9
            \item Infosammlung 8
            \item Thesen aus Dresden 0 $\rightarrow$ hierzu soll es wieder einen AK in Würzburg geben
          \end{itemize}
      \end{itemize}

      Im Back-Up AK werden die Richtlinien erstellt und die Intosammlung überarbeitet.

    \subsubsection*{Back-Up AK}
    Offene Fragen:
    \begin{itemize}
      \item Wie läuft die Kommunikation zwischen KommGrem und CHE, wann und wie trifft sich der Fachbeirat?
    \end{itemize}

    \paragraph{Orientierung für die weitere Arbeit am CHE Ranking}

      Das CHE Ranking wird von Studieninteressierten als Entscheidungshilfe und Informationsquelle genutzt, und hat dabei eine Monopolstellung, was sich erst mal auch nicht ändern wird. Daher ist es unser Anliegen, dass das Ranking diese Aufgabe möglichst gut erfüllt, das heißt:
      \begin{itemize}
        \item hoher, differenzierter und fürs Studium relevanter Informationsgehalt
        \item möglichst wenig vereinfachende Darstellung
        \item Einteilung in Gewinner und Verlierer ist nicht sinnvoll oder hilfreich
        \item zusätzliche und Hintergrundinformationen
        \item Nutzerkompetenzen fördern: Wie nutze ich das Ranking sinnvoll?
        \item Die Studieninteressierten sollen animiert werden, sich aktiv mit den Unterschieden auseinanmderzusetzen und zu überlegen, welche Indikatoren für sie warum interessant sind.
        \item bessere Datengrundlage, höherer Rücklauf, höherer Mindest-Rücklauf
        \item subjektive Fragen können sinnvoll sein, "Studiensituation allgemein" aber nicht
      \end{itemize}

      Durch seine hohe Verbreitung hat das Ranking über die Information von Studieninteressierten hinaus Einfluss auf die hochschulpolitische Diskussion in Deutschland. Das findet die ZaPF problematisch, unser Anliegen ist daher, dass das Ranking sich nicht leicht instrumentalisieren lässt
      \begin{itemize}
        \item wieder: möglichst differenzierte Informationen, keine vereinfachung
        \item keine Werbung mit Ergebnissen nach der Art \glqq\,Uni XYZ hat gewonnen\grqq
      \end{itemize}

    \paragraph{Ideen für die Zwischenzeit und die nächste ZaPF}
      \begin{itemize}
        \item Orientierung von oben und "Auswertung" der Entwicklung der letzten Jahre anhand der Kritikpunkte $\rightarrow$ Thomi und Valentin im Wiki verankern("Zusammenfassung" und "Kritikpunkte"). Dann wird ein Mumble Treffen einberufen
        \item Wie wurde die Befragung durchgeführt, warum do unterschiedlich zwischen den Unis, läst sich das besser machen?
        \item Studienführer anschauen und Darstellung und Erklärungen zur Methodik verbessern
        \item Position zu Rankings fundiert ausarbeiten
        \item Aktuellen Infotext für Studieninteressierte erstellen
        \item Informationen zu Rankings und CHE im Studienführer der ZaPF
      \end{itemize}

    \paragraph{Ideen für die Zukunft}
      \begin{itemize}
        \item Wie läuft die Kommunikation zwischen KommGrem und CHE, wann und wie trifft sich der Fachbeirat? Daürber könnte man mit dem CHE reden.
        \item Eventuell aktiver an die Medien gehen mit unserer Kritik/Position.
        \item Wenn wir eine kritische Position formuliert haben, eventuell zur Teilnahme aufrufen, um eine bessere Datenbasis zu schaffen.
      \end{itemize}

    \paragraph{Infotext für Studieninteressierte}
      Ideen für einen Text: (Abgeändert vom alten Text aus 2012) \\

      Vorsicht mit dem CHE-Ranking im ZEIT-Studienführer! \\
      Viele von euch werden sich wohl im Hinblick auf ein Physikstudium über die verschiedenen Universitäten und ihre Vor- und Nachteile informieren. Ein sehr weit verbreitetes Infomaterial ist der sogenannte Studienführer der ZEIT. Darin befindet sich das CHE-Ranking, bei dem Universitäten nach verschiedenen Aspekten bewertet und in Ranggruppen eingeordnet werden.
      Eine Tabelle mit nur drei Farben liest sich verlockend einfach. Aber dadurch wird eine Aussagekraft suggeriert, die in Wahrheit nicht erreicht wird.
      Es gibt nicht die eine beste Uni -  welche Uni für euch am besten ist, hängt von ganz vielen und unterschiedlichen Faktoren ab.
      Im Studienführer werden leider auch nur fünf Indikatoren abgedruckt - das Ranking selbst erhebt aber viel mehr! Wenn ihr das Ranking nutzt, dann schaut euch auf jeden Fall online alle Indikatoren an und überlegt euch, welche für euch persönlich interessant und wichtig sind. \\

      Setzt euch bitte kritisch mit solchen Rankings auseinander! \\

      Die Frage, sich für eine Uni zu entscheiden, ist ein persönliche und sollte nicht von einer Tabelle beantwortet werden. Aber keine Sorge, es gibt alternative Entscheidungshilfen: Im Studienführer der Physik-Fachschaften (\url{https://studienführer-physik.de}) findet ihr eine wertungsfreie, ausformulierte Übersicht über Physikstudiengänge an vielen deutschen Hochschulen, zum Beispiel zu inhaltlichen Schwerpunkten.
      Habt ihr eine Vorauswahl von einigen wenigen Universitäten getroffen, empfehlen wir euch, Kontakt mit den Fachschaften aufzunehmen. Schreibt uns eine Mail, wir helfen euch gern! 

      \underline{ToDo:} Kritikpunkte aufführen, um das Ranking kritisch einordnen zu können!
      Kommentare: Bringen wir Studieninteressierte dazu, das Ranking zu benutzen, die es noch nicht kennen?
      Wir wollen nicht für das Ranking werben, deswegen müssen wir Kritik klar formulieren und nicht nur schreiben, wie man das Ranking einigermaßen vernünftig werwenden kann. Für diese Kritik brauchen wir eine Position der ZaPF.
      Eventuell den Link zu den Fehlerbalkendiagrammen und die Methodik ein bisschn erklären.

    \paragraph{Neue LEUTe zur SACHE}
      Die Taskforce LEUTE (für Lieblings Engagierte in Ungewählter TaskforcE) zur Sacharbeit zum CHE-Ranking (kurz: SACHE), bestehend aus Mirja Granfors(TU Dresden), Jacob Brunner (), Stephan Hagel (Uni Gießen), Kathrin Rieken (Uni Augsburg) und Andre Jakubowski (Uni Bonn) wird bis zur Winter-ZaPF 2018 eingerichtet. \\

      Gemeinsam mit dem Kommunikationsgremium hat sie folgende Aufgaben:
      \begin{itemize}
        \item Kontakt zum CHE halten
        \item Kritik und Verbesserungsvorschläge zum Ranking (z.B. zum Fragebogen) erarbeiten und diskutieren
        \item Bei zeitkritischen Anfragen die ZaPF in Verhandlungen zum CHE-Ranking vertreten
        \item Einen entsprechenden AK zur nächsten ZaPF vorbereiten.
        \item Sie berichtet dem StAPF und auf der nächsten ZaPF
      \end{itemize}

  \subsection*{Zusammenfassung}
    Das CHE-Ranking allgemein und die bisherige Zusammenarbeit der ZaPF mit dem CHE wurden vorgestellt, sowie vom letzten Fachbeiratstreffen berichtet. Es besteht weiterhin Interesse an einer Diskussion darüber und mit dem CHE und Vertreter der ZaPF sollen auf jeden Fall am Fachbeirat teilnehmen. Als nächste Schritte sollen Richtlinien für die Vertreter der ZaPF im Fachbeirat zusammengestellt, die Infosammlung der ZaPF zum Ranking überarbeitet und ein Positionspapier aus den Thesen aus Dresden zu Rankings allgemein erstellt werden.
