% !TEX TS-program = pdflatex
% !TEX encoding = UTF-8 Unicode
% !TEX ROOT = main.tex

\section{AK barrierefreie Hochschule}

  \textbf{Protokoll vom:} 31.05.2018,
  Beginn: 14:00 Uhr,
  Ende: 15:10 Uhr \\
  \textbf{Redeleitung:} Peter Steinmüller (KIT)\\
  \textbf{Protokoll:} Peter Steinmüller (KIT) \\
  \textbf{anwesende Fachschaften:} Technische Universität Berlin, Technische Universität Darmstadt, Technische Universität Dresden Georg-August-Universität Göttingen, Universität zu Köln, Ludwig-Maximilians-Universität München Philipps-Universität Marburg, Universität Rostock, Karlsruher Institut für Technologie, Julius-Maximilians-Universität Würzburg, Universität Wien

  \subsection*{Informationen zum AK}
    \begin{itemize}
      \item \textbf{Ziel des AKs}: Vermittlung von Best Practices im Umgang mit jDPG-RGs
      \item \textbf{Folge-AK}: nein
      \item \textbf{Zielgruppe}: Alle, bei denen es eine jDPG-Regionalgruppe vor Ort gibt
      \item \textbf{Ablauf}: Austausch
      \item \textbf{Voraussetzungen}: Informieren unter \url{https://jdpg.de/rg}
    \end{itemize}

  \subsection*{Einleitung}
    Auf der ZaPF in Siegen wurde besprochen, welche Probleme Studierende mit Kind im Studium vorfinden. Dabei wurde angesprochen, dass manche Probleme nicht auf diese Gruppe reduziert werden können, sondern auch andere Studierende betreffen. Daher wurde angedacht einen AK zur Barrierefreien Hochschule zu machen, der sich mit körperlichen, geistigen und privat benachteiligten Studierenden befassen soll. Privat bezieht sich hierbei auf finanzielle oder familiäre Beeinflussung, wie beispielsweise die Versorgung von Verwandten.

  \subsection*{Protokoll}
    Der AK wird vorgestellt. \\
    Dabei wird angemerkt:
    \begin{itemize}
      \item Marburg: Stadt ist gut für Sehbehinderte, ist aber eher der Stadt zuzuschreiben.
      \item Köln: Aktuelle Bausituation ist eher schlecht für beispielsweise Sehbehinderte.
      \item Darmstadt: Situation von Gebäude zu Gebäude unterschiedlich. Für das Studium ist die Aufstellung deutlich besser.
    \end{itemize}

    An ein paar Unis sind Fälle bekannt. Ob die Fälle gut abgefangen werden ist aber eher unklar.
    \begin{itemize}
      \item Göttingen: wird über Mentoring Programm versucht abzufangen. Allerdings auch Fälle bekannt, die das Programm nicht annehmen.
      \item LMU: Sichtbarkeit an Unis erhöhen.
      \item Köln: Es ist auch für manche unklar, dass sie einen Ausgleich bekommen würden.
      \item Wien: Die Studierendenschaft kümmert sich um ein enstprechendes Angebot und bewirbt dieses zu Beginn der Semester.
      \item Darmstadt: Datenbank anlegen, in der ein aktiver Austausch statt finden soll, was an welcher Uni bereits existiert und was eventuell für Andere interessant sein kann.
      \item München: LMU ist größtenteils barrierefrei ausgebaut, aber mit Umwegen verbunden, das kostet Zeit.
      \item Karlsruhe: ähnliche Probleme
      \item Rostock: Wie sieht es im Brandfall aus, wenn sich Personen in Räumen aufhalten, die nur über Fahrtstuhl erreichbar sind?
    \end{itemize}

    Von Interesse für diesen AK soll neben den körperlichen Barrieren auch die soziale Komponente sein (Studieren mit Kind, Pflege eines nahen Verwandeten, etc.)
    \begin{itemize}
      \item München: Teilzeitstudium gibt die Möglichkeit nebenher zu arbeiten, um das Studium zu finanzieren (eine kurzfristige Umstellung zum Vollzeitstudium ist aber nicht möglich). Ein technisches System hilft bis zu zwei Gehörlosen einer Vorlesung zu folgen.
      \item Darmstadt: Wechsel von Teil- zu Vollzeitstudium ist ohne großen Aufwand möglich.
    \end{itemize}

    \paragraph{Idee:}
      Datenbank im Studienführer, dazu Sammlung erstellen im ZaPF-Wiki (siehe Zusammenfassung)
      München: eine Kategorisierung der Barrierefreiheiten je Uni.
      Bitte im Wiki angeben, welche Methoden zur Barrierefreiheit genutzt werden.
      Gegebenenfalls vorher an der Uni informieren. \\

      Vorschläge, um entsprechende Studierende zu erreichen sind: Vorstellung in Vorlesungen, Sprechzeiten, Beauftragter für Barrierefreiheit in der Fachschaft. \\

      Es hilft, wenn aktive Fachschaftler ihre eigenen Situationen erklären, um so Hemmungen abzubauen.

    \subsection{Zusammenfassung}
      Zur Beantwortung der oben genannten Punkte wurde eine Datenbank im Wikipedia der ZaPF eingerichtet.
      In dieser Datenbank sollen Punkte gesammelt werden, welche Projekte Universitäten bereits haben, um Menschen mit Handicap oder familiärer Verantwortung zu unterstützen und ein problemloses Studium zu ermöglichen.
