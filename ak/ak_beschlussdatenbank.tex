% !TEX TS-program = pdflatex
% !TEX encoding = UTF-8 Unicode
% !TEX ROOT = main.tex

\section{AK Beschlussdatenbank}

	\textbf{Protokoll vom:} 31.05.2018,
	Beginn: 10:30 Uhr,
	Ende: 12:20 Uhr \\
	\textbf{Redeleitung:} Opa (Alumni) \\
	\textbf{Protokoll:} Anna (Uni Kiel) \\
	\textbf{anwesende Fachschaften:} Uni Freiburg, Uni Tübingen, Alumni, Uni Kiel, FU Berlin, Uni Frankfurt

	\subsection*{Informationen zum AK}
		\begin{itemize}
			\item \textbf{Ziel des AKs}: Wie arhivieren wir die Beschlüsse der ZaPF?
			\item \textbf{Folge-AK}: ja, (WiSe `17 Siegen \url{https://zapf.wiki/WiSe17_AK_Beschlussdatenbank})
			% \item \textbf{Vorwissen}: altes Protokoll lesen (\url{https://zapf.wiki/WiSe17_AK_Vorläufige_Verträge_für_Abschlussarbeiten})
      \item \textbf{Materialien}: siehe altes Protokoll
			\item \textbf{Zielgruppe}: alle, die die Ergebnisse der ZaPF nicht für völlig egal erachten und eh vergessen
			% \item \textbf{Ablauf}: Thematik auffrischen, holistischen Lösungsansatz suchen, Resolution schreiben, Vorlage für Vertrag schreiben
			% \item \textbf{Voraussetzungen}: keine
		\end{itemize}

  \subsection*{Protokoll}
    Zunächst wird die Problematik besprochen: Es ist schwer, einen Überblick über die verabschiedeten Resolutionen und Arbeitskreise (AK) zu behalten. Auf der letzten ZaPF wurde auch beschlossen, dass dies bis zur nächsten ZaPF verbessert wird, dies ist leider noch nicht geschehen.
    Mitten im AK kam die Nachricht von Jan (FU Berlin), dass er bereits am Erarbeiten eines Systems ist.

    \paragraph{Diskussion}
      Es wird festgestellt, dass auf jeden Fall eine bessere Struktur notwendig ist. Ob eine komplett neue Datenbank die sinnvollste Lösung ist, wurde nicht beantwortet. Da dies mit einem sehr hohen Arbeitsaufwand verbunden ist, wurde weiter überlegt, wie man das Wiki verbessern könnte.
      Allgemein ist die Struktur des Wikis schon gut, allerdings ist der Inhalt teilweise nicht up-to-date, beziehungsweise gibt es Überlappe. \\

      Folgende Struktur wird festgestellt: \\

      \begin{itemize}
        \item Die Seite zur aktuellen ZaPF: Hier gibt es Verlinkungen zu den einzelnen AKs mit Protokollen/Zielsetzung.
        \item ``Resolutionen'' ist die chronologische Auflistung der Resolutionen.
        \item ``Themen und Projekte'' umfasst eine Sammlung von größeren Themen, die mehr oder weniger gut zusammenfassen, was in diesem Bereich bereits passiert ist.
      \end{itemize}
      Weiter gibt es dann noch die Möglichkeit, Einträgen eine oder mehrere Kategorien zuzuteilen. Hier kommen wir zu den genannten Überschneidungen. Es gibt zum einen redundante Kategorien und dann wiederum Themen und gleichnamige Kategorien.
      An sich ist die Struktur wie gesagt gut.
