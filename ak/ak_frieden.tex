% !TEX TS-program = pdflatex
% !TEX encoding = UTF-8 Unicode
% !TEX ROOT = main.tex

\section{AK Friedensanliegen}

	\textbf{Protokoll vom:} ???,
	Beginn: 16:30 Uhr,
	Ende: 18:30 Uhr \\
	\textbf{Redeleitung:} ??? \\
	\textbf{Protokoll:} ??? \\
	\textbf{anwesende Fachschaften:} ???

	\subsection*{Informationen zum AK}
		\begin{itemize}
			% \item \textbf{Ziel des AKs}:
			\item \textbf{Folge-AK}: Teilweise
			\item \textbf{Vorwissen}: alte Protokolle lesen (\url{https://zapf.wiki/Vortrag_und_Diskussion_mit_Ican_(Friedensnobelpreisträger)_über_Atomwaffenverbot}, \url{https://zapf.wiki/SoSe17_AK_gesellschaftliche_Verantwortung_und_Zivilklausel})
      		\item \textbf{Materialien}: \url{https://zapfev.de/resolutionen/sose17/gesellschaftlich_verantwortung/PosPapier_gesellschaftliche_verwantwortung.pdf}
			\item \textbf{Zielgruppe}: Friedensfreunde
			% \item \textbf{Ablauf}:
			% \item \textbf{Voraussetzungen}:
		\end{itemize}

  \subsection{Einleitung}
    Die Mehrheit der deutschen Bevölkerung ist gegen die Auslandseinsätze der Bundeswehr, troztdem werden sie immer wieder verlängert. Auch der im letzten Jahr ausgehandelte internationale Atomwaffenverbotsvertrag wird von einem Großteil der Bevölkerung befürwortet, aber bislang nicht von der Bundesregierung unterzeichnet. Die ZaPF hat in Berlin ein Positionspapier beschlossen, in dem sie sich dafür ausspricht, dass ``die Hochschulen ihren Beitrag zu einer gerechten, nachhaltigen, friedlichen und demokratischen Welt'' entwickeln; dennoch macht sich dies im Alltagsbetrieb selten bemerkbar. \\

    Wie kommt das und wie können wir das Friedensanliegen mehr zur Geltung bringen?

  % \subsection*{Protokoll}
  %   \textbf{NICHT vorhanden}
