% !TEX TS-program = pdflatex
% !TEX encoding = UTF-8 Unicode
% !TEX ROOT = ../main.tex

\section{AK barrierefreie Hochschule}

  \textbf{Protokoll vom:} 31.05.2018,
  Beginn: 14:00 Uhr,
  Ende: 15:10 Uhr \\
  \textbf{Redeleitung:} Jan Naumann(FU Berlin) \\
  \textbf{Protokoll:} Niklas Westermann(FU Berlin) \\
  \textbf{anwesende Fachschaften:} FU Berlin, Würzburg, TU Berlin, Potsdam, Bonn, Konstanz, Bochum, Rostock, Dresden, Uni Wien

  \subsection*{Informationen zum AK}
    \begin{itemize}
      \item \textbf{Ziel des AKs}: ZaPF-IT-Arbeit besprechen
      \item \textbf{Folge-AK}: nein
      \item \textbf{Materialien}: \url{https://zapf.wiki/TOPF}
      \item \textbf{Zielgruppe}: IT-Menschen, Orgika zukünftiger ZaPFen und andere Leute, die Ideen für die IT einbringen wollen
      \item \textbf{Ablauf}: Offene Diskussion und Vorstellung des TOPFs
      \item \textbf{Voraussetzungen}: Laptop ist hilfreich
    \end{itemize}

  \subsection*{Protokoll}
    \paragraph{Diverse Infos aller Art}
      \begin{itemize}
        \item Server bei Strato und Hetzner (bei diesem fast alles).
        \item Klemens tritt zurück, dementsprechend ist das Amt vakant
        \item Henkel werden nicht gewählt
        \item Aufgaben sind Administration, Fragen beantworten,…
        \item Zeitaufwand ist eher punktuell
        \item Eingewiesen wird gerne, man muss sich nur auf einem der Kanaäle melden
        \item Die Container werden über ensembl gemanagt
      \end{itemize}

    \paragraph{Wieso wurde das Anmeldesystem nicht von HD genutzt?}
      \begin{itemize}
        \item Sie haben es selbst ausgewählt
        \item Es sind Mails verschwunden, deshalb ist eine Mail untergegangen
      \end{itemize}

    \paragraph{Wie kommen wir an die Domains/Dienste (Fragen von Bonn)?}
      \begin{itemize}
        \item Wir haben das zapf.wiki, zapf.in, studienführer-physik.de, zapfev.de
        \item Es gibt ein Anmeldesystem und das Engelsystem
        \item Außerdem gibts die app (app.zapf.in) oder die von Lennart (HD)
        \item Hauptrepo ist das ZaPF-Git-Repo auf github
        \item Protokollieren der AKs über Pad und/oder Wiki
        \item Im Plenum wurde Openslides verwendet
      \end{itemize}
    \paragraph{Diverse weitere Punkte}
      \begin{itemize}
        \item Es gibt auf Github Private Repos mit Daten des StAPF
        \item Die Anlegung einer Anmeldung ist noch nicht besonders voran gekommen, das Wiki ist das größte Problem.
        \item Jan schreibt E-Mails, falls Aufgaben anfallen und weist die Henkel entsprechend ein
      \end{itemize}
      
      Jan stellt das Handlungspapier vor, es gibt keinen großen Widerspruch.
