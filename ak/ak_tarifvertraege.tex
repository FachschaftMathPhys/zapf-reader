% !TEX TS-program = pdflatex
% !TEX encoding = UTF-8 Unicode
% !TEX ROOT = ../main.tex

\section{AK studentische Tarifverträge}

	\textbf{Protokoll vom:} 02.06.2018,
	Beginn: 09:07 Uhr,
	Ende: 11:00 Uhr \\
	\textbf{Redeleitung:} Jenny \& Jan (FU Berlin) \\
	\textbf{Protokoll:} Manuel Längle (Uni Wien) \\
	\textbf{anwesende Fachschaften:} FU Berlin, TU Berlin, HU Berlin, Goethe Universität Frankfurt, KIT, Uni Köln, TU München, Uni Münster, Uni Potsdam, Uni Siegen, Uni Tübingen, Uni Wuppertal, Johannes Guthenberg-Universität Mainz, Uni Darmstadt, Uni Würzburg, Uni Jena, Uni Marburg, Ruhr Uni Bochum

	\subsection*{Informationen zum AK}
		\begin{itemize}
			\item \textbf{Ziel des AKs}: Austausch zum Thema studentische Beschäftigte, vielleicht eine Solidaritätserklärung mit den TVStud in Berlin, Resolution zum Thema "Bezahlung studentischer Beschäftigter"
			\item \textbf{Folge-AK}: nein
      \item \textbf{Materialien}\footnote{\url{https://tvstud.berlin/}, \url{https://zapf.wiki/SoSe15_AK_Hilfskräfte}}
			\item \textbf{Zielgruppe}: Studentische Beschäftigte von Hochschulen und alle Interessierten
			\item \textbf{Ablauf}: kurzer Bericht aus Berlin, dann Diskussionsrunde
			\item \textbf{Voraussetzungen}: keine
		\end{itemize}

  \subsection*{Einleitung}
    Im Berlin findet momentan ein Arbeitskampf der studentischen Beschäftigten (SHKs) zur Erneuerung des bisher einzigen Tarifvertrags für SHKs deutschlandweit zwischen den Hochschulen und den SHKs, organisiert in GEW und ver.di, statt. In dem AK soll zunächst einmal von den Berliner Erfahrungen erzählt werden. Im Weiteren soll dann eine offene Frage- und Diskussionsrunde stattfinden zu dem Thema, bei dem es allgemein über die Arbeitsverhältnisse von SHKs gehen soll. Dort würden wir gerne einen Eindruck erhalten, wie studentische Beschäftigte an anderen Universitäten bezahlt werden.

    Am Ende des AKs könnte eine Reso zum Thema "Studentische Beschäftigte" stehen und eine Solidaritätserklärung mit dem Berliner Arbeitskampf.

  \subsection*{Protokoll}
  In Berlin wird gerade für einen neuen studentischen Tarifvertrag gestreikt. Der zum 01.01.2018 gekündigte Tarifvertrag soll endlich überarbeitet werden. Es gibt auch in anderen Städten Bestrebungen studentische Tarifverträge einzuführen.

    \begin{outline}
      \1 Peter(KIT): Ihr habt einen Tarifvertrag seit den 80ern. Gilt der für alle SHKs?
          \2 Jan: Das gilt für alle SHKs an den Universitäten und Fachhochschulen.
          \2 TUB: Es gibt noch ein paar SHKs, die direkt vom Land beschäftigt werden.

      \1 Marius (TUM): Wie sieht der Vertrag konkret aus?
          \2 Jan: Der Tarifvertrag regelt den Lohn von 10,98€, er wurde das letzte Mal 2001 überarbeitet. Es gibt 30 Tage Urlaub, aber es wird von einer 6 Tage Woche ausgegangen. Lohnfortzahlung im Krankheitsfall gibt es für 6 Wochen. Außerdem sind Vertragslaufzeiten von 4 Semestern, maximal 6 Jahre, und in der Regel 40 Stunden im Monat angesetzt. Zwischendurch wurde in einer einseitigen Maßnahme das Weihnachtsgeld gestrichen. Der ursprüngliche Tarifvertrag von 1996 war an den BachelorarbeitT gekoppelt. Das heißt er war dynamisch und wurde regelmäßig miterhöht. Studierende an anderen Universitäten werden momentan Universitätenpezifisch oder länderspezifisch bezahlt.

      \1 Jan: In Berlin geht es jetzt darum, dass der Lohn seit 2001 bei ~11€ steht. Wenn man alleine die Erhöhung der Lebenshaltungskosten ansieht, dann müsste der Lohn schon bei ~14€ die Stunde liegen. Bei normalen Arbeitnehmern zahlt nach 6 Wochen die Krankenkasse Krankengeld. Dies gilt nicht für Studierende, da ihr Hauptberuf ist zu studieren.

      \1 Paul(Köln): Wie kam es zu dem Tarifvertrag?
          \2 Jan: Es gab einen sehr großen Streik von den Berliner TutorInnen. Dort wurde über Monate hinweg gestreikt. Daher gibt es den Vertrag seit den 80ern. Nur seit 2001 ist nichts mehr passiert.
      \1 Was ist der Vorteil von Tarifverträgen?
          \2 Jan: Es ist eine gewisse Sicherheit. Es ist festgelegt, wie viel Geld man bekommt und welche Rechte und Pflichten jeder hat. Wenn es keinen Vertrag gibt, dann sind die Arbeitgeber zu nichts verpflichtet außer den Mindestlohn zu zahlen. Man kann in Tarifverträgen noch andere Dinge einzeln regeln, so dass sie verbindlich sind.
          \2 TUB: Man hat als Einzelperson keine Macht, zu sagen ich arbeite nur, wenn ich so viel Geld bekomme. Als ganze Belegschaft hat man mehr Druck.
      \1 Paul (Köln): Wie ist der Streik damals finanziert worden?
          \2 TUB: 1996 ging der Streik auch gegen die Gewerkschaften, weil die Gewerkschaften es nicht geschafft haben, zum Streik aufzurufen. Jetzt sind GEW und ver.di am Verhandlungstisch und man bekommt Streikgeld bei Ausfällen.
      \1 Jenny (FUB): Gibt es bei euch studentische Tarifverträge?
          \2 Marius (TUM): Tarifvertrag für die SHK angelehnt an TVL (aktuell 10,90/h, Bachelorarbeit 12,60/h, MA 17,20/h).
          \2 Marburg: Es gibt Bestrebungen aber es gibt keinen aktuellen Tarifvertrag. Es gibt sehr aktive Hilfskräfteinitiativen.
      \1 Wie organisiert ihr euch?
          \2 Christian (Marburg): Vor allem über die Landesastenkonferenz.
          \2 Jena: Es gibt keine Tarifverträge. Mit Abitur, aber ohne Bachelorarbeitchelor erhält man den Mindestlohnm mit Bachelorarbeit 10€, mit MA 14€. Der Lohn wird nicht jährlich angepasst.
          \2 Jens(Siegen): 9,70€/h Das Problem ist, dass Leute häufig mehr arbeiten müssen als im Vertrag steht. Im Krankheitsfall muss nachgearbeitet werden.
          \2 Peter (KIT): Man kann über ver.di nach dem TV-L bezahlt werden, aber nur wenn man in der Verwaltung arbeitet. Ansonsten hat das KIT einen eigenen Satz 9€ vor dem Bachelorarbeit und 11€ vor dem MA.
          \2 Jan: Leute, die in der Verwaltung arbeiten und nicht in Forschung oder Lehre eingesetzt werden, müssen eigentlich nach TV-L bezahlt werden. Das machen momentan fast alle Hochschulen illegal. In Berlin gab es auch schon eine erfolgreiche Einklagung, weil IT Hilfskräfte auch nach TV-L bezahlt werden müssten.
      \1 Jan: Es gibt eine Novellierung des Hochschulrechts in NRW. Was steht da zu den SHK Räten drin?
          \2 Jens: Es ist keine richtige Personalvertretung. Man soll sie vertreten aber bekommt nicht so richtig Zugang zu den SHKs.
      \1 Jan: Wo gibt es Personalvertretungen?
          \2 Jena: In Thüringen gelten studentische Hilfskräfte nicht als Personal, sondern als Sachmittel.
          \2 Frankfurt: In Hessen ging es auch lange um diesen Punkt, da das Geld aus dem Sachmitteltopf kommt. Die SHKs sind aber auch nicht wahlberechtig zum Personalrat.
          \2 Christian(Marburg): Ist auch aus Hessen; man versucht ein bisschen parallele Strukturen zu den Festangestellten zu schaffen, aber es gibt keine offizielle Personalvertretung.
          \2 Würzburg: Bei ihnen gibt es auch keine Vertretung. Der Stundensatz wird von der Uni festgelegt. Allerdings setzen die Professoren fest, wie viele Stunden für die Arbeit abgerechnet werden. Ohne Bachelorarbeit verdient man den Mindestlohn. Mit Bachelorarbeit sind es 10€ mit Masterabschluss zwischen 13-14€.
          \2 Jenny(FUB): Es gibt Arbeitsstellen, an denen mehr Stunden angesetzt werden und Fälle in denen weniger Stunden als notwendig angesetzt werden.
          \2 Kathi(FFM): In Frankfurt gab es viele Tarifdiskussionen. Da war der "UnterBachelorarbeitu" sehr aktiv.
      \1 Jenny: Worum geht es euch in diesem Arbeitskreis? Worauf wollen wir hinaus?
          \2 Marius (TUM): Wollen wir es schaffen, dass es deutschlandweit vereinheitlicht wird?
          \2 Marburg: Wir sind nicht alle auf dem gleichen Wissensstand. Es wäre gut, wenn wir uns darüber noch austauschen. Auch die Arbeitsrechtlichen Grundlagen.
          \2 Jan zu TUM: Noch haben wir keine direkten Forderungen für diesen AK, aber sein persönliches Ziel wäre es, dass alle SHKs nach dem jeweiligen TV-L bezahlt werden, wie die anderen Beschäftigten.
          \2 TUM: Vielleicht wäre es gut eine Tabelle mit Infos anzulegen, wie Dinge an den einzelnen Universitäten geregelt sind.
              \3 Wie viel Geld gibt es pro Stunde?
              \3 Gibt es einen Tarifvertrag oder nicht?
              \3 Was passiert im Krankheitsfall? \\
              \3 Noch einen Punkt, in dem Kommentare festgehalten werden.
          \2 Würzburg: Vielleicht noch Beispiele hinzufügen?
          \2 TUM: Das hängt auch davon ab an welchem Fachbereich man ist.
      \1 Wupperthal: Es gibt Ist- und Sollstunden. Hierbei gibt es oft die Regelung, dass man nicht mehr als 50\% mehr arbeiten darf, als im Vertrag festgehalten ist.
          \2 Peter (KIT): Das kommt daher, dass Universitäten Gefahr laufen unter die Mindestlohngrenze zu fallen. Daher soll man das dann für den nächsten Monat aufschreiben.
          \2 Kathi (FFM): Es werden 30 Minuten Mittagspause eingerechnet. Die Stundenzettel werden so eingetragen, dass es passt.
          \2 Peter (KIT): Findet die Stundenzettel bei ihnen gut, weil dort schon die Urlaubsstunden eingerechnet werden. Man muss eine Woche/zwei Wochen am Stück Urlaub nehmen. Damit man nicht nur 20 Mal im Jahr einen Tag Urlaub nehmen kann. In Baden-Württemberg gab es einen neuen Höchstsatz, der vom Finanzministerium vorgegeben wurde. Warum sollten die Universitäten den Höchstsatz bezahlen? Kann sein, dass es nicht das Unibudget ist, weil das Landesamt für Besoldung und Versorgung bezahlt.
    \end{outline}

    \paragraph{Wunsch:}
      Folge AK. \\
      Es soll eine Position gefunden werden, zu der die ZaPF Stellung beziehen kann. Was fordern wir. \\

      Jenny: Aktuelle Argumente in Berlin sind: studentische Beschäftigte sind keine richtigen Beschäftigten. Grade bei Jan: Wir arbeiten alle an einer Uni. Warum sollen wir überhaupt speziell bezahlt werden. Können wir nicht noch etwas festhalten, um den Arbeitskampf in Berlin zu unterstützen. Grundforderung, dass Universitäten die Rechte einhalten. Solidaritätserklärungen mit Berlin, von Einzelpersonen und von der ZaPF wäre sehr hilfreich. \\
      Wer hätte Lust noch an einer Solidaritätsbekundung mitzuarbeiten: Christian, Jenny, Jan.
