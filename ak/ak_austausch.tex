% !TEX TS-program = pdflatex
% !TEX encoding = UTF-8 Unicode
% !TEX ROOT = main.tex

\section{Austausch-AK}

  \textbf{Protokoll vom:} 30.05.2018,
  Beginn: 10:30 Uhr,
  Ende: 12:30 Uhr \\
  \textbf{Redeleitung:} Tobias Löffler (Düsseldorf) \\
  \textbf{Protokoll:} Anna (Kiel), Johannes (Tübingen) \\
  \textbf{anwesende Fachschaften:} Uni Konstanz, Uni Augsburg, KIT, Uni Bielefeld, Uni Innsbruck, Uni Wien, TU München, Uni Bonn, BTU Cottbus, Uni Göttingen, Uni Halle, Uni Chemnitz, TU Darmstadt, Uni Freiburg, Uni Marburg, Uni Frankfurt, Uni Würzburg, HU Berlin, Uni Saarland, Uni GreiFachschaftwald, RWTH Aachen, Uni Jena, Uni Giessen, TU Braunschweig, Uni Osnabrück, LMU München, WWU Münster, Uni Mainz, Uni Dortmund, Uni Dresden, Uni Kaiserslautern, Uni Würzburg, Uni Siegen, Uni Potsdam, Uni Ilmenau, Uni Tübingen, Uni Mainz, Uni Erlangen, Uni TU Berlin, Uni Essen, Uni Rostock, Uni Freiberg, Uni Saarland, Uni Köln, Freie Uni Berlin

  \subsection*{Informationen zum AK}
    \begin{itemize}
      \item \textbf{Ziel des AKs}: Erfahrungsaustausch der Fachschaften
      \item \textbf{Folge-AK}: ja
      \item \textbf{Zielgruppe}: alle
      \item \textbf{Ablauf}: Austausch
      \item \textbf{Voraussetzungen}: Informieren auf der AK-Website nach eingereichten Fragen
    \end{itemize}

  \subsection*{Einleitung}
    Im Austausch-AK können alle Fachschaften Fragen stellen, die an alle oder größere Gruppen gerichtet sind und nur schwer in Einzelgesprächen zu beantworten sind.

    Damit sich die Leitung, wie auch die Teilnehmika darauf vorbereiten können, sollen alle Fragen bereits im Vorfeld ins Wiki eingetragen werden.
    Dazu gehört:
    \begin{itemize}
      \item Die Frage
      \item Der/Die Verantwortliche
      \item nötige Zusatzinformationen
    \end{itemize}

  \subsection*{Protokoll (Fragen)}
    \subsubsection*{Vorbereitungsschriften vor Bachelorarbeit}
      \textbf{Kommt von:} Karola (UP)

      Wie bereitet ihr euch schriftlich auf die Bachelorarbeit vor? \\

      Gibt es Hausarbeiten während des BAs, die eine Übung sind?
      Gibt es Schreibseminare oder ähnliches?
      Ich habe von einigen Studierenden in Potsdam gehört, dass das Schreiben einer Abschlussarbeit schwer fällt, da in Potsdam davor nur Protokolle geschrieben werden. Was gibt es in anderen Unis zur Vorbereitung?

      \paragraph{Antworten}
        \begin{itemize}
          \item (Düsseldorf) Ja, Angebote von der Uni zu theoretischen Schreibseminaren. Außerdem gibt es Protokolle die man schreiben muss mit einer Note und einem Zettel was man falsche gemacht hat, aber nicht die Möglichkeit einer Korrektur
          \item (Bonn): Seminar Präsentationstechniken. An einem Beispiel lernen, wie das geht. Wird als unnötig empfunden. Wie macht man einen Vortrag und schreibt eine Ausarbeitung. Vorbereitung als Pflicht
          \item Bielefeld: Frewilliges Seminar in jedem Semester von einer extra Stelle für solche Kompetenzen, mehr Details nicht vorhanden
          \item Kiel: Wie Bonn
          \item Frankfurt: Seminare, nicht eingebunden in Studienverlauf. auf wissenschaftliche Präsentation ausgerichtet. Sehr allgemein gehalten
          \item TU Darmstadt: Von der Uni-Bib Schreibseminare für Abschlussarbeiten. Zusätzlich Möglichkeit für eine “Mini-Forschung”, die als Vorbereitung für Abschlussarbeit gesehen werden kann.
          \item Dortmund: Protokolle für Praktikum, die auch kontrolliert werden, und bei denen man durchfallen kann.
          \item TUM: Zwei Dinge, (1) für alle Bachelorstudierende bietet der Übungsleiter ein eigenes Seminar an, was beachtet werden muss. Zusätzlich ein Seminar ein Seminar ein Seminar
          \item Göttingen: Freiwillige Angebote, zentrale Schlüsselqualifikationen. Fakultät HilFachschaftseminar a Sprechstunden. Projektpraktikum (?)
          \item Kaiserslautern: Softskill Modul
          \item Saarland: keine Pflichtveranstaltungen im Bachelor, aber im Zentrum für Schlüsselkompetenzen ein Seminar (freiwillig), in dem man über Nacht Nacht das ganze erlernt.
          \item Halle: Schlüsselqualifikation freiwillig, scientific writing. Im Studiengang keine Angebote eingebunden
          \item Freiberg: Extra-Modul für Literaturrecherche (zum Erlernenm Erlernenm Erlernen)
          \item Erlangen: Ab Beginn des dritten Semesters vielfache Vortestate mit Korrekturen, die eingearbeitet werden müssen. Seminar durch Maxplanck–Institut für Bachelor-Master-Studenten, Auch Seminare im Studiengang(?)
          \item Essen: Angebot “Schreibwerkstatt” von der Universität für Hilfe bei Abschlussarbeiten, aber auch andere schriftlichen Arbeiten. Zusätzlich Seminare.
          \item Rostock: Ein Seminar zum Vortragen üben, die Bachelorverteidigung als Modul angeboten.
          \item Umfrage: Bei wem gibt es ein Seminar zum Einüben des wissenschaftlichen Vortrags im Bachelor $\rightarrow$ 16-20 Meldungen
        \end{itemize}

    \subsubsection*{Verteilungsschlüssel Semestergelder}
      \textbf{Kommt von:} Tobi (Düsseldort)

Wir haben bei uns aktuell ein System zur Verteilung der Semestergelder nach folgendem System: \\

      \begin{itemize}
        \item Pro Student gehen x Euro in einen Topf.
        \item Aus diesem Topf gehen dann an jeden Fachschaftsrat n Euro als Sockelbetrag (n =ca 500 Euro)
        \item Die restlichen Gelder im Topf werden dann durch die Anzahl der Studenten geteilt und dieses Geld wird dann folgendermaßen verteilt. Dieser Betrag wird als Vollzeitäquivalent bezeichnet:
          \begin{itemize}
            \item Philosophische Fakultät: Hier gibt es einen Verteilungsschlüssel bei dem die Studenten im Hauptfach 2/3 an das Hauptfach und 1/3 Nebenfach (2-Fach Studiengänge) (Magister und Lehramt Ignoriere ich mal fleißig. Warum? Weil es dort so gut wie keine Studenten mehr gibt)
            \item Alle anderen: 1 zu 1 an den jeweiligen Fachschaftsrat
          \end{itemize}
      \end{itemize}
      Nun die Frage an die Fachschaften, die Gelder von ihren Studierenden bekommen: \\
      \begin{enumerate}
        \item Wie viel ist das pro Studi?
        \item Wie ist der Verteilungsschlüssel?
      \end{enumerate}

      \paragraph{Antworten}
        \begin{itemize}
          \item Wie viele Fachschaften haben einen Festbetrag (unabhängig von der Anzahl an Studierenden die eingeschrieben sind; ausschließlicher Betrag; auch wenn das Geld extra beantragt werden muss)? \\
          $\rightarrow$ Aachen, Braunschweig, Ilmenau, Köln, Würzburg, Halle, Konstanz
          \item Wer hat gar kein Geld? $\rightarrow$ Erlangen, Augsburg, FUB
          \item Wer hat einen Sockelbetrag und einen Anteil der durch Studierendeanzahl kommt? $\rightarrow$ Österreich, LMU, Greifswald, Frankfurt, Saarland, Gießen, Siegen, Münster, Dortmund, Göttingen, Karls-Marx-Stadt, Potsdam, Kit, Cottbus, Essen,
          \item Bei wem dieser Unis macht der Sockelbetrag mehr als die Hälfte aus? $\rightarrow$ Greirswald, Siegen, Saarland, Chemnitz, Potsdam.
          \item Wo wird das Geld ausschließlich durch die Studianzahl bestimmt? $\rightarrow$ Rostock, Freiberg, Jena
          \item Jena, Freiburg: Sondermodelle Mindestgeld, wenn man eine bestimmte Anzahl an Studierenden erreicht.
          \item In Gießen hängt das Geld von der Wahlbeteiligung ab. Es gibt für jede Fachschaft einen Sockelbetrag von 500€ + pro Wahlstimme 1€. Für den Rest (bis ca 42k€ zusammen kommen) wird von der Fachschaftskonferenz ein Antrag an den AStA gestellt, der von diesem nochmal überarbeitet wird. So kommt das Budget zusammen.
          \item Wer bekommt ausschließlich auf Beantragung Geld? $\rightarrow$ Bielefeld, Darmstadt, ganz Berlin, Mainz
          \item Halle: Gelder beim Stura beantragen, der Kassenprüfungsausschuss prüft die Kassen der Fachschaften. Im Folgejahr nach Prüfung (gemeinnützig?) erhält man Geld
          \item Siegen: 1/3 des gesamten AStA Haushaltes wird an alle Fachschaften verteilt.
          \item Marburg: Geld beantragen, es gibt einen Schlüssel, aber man kann auch mehr beantragen
          \item Innsbruck: Was passiert bei Überschuss bei beantragten Geldern? \\
            Wer darf Überschüsse behalten? $\rightarrow$ Greifswald Frankfurt, Göttingen, Braunschweig, WWU, Bonn, Siegen, Halle, Dresden, Cottbus, KiT, Erlangen, Freiburg, Essen
          \item Hat jemand noch eine nicht-genannte Lösung für \flqq kleine Fachschaften brauchen auch mal Geld, aber basierend auf den Studierendenzahlen funktioniert das nicht\frqq?
            \begin{itemize}
              \item Augsburg: Sind vom Institut angestellt für Umfragen und haben deswegen auch einen Raum. Dieses Geld reicht ihnen aus.
              \item TU Wien: Finanzierung durch Partys.
              \item Siegen: Kleinere Fachschaften werden durch Große gegessen.
            \end{itemize}
        \end{itemize}

    \subsubsection*{Studentische Wahlen}
      \textbf{Kommt von:} Kevin Postler (KaWuM, Karlsruhe), \textbf{weitergeleitet durch:} Tobias (Düsseldorf) \\

Diese Frage bezieht sich sowohl auf Fachschaftswahlen, als auch auf Wahlen zum Studierenden Parlament/Studentenrat oder einem anderen zentralen Studentisches Gremium (ZSG)

\begin{itemize}
          \item Ist die Fachschafts-Wahl gemeinsam mit der ZSG-Wahl?
          \item Wie hoch ist eure Wahlbeteiligung (Bei getrennten Wahlen für Fachschaft und ZSG gesondert)
          \item Habt ihr ein Budget/Wenn Bekannt das Budget der Uni-Weiten Wahl
          \item Gibt es eine Aufwandsentschädigung/Erfrischungsgeld für die Wahlhelfer/den Wahlvorstand (Fachschaft/ZSG)
          \item Wie wird die Wahl Promoted/Versucht die Wahlbeteiligung zu erhöhen?
          \item An wieviele Urnen wird gewählt?
          \item Wieviele Wähler (Fachschaft/ZSG) gibt es?
        \end{itemize}

      \paragraph{Antworten}

        Messfehler: +/- Würzburg

        \begin{itemize}
          \item Bei wem fallen diese Wahlen zusammen?
            $\rightarrow$ 25 Universitäten (Deutschland + Österreich)

          \item Wie hoch sind bei den Fachschaften, wo die Fachschaft-Wahlen einzeln durchgeführt werden, die Wahlbeteiligung? (Wieviel von den Studierenden die wahlberechtigt sind, wählen auch bei den Fachschaften.):
            \begin{itemize}
              \item 0-10\%: 5
              \item 10-20\%: 8
              \item 20-30\%: 3
              \item > 30\%: 4
              \item > 40\%: 3
              \item Maximum: 43\% / 70\% (inoffiziell ohne Parkstudis, eigene Statistik)
            \end{itemize}

          \item Wahlbeteiligung für Gremienwahlen (Fachschaft und ZSG nicht gleichzeitig)
            \begin{itemize}
              \item 0-10\%: 4
              \item 10-20\%: 11
              \item 20-30\%: 0
              \item > 30\%: 0
              \item > 40\%: 0
            \end{itemize}

          \item Zusammenwählen Wahlbeteiligung:
            \begin{itemize}
              \item 0-10\%: 2
              \item 10-20\%: 4
              \item 20-30\%: 8
              \item 30-40\%: 4
              \item > 40\%: 4
            \end{itemize}

          \item Welche Maßnahmen unternehmt ihr, um die Wahlbeteiligung zu erhöhen?
            \begin{itemize}
              \item Bei wem gibt es eine Belohnung für erfolgte Wahl? (Wahlnüsse, Wahlwaffeln, Wahleis, …) $\rightarrow$ 21
              \item Braunschweig: Jeder wird persönlich angesprochen (kleiner Fachbereich)
              \item Bielefeld: Die Fachschaft stellt sich kurz in den Vorlesungen vor, kurt vor den Wahlen. Hat dafür keine Plakate.
              \item FUB: Geht in die VLen.
              \item Wer sonst noch: persönlich in die Vorlesungen gehen und sagen: \flqq Geht wählen\frqq? Die meisten.
              \item Halle: Grillparty mit Aufruf zum Wählen gehen.
              \item KIT: Mobile Wahlurnen vor Grundvorlesungsräumen
              \item Giessen: Online Wahl
              \item Siegen: Professoren die zur Wahl auffordern
              \item Dresden: Werbung in Strassenbahnen, und Werbeprodukten
              \item Mainz: Kaffeebecher werden in der Mensa verteilt mit \flqq Geht wählen\frqq. Bringt aber nix.
              \item LMU: Emails über zentralen Verteiler
              \item Cottbus: Wahl in einer Vollversammlung die immer vor/während der Weihnachtsfeier ist.
              \item Darmstadt: Vollversammlung. Wird nicht gut besucht, außerdem werden Fachschaftsmagazine verteilt
            \end{itemize}

          \item Wieviele Wahlurnen gibt es für die Wahl (nur Fachschaft-Wahl exklusiv):
            $\rightarrow$ keine Fachschaft hat in diesem Fall mehr als eine Urne

          \item Wieviele Urnen für die uniweiten Wahlen?
            \begin{itemize}
              \item LMU: Mehrere, pro Fakultät aber nur eine
              \item Insgesamt: 9 Hochschulen
            \end{itemize}

          \item Gibt es bei euch ein Budget für die Wahlen (für was auch immer, Wahlhilfe, Helfer, Werbung, …) ?
            $\rightarrow$ Ja: 21

          \item Gibt es Erfrischungsgeld für die Wahlhelfer bei Wahlen (ZSG + Fachschaft Wahl gemeinsam)?
            $\rightarrow$ Ja: 16

          \item Gibt es Geld für den Wahlvorstand (ZSG + Fachschaft Wahl)?
            $\rightarrow$ Ja: 15

          \item Bei wem wird die Wahl promotet (ZSG + Fachschaft Wahl)?
              $\rightarrow$ Ja, Plakatkampagnen mit Hinweis auf die Wahl (keine Plakate der Kandidaten): 26 \\
              Wer macht keine Werbung für sich selbst bei der Wahl (“Wählt mich”)? $\rightarrow$ Niemand \\
              Ja, Plakatkampagnen (beliebig): 0
        \end{itemize}

    \subsubsection*{Masterstudienordnung/ Orientierungsstudiengang}
      \textbf{Kommt von:} Sven (Greifswald)

      \begin{itemize}
        \item Wie sieht bei euch die Gestaltung des Masterstudienganges aus vor allem im Bezug auf das 3. Semester?
        \item Gibt es Vorlesungen/Übungen oder ähnliches?
        \item Oder sind alle dortigen Veranstaltungen Scheinmodule?
        \item Gibt es an eurer Uni einen Orientierungsstudiengang insbesonderen im Mat.-Nat.- Bereich?
        \item Was für Meinungen habt ihr zu diesem?
        \item Gibt es generelle Probleme mit diesem?
      \end{itemize}
      Wir arbeiten zurzeit an einer Neuausarbeitung unseres Masterstudienganges, der noch aus alten Zeiten stammt. Ein großer Punkt dabei ist das dritte Mastersemester, welches bei uns nur der Masterarbeit dient und nur Scheinmodule beinhaltet. Gibt es vielleicht an anderen Universitäten eine bessere Ausgestaltung?
      Zusätzlich soll ein Orientierungsstudium an der Mathematisch-Naurwissenschaftlichen Fakultät bei uns eingeführt werden. Da wir auf diesem Gebiet keine Erfahrung haben, würden wir uns über Meinungen, Anregungen und ähnliches freuen.

      $\Rightarrow$ Masterstudiengang soll überarbeitet werden. Gibt es Universitäten, wo es keine Scheinmodule im Masterstudiengang im dritten Semester gibt für die Masterarbeit?

      \paragraph{Antworten}
        \begin{itemize}
          \item Wer hat denn im dritten Mastersemester wirkliche Module und nicht Scheinmodule in Vorbereitung auf die Masterarbeit?
            $\rightarrow$ Göttingen, Augsburg, Marburg, Freiberg (Insert your Answers here)
          \item Wo gibt es Orientierungsstudiengänge im MINT Bereich?
            $\rightarrow$ TUM, TUB, Düsseldorf, Würzburg
        \end{itemize}

    \subsubsection{Auswertung von Evaluationsergebnissen}

      \textbf{Kommt von:} Jenny (FU Berlin)

      \begin{itemize}
        \item Wer erhält die Evaluationsergebnisse? Professoren? Ausbildungskommission?..
        \item Werden sie veröffentlicht? Wenn ja, wie?
      \end{itemize}

      \paragraph{Antworten}
        \begin{itemize}
          \item Bei wem gibt es Evaluationsbeauftragte von der Uni?
            \begin{itemize}
              \item Ja: 0
              \item Nein, ergo Fachschaft alleine: 9
              \item Jein, Fachschaft und Uni machen beide Evaluationen: 0
            \end{itemize}
          \item Bei wem bekommen nur die Professoren die Ergebnisse (keine Veröffentlichung der Ergebnisse)? $\rightarrow$ 18
          \item Bei wem bekommt es irgendeine irgendwie geartete Kommission (zusätzlich)? $\rightarrow$ 11
          \item Bei wem bekommt jeder die Ergebnisse zu sehen? $\rightarrow$ 10
          \item Anmerkungen: Teilweise Modelle, dass der Evaluierte die Möglichkeit einer Zustimmung zur Veröffentlichung hat.
          \item Werden bei Leuten die Abschlussarbeitsbewerter bewertet (a la “Rate my Prof.”)?
            $\rightarrow$ TUM hat das. Ist ein ähnlicher Prozess wie bei der Vorlesungsevaluation.
          \item Bei wem werden Praktika evaluiert?: $\rightarrow$ 30 “viele”
            Hier nicht: 5-6 “nicht so viele”
          \item Bei wem gibt es etwas in der Richtung einer \flqq Studiengangsevaluation\frqq oder einer Evaluation des Studiums an der eigenen Uni generell? $\rightarrow$ 15 (Pharmezeuten empfinden die gleichen Vorlesungsräume wie Physiker allgemein als hässlicher)
        \end{itemize}

    \subsubsection*{Einbindung von internationalen Studierenden in die (aktive) Fachschaft}
      \textbf{Kommt von:} Jakob (Göttingen)

      Fachschaftler rekrutieren sich (in Göttingen) hautpsächlich im Bachelor. Da es dort wenig internationale Studierende gibt, sind in der Fachschaft auch wenige. Möglicherweise habe im Speziellen aber besondere Anforderungen, bei denen wir wegen Unkenntnis nicht helfen, oder Bereicherungen, die uns entgehen. Gibt es (funktionierende!) Konzepte, um speziell in die Fachschaft einzubinden oder ist das unnötig?
      Gibt es besondere Anforderungen, von denen ihr wisst?

      \paragraph{Antworten}
        \begin{itemize}
          \item Bonn: English zu reden, hat leider nicht funktioniert, Filmabende
          \item LMU: International Dinner
          \item Jena: Einladung aller Studierende zu einem Grillabend, insbesondere auch der internationalen Studierenden.
        \end{itemize}

    \subsubsection*{Größe des Fachschaftsrates}
      \textbf{Kommt von:} Hubert Lam (Saarland)

      Hintergrund: Mein Fachschaftsrat hat für die letzte Wahl die maximale Anzahl der Vertreter von 15 auf 19 gehoben. Einige fanden diese Entscheidung nicht gut. Daher wollten sie wissen, wie es landesweit so aussieht. \\

      \begin{itemize}
        \item Wie groß ist euer Fachschaftsrat bzw. wie groß ist die Anzahl der aktiven Fachschaftler?
        \item Legt ihr die Zahl fest? Wenn ja, wie?; Wenn nein, warum?
        \item Welche Erfahrung habt ihr mit großen Räten gemacht?Mir ist bewusst, dass es Fachschaftsräte gibt, die - ich nenne sie mal - freie Helfika beschäftigen. Hier würde mich daher interessieren, ob diese Helfika auch ein Stimmrecht in Abstimmungen besitzen.
      \end{itemize}

      \paragraph{Antworten}
        \begin{itemize}
          \item Definition (Fachschaftsrat): Das Konstrukt der gewählten Leute die Arbeit leisten. [unvollständige Definition, Anm. des Protokolls]
          \item Bei wem gibt es eine Begrenzung der Anzahl der Stimmrechte auf den Fachschaftssitzungen?: $\rightarrow$ 18
          \item Bei wem darf jeder ein Stimmrecht in der Fachschafts-Sitzung ausüben? $\rightarrow$ 20
          \item Bei denen, die eine Begrenzung der Anzahl der möglichen Fachschaftsräte hat, ist desen Anzahl abhängig von der Anzahl der Studierenden? $\rightarrow$ 4, Es gibt eine Mindestzahl und der Rest ist irgendwie abhängig
          \item Wer hat eine feste Zahl im Moment??
            \begin{itemize}
              \item $\leq$ 10: 4 %kleiner gleich
              \item 11-20: 8
              \item >20: 4
            \end{itemize}
          \item Welcher Fachschaftsrat legt die eigene Größe selbst fest? $\rightarrow$ Göttingen, Essen, Rostock, Dortmund, Marburg, Saarland, Dresden
          \item Wieviele Hochschulen haben einen gewählten Fachschaftsrat (der Studenten!)? $\rightarrow$ 28
        \end{itemize}

    \subsubsection*{Handhabung von Problem-Professoren}
      \textbf{Kommt von:} Hubert Lam (Saarland)

      Wer kennt Sie nicht? Professoren, die trotz längeren Gesprächen und einschalten höherer Instanzen, eine schlechte Veranstaltung halten. Sei es durch fehlerhafte Übungsblätter, unkoperative Übungsleiter oder ähnlichem. Sie verbessern ihre Lehre nicht und wiederholen ihr Programm jedes Jahr/Semester. Uns würde es daher interessieren, wie ihr solche Professoren handhabt. \\

      \paragraph{Antworten}
        Verweis auf Austausch-AK in Siegen (\url{https://zapf.wiki/WiSe17_AK_Austausch#Umgang_mit_Kritik_an_Professoren}) \\
        Verweis auf AK aus Konstanz (Anmerkung des Protokolls: Ein entsprechender AK konnte nachträglich in den Aufzeichnungen der ZaPF Konstanz nicht gefunden werden).

    \subsubsection*{Willkommensveranstaltung für internationale Studierende}
      \textbf{Kommt von:} Lina (Innsbruck)

In welcher Uni gibt es eine Willkommensveranstaltung (Erstsemestereinführung/Orientierungswoche…) für internationale Studierende?
Ich freue mich über Material (Präsentationen, Informationsblätter…), per Telegram, auf einem Stick oder per Mail an lina at siegen.zapf.in

    $\Rightarrow$ Verweis auf eigenen AK

    \subsubsection*{Fragensammlung Uni Marburg}
      \textbf{Kommt von:} Christian (Marburg)
        \begin{itemize}
          \item Wie gut seht ihr die Zusammenarbeit von Studierenden und Dekan/Studiendekan? Wie werden Probleme behandelt?
          \item Werden Stipendien aktiv beworben?
          \item Ist es möglich über 180/240 CP hinaus Module zu hören und anerkennen zu lassen?
          \item Wie sind (Bachelor-) Master-Seminare ausgestaltet? Kaum Anwesenheit und nur Vorträge?!
        \end{itemize}

      \paragraph{Antworten}
        \begin{itemize}
          \item Werden Stipendien direkt beworben? $\rightarrow$ Ja: 19
          \item Ist es möglich Module über die 180/240 CP hinaus zu hören?
            \begin{itemize}
              \item Möglich diese zu hören: Ganz viele
              \item Nicht möglich diese zu hören (Prüfung darf nicht absolviert werden): 0
              \item Bei wem dürfen diese gehörten Module nicht ins Zeugnis aufgenommen werden?
                \begin{itemize}
                  \item Hier: Greifswald
                  \item Nicht automatisch, aber auf Antrag möglich: ein paar
                \end{itemize}
            \end{itemize}
          \item Bei wem gibt es Veranstaltungen, für die man zum Bestehen nur einen Vortrag halten muss? $\rightarrow$ 21
          \item Bei wem klappt die Zusammenarbeit mit dem Studiendekan gut? $\rightarrow$ nicht gut: 3 \\
            Umgang damit: Greifswald hat keine Strategie;
              Marburg: Problem mit Studiendekan, in viele Dinge nicht involviert. Aber guter Kontakt mit Dekan, das hilft.
        \end{itemize}

    \subsubsection*{Identitäre Bewegung an Hochschulen (Allgemein Nationalisten)}

In Düsseldorf sind uns Aktionen der sogenannten Identitären Bewegung aufgefallen. Auch sollte der Shitstorm, der den AStA Köln vor etwa einem Monat getroffen hat in Erinnerung geblieben sein. Wir wollen nun wissen:
      \begin{itemize}
        \item Gibt es auch an anderen Universitäten Aktionen von Peronen mit nationalistischen Bestrebungen?
        \item Wenn ja, wie geht man bei euch damit um?
      \end{itemize}

      \paragraph{Antworten}
        \begin{itemize}
          \item Definition: Es gibt Netztrolle die fordern, wir sollen um Deutschland wieder Grenzen ziehen und Deutschland besser machen. [unvollständige Definition, Anm. des Protokolls]
          \item Wer hat mit denen auch Probleme? $\rightarrow$ 8 Hochschulen haben dazu Vorkommnisse
          \item Wie geht ihr damit um?
            \begin{itemize}
              \item Halle: Verweis auf AfD-AK, Campus-Partei: Campus-Alternative, wurden auch in den Studierenden-Rat reingewählt.
              \item Bielefeld: Probleme mit türkischen Nationalisten während Türkei-Wahlen
              \item Mainz: Probleme mancher Fachschaften mit Graffiti, wird dann wieder weggemacht und \flqq wir sind gegen rechts\frqq proklamiert.
              \item Braunschweig: Eine Burschenschaft, welche regelmäßig Seminare startet und sich beschwert, dass keine Linken dazu kommen. Letztes Jahr Gegendemonstration mit T-Shirts \flqq Privatperson\frqq (getragen durch Personen auf höheren Uni-Ämtern), die rechte Szene ist dann auf diese T-Shirts eingegangen und hat die Personen gezielt \flqq verfolgt\frqq im Sinne von regeläßigen Nachfragen, was das soll.
              \item Rostock: Vortrag vom AStA sabotiert durch Zwischenrufe, Gegenmaßnahme: einfach mehr Vorträge zu diesem Thema
              \item Greifswald: Gedenkstein wurde von der identitären Bewegung vor der Uni gelegt (in Nacht und Nebel Aktion). Teilweise sind Denunziationslisten (gegen die identitäre Bewegung) mit Adressen etc. aufgetaucht
              \item Dresden: Es werden Sticker verklebt, diese werden wieder entfernt. Burschenschaften, gegen die der Stura HowTos schreibt, wie man mit ihnen umgehen soll.
              \item Potsdam: Hat Nazi-Kleber-Überklebe-Aufkleber
            \end{itemize}
          \item Wie kann man sich darauf vorbereiten? $\rightarrow$ Bitte mit den Betroffenen kurzschließen, gegebenenfalls im Wiki Ideen ergänzen.
        \end{itemize}

    \subsubsection*{internationale Studierende}
      \textbf{Kommt von:} Max (Uni Rostock)

      \begin{itemize}
        \item Wie viele internationale Studierende gibt es an eurer Universität?
        \item (Wie) wird es an eurer Uni gefördert, dass es mehr internationale Studierende gibt?
      \end{itemize}

      \paragraph{Antworten}
        \begin{itemize}
          \item Wer weiß wie viele Internationale Studierende an seiner Uni hat? $\rightarrow$ sehr wenige (4)
          \item Förderprogramme um internationale Studierende herzuholen? $\rightarrow$ LMU, Augsburg, Bielefeld, Greifswald, Frankfurt, Darmstadt, Konstanz, Saarland, Marburg, Bonn, Siegen, TUM, Braunschweig, Jena, FUB, Göttingen, Karl-Marx-Stadt, Potsdam, Halle, Freiberg, Mainz, KIT
        \end{itemize}

    \subsubsection*{Benennungen durch Statusgruppen}
      \textbf{Kommt von:} Fabs (TU Berlin)

Hintergrund: Bei uns gibt es bei der Benennung der Mitglieder von Berufungskommissionen das Problem, dass die Professoren massiv Druck ausüben, um die Benennung der studentischen und WiMi-Mitglieder zu \flqq übernehmen\frqq. Dies geschieht ohne Rücksprache mit den Benannten, die zum Teil diese Aufgabe gar nicht wollen, oder mit deren Statusgruppen. \\

Klärung des genauen Problems:
Bei manchen Berufungskomissionen sollten Studierende und wissenschaftliche Mitarbeiter die Vertreter benennen.
Faktisch ist es aber aktuell so, dass die Professoren Wünsche \flqq vorschlagen\frqq und dann versucht wird dies zu \flqq begründen \frqq, damit diese Personen in die Kommission rein kommen. Die Personen, die rein sollen, haben oft selbst gar keine Lust auf die Kommission. \\

      Weitere Fragen: \\
      \begin{itemize}
        \item An welchen anderen Universitäten besteht dieses Problem auch (gegebenenfalls auch in anderen benannten Kommissionen/Gremien)
        \item Wie geht ihr damit um?
      \end{itemize}

      \paragraph{Antworten}
        \begin{itemize}
          \item Essen: Problem einmalig in einem Fachbereich, dann an das Dekanat diese Information weitergegeben. Seitdem ist dies nicht wieder geschehen.
          \item Siegen: Eigene Nachfolge-Kommission komplett selber besetzt (auch nur manchmal) durch den Professor, der ausscheidet.
        \end{itemize}

    \subsubsection*{Änderungen der Struktur von bayrischen Studierendenvertretungen}
      \textbf{Kommt von:} Andy (Würzburg)

Hintergrund: Durch Änderungen im Bayerischen Hochschulgesetz können die Universitäten ab diesem Sommer die Struktur ihrer Studierendenvertretung (weitestgehend) selbst geben. In Würzburg wurden diese Gestaltungsmöglichkeiten an die Studierenden weitergegeben.
      \begin{itemize}
        \item Wie handhaben die anderen bayrischen Universitäten diese Umstellungen?
        \item Welche Änderungen werden angestrebt, und durch welches Gremium/Statusgruppe?
      \end{itemize}

      \paragraph{Antworten}
        \begin{itemize}
          \item LMU: Nie was davon gehört.
          \item Würzburg: Die Universiät hat jetzt mehr Feiheit, wie sie die Gremien der Hochschulpolitik gestalten wollen.
        \end{itemize}
        $\Rightarrow$ die Bayern sollen dazu einen Bieraustausch-AK veranstalten!

    \subsubsection*{Transparenzklausel}
      \textbf{Kommt von:} Andy (Würzburg)

      Gibt es an anderen Universitäten Regelungen oder Richtlinien zum transparenten Umgang mit Drittmittelforschung? Wenn ja, in welcher Form und mit welchem Inhalt?

      \paragraph{Antworten}
        Wer hat keine Ahnung, ob es sowas bei Ihnen gibt?
        \begin{itemize}
          \item Ja: Ganz viele
          \item Nein: Göttingen
          Anmerkung: In Niedersachen ist dies im Gesetz verankert! (Seit der letzten Novelle)
        \end{itemize}

    \subsubsection*{Projektpraktika}
      \textbf{Kommt von:} Andy (Würzburg)

      An welchen Universitäten ist ein Projektpraktikum Teil der Grund- oder Fortgeschrittenenpraktika?

      \paragraph{Antworten}
        \begin{itemize}
          \item Definition: Ein Projektpraktikum ist ein Praktikum, bei dem man sich einen Fachbereich aussucht und einen größeren Versuch/Projekt zu diesem Thema bearbeitet. Auch Praktikum in welchem eigene Versuche vorgeschlagen und durchgeführt werden. \\
          Hier gibt es das:
          Bonn, Düsseldorf, Siegen, Dortmund, Wien, Konstanz, TUB, FUB, Göttingen, Erlangen, Marburg, Bochum, Rostock, Würzburg
          \item Göttingen: War lange Jahre lang Pflicht. Danach Beschwerden, die gar nicht richtig Lust darauf hatten. Seitdem ist es ein Wahlmodul geworden. Durch die Wahl ist die Anzahl an Gruppen signifikant gesunken. Befürchtung: dass es dort bald gar nichts mehr gibt.
          \item TU Berlin: Es gibt ein Projektversuch innerhalb des Anfängerpraktikums, zieht sich aber nicht über ein Semester. Gibt auch eine Alternative zum Anfängerpraktikum, das Projektlabor: Hier erfolgt die Bearbeitung in Gruppen.
          \item Dortmund: Anfängerpraktikum mit 24 Versuchen. Man kann sich ein paar Versuche sparen, und dafür ein Projekt in einer Zweier-Gruppe machen.

        \end{itemize}

    \subsubsection*{Teilzeitstudium}
      \textbf{Kommt von:} Alex (KIT)
      \begin{itemize}
        \item Habt ihr das (explizit in der Physik)?
        \item Wenn nein, warum nicht?
        \item Wenn ja: Unter welchen Bedingungen?
        \item Bedingungslos für alle?
        \item Nur für Benachteiligte?
        \item Wurden eure PO’s dafür umgeschrieben?
      \end{itemize}
      Besonders interessant wären Universitäten aus Baden-Württemberg.

      \paragraph{Antworten}
        \begin{itemize}
          \item Bei wem gibt es ein Teilzeit-Studium? $\rightarrow$ Frankfurt, FUBM Saarland, Darmstadt, Marburg, Tübingen, GreiFachschaftwald, Essen, Chemnitz, Düsseldorf, Potsdam \\
            Verweis auf Studienführer, dort dürfte es auch hinterlegt sein.
          \item Gibt es bei irgendjemandem eine Bedingung, wer das machen darf?
            \begin{itemize}
              \item Berlin: Studierende mit Kind, BeruFachschafttätige, Krankheit, Wegen Pflege von Angehörigen
              \item Tübingen: nur wenn man ein Kind hat
            \end{itemize}
          \item Bei wem wurde die PO angepasst (also nicht einfach “Mach die Hälfte”)? $\rightarrow$ am KIT soll umgeschrieben werden
        \end{itemize}
