% !TEX TS-program = pdflatex
% !TEX encoding = UTF-8 Unicode
% !TEX ROOT = main.tex

\section{AK E-Learning}

\textbf{Protokoll vom:} 02.06.2018,
Beginn: 09:15 Uhr,
Ende: 11:00 Uhr \\
\textbf{Redeleitung:} Jakob Brenner (LMU M\"unchen) \\
\textbf{Protokoll:} Manuel Längle (Uni Wien) \\
\textbf{anwesende Fachschaften:} LMU München, Uni Innsbruck, Uni Wien, Uni Tübingen, RWTH Aachen, Uni Köln, Uni Wuppertal, Uni Jena, Uni Bonn, TU Graz, Uni Bielefeld, Uni Oldenburg, Uni Konstanz, Uni Dresden, TU Darmstadt, Uni Potsdam, Uni Bochum, Uni Würzburg

\subseection*(Informationen zum AK}
\begin{itemize}
	\item \textbf{Ziel des AKs}: Materialsammlung \& Austausch, bei Interesse ein Positionspapier
	\item \textbf{Folge-AK}: nein
	\item \textbf{Zielgruppe}: Leute von möglichst vielen Unis, gerade Leute, deren Unis in diesem Bereich Engagement zeigen.
	\item \textbf{Ablauf}: Kurze Definition der Themen, Grundsatzdiskussion über E-Learning, danach Austausch über Vorgehensweisen, Material...
	\item \textbf{Voraussetzungen}: Inforamtionen über die Situation an deinen oder anderen Unis

\subsection*{Protokoll)
	\begin{itemize}
		\item Aachen: Haben ein gutes Angebot. Sind interessiert eine Materialsammlung zu machen.

		\item Graz: Verwenden Moodle und wünschen sich mehr Interaktivität.

		\item Köln: Haben wenig E-Learning. Werfen die Frage auf, ob E-Learning gut ist.

		\item Aachen: Coole Frage ob E-Learning sinnvoll ist.

		\item Mainz: Haben Lehrvideos und gutes E-Learning. Heben Vorteile von E-L hervor, wenn es gut mit den VOs abgestimmt ist.

		\item Wuppertal: Dozentenabhängiges Angebot, z.B. VO-Unterlagen.

		\item Jena: Dünnes bzw. fehlendes Angebot.

		\item Aachen: alles online, alle Skripten, viele Vorlesungsvideos;
		Stellt sich unter E-Learning mehr als nur eine Unterlagensammlung vor. Modell mit online Arbeitsaufträgen, die auch benotet werden. (Aachen scheint bereits ein fortgeschrittenes Angebot zu besitzen.)

		\item Wien: Sämtliche Unterlagen online. Vorstellung von E-L wie Aachen
	\end{itemize} \\

	Es wird ein Stimmungsbild gemacht, welche Resourcen verwendet werden. Alle bekommen Übungszettel und Skripte online außer Konstanz. Die bekommen keine hochgeladenen Skripte von Professoren. An der Uni Wien gibt es YouTube Videos zur Praktikumsvorbereitung und Bedienung der Ger\"ate.

	Stimmungsbild: Mehr als die Hälfte findet, dass sich die LV-Leiter sich mehr mit E-Learning auseinandersetzen sollten

	\begin{itemize}
		\item Aachen: Haben eine nette App bei der man mit Physik spielen kann. Phyphox. Gab Übungen als Bonuspunkteabgaben. Ist eine Super app sagen viele Leute. Moodle war scheiße.
		\item Graz: Bereits so viel online, mehr E-Learning ist nur konsequent.
		\item Tübingen
		Implizite Kompetenzen nicht abschaffen. Es ist eine kompetenz ohne Lernprogramme klarzukommen. Man sollte aus einem Paper einen versuchsaufbau machen können, das sollte nicht verlernt werden.

		\item Wien
		Sinnhaftigkeit hängt stark von der Qualität der Umsetzung ab.

		\item Aachen
		Sicherheitsrisiken. Gehackte Accounts. Onlinetests sind unflexibel.

		\item Köln
		Vorteil von E-Learning?
		Vorlesungenen sind nicht für alle was. Tempo passt oft nur für einen bruchteil der Studierenden.
		Inverted Classroom ist toll.

		\item Wien
		Inverted Classroom ist toll. Stoff vorher hochladen ist auch gut.

		\item Tübingen
		Möchte Vor- und Nachteile evaluieren. Man hat sich Tutoriuen für Mediziner durch E-Learning abschaffen, Ressourcen frei schaufeln
	\end{itemize}

\subsubsection*{Materialsammlung}
	\begin{itemize}
		\item LMU: Mathe für Nichtfreaks (https://de.wikibooks.org/wiki/Mathe_für_Nicht-Freaks), kleine, öffentlich frei zugängliche Seiten zu mathematischen Grundbegriffen und \"ahnliches.
			Entwurf vom Studiendekan, Sammlung von Werkzeugen
			http://www.physik.uni-muenchen.de/lehre/elearning/\\

			Fachschaft:
			Explizit selbstgeschriebene Skripte werden hochgeladen. Weil dann hat der Studi das Recht es hochzuladen, sonst hat der Professor das Copyright.\\

			Moodle wird nicht verwendet, Professoren laden alles auf eigenen Websites hoch
			Klausurensammlung\\

			Neues Konzept:
			Semestersprecher
			Man geht am Anfang des Semesters in 1.-4.-Semester Vorlesungen um Semestersprecher zu ernennen.
			Reden mit Professoren über Kompetenzen, die die Studis haben sollen und Werkzeuge, die sinnvoll zu erreichen sein könnten.
			Abstimmungen in Vorlesungen am Beamer - pingo \\

			Sie haben einen sehr engagierten Professor.
			Der nimmt sich selbst auf während er auf einem Tablet die Vorlesung schreibt
			und online ein Diskussionsforum mit Tutoren, die angestellt sind, um da Dinge zu machen
			Das Diskussionsforum ist anonym. \\

			keine Zentrales hoch und runterladeportal.

			Es gibt Probleme mit der DSGVO.
			Anmerkung: Zentrale Systeme der Uni sollen verwendet werden, dann ist die DSGVO das Problem von Profis.

			Es gibt keine Anmeldung für Prüfungen und ein extrem schlechtes Vorlesungsverzeichnis, das niemand verwendet. Jede LV hat eine eigene Webseite. Es gibt kein funktionierendes, zentrales System.
 \\
Uni Potsdam: Moodle, funktioniert gut
 \\
Aachen: Alles Online an einer Stelle.
 \\
Oldenburg: Projekt Praktika sollen digitalisiert werden. Phyphox wurde nciht nagenommen.

**Allgemeines Stimmungsbild: Wer hat ein Zentrales System.
 \\
München und Mainz hat kein zentrales E-Learnings System:  \\
Bielefeld: hatten anfang des E-Learning Zeitalters 6 verschiedene E-Learning Systeme
Jetzt, am Ende haben sie ein System das funktioniert.
**Jede uni sagt jetzt mal was sie hat
 \\
Darmstadt: Zentrales System, Moodle, manche Profs verwenden eigene Webseite aber meiste VO haben Moodle webseite \\
Tübingen: Campus für Vorlesungsverzeichnis. Ilias wird verwendet, Moodle auch. Soll jetzt vereinheitlicht werden. Video aufzeichnungen von vorlesungen (meißt aber veraltet). \\
Aachen: vorlsungesaufzeichnungen von der Fachschaft, E-tests, erklärvideos, online abgaben, sie verwenden nur manchmal moodle, sonst wird das Kampussystem verwendes, welches besser ist \\
Graz: Besteht aus zwei Unis die sich ein Studium teilen. Moodle verwendet die Hauptuni Graz. Die TU Graz verwendet ein Teach center, moodle abklatsch. Manche Profs verwenden private Webseites \\
Bochum: Zentral verwaltetes System. Digitale Abgaben in Theorie. Grundpraktikum - Zentrale Datenbank. Dann kann man Daten von anderen Gruppen auch verwenden. \\
Wien: verwenden Moodle, großteils, klappt ganz gut, es gibt videos in praktika, Onlinetests und gewartete Foren für manche frühe Vorlesungen \\
Innsbruck: Zentrales Anmeldesystem. E-Learning auf zentraler Seite, basierend auf OpenOLAT. Dort zu finden sind u.U. VO-Unterlagen, Übungsblätter mit Möglichkeit der Onlineabgabe, etc. System wird unterschiedlich intensiv genützt, funktioniert aber gut. \\
Jena: Schlechtes zentrales System. Webseite mit Skripten und Übungszetteln.
Abgaben per Mail oder gedruckter Code. \\
bochum: Moodle wird verwendet. Blackboard vorher parallel. Moodle wird Prof-abhängig genutzt. Gibt die Idee ein Online Seminar zu machen, also ganz ohne anwesenheit. Noch in Anfangsplanung, aber Finanzierung steht \\
Mainz: anmelden funktioniert nicht, jeder prof verwendet was anderes, funktioniert einigermaßen bis ganz gut \\
Wuppertal: Zentrales Vorlesungssystem Wusel mit Moodle dazu gekoppelt
gute erfahrung, Praktikumsprotokolle werden manchmal online über Moodle abgegeben
Skibu - das ist wie Dropbox nur viel größer, Profs erstellen eigene Ordner, studis können das auch nutzen, ist NRW intern
Studierenden sind zufrieden \\
Köln: Ilias ist die Plattform für Materialien und so
Virtueller schreibtisch, bei dem auch eigene Daten abgelegt werden können. VO Videomitschnitte, kleine Tests. Von GeiWi mehr genützt als von der Physik.
Onlineassessment vor der Immatrikulation (finden alle Kacke, muss allerdings erst an der Masse getestet werden). \\
Giessen: Ilias und Stud.IP
Prüfungsanmeldung online. Hochschuldidaktikzentrum: Coaching für Lehrende auf 1:1 Basis. \\
TU Dresden: gibt zentrales system
anmeldung für Übungsgruppen und Übungsblätter werden hochgeladen
Manche Profs verwenden eigene Webseiten \\
Potsdam: Moodle + eigene Webseiten
auch was Dropbox ähnliches \\
Konstanz: Ilias für Übungsblätter, Skripten werden oft nicht hochgeladen; Streamingseite in der Physik nicht verwendet, da niemand sich filmen lassen will; \\
Würzburg: zwei systeme, veranstaltungsanmeldung meldet einen direk in dem anderen system an
abgaben können hochgeladen werden in dem system
dozenten verwenden noch eingene webseites
\subsubsection*{Umsetzung}
	\paragraph{Was wird verwendet?}
		\begin{itemize}
			\item live abstimmungen in Vorlesungen - selber machen, Tools: hmind.org, pingo.com,
			kann sehr gut sein
			kann zeitverschwendung sein und nicht ernstgenommen werden - analogabstimmungen manchmal besser (farbige karten)

			\item online tests - zum benoten und zur selbsteinschätzung, Tools: moodle,
			hier kommt die frage auf, ob die fragen im nachhinein hergezeigt wird
			http://hmind.org/
			Gute Möglichkeit zur Einbindung eines Quizzes: OpenOLAT

			\item Videos von vorlesungen
			sehr viele - 1
			viele - 0
			wenige - eine schwache hälfte
			eigentlich keine - eine starke hälfte

			\item streamvorlesungen
			niemand hat das
		\end{itemize}

\paragraph{Wer nimmt auf?} \\
bei den meisten die fachschaften
https://video.fsmpi.rwth-aachen.de/
viele profs haben Probleme damit, sich aufnehmen zu lassen \\

Aufgenommen wird zur hälfte von Fachschaften zur anderen hälfte von Studies
manche Unis bieten material an
manche fachschaften bieten kameras und mikros an um aufzunehmen\\

Aachen bietet den profs an, dass sie selber entscheiden können wer es sieht, fachintern, uniintern, öffentlich\\

wenn sich manche profs filmen lassen, lassen sich andere leichter filmen
warum wills keiner machen:
profs haben angst, dass studies nicht mehr in die Vorlesung kommen
warum wollen profs, dass stidis in die VOs kommen? - damit sie sehen ob die Studies mitkommen oder ob die Augen glasig werden, fehlendes feedback\\

Befürchtung zu aufzeichnungen: dozent mag die Vorlesung nicht halten, muss aber, spielt dann einfach nur videos
besser als ein gutes video als eine schlechte vorlesung\\

Uni Graz: Preis für E-Learning, Uniweit, cool\\

\subsubsection{Träume}
	\begin{itemize}
		\item ordentliches anmeldungssystem
		\item zentrale materialquelle
		\item aufnahmesystem für videos
		\item online diskussionsforen
		\item profs solln vor der volresung literatur zu der vorlesung online stellen
		\item interaktiveres E-learning, upload von beispielen die von anderen studies bewertet werden, programmierübung
		\item ein Netzwerk das nicht zusammenbricht
		\item online tutorien zum fragen beantworten
		\item touchscreen wo leute drauf schreiben das an die tafel projeziert werden, sollen keine skripten ersetzen
		\item tafeln automatisch abfotografieren
		\item live streams zu vorlesungen soll es geben, verpasst ma krank nichts
		\item quizze wären cool
		\item klausuren online stellen
		\item inhaltsangaben von vorlesungen angeben
		\item mehr ausprobierwille zum e-learning
		\item alle sollen sich gedanken machen wie man e-learning integrieren kann, es soll nichts anderes ersetzen sondern ein teil des ganzen werden, freies studium sollte das ziel sein! balanced use
		\item wir sollen selber online suchen und das dann unseren dozenten vorschlagen wie ma das verwendet
		\item zentrales system - wie in aachen, aachen ist supergut
		\item wir sollten skripten videos und alles miteinander teilen - unis connected
		\item gemeinsame plattform, wir sollten teilen!
	\end{itemize}

\subsubsection*{Weiteres Vorgehen}
folge auf der nächsten zapf wäre super \\
aufgabe bis zur nächsten zapf: jede fachschaft bring was sie hat und bringen kann \\
schwierigkeiten und erfolge teilen \\

\paragraph{Ziel für den nächsten AK}
fachschaften sammeln material \\
anderes ziel: wir sollen einen leitfaden für E-Learning erstellen \\

wie überzeugt ma profs, was kann ma machen, was ist cool, yeah!

\subsubsection*{Wünsche}
links zu Videos
links auf Übungsblättern
animationen für E-Dynamik oder so
**Soll eine Vorteilnachteilsammlung gemacht werden?

\subsection*{Zusammenfassung}
Am Anfang war der AK eher auf den Austausch und Vergleich konzentriert. Im pad sind die Aussagen der einzelnen Unis zu finden.

Im Allgemeinen war die Stimmung positiv bezüglich Elearning Angeboten ergänzend aber nicht ersetzend für Vorlesungen.

Im generellen zeigte sich, dass die fast alle das Angebot ihrer Uni für unzureichend empfanden, jedoch einige bereits deutlich mehr aufwiesen als die anderen.

Weit verbreitet vorhanden waren zentrale Systeme zum Anmelden, welche in vielen Fällen auch Materialsammlungen und Diskussionsräume für Lehrveranstaltungen zumindest in der Infrastruktur bereitstellen. Die verbreitesten Systeme waren Moodle und Ilias.

Weniger weit verbreitet waren Vorlesungsaufzeichnungen, wo sie bestanden wurde es zumeißt von den Fachschaften organisiert.
\subsection*{Fazit}
Es kam der Wunsch nach einem Folge-AK in Würzburg auf, zur Vorbereitung ist die Materialsammlung in eueren Fachschaften bezüglich Werkzeugen, aber auch erfolgreichen Vorgehensweisen vorgesehen, diese werden dann dort zu einer Materialsammlung kombiniert.
