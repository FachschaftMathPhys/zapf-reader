% !TEX TS-program = pdflatex
% !TEX encoding = UTF-8 Unicode
% !TEX ROOT = main.tex

\section{AK E-Learning}

	\textbf{Protokoll vom:} 02.06.2018,
	Beginn: 09:15 Uhr,
	Ende: 11:00 Uhr \\
	\textbf{Redeleitung:} Jakob Brenner (LMU München) \\
	\textbf{Protokoll:} Manuel Längle (Uni Wien) \\
	\textbf{anwesende Fachschaften:} LMU München, Uni Innsbruck, Uni Wien, Uni Tübingen, RWTH Aachen, Uni Köln, Uni Wuppertal, Uni Jena, Uni Bonn, TU Graz, Uni Bielefeld, Uni Oldenburg, Uni Konstanz, Uni Dresden, TU Darmstadt, Uni Potsdam, Uni Bochum, Uni Würzburg

	\subsection*{Informationen zum AK}
		\begin{itemize}
			\item \textbf{Ziel des AKs}: Materialsammlung \& Austausch, bei Interesse ein Positionspapier
			\item \textbf{Folge-AK}: nein
			\item \textbf{Zielgruppe}: Leute von möglichst vielen Unis, gerade Leute, deren Unis in diesem Bereich Engagement zeigen.
			\item \textbf{Ablauf}: Kurze Definition der Themen, Grundsatzdiskussion über E-Learning, danach Austausch über Vorgehensweisen, Material...
			\item \textbf{Voraussetzungen}: Inforamtionen über die Situation an deinen oder anderen Unis
		\end{itemize}

\subsection*{Protokoll}
	\begin{itemize}
		\item Aachen: Haben ein gutes Angebot. Sind interessiert eine Materialsammlung zu machen.

		\item Graz: Verwenden Moodle und wünschen sich mehr Interaktivität.

		\item Köln: Haben wenig E-Learning. Werfen die Frage auf, ob E-Learning gut ist.

		\item Aachen: Coole Frage ob E-Learning sinnvoll ist.

		\item Mainz: Haben Lehrvideos und gutes E-Learning. Heben Vorteile von E-L hervor, wenn es gut mit den VOs abgestimmt ist.

		\item Wuppertal: Dozentenabhängiges Angebot, z.B. VO-Unterlagen.

		\item Jena: Dünnes bzw. fehlendes Angebot.

		\item Aachen: alles online, alle Skripten, viele Vorlesungsvideos;
		Stellt sich unter E-Learning mehr als nur eine Unterlagensammlung vor. Modell mit online Arbeitsaufträgen, die auch benotet werden. (Aachen scheint bereits ein fortgeschrittenes Angebot zu besitzen.)

		\item Wien: Sämtliche Unterlagen online. Vorstellung von E-L wie Aachen
	\end{itemize}

	Es wird ein Stimmungsbild gemacht, welche Resourcen verwendet werden. Alle bekommen Übungszettel und Skripte online außer Konstanz. Die bekommen keine hochgeladenen Skripte von Professoren. An der Uni Wien gibt es YouTube Videos zur Praktikumsvorbereitung und Bedienung der Ger\"ate.

	Stimmungsbild: Mehr als die Hälfte findet, dass sich die LV-Leiter sich mehr mit E-Learning auseinandersetzen sollten

	\begin{itemize}
		\item Aachen: Haben eine nette App bei der man mit Physik spielen kann. Phyphox. Gab Übungen als Bonuspunkteabgaben. Ist eine Super app sagen viele Leute. Moodle war scheiße.
		\item Graz: Bereits so viel online, mehr E-Learning ist nur konsequent.
		\item Tübingen
		Implizite Kompetenzen nicht abschaffen. Es ist eine kompetenz ohne Lernprogramme klarzukommen. Man sollte aus einem Paper einen versuchsaufbau machen können, das sollte nicht verlernt werden.

		\item Wien
		Sinnhaftigkeit hängt stark von der Qualität der Umsetzung ab.

		\item Aachen
		Sicherheitsrisiken. Gehackte Accounts. Onlinetests sind unflexibel.

		\item Köln
		Vorteil von E-Learning?
		Vorlesungenen sind nicht für alle was. Tempo passt oft nur für einen bruchteil der Studierenden.
		Inverted Classroom ist toll.

		\item Wien
		Inverted Classroom ist toll. Stoff vorher hochladen ist auch gut.

		\item Tübingen
		Möchte Vor- und Nachteile evaluieren. Man hat sich Tutoriuen für Mediziner durch E-Learning abschaffen, Ressourcen frei schaufeln
	\end{itemize}

\subsubsection*{Materialsammlung}
	\begin{itemize}
		\item LMU: Mathe für Nichtfreaks (\url{https://de.wikibooks.org/wiki/Mathe_für_Nicht-Freaks}), kleine, öffentlich frei zugängliche Seiten zu mathematischen Grundbegriffen und ähnliches.
			Entwurf vom Studiendekan, Sammlung von Werkzeugen (\url{http://www.physik.uni-muenchen.de/lehre/elearning/})

			Fachschaft:
			\begin{itemize}
				\item Explizit selbstgeschriebene Skripte werden hochgeladen. Dann hat der Studierende das Recht es hochzuladen, sonst hat der Professor das Copyright.\\

				\item Moodle wird nicht verwendet, Professoren laden alles auf eigenen Websites hoch
				\item Klausurensammlung

				\item Neues Konzept:
				Semestersprecher \\
				Man geht am Anfang des Semesters in 1.-4.-Semester Vorlesungen um Semestersprecher zu ernennen.
				Reden mit Professoren über Kompetenzen, die die Studis haben sollen und Werkzeuge, die sinnvoll zu erreichen sein könnten.
				Abstimmungen in Vorlesungen am Beamer - pingo

				\item Sie haben einen sehr engagierten Professor.
				Der nimmt sich selbst auf während er auf einem Tablet die Vorlesung schreibt
				und online ein Diskussionsforum mit Tutoren, die angestellt sind, um da Dinge zu machen.
				Das Diskussionsforum ist anonym.

				\item kein zentrales Hoch- und Runterladeportal

				\item Es gibt Probleme mit der DSGVO.
				Anmerkung: Zentrale Systeme der Uni sollen verwendet werden, dann ist die DSGVO das Problem von Profis.

				\item Es gibt keine Anmeldung für Prüfungen und ein extrem schlechtes Vorlesungsverzeichnis, das niemand verwendet.
				Jede Lehrveranstaltung hat eine eigene Website. Es gibt kein funktionierendes, zentrales System.
			\end{itemize}
			\item Uni Potsdam: Moodle, funktioniert gut
			\item Aachen: Alles Online an einer Stelle
			\item Oldenburg: Projektpraktika sollen digitalisiert werden. Phyphox wurde nicht angenommen.
	\end{itemize}

			 \paragraph{Allgemeines Stimmungsbild}
			   Wer hat ein Zentrales System. \\

				\begin{itemize}
					\item München und Mainz haben kein zentrales E-Learning System.  \\
					\item Bielefeld: hatten am Anfang des E-Learning Zeitalters 6 verschiedene E-Learning Systeme;
						Jetzt, am Ende haben sie ein System das funktioniert. \\
					\item Uni Darmstadt: Zentrales System, Moodle, manche Professoren verwenden eigene Website, aber meiste Vorlesungen haben Moodle Website \\
					\item Uni Tübingen: Campus für Vorlesungsverzeichnis. Ilias wird verwendet, Moodle auch. Soll jetzt vereinheitlicht werden. Video-Aufzeichnungen von Vorlesungen (meißt aber veraltet). \\
					\item Uni Aachen: Vorlesungesaufzeichnungen von der Fachschaft, E-Tests, Erklärvideos, Online-Abgaben, sie verwenden nur manchmal Moodle, sonst wird das Campussystem verwendet, welches besser ist \\
					\item Uni Graz: Besteht aus zwei Unis, die sich ein Studium teilen. Moodle verwendet die Hauptuni Graz. Die TU Graz verwendet ein Teach-Center, Moodle abklatsch. Manche Professoren verwenden private Websites \\
					\item Uni Bochum: Zentral verwaltetes System. Digitale Abgaben in Theorie. Grundpraktikum - Zentrale Datenbank. Dann kann man Daten von anderen Gruppen auch verwenden. \\
					\item Uni Wien: verwenden Moodle, großteils, klappt ganz gut, es gibt Videos in Praktika, Onlinetests und gewartete Foren für manche frühe Vorlesungen \\
					\item Uni Innsbruck: Zentrales Anmeldesystem. E-Learning auf zentraler Seite, basierend auf OpenOLAT. Dort zu finden sind u.U. Vorlesungs-Unterlagen, Übungsblätter mit Möglichkeit der Onlineabgabe, etc. System wird unterschiedlich intensiv genützt, funktioniert aber gut. \\
					\item Uni Jena: Schlechtes zentrales System. Website mit Skripten und Übungszetteln.
						Abgaben per Mail oder gedruckter Code. \\
					\item Uni Bochum: Moodle wird verwendet, Blackboard vorher parallel, Moodle wird abhängig vom Professor genutzt. Gibt die Idee ein Online Seminar zu machen, also ganz ohne Anwesenheit. Noch in Anfangsplanung, aber Finanzierung steht \\
					\item Uni Mainz: Anmelden funktioniert nicht, jeder Professor verwendet was anderes, funktioniert einigermaßen bis ganz gut \\
					\item Uni Wuppertal: Zentrales Vorlesungssystem Wusel mit Moodle dazu gekoppelt,
						gute Erfahrung, Praktikumsprotokolle werden manchmal online über Moodle abgegeben \\
						Skibu - das ist wie Dropbox nur viel größer, Professoren erstellen eigene Ordner, Studierende können das auch nutzen, ist NRW intern
						Studierende sind zufrieden \\
					\item Uni Köln: Ilias ist die Plattform für Materialien und so
						Virtueller Schreibtisch, bei dem auch eigene Daten abgelegt werden können. Vorlesungs-Videomitschnitte, kleine Tests. Von Geisteswissenschaftlern mehr genutzt als von der Physik.
						Onlineaccessment vor der Immatrikulation (finden alle unpraktisch, muss allerdings erst an der Masse getestet werden). \\
					\item Uni Giessen: Ilias und Stud.IP,
						Prüfungsanmeldung online, Hochschuldidaktikzentrum: Coaching für Lehrende auf 1:1 Basis. \\
					\item Uni TU Dresden: gibt zentrales System,
						Anmeldung für Übungsgruppen und Übungsblätter werden hochgeladen,
						manche Professoren verwenden eigene Websiten \\
					\item Uni Potsdam: Moodle und eigene Websiten,
						auch was Dropbox ähnliches \\
					\item Uni Konstanz: Ilias für Übungsblätter, Skripte werden oft nicht hochgeladen; Streamingseite in der Physik nicht verwendet, da niemand sich Filmen lassen will; \\
					\item Uni Würzburg: zwei Systeme, Veranstaltungsanmeldung meldet einen direk in dem anderen System an,
						Abgaben können hochgeladen werden in dem System,
						Dozenten verwenden noch eingene Websites
				\end{itemize}

	\subsubsection*{Umsetzung}
		\paragraph{Was wird verwendet?}
			\begin{itemize}
				\item Live-Abstimmungen in Vorlesungen - selber machen, Tools: hmind.org, pingo.com,
				kann sehr gut sein,
				kann Zeitverschwendung sein und nicht ernstgenommen werden - Analogabstimmungen manchmal besser (farbige Karten)

				\item Online-Tests - zum Benoten und zur Selbsteinschätzung, Tools: Moodle,
				hier kommt die Frage auf, ob die Fragen im Nachhinein angezeigt werden.
				http://hmind.org/
				Gute Möglichkeit zur Einbindung eines Quizzes: OpenOLAT

				\item Videos von Vorlesungen
					\begin{itemize}
						\item sehr viele - 1
						\item viele - 0
						\item wenige - eine schwache Hälfte
						\item eigentlich keine - eine starke Hälfte
					\end{itemize}

				\item Streamvorlesungen: niemand hat das
			\end{itemize}

	\paragraph{Wer nimmt auf?}
		bei den Meisten die Fachschaften
		https://video.fsmpi.rwth-aachen.de/
		viele Professoren haben Probleme damit, sich aufnehmen zu lassen \\

		Aufgenommen wird zur Hälfte von Fachschaften zur anderen Hälfte von Studierenden,
		manche Unis bieten Material an,
		manche Fachschaften bieten Kameras und Mikros an, um aufzunehmen. \\

		Aachen bietet den Professoren an, dass sie selber entscheiden können, wer es sieht - fachintern, uniweit, öffentlich\\

		Wenn sich manche Professoren filmen lassen, lassen sich andere leichter filmen. \\
		Warum will's keiner machen: Professoren haben angst, dass Studierende nicht mehr in die Vorlesung kommen \\
		Warum wollen Professoren, dass Studierende in die Vorlesungen kommen? - Damit sie sehen, ob die Studierenden mitkommen oder ob die Augen glasig werden, fehlendes Feedback\\

		Befürchtung zu Aufzeichnungen: Dozent mag die Vorlesung nicht halten, muss aber, spielt dann einfach nur Videos;
		besser als ein gutes Video als eine schlechte Vorlesung\\

		Uni Graz: Preis für E-Learning, Uniweit, cool

\subsubsection{Träume}
	\begin{itemize}
		\item ordentliches anmeldungssystem
		\item zentrale materialquelle
		\item aufnahmesystem für videos
		\item online diskussionsforen
		\item Professoren solln vor der volresung literatur zu der vorlesung online stellen
		\item interaktiveres E-learning, upload von beispielen die von anderen studies bewertet werden, programmierübung
		\item ein Netzwerk das nicht zusammenbricht
		\item online tutorien zum fragen beantworten
		\item touchscreen wo leute drauf schreiben das an die tafel projeziert werden, sollen keine skripten ersetzen
		\item tafeln automatisch abfotografieren
		\item live streams zu vorlesungen soll es geben, verpasst ma krank nichts
		\item quizze wären cool
		\item klausuren online stellen
		\item inhaltsangaben von vorlesungen angeben
		\item mehr ausprobierwille zum e-learning
		\item alle sollen sich gedanken machen wie man e-learning integrieren kann, es soll nichts anderes ersetzen sondern ein teil des ganzen werden, freies studium sollte das ziel sein! balanced use
		\item wir sollen selber online suchen und das dann unseren dozenten vorschlagen wie ma das verwendet
		\item zentrales system - wie in aachen, aachen ist supergut
		\item wir sollten skripten videos und alles miteinander teilen - unis connected
		\item gemeinsame plattform, wir sollten teilen!
	\end{itemize}

	\subsubsection*{Weiteres Vorgehen}
		Folge-AK auf der nächsten ZaPF wäre super \\
		Aufgabe bis zur nächsten zapf: jede Fachschaft bringt was sie hat und bringen kann \\
		schwierigkeiten und erfolge teilen

		\paragraph{Ziel für den nächsten AK}
			Fachschaften sammeln Material \\
			anderes Ziel: wir sollen einen Leitfaden für E-Learning erstellen \\

			Wie überzeugt man Professoren, was kann man machen, was ist cool?

	\subsubsection*{Wünsche}
		\begin{itemize}
			\item Links zu Videos
			\item Links auf Übungsblättern
			\item Animationen für E-Dynamik oder so
		\end{itemize}

	\subsection*{Zusammenfassung}
		Am Anfang war der AK eher auf den Austausch und Vergleich konzentriert. Im Protokoll sind die
		Aussagen der einzelnen Unis zu finden. \\

		Im Allgemeinen war die Stimmung positiv bezüglich E-Learning Angeboten, ergänzend aber nicht ersetzend
		für Vorlesungen. \\

		Im Generellen zeigte sich, dass fast alle das Angebot ihrer Uni als unzureichend empfanden,
		jedoch einige bereits deutlich mehr aufweisen als andere. \\

		Weit verbreitet vorhanden waren zentrale Systeme zum Anmelden, welche in vielen Fällen auch
		Materialsammlungen und Diskussionsräume für Lehrveranstaltungen, zumindest in der Infrastruktur,
		bereitstellen. Die verbreitesten Systeme waren Moodle und Ilias. \\

		Weniger weit verbreitet waren Vorlesungsaufzeichnungen, wo sie bestanden, wurde es zumeißt von den Fachschaften organisiert.

	\subsection*{Fazit}
		Es kam der Wunsch nach einem Folge-AK in Würzburg auf, zur Vorbereitung ist die Materialsammlung in
		den Fachschaften bezüglich Werkzeugen, aber auch erfolgreichen Vorgehensweisen vorgesehen,
		diese werden dann dort zu einer Materialsammlung kombiniert.
