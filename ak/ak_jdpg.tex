% !TEX TS-program = pdflatex
% !TEX encoding = UTF-8 Unicode
% !TEX ROOT = main.tex

\section{AK jDPG und Fachschaft}

  \textbf{Protokoll vom:} 02.06.2018,
  Beginn: 09:15 Uhr,
  Ende: 11:00 Uhr \\
  \textbf{Redeleitung:} Merten (jDPG)\\
  \textbf{Protokoll:} Niklas (Oldenburg) \\
  \textbf{anwesende Fachschaften:} Uni Oldenburg, Uni Bonn, Uni Götteningen/jDPG, Uni Münster, Uni Darmstadt, Uni Braunscheig, Uni Bochum, Uni Würzburg, Uni Rostock

  \subsection*{Informationen zum AK}
    \begin{itemize}
      \item \textbf{Ziel des AKs}: Vermittlung von Best Practices im Umgang mit jDPG-RGs
      \item \textbf{Folge-AK}: nein
      \item \textbf{Zielgruppe}: Alle, bei denen es eine jDPG-Regionalgruppe vor Ort gibt
      \item \textbf{Ablauf}: Austausch
      \item \textbf{Voraussetzungen}: Informieren unter \url{https://jdpg.de/rg}
    \end{itemize}

  \subsection*{Protokoll}
    \paragraph{Was ist die jDPG?}
      \begin{itemize}
        \item Arbeitskreis der DPG (Deutsche Physikalische Gesellschaft)
        \item angebotene Programme
          \begin{itemize}
            \item Schulbegleitendes Programm
            \item Wissenschaftliches Programm
            \item Berufsvorbereitendes Programm
            \item internationale Vernetzung in der ICPS (international conference of physics students)
            \item Hochschule und Gesellschaft
          \end{itemize}
        \item Der Unterschied zwischen Fachschaft und jDPG-Regionalgruppen wird erläutert:
          \begin{itemize}
            \item Finanzen müssen für eine Veranstaltung von der DPG beantragt werden
            \item jDPG vornehmlich Veranstaltungsorganisation $\leftrightarrow$ Fachschaft vornehmlich Gremienarbeit
          \end{itemize}
      \end{itemize}

    \paragraph{Austausch}
      \begin{itemize}
        \item teils Personalunion, teils gegenseitiges Ignorieren
        \item Konflikte selten
        \item Kommunikation häufig mangelhaft
        \item gegenseitige Bewerbung von Veranstaltungen kommt vor, vornehmlich in Orientierungsphase, enge Zusammenarbeit eigentlich nur bei Personalunion
        \item Entscheidende Frage: wie verbessert man die Kommunikation?
        \item Lokale Person zur jDPG als Ansprechpartner in der Fachschaft und anders herum
        \item Gemeinsamer Newsletter
      \end{itemize}
      Es werden Stichpunkte gesammelt, wie die Zusammenarbeit verbessert werden kann (s.u.). Diese Stichpunkte sollen gemeinsam mit der jDPG weiter ausgearbeitet und als Handreichung an die Fachschaften geschickt werden.

    \paragraph{Nächste Schritte}
      \begin{itemize}
        \item Stichwortsammlung (s.u.) an jDPG-Bundesvorstand, dort Gemeinsamen AK auf jDPG-Mitgliederversammlung und Winter-ZaPF vorschlegen, in dem gemeinsam über diese Punkte gesprochen wird.
        \item Wiki-Seite “Was ist eigentlich diese jDPG und was will die von uns” erstellen, in der beschrieben wird, inwiefern Fachschaften von der Zusammenarbeit profitieren.
      \end{itemize}

    \paragraph{Stichwortsammlung}
      \begin{itemize}
        \item Es sollte eine regelmäßige Kommunikation zwischen Fachschafts- und jDPG-Mitgliedern stattfinden.
          \begin{itemize}
            \item Regionale Ansprechpartner auf beiden Seiten
            \item Gegenseitige Einladung auf die Sitzung oder zum Stammmtisch
          \end{itemize}
        \item gemeinsame Veranstaltungen
          \begin{itemize}
            \item intern: Stammtisch, Kennlerngrillen von Fachschaft \& jDPG-Regionalgruppe
            \item gemeinsame Veranstaltungen in Einführungs-Phasen, gemeinsam organisierte Exkursionen
            \item gegenseitiges Bewerben von Veranstaltungen über Newsletter
          \end{itemize}
        \item Beachtung von Terminüberschneidungen bei der Planung von Veranstaltungen
        z.B. nicht gleichzeitiges Legen von regelmäßien jDPG- und Fachschaftsterminen
      \end{itemize}

      $\Rightarrow$ jDPG und Fachschaft stehen nicht in Konkurrenz zueinander, sondern bieten einander ergänzende Angebote
