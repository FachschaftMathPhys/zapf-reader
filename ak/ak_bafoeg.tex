% !TEX TS-program = pdflatex
% !TEX encoding = UTF-8 Unicode
% !TEX ROOT = ../main.tex

\section{AK BAFöG}

  \textbf{Protokoll vom:} 31.05.2018,
  Beginn: 08:10 Uhr,
  Ende: 09:55 Uhr \\
  \textbf{Redeleitung:} Peter Steinmüller (KIT) \\
  \textbf{Protokoll:} Elisa (Darmstadt), Mandy (Potsdam), Lydia (TU Dresden) \\
  \textbf{anwesende Fachschaften:} Brandenburgische Technische Universität Cottbus, Heinrich Heine Universität Düsseldorf, Technische Universität Darmstadt, Friedrich-Alexander-Universität Erlangen-Nürnberg, Technische Universität Bergakademie Freiberg, Ludwig-Maximilians-Universität München, Philipps-Universität Marburg, Universität Potsdam, Universität Rostock, Universität des Saarlandes, Karlsruher Institut für Technologie, Julius-Maximilians-Universität Würzburg, Technische Universität Dresden, Ruhr Uni Bochum

  \subsection*{Informationen zum AK}
    \begin{itemize}
      \item \textbf{Ziel des AKs}: Reso im nächsten AK vorbereiten
      \item \textbf{Folge-AK}: ja
      \item \textbf{Zielgruppe}: alle ZaPFika, die an der Verbesserung von BaFöG mitarbeiten wollen
      \item \textbf{Ablauf}: Austausch
      \item \textbf{Voraussetzungen}: Protokoll der ZaPF in Siegen\footnote{\url{https://zapf.wiki/WiSe17_AK_BAföG}}
    \end{itemize}

  \subsection*{Protokoll}
    \subsubsection*{Elternunabhängig}
      \begin{itemize}
        \item Problem, BaföG wird schnell gestrichen, weil die Eltern etwas zu viel verdienen, dabei sind die Freibeträge oft nicht an die reale Lebenserhaltungskosten der Eltern angepasst, Existenzlücken
        \item (Marburg):fördert Bürokratieabbau und steht im Einklang mit anderen Initiativen
        \item sollte erster Ansprechpunkt sein
        \item derzeit kann nur elternunabhängiges BAföG bei abgeschlossener Erst-Ausbildung erfolgen, das ist im Normalfall Ausbildung + 3 Jahre Arbeiten
        \item Eltern sind nur verpflichtet, erdten Bildungsweg zu bezahen, gibt aber oft Rechtsstreite, BaföG für zweiten Bildungsweg wird oft von Eltern später eingeklagt
      \end{itemize}

    \subsubsection*{Höhere Freibeträge (bei Berechnung)}
      \begin{itemize}
        \item um Existenzlücken zu schließen
        \item (KIT):Wenn Elternunabhängigkeit gefordert wird, sollte dieser Punkt vielleicht gestrichen werden, da dieser im Widerspruch dazu steht.
        \item (Bonn?): Es gibt nicht nur Freibeträge bezüglich der Eltern.
        \item (München): angepasste Freibeträge stehen nicht in Widerspruch zu Elternunabhängigkeit
        \item (München): Zusätzliche Forderung statt alternativer Vorschlag
        \item eine einmalige Erhöhung der Freibeträge ist nur eine kurzfristige Lösung, daher ist es sinnvoller eine regelmäßige Aktualisierung zu fordern
        \item Zusammenlegung mit Aktualisierung der Beträge für die Reso
      \end{itemize}

    \subsubsection*{weniger Bürokratie/mehr Datenschutz}
      \begin{itemize}
        \item nur eigene Personenbezogene Daten, keine Daten von ELtern o.ä.
        \item Streichung/Kürzung innerhalb des Studiums nach 4. Fachsemester, falls das Studium in Regelstudienzeit nicht schaffbar, wird kritische gesehen
        \item (KIT): Einsparung von sehr viel Bearbeitungszeit in Studierendenwerken.
        \item (München):Folgt dieser Punkt nicht aus der Elternunabhänigkeit?
        \item (Marburg, KIT): Zum Teil, aber nicht vollständig, da es auch andere Punkte, wie den Leistungsnachweis nach 4 Semestern umfasst.
      \end{itemize}

    \subsubsection*{interne Regeln (Stadt, Studentenwerk)}
      \begin{itemize}
        \item Einheitliche Regelung bzw. Auslegung bzgl. Anrechnung von Gremiensemestern und so weiter
        \item Anpassung der Beträge an bsp. lokale Mietspiegel
        \item (Marburg): zum Teil ist die Auslegung auch im selben Studierendenwerk bei verschiedenen Bearbeitern unterschiedlich
        \item Stadtabhängigkeit (andere Lebenswerhaltungskosten)
        \item Freibeträge an Stadt anpassen
        \item Vorschlag: Zusammenlegung mit Studiengangswechsel, ist auch interne Regelung
        \item Ausklammerung des Punktes für die Reso
      \end{itemize}

    \subsubsection{Maximale Förderungsdauer}
      \begin{itemize}
        \item es soll nicht unendlich lang sein, Begrenzungen und Streichungen sind wichtig
        \item es sollten trotzdem mehr als nur ein zusätzliches Semester gezahlt werden, Durchschnitt braucht etwa 2 Semester länger
        \item Anpassung der Regelstudienzeit, wenn der Großteil der Studierenden länger braucht
        \item Frage: in welchem Umfang soll erhöht werden? Multipikator passt sich besser an Regelstudienzeit an, steht immer im gleichen Verhältnis
        \item klingt zunächst nicht viel (x 1,5), entspricht in manchen Fällen aber der Maximalstudiendauer
        \item (Cottbus): Sollte eine gerade Zahl sein, da viele Module im Wintersemester ODER Sommersemester angeboten werden
        \item Antrag auf Verlängerung der Förderungsdauer wird in verschiedenen Studierendenwerken sehr verschieden ausgelegt (also interne Regelungen)
        \item meist nur Beteiligung in gewählten Gremien für Weiterförderung (nötig), ehrenamtlichen Engagement sollte mehr gewürdigt werden
        \item Umbenennung der Forderung in: realistischere Förderungsdauer
        \item Skandinavische Förderungssysteme als Vorbild einer entsprchenden Regelung
        \item Zusammenlegung mit Studiengangswechsel
      \end{itemize}

    \subsubsection{(Dresden): Studiengangswechselß}
      \begin{itemize}
        \item (Dresden): BAföG Anspruch nur bei Studiengangswechsel bis zum 2. Fachsemester
        \item man kann noch bis 4. Fachsemester rausgeprüft werden
        \item Fortzahlung abhängig von Studiengang, zu dem gewechselt wird
        \item Wie kann das mit den anderen Punkten zusammengetragen werden?
        \item Vorschlag: bei Förderungsdauer miteinbringen, da es auch eine Anpassung an reale Situation ist
        \item Jedoch hier kein Zurücksetzen der Förderungshöchstdauer, sodass ggf. eine Lücke entsteht durch die endlos studiert und gezahlt wird.
        \item Mehrfacher Studiengangwechsel zur Neuorientierung muss möglich sein.
        \item Besonders wichtig: “Wechsel bis” aus dem Gesetz streichen
        \item (Bonn)Vorschlag: Förderung für gewisse Zeit, unabhängig davon, was ich studiere und wie oft ich wechsel
        \item (Würzburg): spricht aber gegen den Punkt, das volle Studium zu bezahlen, nach Wechsel fängt Regelstudienzeit von vorn an
        \item (Bonn): man muss aber Grenze setzen, um Missbrauch zu verhindern
        \item Förderungsdauer wird bei sinnvollen/begründeten Wechsel erhöht
        \item (Würzburg): wie wird “begründet” definiert? Subjektive Entscheidung
        \item (Darmstadt): Unterscheidung zwischen freiwilligem und unfreiwilligem Wechsel?
        \item skeptisch, das in die Reso zu schreiben, ist ein zu heikles Thema
        \item (Rostock): Statistiken zu Studiengangswechseln ansehen und einfließen lassen
        \item Kompromiss: Studiengangsdauer wird voll bezaht, aber nur für einen Wechsel (allgemeine Zustimmung)
        \item Beim ersten Wechsel beginnt Förderungsdauer neu, bei weiteren Wechseln nicht mehr
        \item (Würzburg): Durch Neustrukturierungen oder einfach spätere Vorlesungen und Module in höheren Semestern können auch ein Grund für Wechsel sein
        \item Kritik (Bonn): Wechsler sollten nicht bevorzugt werden gegenüber denen, die von Anfang an durchziehen
        \item (Dresden \& Bonn): Wechsler sind dadurch nicht zwingend bevorteilt, sie haben den selben Abschluss am Ende und haben die Zeit für dn anderen Studiengang quasin “umsonst” investiert
        \item in einer Zeit mit so viel Berufsauswahl ist eine erste Fehlentscheidung stark gerechtfertigt
        \item Zusammenlegung mit realistischer FHD für die Reso
        \item Wie der Absatz in der Reso konkret formuliert/umgesetzt wird, muss nochmals diskutiert werden.
      \end{itemize}

    \subsubsection{öftere Aktualisierung der Beträge/des Gesetzes}
      \begin{itemize}
        \item Zusammenlegung mit der Erhöhung der Freibeträge.
        \item Anpassung an Mietspiegel, Mietpreise ändern sich oft
        \item Wohngeldbetrag ist unrealistisch, Wohnungen/Zimmer kosten in den meisten Fällen deutlich mehr
      \end{itemize}

    \subsubsection{Priorisierung der Forderungen und Struktur der Resolution}
      \begin{itemize}
        \item (KIT): Reso soll maximale Forderung sein.
        \item Struktur der Reso:
          \begin{itemize}
            \item öftere Aktualisierung
            \item Elternunabhängigkeit
            \item weniger Bürokratie (evtl. mehr Datenschutz)
            \item Förderungsdauer/Studiengangswechsel
          \end{itemize}
      \end{itemize}
\vspace{5mm}

Der AK spricht sich mehrheitlich für diese Reiheinfolge aus.
Peter bereitet für Würzburg eine konkrete Struktur, vielleicht auch schon einige Sätze vor, sodass dort konkret diskutiert, geschrieben werden kann. Adressaten werden auch dann festgelegt.
