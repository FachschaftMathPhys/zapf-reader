% !TEX TS-program = pdflatex
% !TEX encoding = UTF-8 Unicode
% !TEX ROOT = main.tex

\section{AK Umgang mit Förderabsagen}

	\textbf{Protokoll vom:} 31.05.2018,
	Beginn: 10:30 Uhr,
	Ende: 12:00 Uhr \\
	\textbf{Redeleitung:} Björn (RWTH Aachen) \\
	\textbf{Protokoll:} Patrick (Uni Konstanz) \\
	\textbf{anwesende Fachschaften:} RWTH Aachen, Rheinische Friedrich-Wilhelms-Universität Bonn, Friedrich-Alexander-Universität Erlangen-Nürnberg, Universität Konstanz, Universität Rostock, Universität Siegen, Julius-Maximilians-Universität Würzburg

	\subsection*{Informationen zum AK}
		\begin{itemize}
			\item \textbf{Ziel des AKs}: Wie geht man mit Absagen von großen Fördereren für die ZaPF um?
			\item \textbf{Folge-AK}: ja (\url{https://zapf.wiki/WiSe17_AK_Förderungsabsagen})
      \item \textbf{Materialien}: Protokoll aus Siegen
			\item \textbf{Zielgruppe}: jeder, der an zukünftigen ZaPFen interessiert ist
			\item \textbf{Ablauf}: Bericht zur aktuellen Lage, Ideensammlung
			% \item \textbf{Voraussetzungen}:
		\end{itemize}

  \subsection*{Protokoll}
    \paragraph{Ausgangslage}
      Siegen wurde nicht durch das BMBF gefördert. Für Heidelberg war lange Zeit nicht klar, ob die Förderung durch das BMBF ebenfalls abegsagt wird.
      Durch eine Großspende konnte jedoch auf die BMBF-Förderung komplett verzichtet werden.

    \paragraph{Aktuelle Lage}
     Gesamtfördersumme geht zurück, bei steigender Anzahl an Förderanfragen. Die Chance, nicht mehr gefördert zu werden, steigt demnach. \\
     Die nächsten ZaPFen in Würzburg und Bonn berichten: Haben Probleme, wenn der Bundeshaushalt nicht vor der Sommerpause beschlossen wird.
     Die Bewerbungsfristen für die Förderung werden dieses Jahr nach hinten verschoben, da auf den Beschluss des Bundeshaushaltes gewartet wird.
     Bei Würzburg wird es auch so schon sehr eng mit der Förderung, wegen dem Zeitplan der Haushaltsdebatte. \\

    \paragraph{Langfristige Planung}
      Notfallfinanzierung über Fördermitglieder soll dauerhaft stehen. Auch könnte ein Sponsoringkreis über Alumni aufgebaut werden. \\

      Weitere Möglichkeiten für (besseres) Sponsoring:
      \begin{itemize}
        \item Sponsor-Werbe-Mappe
        \item T-Shirts mit Werbung bedrucken
        \item Anscheinend gibt es auch europäische Mittel, die man anfragen könnte. Johannes (Bonn) informiert sich im Rahmen der ZaPF in Bonn, welche Möglichkeiten es gibt und mit welchem Aufwand diese verbunden sind.
      \end{itemize}
