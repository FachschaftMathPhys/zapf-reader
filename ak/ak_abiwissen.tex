% !TEX TS-program = pdflatex
% !TEX encoding = UTF-8 Unicode
% !TEX ROOT = main.tex

\section{AK Abiwissen}

	\textbf{Protokoll vom:} 02.06.2018,
	Beginn: 11:30 Uhr,
	Ende: 13:30 Uhr \\
	\textbf{Redeleitung:} Leon Nutzinger (FU Berlin) \\
	\textbf{Protokoll:} Laura Weber (Uni Würzburg) \\
	\textbf{anwesende Fachschaften:} FU Berlin, Uni Bielefeld, TU Braunschweig, Ruhr-Uni Bochum, Uni Cottbus, Uni Darmstadt, Uni Dortmund, Uni Duisburg-Essen, Uni Frankfurt, Uni Freiburg, Uni Gießen, Uni Greifswald, Uni Halle, Uni Innsbruck, Uni Jena, TU Kaiserslautern, Karlsruher IT, Uni Mainz, LMU München, Uni Siegen, Uni Tübingen, Uni Wuppertal, Uni Würzburg,

	\subsection*{Informationen zum AK}
		\begin{itemize}
			\item \textbf{Ziel des AKs}: Diskussion der in der Einleitung erwähnten Materialien
			\item \textbf{Folge-AK}:  ja, zuletzt in Siegen (\url{https://zapf.wiki/WiSe17_AK_Abiwissen})
			\item \textbf{Vorwissen}: Teilnahme an vergangenen AKs hilfreich
      \item \textbf{Materialien}: Protokolle und Positionspapiere vorheriger AKe
			\item \textbf{Zielgruppe}: alle an der Thematik Interessierten
			\item \textbf{Ablauf}: Sichtung und Diskussion der Materialien
			\item \textbf{Voraussetzungen}: keine
		\end{itemize}

  \subsection*{Einleitung}
    Geschichtlicher Abriss: Sonja berichtet was bisher geschah \url{https://zapf.wiki/Bachelor-Master-Umfrage}. \\
Zunächst wurde die Bachelor-Master-Umfrage durch ZaPF und jDPG  durchgeführt, um zu erruieren, wie das ganze implementiert wurde.
Nach 4 Jahren wurde sie wiederholt, um die Entwicklung anzuschauen. Neuer Schwerpunkt ist der Studieneinstieg.
Umfrageergebnisse wurden auch für ZaPF-AKs genutzt
Bislang gab es keine große Veröffentlichung der Ergebnisse, da diese zu umfangreich sind.
Fachschaften können die Daten von ihrer eigenen Hochschule erfragen
Es wurde ein Positionspapier verabschiedet, dass die ZaPF die Umfrage auch in Zukunft alle vier Jahre wiederholen möchte.
Es soll einerseits Kernfragen geben, die immer wieder gefragt werden sollen, um langfristige Entwicklungen sehen zu können, andererseits Spezialfragen, die spezielle einzelne Themen thematisieren.
In den letzten Wochen wurde relativ viel Zeit investiert, um auf Grundlage der bisherigen Fragebögen, neue zu erstellen.

  \subsection*{Protokoll}
    Dies ist ein Nachfolge-AK aus Siegen, Berlin, Dresden, Konstanz und Aachen. Letzerer sieht sich selbst als Folge-AK eines AK auf der ZaPF in Karlsruhe 2012. \\

    Der AK beschäftigt sich mit dem Wissen, dass die Studierenden aus dem Abitur ins Physikstudium einbringen. Die immer häufiger und immer länger werdenden Brückenkurse legen nah, dass das Abiturwissen nicht mehr auszureichen scheint. Gleichzeitig sind \flqq de facto\frqq verbindliche Brückenkurse problematisch, da sie meistens keine Leistungspunkte geben und den Semesterbeginn inoffiziell nach vorne verlegen. In diesem Frühling waren einige offene Briefe zu diesem Thema von Hochschulprofessoren und Didaktikvereinigungen verschickt worden (siehe unten), zu denen wir in Berlin in einem Positionspapier Stellung genommen hatten, siehe Reader zur \flqq ZaPf in Berlin\frqq S.76f. \\

Die Datenlage zum Vorwissen der Erstsemester ist leider relativ klein. Da an vielen Universitäten der Brückenkurs auch von den Fachschaften organisiert wird, kam die Idee auf, dort einen Test durchzuführen, nach aktuellem Plan zu Beginn des nächsten Wintersemesters im Oktober 2018. Dazu wurden bei der letzten ZaPF die Fachschaften angefragt, ob bei ihnen solche Tests gemacht werden und ob sie uns eine Kopie zuschicken könnten, siehe Reader \flqq ZaPF in Berlin\frqq, S.77f. \\

In diesem AK soll nun die weitere Planung besprochen werden. Ebenso sollen die zugeschichten Tests gesichtet werden.

Im Anschluss sind einige Dateien dazu verlinkt. Besonders interessant sind

\begin{itemize}
\item Die Studie von Borowski \footnote(\url{https://link.springer.com/article/10.1007/s40573-016-0041-4}), die aussagt, dass sich heutige Erstsemester in ihren mathematischen Kenntnissen nicht signifikant von den Erstemestern von vor 40 Jahren unterscheiden.
\item Der offene Brief vom 17.3.17 von über 50 ProfessorInnen an die Kultusminister \footnote{\url{https://zapf.wiki/Datei:Offener_Brief_Mathematikunterricht_Kompetenzorientierung.pdf}}, in denen sie den aktuellen Mathematikunterricht, insbesondere die Kompetenzorientierung scharf kritisieren.
\item Die Antwort auf diesen offenen Brief von Mathematikdidaktik-Professuren \footnote{\url{https://zapf.wiki/Datei:Mathematiker-distanzieren-sich-vom-mathematiker-brandbrief.pdf}}
\item Die Erwiderung der Deutschen Mathematiker-Vereinigung (DMV), der Gesellschaft für Didaktik der Mathematik (GDM) und des Verbands zur Förderung des MINT-Unterrichts (MNU) \footnote{\url{https://zapf.wiki/Datei:Stellungnahme_DMV_GDM_MNU_20.04.2017.pdf}}
\end{itemize}

\subsubsection*{Besprechung der letzten AKe}
\begin{outline}
\1 \textit{Anmerkung: Brückenkurse=online, Vorkurs=vor Ort}
\1 Kurze geschichtliche Einleitung. Wesentliche Punkte:
\2 Es wurde vor ca. 30 Jahren und 5 Jahren evaluiert (Borowski), wie viel mathematisches Wissen ein Absolvent des Abiturs besitzt.
\2 Idee war, einen Test selbst zu erstellen, der über alle Fachschaften verteilt werden soll. Die Inhalte sollten den selben Zweck haben, wie der oben genannte.
\2 Im letzten AK wurden mehrere wesentliche Ziele herausgearbeitet:
\3 Ziel 1: Fähigkeiten sammeln, um Brückenkurse anzupassen (zu optimieren)
\3 Ziel 2: Vermittlung in der politischen Diskussion um den Bildungsstand
\3 Ziel 3: Vergleich zwischen Kerncurriculum und tatsächlichen Fähigkeiten
\3 Ziel 4: Vergleich zwischen tatsächlichen Fähigkeiten und an der Uni geforderten Fähigkeiten
\3 Ziel 5: Abfrage des Verständnisses von mathematischen Konzepten
\1 Datensammlung über BaMa-Umfrage
				\1 Vor kurzem veröffentliche MINT Studie (\url{https://www.ipn.uni-kiel.de/de/das-ipn/abteilungen/didaktik-der-mathematik/forschung-und-projekte/malemint}):
					\2 Abfrage, was Absolventen eines Abiturs können müssen / Grundwissen für MINT-Studiengänge
					\2 Grobes Ergebnis war, dass Grundrechenarten, sowie Bruchrechnen vermittelt werden sollten
				\1 Vorkurse funktionieren an manchen Unis besser als an anderen
					\2 Sollen Vorwissen auffrischen, nicht neuen Stoff einführen
					\2 Konzepte von Null an Einführen kaum möglich
					\2 Wissen um mathematische Werkzeuge wird vorausgesetzt
				\1 Vorschlag: Liste von Standards erarbeiten, was Abiturienten können sollten
				\1 Herausfinden, was die Differenzen zwischen Lehrplanzielen und den Voraussetzungen der Hochschulen sind
				\1 Mögliche Ziele des AKs:
					\2 Ampel-Liste kommentieren
					\2 \flqq Test\frqq, der an Fachschaften verteilt wird
				\1 Idee: Die Ampel-Liste evaluieren und daraus eventuell den Test gestalten
				\1 Evaluation der Ziele des Protokolls der letzten ZaPFen
				\1 Frage: Was können wir erreichen bzw. was machen wir mit den Ergebnissen von Umfragen, die wir machen?
				\1 Anmerkung: Wir können die Schule nicht ändern, aber die Universitäten darauf hinweisen, dass sich bei den Schülern die tatsächlichen Fähigkeiten signifikant von denen im Lehrplan vorausgesetzten unterscheiden.
			\end{outline}

			\paragraph{Meinungsbild zu heute zu diskutierendem Thema aus oben genannten Zielen:}
				\begin{tabular}{c|r|r|r}
					Ziel & Für Ziel	& Für Auslagern	& Gegen Auslagern  \\ \hline
					1 &	10 & 12	& 1 \\
					2	& 3	& 0	& 1 \\
					3	& 6	& keine Abstimmung & keine Abstimmung \\
					4 &	Mehrheit & keine Abstimmung &	keine Abstimmung \\
					5 &	9	& 0	& 1
				\end{tabular}

				\textbf{Ergebnis:}
				\begin{itemize}
					\item Wir konzentrieren uns heute auf das Ziel 4
					\item Ziel 1 wird in einen eigenen Folge-AK ausgelagert
				\end{itemize}

		\subsubsection*{Vergleich zwischen tatsächlichen Fähigkeiten und an der Uni geforderten Fähigkeiten}
			\begin{outline}
				\1 Diskussion über den tatsächlichen Adressaten, das Format und die Aussagekraft eines Fragebogens
				\1 Fragebogen von Berlin als Beispiel (wurde von einem Didaktiker kontrolliert)
				\1 Problem: Ein solches Projekt ist zu groß für die ZaPF
					\2 Möglich Lösung wäre, die Thematik an eine "Dauerstelle" (z.B. Promovierender der Didaktik) auslagern. Die ZaPF würde die Person unterstützen.
					\2 Uni Duisburg-Essen und Bochum hatten eine Umfrage dorthin möglicherweise eine Kooerationsumfrage schicken
					\2 KIT: Es wird am Anfang und am Ende eines Vorkurses ein Test gemacht, Ergebnisse sind vorhanden, es gibt die Möglichkeit dieses System auch an anderen Unis einzuführen
					\2 KIT: statt selbst einen riesen Arbeitsaufwand mit einer eigenen Umfrage zu haben, einfach vorhandene Ergebnisse verwenden und Uniresourcen nutzen
					\2 Bisheriges Ergebnis: Suche nach Kooperation mit Fachdidaktikern die möglicherweise bereits auf diesem Gebiet forschen
					\2 Vorschlag dazu: Wir geben unseren Inhalt der helfenden Person vor, damit genau unsere Wünsche evaluiert werden.
						\3 Anmerkung: Lieber nicht zu viel vorgeben
				\1 Frage: Kann man eine derartige Umfrage überhaupt deutschlandweit machen? da Bildung ja Ländersache ist
					\2 Umfrage länderspezifisch oder so allgemein wie möglich
					\2 Zeitlicher Aufwand könnte dazu führen, dass die Thematik sich verliert
				\1 Hilfsangebot für das Auswerten stellen
				\1 Diskussion über das Format der möglichen Umfrage. Ankreuzen würde alleine nicht reichen.
				\1 Wann und wie bringt man den Test zu den Studierenden?
			\end{outline}

		\subsubsection*{Endergebnis}
			\begin{itemize}
				\item Ziel 1 wird ausgelagert
				\item Ziel 4 und 5 wird in diesem AK aktiv besprochen
				\item Kooperation mit (idealerweise Mathematik-) Didaktikern wird angestrebt
			\end{itemize}

		\subsubsection*{Unsere Aufgaben}
			\begin{outline}
				\1 Antragstext für das Plenum mit dem Inhalt: Der StaPF schickt Mail rum. Inhalt wird im Back-Up AK formuliert.
					\2 Begründung für Antrag
				\1 Anschreiben muss formuliert werden
					\2 möglicherweise mit beiligenden Fragen und/oder Metadaten die wir gerne erhoben hätten
						\3 an Fachschaften
						\3 an Didaktiker
			\end{outline}


	\subsection*{Protokoll: Backup-AK}
		\begin{itemize}
			\item Andere MINT BuFaTas einbeziehen (später bei Bedarf)
			\item Mit KFP abstimmen
		\end{itemize}

		\paragraph{Was wir bieten können}
			\begin{itemize}
				\item Wir können die Einholung der Daten übernehmen
				\item Manpower zur Verarbeitung der Ergebnisse (z. B. Digitalisierung, Korrektur)
				\item Outreach an die meisten Hochschulen/Universitäten Deutschlands, Österreich \& der Schweiz
				\item Input in Form von bestehenden Bögen und bisherigen Ergebnissen der Befragungen an einzelnen Universitäten/Hochschulen
			\end{itemize}

		\paragraph{Wen wir suchen}
			Mathematikdidaktiker \\
			Bereit zu Open Data (öffentl. Datenbank der Ergebnisse und der Fragebögen)

		\paragraph{Was wir wünschen}
			Erstellen eines forschungsrelevanten Tests, in Kooperation, mit folgenden Inhalten und Zielen:
			\begin{outline}
				\1 Vor Beginn sämtlicher universitärer Veranstaltungen
				\1 Unser Ziel ist es, eine möglichst große Datenmenge zu sammeln, um flexible Auswertungensarten zu ermöglichen
				\1 Vergleich der \textit{tatsächlichen} und \textit{an der Uni geforderten} Kenntnisse:
					\2 kurzfristiger zeitlicher Rahmen (<5 Jahre) für die erste Erhebung
					\2 evtl. langfristige Projekte mit anderen BuFaTas aus dem MINT-Bereich
				\1 Potentielle Forschungsfragen:
					\2 Welche der als notwendig erachteten Kenntnisse sind in besonderem Maße nicht vorhanden?
					\2 Ist ein angehender Physikstudent durch seine Schulbildung insgesamt adäquat auf dieses Studium vorbereitet?
					\2 Gibt es signifikante Korrelationen der Ergebnisse mit einzelnen Kriterien der Metadaten-Abfrage?
				\1 Metadaten-Abfrage nach Vorlage von Berliner Entwurf
				\1 Beachte die als wichtig erachteten Punkte der KFP (\url{https://www.kfp-physik.de/dokument/KFP-Empfehlung-Mathematikkenntnisse.pdf})
				\1 Nochmal anschauen:
					\2 Kurvenscharen $\sin$, $\exp$, Polynome (gegebenenfalls nicht notwendig)
					\2 Partielle Integration (gegebenenfalls notwendig)
					\2 Substitutionsregel bei Integralen (gegebenenfalls notwendig)
				\1 Zusätzlicher Punkt:
					\2 Aussagenlogik
			\end{outline}

		\paragraph{Möglichkeiten ("Verhandlung")}
			\begin{itemize}
				\item Karenzzeit (vor dem Zugänglichmachen für die Allgemeinheit/Dritte) zur Erstverabeitung der Daten anbieten
				\item Zusätzliche Aufgaben für eigene Forschungsfragen
			\end{itemize}

		\paragraph{Modus}
			\begin{itemize}
				\item Fachschaften bitten, persönlich und direkt bei ihren Mathedidaktikern anzufragen
				\item Anhang mit Anschreiben direkt an die Mathematikdidaktiker
			\end{itemize}
