% !TEX TS-program = pdflatex
% !TEX encoding = UTF-8 Unicode
% !TEX ROOT = main.tex

\section{AK Hörsaal-Sponsoring}

	\textbf{Protokoll vom:} 31.05.2018,
	Beginn:10:30 Uhr,
	Ende: 12:30 Uhr \\
	\textbf{Redeleitung:} Lisanne (TU Darmstadt) \\
	\textbf{Protokoll:} Elisa (TU Darmstadt) \\
	\textbf{anwesende Fachschaften:} Technische Universität Chemnitz, Technische Universität Darmstadt, Ernst Moritz Arndt Universität Greifswald, Friedrich-Schiller-Universität Jena, Christian-Albrechts-Universität zu Kiel, Universität Konstanz, Fachhochschule Lübeck, Universität Rostock, Julius-Maximilians-Universität Würzburg, Universität Wien, Technische Universität Dresden

	\subsection*{Informationen zum AK}
		\begin{itemize}
			\item \textbf{Ziel des AKs}: Resolution gegen Hörsaalsponsoring und ständige Werbung auf dem Campus
			\item \textbf{Folge-AK}:  ja, es gab dazu einen AK in Siegen (\url{https://zapf.wiki/WiSe17_AK_HoersaalBranding})
			\item \textbf{Vorwissen}: Infos zu Hörsaalsponsoring an eurer Uni (siehe oben), Definition von Werbung/Sponsoring aus dem Siegener AK
      \item \textbf{Materialien}: \url{https://kif.fsinf.de/wiki/KIF460:Resolutionen/Ablehnung_Sponsoring_und_Werbung_Lehr/Lernräume}, \url{https://www.tu-darmstadt.de/verbunden_bleiben/spenden_sponsoring/werbung_und_promotion/index.de.jsp}: Website der TU Darmstadt, die für Raumsponsoring wirbt
			\item \textbf{Zielgruppe}: alle an der Thematik Interessierten
			\item \textbf{Ablauf}: Input aus Darmstadt, Reso-Idee, Diskussion, Resolution schreiben
			\item \textbf{Voraussetzungen}: keine
		\end{itemize}

  \subsection*{Einleitung}
    Geschichtlicher Abriss: Sonja berichtet, davon was bisher geschah \url{https://zapf.wiki/Bachelor-Master-Umfrage}. \\
    \begin{itemize}
      \item Umfrage zuerst durchgeführt (durch ZaPF und jDPG), als die meisten Unis Bachelor/Master eingeführt haben, um zu “erforschen”, wie das ganze implementiert wurde
      \item Nach 4 Jahren wurde sie wiederholt, um die Entwicklung anzuschauen. Neuer Schwerpunkt: Studieneinstieg
      \item Umfrageergebnisse wurden auch für ZaPF-AKs genutzt
      \item Bislang keine große Veröffentlichung der Ergebnisse, da viel zu umfangreich
      \item Fachschaften können die Daten von ihrer eigenen Hochschule erfragen
      \item Es wurde PosPapier verabschiedet, dass wir die Umfrage auch in Zukunft alle vier Jahre wiederholen wollen
      \item Es soll einerseits Kernfragen geben, die immer wieder gefragt werden sollen, um langfristige Entwicklungen sehen zu können, andererseits Spezialfragen, die spezielle einzelne Themen thematisieren.
      \item In den lezten Wochen wurde relativ viel Zeit investiert, um auf Grundlage der bisherigen Fragebögen, neue Fragebögen zusammenzustellen
    \end{itemize}

  \subsection*{Protokoll}
	 	Einleitung von Lisanne (Darmstadt): Folge-AK zum AK in Siegen. Darlegung der Situation an der TU Darmstadt: Seit WS 16/17 gibt es bei uns zwei gesponsorte Hörsäle, den Bosch Hörsaal und den Software AG Hörsaal. Die betroffenen Fachbereiche und die Hochschulöffentlichkeit wurden dabei lediglich informiert, nachdem die Entscheidung des Sponsorings gefällt wurde. Seitdem beschäftigen wir uns in der zentralen und dezentralen Hochschulpolitik mit dem Thema und positionieren uns zum Einen klar gegen das Sponsoring, kritisieren zum Anderen aber auch das Verfahren der Einführung und versuchen dieses in die Gremien der akademischen Selbstverwaltung einzubringen.
		Wir könnten den AK auch ausweiten, über das Hörsaalsponsoring hinaus, zu Werbung (von Unternehmen) am Campus. \\
		Frage: Wie sieht Hörsaalsponsoring an anderen Unis aus? Wie werden eingenommene Mittel verwendet?
		\begin{outline}
			\1 (Bonn): Gesponsorte Professur durch die Telekom. Mittel, die durch solche Sponsoren eingenommen wurden, sind direkt in die Lehre geflossen.
			\1 (Lübeck): Kein Hörsaalsponsoring. Wurde diskutiert, wobei von studentischer Seite ein Veto eingelegt wurde. Es ist aber unklar, ob das Präsidium dies weiterhin anstrebt.
			\1 (Jena): Es gibt einen gesponserten Hörsaal.
			\1 (Würzburg): Wenn die Mittel sinnvoll verwendet werden, kann das Sponsoring auch eine gute Sache sein.
			\1 (Bonn): Im Positionspapier sollten wir als ZaPF auf keinen Fall anklingen lassen, dass wir Hörsaalsponsoring befürworten.
			\1 (Darmstadt): Die Mittel aus Sponsoring sollten nicht essentiell für die Uni sein. Instandhaltung von Hörsälen etc. muss von der Uni finanziert werden können.
			\1 (Rostock): Einige Werbemaßnahmen, z.B. für Plasmaspenden sind positiv zu sehen.
			\1 (Darmstadt): Anmerkung, dass es vornehmlich um kommerzielle Werbung gehen soll.
			\1 (Wien): Hörsaalsponsoring ist keine dezente Werbung, sondern Manipulation.
			\1 (Kiel): Hörsaalsponsoring wird nicht ganz so kritisch gesehen, soweit es keine riesigen Werbebanner gibt.
			\1 (Darmstadt): Durch Hörsaalsponsoring werden Studenten einseitiger Werbung ausgesetzt. An der TU Darmstadt sollen bis zu 12 Hörsäle gesponsort werden. Das Sponsoring umfasst Namensschild, Einblendung des Logos beim EInschalten des Beamers und Name im Verwaltungssystem.
			\1 (Jena): In Jena gibt es im Verwaltungssystem keine Namenseinblendung des gesponsorten Hörsaals.
			\1 (Wien): Namensgebung des Hörsaals ist wesentlich kritischer, als Werbebanner auf dem Campus. Da die Universität eine staatliche Einrichtung ist.
			\1 (Bonn): Wie ist das dann bei nicht staatlichen, privaten Universitäten?
			\1 (Greifswald): Lehre sollte frei von wirtschaftlicher Werbung sein! \\
			\textit{Dieses Thema nehmen wir aus der weiteren Diskussion heraus, da dies hier nicht zielführend ist und wir nicht genug Information über die Organisation von privaten Unis haben.}
			\1 (Darmstadt): Zusammenfassung bis jetzt: Das aus der Diskussion hervorgehende Bild der Positionen reicht von gesamter Ablehnung von Werbung auf dem Campus bis zur Ansicht, dass Sponsoring, wenn es einen konkreten Nutzen für die Uni hat, okay ist. \\
			\textit{Weiteres Vorgehen: Positionspapier gegen dauerhafte Werbung am Campus (Definition siehe letzter AK), insbesondere Hörsaalsponsoring (Name Hörsaalsponsoring vielleicht nicht eindeutig, nähere Definition)}
			\1 (Greifswald): generell dauerhafte Werbung sollte noch näher diskutiert werden. Werbung in Lern- und Lehrräumen generell sollte kritisiert werden.
			\1 (Wien): Probleme mit möglichen Monopolen, eventuell nur Förderung von Monopolen.
			\1 (Bonn): Es gibt Bereiche, die werbefrei sein sollten (Bibiliotheken, Übungsräume, Hörsaale) insbesondere auch von Werbeeinblendungen in Lehrveranstaltungen. Die Universität als staatlicher Ort des Lernens \& der Bildung, sollte nicht auf die Einwerbung von Geldern durch solche Werbung angewiesen sein. Falls dies aber der Fall ist, sehen wir die Probleme und Lösungsansätze an anderer Stelle.
			\1 (Chemnitz): Näheres Eingehen auf Art der Werbung, Kriterien und eventuelle auch auf die  Mittelverwendung
			\1 (Bonn, Würzburg): Priorisierung Werbung im Raum selbst, gegenüber Werbung außerhalb der Räume.
			\1 (Kiel): Reine Namensgebung wird weniger kritisch gesehen, als ein Einblenden von Logos
			\1 \underline{Meinungsbild}: Wollen wir ein Positionspapier zu Werbung auf dem Campus? \\ $\rightarrow$ einstimmige Zustimmung
			\1 (Bonn ): Zunächst allgemeine Positionierung, dann weitere Spezifizierungen.
			\1 (Jena): Wichtig ist es, die Art der Werbung auf kommerzielle Werbung zu spezifizieren.
			\1 (Bonn): Weiche Formulierung für Positionierung allgemein gegen Werbung.
			\1 (Darmstadt): Positionierung im Positionspapier sollte nicht schärfer sein, als die in der Reso, die auf der nächsten ZaPF entstehen soll. Allgemeine Positionierung gegen Werbung findet eventuell keinen Konsenz auf der ZaPF.
			\1 (Greifwald):  Es sollte erstmal eine Positionierung gegen Werbung in Lehr- und Lernräumen geben.
			\1 (Wien): Im Positionspapier kann durchaus stehen, dass kommerzielle, allgemeine Werbung auf dem Campus abgelehnt wird.
			\1 \underline{Abstimmung}: Wollen wir uns auch allgemein gegen kommerzielle Werbung positionieren? \\ $\rightarrow$ Zustimmung mit Enthaltung
		\end{outline}
		Aufteilung auf zwei Kleingruppen, im Folgenden getrennt protokolliert.

		\subsubsection*{Positionierung zur allgemeinen, kommerziellen Werbung auf dem Campus}
			\begin{outline}
				\1 Gegen kommerzielle Werbung, deren einziger Zweck Bildung von Markenbewusstsein
				\1 Wenn es keinen sachlichen Mehrwert gibt \\
				$\rightarrow$ Bspw. nicht gegen Hörsaalbau; Hörsaalbranding/-sponsoring schon
				\1 Auf Hochschulgelände und -gebäude
			\end{outline}

			\paragraph{Formulierung}
				Die ZaPF spricht sich gegen kommerzielle Werbung auf dem Gelände und in den Gebäuden der Hochschule aus, welche nur dem Zweck des Markenbewusstseins dient und keinen sachlichen Mehrwert hat. Dieses folgt daraus, dass Hochschulen ein Ort freier Bildung und Forschung sein sollen.

 			\paragraph{möglicher Endsatz}
 				Der Aufbau von Markenbewusstsein durch die im Vorigen erwähnte Werbung widerspricht der Freiheit der Studierenden eine eigene, nicht vorurteilsbehaftete Karriereentscheidung zu fällen.

		\subsubsection*{Positionierung zu kommerzieller Werbung in Lehr- und Lernräumen}
			\begin{outline}
				\1 Definition von Lehr- und Lernräumen: Bibliotheken, Übungs- \& Seminarräume, Hörsäle
				\1 Lernzentren ausnehmen, da solche Arbeits- und Aufenthaltsräume
				\1 Lehre und lernen sollte an Universitäten (staatliche Einrichtungen) nicht solch kommerzieller Werbung ausgesetzt werden
				\1 in Lernräumen: Aspekte Ablenkung und Beeinflussung. In Lernräumen eindeutig Konzentration auf Lenen!!
				\1 uniinterne Kommunikation ist davon ausgenommen, aber keine andauernde Werbung (dh Plakate heißen wir nicht gut, Ankündigungen vor der Vorlesung sind okay)
			\end{outline}

			\paragraph{Formulierung}
				Insbesondere sind wir der Meinung, dass in Räumen der Lehre und des Lernens (Bibliotheken, Hörsäle, Übungsräume) bei Lehr- und Lernbetrieb das Arbeiten ohne Beeinflussung durch dauerhafte Werbung möglich sein muss. Sinn der Lehrveranstaltungen und des Lernbetriebs ist es, dass Studierende frei von Interessen Dritter Fachinhalte erlernen und diskutieren können und Lehrende Lehrinhalte frei vermitteln können. Diese Arbeitsatmosphäre wird durch Werbung beeinträchtigt.
				Lediglich die temporäre Kommunikation von universitätsinternen Informationen sollte weiterhin möglich sein.
				Hingegen ist kommerzielle Werbung (erklärende Fußnote), insbesondere Hörsaal- und Raumbranding (definierende Fußnote) nicht hinnehmbar.

		\subsubsection*{Weiteres Vorgehen}
			Wir treffen uns im Kreise aller Freiwilligen heute Abend um 20 Uhr vor Raum 1.404, um die Formulierung weiter auszufeilen. Diese wollen wir morgen mit in die AK-Vorstellung nehmen und anschließend in der Postersession diskutieren.

		\subsubsection*{Vorschlag für die Vorstellung der AKs und die Postersession}
			Die ZaPF spricht sich gegen kommerzielle Werbung auf dem Gelände und in den Gebäuden von Hochschulen aus, welche nur dem Zweck des Markenbewusstseins dient und keinen sachlichen Mehrwert hat. Dies folgt daraus, dass Hochschulen ein Ort freier Bildung und Forschung sein sollen.
			Der Aufbau von Markenbewusstsein durch die im Vorigen erwähnte Werbung widerspricht der Freiheit der Studierenden eine eigene, nicht vorurteilsbehaftete Karriereentscheidung zu fällen.
			Die ZaPF spricht sich dafür aus, dass Universitätsgelände und -gebäude möglichst frei von kommerzieller Werbung zu halten sind.
			Insbesondere spricht sich die ZaPF dafür aus, dass in Räumen der Lehre und des Lernens (z.B. Bibliotheken, Hörsäle, Übungsräume) bei Lehr- und Lernbetrieb das Arbeiten ohne Beeinflussung durch Werbung möglich sein muss. Sinn der Lehrveranstaltungen und des Lernbetriebs ist es, dass Studierende unbeeinflusst von Interessen Dritter Fachinhalte erlernen und diskutieren, sowie Lehrende Lehrinhalte frei vermitteln können. Diese Arbeitsatmosphäre wird durch Werbung beeinträchtigt.
			Lediglich die Kommunikation von universitätsinternen Informationen sollte weiterhin möglich sein.
			Hingegen ist kommerzielle Werbung \footnote{Werbung meint hier Maßnahmen zur Öffentlichkeitswirkung von kommerziellen Einrichtungen, Drittmittel, die in Forschung fließen, sind hier nicht gemeint.}, insbesondere Hörsaal- und Raumbranding \footnote{Hörsaal- und Raumbranding meint hier den Verkauf von Namensrechten von Hörsälen und anderen Lehr- und Lernräumen. In konkreten Fällen kann dies das Anbringen von Firmenlogos am und im betroffenen Raum und an der Rauminfrastruktur, sowie die Eintragung des Namens ins Raumverwaltungssystem der Hochschule bedeuten.} nicht hinnehmbar.

		\subsubsection*{Ergebnis aus der Postersession}
			Neue Formulierung des Positionspapiers:

			Die ZaPF spricht sich dafür aus, dass in Räumen der Lehre und des Lernens (z.B. Bibliotheken, Hörsäle, Übungsräume, Praktikumsräume) bei Lehr- und Lernbetrieb das Arbeiten ohne Beeinflussung durch Werbung stattfinden soll. Sinn der Lehrveranstaltungen und des Lernbetriebs ist es, dass Studierende unbeeinflusst von Interessen Dritter Fachinhalte erlernen und diskutieren, sowie Lehrende Lehrinhalte frei vermitteln können. Diese Arbeitsatmosphäre wird durch Werbung beeinträchtigt.
			Kommerzielle Werbung \footnote{Werbung meint hier Maßnahmen zur Öffentlichkeitswirkung von kommerziellen, außeruniversitären Einrichtungen.} in diesen Räumen, insbesondere Hörsaal- und Raumbranding \footnote{Hörsaal- und Raumbranding meint hier den Verkauf von Namensrechten von Hörsälen und anderen Lehr- und Lernräumen. In konkreten Fällen kann dies das Anbringen von Firmenlogos am und im betroffenen Raum und an der Rauminfrastruktur, sowie die Eintragung des Namens ins Raumverwaltungssystem der Hochschule bedeuten. }, ist nicht hinnehmbar. \\

			Begründung für das Positionspapier:
			geht aus dem Papier und dem AK-Protokoll hervor. \\

			Begründung, dass abweichend vom AK-Ergebnis der allgemeine Teil gestrichen werden soll:
			Die ZaPF hat sich noch nicht ausreichend mit dem sehr umfassenden Thema Werbung auf dem Campus beschäftigt. Eine Verschriftlichung zum Thema „kommerzielle Werbung auf dem Campus“, die allgemeiner gefasst ist, sollte in einem neuen AK auf einer kommenden ZaPF bearbeitet werden, da in der Postersession aufkam, dass dieses Thema noch viele offene Fragen hat. \\

			Begründung, dass Satz: „Lediglich die Kommunikation von universitätsinternen Informationen sollte weiterhin möglich sein.“: ist vielleicht zu schwammig, Dass diese nicht betroffen sind geht eigentlich aus der Fußnote 1 hervor.

		\subsubsection{Finale Version}
			\textbf{Positionspapier gegen Werbung in Lehr- und Lernräumen}
				Die Zusammenkunft aller Physik Fachschaften (ZaPF) spricht sich dafür aus, dass in Räumen der Lehre und des Lernens (z.B. Bibliotheken, Hörsäle, Übungsräume, Praktikumsräume) bei Lehr- und Lernbetrieb das Arbeiten ohne Beeinflussung durch Werbung stattfinden soll. Sinn der Lehrveranstaltungen und des Lernbetriebs ist es, dass Studierende unbeeinflusst von Interessen Dritter Fachinhalte erlernen und diskutieren, sowie Lehrende Lehrinhalte frei vermitteln können. Diese Arbeitsatmosphäre wird durch Werbung beeinträchtigt.
				Kommerzielle Werbung \footnote{Werbung meint hier Maßnahmen zur Öffentlichkeitswirkung von kommerziellen, außeruniversitären Einrichtungen.} in diesen Räumen, insbesondere Hörsaal- und Raumbranding \footnote{Hörsaal- und Raumbranding meint hier den Verkauf von Namensrechten von Hörsälen und anderen Lehr- und Lernräumen. In konkreten Fällen kann dies das Anbringen von Firmenlogos am und im betroffenen Raum und an der Rauminfrastruktur, sowie die Eintragung des Namens ins Raumverwaltungssystem der Hochschule bedeuten.} ist von daher nicht hinnehmbar. \\

				Begründung für das Positionspapier:
				geht aus dem Papier und dem AK-Protokoll hervor. \\

				Begründung, dass abweichend vom AK-Ergebnis der  Teil zu  Werbung auf dem Campus im Allgemeinen gestrichen wurde:
				Die ZaPF hat sich noch nicht ausreichend mit dem sehr umfassenden Thema Werbung auf dem Campus beschäftigt. Eine Verschriftlichung zum Thema „kommerzielle Werbung auf dem Campus“, die allgemeiner gefasst ist, sollte in einem neuen AK auf einer kommenden ZaPF bearbeitet werden, da in der Postersession aufkam, dass dieses Thema noch zu viele offene Fragen hat, insbesondere welche Art von Werbung kritisch gesehen wird.

	\subsection{Zusammenfassung}
		Im AK wurde zunächst kurz der Verlauf des Siegener AKs (siehe dessen Protokoll) dargelegt und danach die Situation an der TU Darmstadt dargelegt. Danach gab es eine kurze Diskussion zum Thema Hörsaalsponsoring und Werbung auf dem Campus, wobei die Meinungen zwischen strikter Ablehnung und Toleranz unter bestimmten Bedingungen (z.b. Unternehmen übernimmt Hörsaalsanierung) variierten. Es wurde sich auf ein Positioinspapier (mit möglicher Reso auf nächster ZaPF) gegen Werbung auf dem Campus allgemein und insbesondere Werbung in Lern- und Lehrräumen mit großer Mehrheit ausgesprochen. Es wurde in der Diskussion auch auf den Unterschied zwischen kommerzieller Werbung und jeglicher Werbung, sowie die Abgrenzung zu Sponsoring eingegangen. Danach wurde ein Vorschlag für ein Positionspapier ausgearbeitet, dass nach dem AK noch redaktionell überarbeitet wird, sodass es in der AK-Vorstellung präsentiert werden kann. Diesem Vorschlag wurde mit einer Enthaltung zugestimmt.
