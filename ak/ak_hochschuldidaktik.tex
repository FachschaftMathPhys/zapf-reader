% !TEX TS-program = pdflatex
% !TEX encoding = UTF-8 Unicode
% !TEX ROOT = main.tex

\section{AK Hochschuldidaktik und DPG}

  \textbf{Protokoll vom:} 02.06.2018, %???
  Beginn: 09:15 Uhr,
  Ende: 11:00 Uhr \\
  \textbf{Redeleitung:} Stefan (Uni Köln) \\ %???
  \textbf{Protokoll:} %??? \\
  \textbf{anwesende Fachschaften:} Uni Würzburg, Uni Bonn, Uni Münster, Uni Mainz, FU Berlin, Alumni, Uni Wuppertal, Uni Bielefeld, Uni Duisburg/Essen, Uni Konstanz, TU Berlin, Uni Dresden, TU Braunschweig, Uni Köln

  \subsection*{Informationen zum AK}
    \begin{itemize}
      \item \textbf{Ziel des AKs}: Positionspapier, gemeinsame Gestaltung eines Splinter-Meetings bei der DPG-Fruehjahrstagung
      \item \textbf{Folge-AK}: nein
      \item \textbf{Materialien}: Link zu Protokollen, Artikeln, Gesetzen etc. angeben, Dateien hochladen
      \item \textbf{Zielgruppe}: Alle, die an der wissenschaftlichen Auseinandersetzung mit Weiterentwicklung von Studiengaengen interessiert sind
      \item \textbf{Ablauf}: Vorstellung des Beitrages der Kölner Fachschaft zur Frühjahrstagung des Fachbereichs Didaktiken der DPG 2018; Diskusion des Angebots, dort im nächsten Jahr ein Splinter-Meeting zu organisieren
    \end{itemize}

    \subsection*{Protokoll}
      \subsubsection*{Einleitung}
        Im Rahmen der Kölner Bemühungen zur Weiterentwicklung der Studiengänge hat sich immer wieder heraus gestellt, dass die Debatten hinter der Reform von Studiengängen weder dokumentiert, noch wissenschaftlich systematisiert sind.
        Auf den vergangenen ZaPFen wurde in den "Rote Fäden der Studienreform"-AKs immer wieder deutlich, wie notwendig es ist, damit zu beginnen. \\
        Angesichts dessen hat die Kölner Fachschaft für die letzte DPG-Frühjahrstagung zur Didaktik der Physik mehrere Beiträge über die Bemühungen in Köln angemeldet. Nach anfänglicher Skepsis der OrganisatorInnen sind die Beiträge auf sehr großes Interesse gestoßen.
        Als Konsequenz daraus wurden wir von den OrganisatorInnen dazu aufgefordert, ein "Studienreform-Forum" bei der Frühjahrstagung 2019 in zu diskutierender Form zu gestalten. \\
        Wir würden gerne diskutieren, welche Form dafür sinnvoll ist und wer Lust hat, sich daran zu beteiligen. \\

      \begin{itemize}
        \item Frage (Bonn): Wer formulierte Hochschulreform?
          \begin{align}
            & \rightarrow Vorsitzender der DPG
            & \rightarrow nicht fest gekoppelt an Postersession, auch andere Form möglich
          \end{align}
        \item Anmerkung (Braunschweig): Physikdidaktik stehe nicht am Anfang
        \item Schwierigkeit an vielen Hochschulen, dass Traditionen kaputt gegangen sind
        \item Frage (Braunschweig): Trennung von Hochschul- und Schuldidaktik?
          \begin{align}
            & \rightarrow vieles übertragbar
            & \rightarrow Übertragung mit Arbeit verbunden
            & \rightarrow Hochschuldidaktik in Physikdidaktik sehr kleines Feld
          \end{align}
        \item Versuch, ein sinnvolles Format für Diskussionen und Anregungen zu finden
        \item Wuppertal: Evaluation als Mittel, um herauszufinden, welche Professoren didaktisch "wertvoller" sind
        \item Braunschweig: Problem im Datenschutz
        \item Wuppertal: Herausfinden, welche Professoren deutschlandweit gut sind
        \item Braunschweig: "Bloßstellen" bestimmter Professoren
        \item Wuppertal: nicht auf einzelne Professoren (deutschlandweit) beziehen, sondern FSen sollen an Dozenten herantreten
        \item Köln: zu viele Beispiele und Anekdoten, alles wird im Detail unübersichtlich, Lösungen für unbekannte Probleme finden $\rightarrow$ finden von Problemen für gegebene Lösungen, Systematisierung und Dokumentation als Ziel, alles soll nicht nur für "Musterdozenten" funktionieren
        \item Vorschlag Bonn: Einrichtung in Uni, die sich um strukturelle, didaktische Probleme kümmert (Übungsbetrieb, E-Learning, ...) und Dozenten unterstützt
        \item Braunschweig: Evaluationsstruktur sowieso schon vorgeschrieben, die Auswertung ist das Problem
        \item Situation Köln: zentrale Evaluationen nicht sinnvoll, Ergebnisse werden nicht veröffentlicht, Standardfragen nicht vergleichbar, nur strukturelle Informationen werden veröffentlicht
        \item Situation Würzburg: Evaluationen werden veröffentlicht, offene Diskussion mit Professoren, Wettbewerb
        \item Ziel: Weiterentwicklung von Studiengängen, Dokumentation aller Änderungen, Systematisierung durch z.B. didaktische Theorien, höheres Niveau schaffen als Erzählen von einzelnen Anekdoten
        \item Frage Alumni: Vernetzung mit DPG? \\
          $\rightarrow$ sinnvolle Dokumentation als Anforderung, noch keine Lösung
        \item Vorschlag Bielefeld: Thema als Masterarbeit in der Didaktik
        \item Braunschweig: Kluft zwischen Theorie und Praxis (Didaktik-Professoren $\leftrightarrow$ Lehrer), Vorschlag: Seminar anbieten von Professoren für Dozenten, Bonns Vorschlag an die Zentren für Lehrerfortbildung weiterleiten
        \item Köln: Fehlen struktureller Fragen, z.B.: Weiterentwicklung von Modulhandbüchern, Studienverlaufsplänen $\rightarrow$ Bedarf an Menschen, die darüber Arbeiten schreiben; kurzfristig Forum auf DPG-Tagung sinnvoll
        \item Frage Wuppertal: fächerübergreifend gute Didaktiker aus anderen Unis holen? \\
          $\rightarrow$ es geht um die Entwicklung von Plänen speziell in der Physik
        \item Braunschweig: früher Vernetzung von Theorie und Praxis viel enger, heute nur Lernen auf Klausuren $\rightarrow$ fließenderer Übergang gewünscht, z.B. Vergleich von Theo- und Experimentalphysik in Thermodynamik
        \item Bonn: Vernetzung sei Systemfrage
        \item Köln: Beispiel Hamburg wird aufgezeigt, dort gibt es wohl sehr große Freiheiten, auch im Bachelor-Master-System, Frage der Kommunikation $\rightarrow$ systematisches Hinterfragen von Regeln im Studienverlauf
        \item Wuppertal: Nachvollziehbarkeit von Änderungen in der Hochschuldidaktik
        \item Braunschweig: Erklären wo die Punkte herkommen, damit sie in der DPG vorgestellt werden können
        \item Köln: "auf Vorrat lernen" abschaffen, Instrumentalisierbarkeit lernen, Sinnhaftigkeit von Gelerntem hinterfragen
        \item Bielefeld: grundlegenden Sinn des AK herausfinden
        \item Wuppertal: Menschen müssten sich zwischen den ZaPFen damit beschäftigen, wenn auf Winter-ZaPF daran gearbeitet werden soll
        \item Köln: bis zum Call for Papers sollte man aber schon einen groben Plan haben, keine Kopplung an ZaPF-Beschlüsse gewünscht, keine Monopolisierung der Orga des AK in Köln
        \item Braunschweig: Ausarbeitung, bevor es im Plenum vorgestellt wird; Ausgliederung des AK aus der ZaPF zu "privatem" AK
        \item Bielefeld: möchte, dass es im Plenum angesprochen wird, kurze Erklärung des AKs, Überlegung der nächsten Schritte auf der nächsten ZaPF
        \item Vorschlag Braunschweig: Beiträge sammeln, um alles reviewen zu können (auf studentischer Basis), Thema reifen lassen, gute Basis auf der man weitere Maßnahmen aufbauen kann, schriftliche Diskussion produktiver
        \item Vorschlag Köln: Aufbau von Unterkonferenzen, offene Podiumsdiskussionen als Abschluss
          \begin{align}
            & \rightarrow könnte an Barrieren scheitern, weil sich jemand angegriffen fühlen könnte
            & \rightarrow außerdem Änderungswunsch bei vielen Professoren nicht vorhanden, bzw. Problem, Interesse zu wecken
            & \rightarrow aber wenig fundierte Kritik an unserem Vorhaben vorhanden
          \end{align}
        \item Vorschlag Braunschweig: Antrag an StAPF zum Kontaktaufbau zur DPG, würde Kontinuität in den Prozess bringen
        \item Wuppertal: Thema: Tabuthema Heidelberg?
        \item Bonn: Dieses Problem ist universal. Vielleicht Erfahrungsberichte?
        \item Bielefeld: Online-Magazin einrichten?
        \item Köln: Call for papers auch über ZaPF-Verteiler
        \item nächste Schritte: Abschlussplenum, Suche nach engagierten Menschen, Telefonkonferenz, Absprache mit DPG
      \end{itemize}

    \subsection*{Zusammenfassung}
      \begin{itemize}
        \item Inhalt:
          \begin{itemize}
            \item Übergang zwischen Veranstaltungen
            \item Geschichte der Studiengänge
            \item Quelle, Gründe, Obsoleszenz von Vorschriften
            \item auf Vorrat lernen überdenken
          \end{itemize}
        \item Strukturelles:
          \begin{itemize}
            \item Dokumentation
            \item Einbeziehung anderer Fächer
            \item Abschlussarbeiten
          \end{itemize}
      \end{itemize}
      Wer sich an der Gestaltung des hochschuldidaktischen Forums bei der nächsten DPG-Frühjahrstagung des Fachbereichs Didaktiken beteiligen möchte, melde sich bitte bei der Kölner Fachschaft.
