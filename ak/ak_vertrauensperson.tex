% !TEX TS-program = pdflatex
% !TEX encoding = UTF-8 Unicode
% !TEX ROOT = main.tex

\section{AK AK Vertrauensperson Wahlprozedere}

  \textbf{Protokoll vom:} 31.05.2018, % ???
  Beginn: 10:30 Uhr, % ???
  Ende: 12:30 Uhr \\ % ???
  \textbf{Redeleitung:} Lisa (Uni Erlangen-Nürnberg) \\
  \textbf{Protokoll:} Jenny (FU Berlin) \\
  \textbf{anwesende Fachschaften:} FU Berlin, Frankfurt, Potsdam, Würzburg, Siegen, Konstanz, Münster, Dresden, Bonn, Oldenburg, Göttingen, Erlangen, TU Berlin

  \subsection*{Informationen zum AK}
    \begin{itemize}
      \item \textbf{Ziel des AKs}: Diskussion und vielleicht Überarbeitung des Wahlprozederes der Vertrauenspersonen
      \item \textbf{Folge-AK}: nein
      \item \textbf{Zielgruppe}: Interessierte
      \item \textbf{Materialien}: Protokoll aus Wien\footnote{\url{https://zapf.wiki/WiSe13_AK_Anti_Harassment_Policy}}
      \item \textbf{Ablauf}: Diskussion
    \end{itemize}

  \subsection*{Einleitung}
    Seit vier Jahren gibt es Vertrauenspersonen auf der ZaPF. Es werden sechs Vertrauenspersonen gewählt und zwei von der ausrichtenden Fachschaft ernannt. Die Wahl der Vertrauenspersonen ist, sofern mehr als sechs Personen kandidieren, etwas kompliziert. \\

    Ist die Zahl der Vertrauenspersonen sechs und zwei gut? Ist das zuviel (beliebig) oder zu wenig (nicht immer verfügbar)? Brauchen oder wollen wir überhaupt eine Beschränkung der Anzahl? \\
    Sind wir mit dem Wahlprozedere glücklich? Funktioniert es in der Praxis gut? \\
    Von den Vertrauenspersonen darf es keine Besprechung oder Rückmeldung geben, auch nicht anonymisiert. Das heißt wir wissen nicht, wie häufig und aus welchen Gründen Vertrauenspersonen angesprochen werden. \\

    Hätten die Vertrauenspersonen Daten, ließe sich daraus eventuell ableiten, welche Fortbildungen/Vorbereitungen/Ansprüche für Vertrauenspersonen hilfreich sind. Oder es ließe sich herausfinden, ob es wiederkehrende "Probleme" gibt, auf die, die Orga schon im Vorfeld eine Lösung finden könnte.
    Wäre es denkbar, Daten hinreichend anynomisiert zu erheben? Beispielsweise in dem sich die Vertrauenspersonen (anonymisiert) während/nach der ZaPF austauschen und gegebenenfalls direkte Schlüsse für die nächste ZaPF ziehen. Das könnte beispielsweise ausschließlich die Anzahl der Anfragen betreffen und gegebenenfalls an dem StaPF genannt werden.
    Die wichtigste Frage ist natürlich: Wollen wir das?

  \subsection*{Protokoll}
    \paragraph{Anzahl der Vertrauenspersonen \& Wahlverfahren}
      \begin{itemize}
        \item Christian (Oldenburg): Erklärung, warum damals diese Zahl und Wahlprozedere gewählt wurde. Kreis sollte etwas eingeschränkt sein, damit man nicht den Überblick verliert. Wahlmodus ist dafür da, sicher zu stellen, dass mindestens jeder eine Person unter den Vertrauenspersonen findet, dem er/sie vertaut.
        \item Benedikt (Münster): Was passiert, wenn sich jemand nicht repräsentiert fühlt?
        \item Karola (Potsdam): Hat sich viele Gedanken gemacht und ist der Meinung, dass eine Begrenzung der VP Anzahl nicht gut ist. Man sollte die Kandidierenden nicht einfach durchwinken, sondern die Leute trotzdem legitimieren. Wenn sich 8 Leute zur Wahl stellen und einer keine Stimmen bekommt, dann wäre das wieder schlecht. Auf jeden Fall sollte eine Begrenzung höher sein, eventuell aber nicht aufgelöst werden.
        \item Marcus (Alumni): Bei der Konzipierung der VP (im Ursprungs-AK in Wien) wurde schnell über Wahlen gesprochen, wie eine Wahl das "Vertrauen" abbilden kann. Wenn Kandidaten nicht gewählt werden heißt das nicht, dass Leute dieser Person nicht vertrauen, sondern dass viele Leute den gleichen Personen vertrauen. Im AK war die Rede, dass die Personen bis zur nächsten ZaPF gewählt sind, das ist in der Satzung nicht so angekommen. Das bedarf einer Satzungsänderung.
        \item Jenny (FUB): Die Wahl legitimiert nicht, dass dieser Person vertraut wird, sondern das Vertrauen gibt jede einzelne Person. Daher ist eine Begrenzung der VP Anzahl nicht gut. Wichtig ist, dass jedeR eine Ansprechperson findet.
        \item Bekka (Konstanz): Man kann die Zahl der Vertrauenspersonen nicht unabhängig von dem Informationsaustausch über Vorfälle diskutieren. Wenn man keinen Austausch hat, dann braucht man auch keine Begrenzung. Wenn dies geändert wird, dann ist eine Begrenzung sinnvoll.
        \item Jakob (Göttingen): Findet es schwierig pauschal zu sagen, dass Vertrauenspersonen miteinander reden dürfen: Es könnte ein Problem mit einer der VPen bestehen. Möchte den Zäpfchenschutz mit einbringen. Zäpfchen können noch nicht entscheiden, welchen Leuten man vertrauen kann. Damit übernimmt die restliche ZaPF die Verantwortung gute Vertrauenspersonen auch für Zäpfchen zu wählen. Falls alle die kandidieren VP sein dürfen, kann das Problem auftreten, dass eine Person sich selbst für \item vertrauenswürdig hält, es aber nicht ist. Pro Wahlen bei Vertrauenspersonen, weil Filterfunktion.
        \item Elli (TUB): das Konzept wurde damals formal sehr durchdacht. Wie findet ihr es in der Praxis, wie es gelebt wird? (Glaubt ihr,) dass es tatsächlich vorkommt, dass mehreren gewählten VP nicht vertraut wird? Bisher gibt es 8 Vertrauenspersonen. Habt ihr das Gefühl, dass immer jemand verfügbar ist? Habt ihr das Gefühl, wer war das nochmal, es ist zu beliebig?
        \item Ludi (Erlangen): Wenn Vertrauenspersonen miteinander reden dürfen, sollte man die Anzahl begrenzen. Die zwei Punkte muss man zusammen behandeln. Angenommen, dass sie nicht mitander reden, sollte man wählen, aber die Anzahl anpassen.
        \item Marcus (Alumni): Ist gegen ein Mehrheitswahlsystem, da sonst Minderheiten ausgeschlossen werden. Das System sorgt dafür, dass es unwahrscheinlich ist, dass jemand keinen Ansprechpartner hat. Er hat das Gefühl, dass Vertrauenspersonen präsent sind und das liegt auch am Wahlsystem.
        \item Daniela (FFM): Es ist ein Vorteil des komplizierten Wahlsystems, dass es Bewusstsein dafür schafft. Und es schafft ein Bewusstsein dafür, dass es auch andere Wahlprinzipien als die Mehrheitswahl gibt. Findet es schwer sich zu merken, wer die Vertrauenspersonen sind. Würde sich sehr dafür interessieren, wie es sich anfühlt, wenn man die Personen nicht persönlich kennt. Man könnte Bilder der Vertrauenspersonen zentral aufhängen.
        \item Jakob (Göttingen): Noch im Plenum, wissen ZäPFchen bereits nicht mehr, wer gewählt wurde. Findet das System ausreichend von der Wahl her. Ist dagegen, dass die Vertrauenspersonen miteinander reden, da man nur wählt, dass es eine Vertrauensperson gibt, der man vertraut und die Situation existiert, anderen VPers nicht zu vertrauen.
        \item Cindy (Dresden): Findet es schon ein bisschen beliebig, grade wenn man noch nicht so oft auf ZaPFen war.Es ist am Anfang schwierig, wenn man nicht weiß, wem man vertrauen kann. Bilder aushängen wäre praktisch oder Bändchen/ auffällige Markierung.
        \item Ludi (Erlangen-Nürnberg): Gefühlt haben in der Vorstellung alle Kandidierenden das Gleiche gesagt.
        \item Karola (Potsdam): Zur Wahlsache stimmt sie Marcus und Daniela zu. Erhofft sich vom System, dass für jeden was dabei ist. Denkt schon, dass viele Leute die Vertrauenspersonen wiedererkennen, bis zu einem gewissen Grad. Könnte man schon kennzeichnen, aber auch nicht übermäßig. Wenn jemand zu ihr kommen würde, dann würde sie alles andere weglegen und da sein. Versucht daher präsent zu sein und sich nicht den ganzen Tag irgendwo versteckt. Vielleicht sollten wir doch erstmal darüber reden, was Leute verstehen, sollten die Vertrauenspersonen miteinander reden. Schlägt vor, dass nichts Persönliches angesprochen wird. Aber grundsätzliche Probleme sollten schon angesprochen werden können. Wenn es eine Feedbackrunde der Vertrauenspersonen geben würde, dann würde es in ihrer Vorstellung nicht um persönliche Dinge, sondern nur um Dinge gehen, die die ZaPf betreffen.
        \item Marcus (Alumni): Es gab einen AK in Berlin, der sich mit der Schweigepflicht der VP befasst hat. Er findet aber das Protokoll des Abschlussplenums nicht, daher weiß er nicht konkret, was beschlossen wurde. Es ist aber rausgekommen, dass Vertrauenspersonen mit Bild, Email und Namen ausgehangen werden sollen.
        \item Jenny (FUB): Vertrauenspersonen können der Person, die sich an sich wendet, offen sagen, dass sie bspw. überfordert ist und sich an andere wenden möchte. Und nach Zustimmung des ZaPFikon darf die VP darüber sprechen. Insofern muss nicht unbedingt die Regelung geändert werden.
        \item Benedikt (Münster): Glaubt nicht, dass es hilfreicher für ZäPfchen ist, wenn andere für sie die Vertrauenspersonen wählen. Glaubt nicht, dass für die Vertrauenpersonen mehr Vertrauen geschaffen wird für ZaPfchen, nur weil andere sie wählen. Würde die Kommunikation zwischen den Vertrauenspersonen eher positiv formulieren.
        \item Karsten (Münster): Ist selbst Zäpfchen, wenn er ein Problem hätte, wüsste er nicht, an welchen von den 8 Personen er sich wenden würde.
        \item Bekka (Konstanz): Man schränkt das für Zäpfchen dadurch ein, dass es 8 Personen gibt, die da stehen. Zur Sichtbarkeit, Vertrauenspersonen werden wahrgenommen und mit Bild finden auch ZäPFchen die VPersonen wieder.
        \item Daniela (FFM): Es wäre schön, wenn die gewählten Leute nach der Wahl nochmal nach vorne kommen und wenn man sie z.B. mit einem andersfarbigen Namensschild kennzeichnet.
        \item Elli (TUB): Danke für die Rückmeldung der ZäPFchen. Wird die Bilder einfordern.
        \item Marcus (Alumni): Hat das Dokument gefunden\footnote{\url{https://zapfev.de/reader/2017_SoSe_Berlin_vorlaeufig.pdf} (S.68f)}. Es gibt keinen weiteren Beschluss zur Schweigepflicht. Dort steht genau drin, wie das mit Bildern der VP aushängen und so sein soll. Da sollte der StAPF auch mit drauf achten, dass die Orga das macht. Kommt auf die StAPF Checkliste.
        \item Jenny (FUB): Habe mit ausgezählt. Es war schade, dass zwischen Personen ausgelost werden musste.
        \item Bekka (Konstanz): Es fehlt einfach eine Klausel, die entscheidet was bei Stimmengleichstand passiert.
        \item Cindy (Dresden): Was bedeutet die Beliebigkeit, die ihr grade angesprochen habt?
        \item Jenny(FUB): Wenn in einer 3. Auszählung Stimmengleichstand ist, dann wird gelost, das ist beliebig.
        \item Andy (Würzburg): Das Wahlverfahren führt nicht dazu, dass wir sagen, das sind Leute denen viele vertrauen. Eine Person gewinnt den ersten Wahlgang mit Mehrheit, die anderen danach kommen auch rein. Wenn es eine Person gibt, bei der zum Beispiel zwei Personen nur dieser Person vertrauen.
        \item Jakob (Göttingen): Für die ZaPFchen ist es eher nur ein Vorschlag, dass sie wissen, wem sie vielleicht mehr vertrauen können. Primäres Ziel ist, dass jeder (der qualifiziert abstimmen kann) mindestens eine Person hat, der er vertrauen kann.
        \item Karola (Potsdam): Vorschlag bei Gleichstand beide Personen als VP zu nehmen. Das gewichtet beide gleich.
        \item Jenny (FUB): Tut es nicht. Da es um die Personen geht, die eben nicht die beiden gewählt hat.
        \item Kurze Diskussion, was die Unterschiede wären.
        \item Sonja (Bonn): Aufwand-Nutzen-Verhältnis des Wahlmodus ist aus ihrer Sicht nicht gut. Die Gründe für die Einführung waren sinnvoll, aber alle "Nichtgewählten" sagen, dass man trotzdem zu Ihnen kommen kann. Daher würde sie lieber nur den zweiten Wahlgang machen über "Vertraut ihr mind. einer Person von allen, die antreten" und alle sind damit gewählt.
        \item Cindy (Dresden): Warum nimmt man bei Stimmengleichstand nicht die absolute Zahl vom Anfang. Man könnte im zweiten Wahlgang zählen wer nur ein Kreuz bei wem gemacht hat.
        \item Jörg (FUB): Sieht den Punkt mit dem Zeitaufwand im Plenum nicht so gravierend, weil es im wesentlichen zwei-drei Leute gibt, die auszählen. Das Plenum muss nur einmal den Zettel ausfüllen.
        \item Daniela (FFM): Findet, dass das Wahlverfahren schön darauf hinweist, dass es auch andere Wahlen als Mehrheitswahlsystem gibt.
        \item Marcus(Alumni): Sieht Danielas Meinung inhaltlich auch so. Findet aber, dass es kein Argument ist für die Sinnhaftigkeit dieses Prozederes.
        \item Sonja (Bonn): Meinte nur, dass das Wahlprozedere sehr umständlich zu erklären ist und ZäPFchen es dann vielleicht nicht ganz verstehen.
        \item Jenny (FUB): Es hat immer den Beigeschmack, dass Leute "nicht" gewählt werden.
        \item Daniela (FFM): Stimmt zu, dass es komisch aussieht, wenn Leute nicht gewählt werden. Und auch dass es komisch Klingt, wenn Leute sagen, dass man dennoch zu ihnen kommen kann.
        \item Jakob (Göttingen): Widerspricht Jenny. Die Wahl soll den Zapfika nahelegen, sich an genau diese VP zu wenden (vor allem denen, die nicht von sich aus jemanden kenn). Das soll aber nicht heißen: Den Nichtgewählten kann nicht vertraut werden. Zusätzlich: Aufgreifen der Fortbildungs-AKs: Nach Möglichkeit die VPers schulen.
        \item Daniela (FFM):Fände es spannend im Wiki aufzuschreiben, was dieses Wahlverfahren bedeutet, so dass ein Bewusstsein dafür geschaffen wird, dass nicht Leute "nicht gewählt" werden, sondern dass es Festhalten, was eigentlich Aufgaben der Vertrauenspersonen sind und Personen präsent wie z.B. "Sprechzeiten" festlegen.
        \item Elli (TUB): Es macht einen Unterschied, ob man sich anonym an Vertrauenspersonen wendet oder ob man mit einer anderen Person einfach so über etwas redet.
        \item Marcus (Alumni): Würde den Punkt von Jakob eher positiv formulieren. Die gewählten Personen haben jetzt die Aufgabe auch präsent zu sein und die anderen können auch angesprochen werden.
        \item Bekka (Konstanz): Die gewählten Personen haben ein anderes Bewusstsein, als die nicht gewählten Personen. Sie haben eine Verantwortung übernommen.
      \end{itemize}

    \paragraph{Informationen über Anfragen/Inanspruchnahme von Vertrauenspersonen}
      \begin{itemize}
        \item Jakob (Göttingen): Vielleicht sind Vertrauenspersonen nicht in der Lage Dinge hinreichend zu anonymisieren. Es kann passieren, das man aus der Situation trotzdem das Ganze auf einen Kreis an Personen reduzieren kann (vor allem, wenn man viele ZaPFika kennt).
        \item Karola (Uni Potsdam): Hat darüber nachgedacht, dass die Person, die ein Problem hat, natürlich vorher gefragt werden muss. Darf ich das anonymisiert ansprechen?
        \item Elli (TUB): Ein großer Kritikpunkt auf ZaPF nach der Einführung der Vertrauensperson (nach Wien) war, dass nachgefragt wurde: Wurden die Vertrauenspersonen gebraucht? Wie viele Anfragen gab es? Das war nach ein paar ZaPFen weg. Gedankenexperiment: Angenommen wir wüssten von allen Personen wie oft und mit welchen Problemen sie angesprochen würden. Was könnten wir dann damit anfangen? Könnte man gezielte Schulungen für VP anbieten? Gibt es Dinge, die die Orga im Vorfeld tun könnte? Schlägt vor, ob man unter den Vertrauenspersonen nach der ZaPF sich grob zumindest über Anzahl der Anfragen und Problembereiche sich austauschen können.
        \item Bekka (Konstanz): Findet es inzwischen wichtig, dass sich Vertrauenspersonen nach der ZaPF ein bisschen austauschen können.
        \item Daniela (FFM): In Absprache mit allen Personen, die davon betroffen sind (Opfer- und Täterschutz). Wenn Vertrauenspersonen evaluieren sollen, ob es ein strukturelles Problem der ZaPF ist, könnte es sogar hilfreich sein, dass Vertrauenspersonen die Situation einordnen können.
        \item Benedikt (Münster): Vertrauenspersonen könnten von sich aus sagen wozu sie gerne Schulungen hätten. Was ist, wenn einen selbst etwas so sehr mitnimmt, dass man mit jemandem darüber reden muss?
        \item Jakob (Göttingen): Auf den Schulungen wurde darüber geredet, dass es dann am besten ist mit einer 3. Person, die keinen ZaPFbezug hat zu sprechen, vor allem sollte die Wahrscheinlichkeit, dass jemand auf er ZaPF davon erfährt gegen Null gehen.
        \item Fluff (Bonn): War damals in Wien auch dabei, hat vergessen warum er gegen die VPen war. Darum hat er auch mal nachgefragt, ob die Vertrauenspersonen in Anspruch genommen wurden.
        \item Marcus (Alumni): Findet es wichtig, dass wir Täter und Opferschutz hochhalten. Dabei ist es schon schwierig in einigen Fällen überhaupt von Tätern zu sprechen. Hat Angst, dass es dazu führen kann, dass Leute an den Pranger gestellt werden. Daher ist er dagegen, dass Vertrauenspersonen grundsätzlich über alles reden.
        \item Es gibt allgemein noch Redebedarf in einigen Punkten. Der Fall mit Gleichstand bei der Wahl sollte aber in der GO geregelt werden.
        \item Karola (Potsdam): In BackUp AK können wir eine Liste erarbeiten - Was dürfen Vertrauenspersonen? Welche Rechte und Welche "Pflichten" haben die Betroffenen, die sich bei den Vertrauenspersonen melden?
        \item Daniela (FFM): Wenn wir so eine Handreichung erarbeiten, dann wäre es gut, wenn die Vertrauenspersonen da nochmal mit dem StAPF oder so gemeinsam rauf gucken.
        \item Elli (TUB): Dabei sollten wir darauf achten, dass in einem neuen Beschluss alle bestehen bleibenden Informationen ebenfalls enthalten sind. Damit wir ein Dokument mit allen notwendigen Informationen haben.
        \item Bekka (Konstanz): Es geht eher darum, ob Vertrauenspersonen auch Stimmungen weitergeben dürfen.
      \end{itemize}

  \subsection*{Zusammenfassung}
    Auf dem AK wurden ausführlich die Anzahl der Vertrauenspersonen und Details des Wahlprozedere diskutiert. Es wurde schnell klar, dass die optimale Anzahl der Vertrauenspersonen nicht unabhängig von einem Informationsaustausch zwischen den Vertrauenspersonen besprochen werden kann. Einige waren der Ansicht, dass bei einer Besprechung der Vertrauenspersonen, die Zahl der VP begrenzt sein sollte, jedoch nicht, falls die VP sich nicht austauschen dürfen. \\
    Das Wahlprozedere wurde überwiegend positiv gesehen. In diesem Fall ist eine Mehrheitswahl keine optimale Lösung. Das komplexe Verfahren macht die Vertrauenspersonen für die ZaPF sichtbar. \\
    Es wurden konkrete Verbesserungen vorgeschlagen, um die Sichtbarkeit der VP für die Zapfika zu erhöhen. Dies soll im Back-Up AK und in einem Folge-AK in Würzburg zu einer Handreichung (gemeinsam mit den bestehenden Beschlüssen) erarbeitet werden.
