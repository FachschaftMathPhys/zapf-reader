% !TEX TS-program = pdflatex
% !TEX encoding = UTF-8 Unicode
% !TEX ROOT = main.tex

\section{AK Novellierung des Polizeiaufgabengesetzes in Bayern}

  \textbf{Protokoll vom:} 31.05.2018, % ???
  Beginn: 14:00 Uhr, % ???
  Ende: 16:00 Uhr \\ % ???
  \textbf{Redeleitung:} Lisa Dietrich (Uni Erlangen-Nürnberg) \\
  \textbf{Protokoll:} Sebastian Schmidt (TU Dresden) \\
  \textbf{anwesende Fachschaften:} Freie Universität Berlin, Rheinische Friedrich-Wilhelms-Universität Bonn, Technische Universität Braunschweig, Technische Universität Chemnitz, Technische Universität Dresden, Friedrich-Alexander-Universität Erlangen-Nürnberg, Gießen Friedrich-Schiller-Universität Jena, Universität zu Köln, Technische Universität Kaiserslautern, Universität Konstanz, Fachhochschule Lübeck, Westfälische Wilhelms-Universität Münster, Hochschule RheinMain Universität Siegen, Eberhard Karls Universität Tübingen, Karlsruher Institut für Technologie, Universität Wien, Bergische Universität Wuppertal

  \subsection*{Informationen zum AK}
    \begin{itemize}
      \item \textbf{Ziel des AKs}: Handreichung für Fachschaften, wie man dagegen protestieren kann, wie man Studierende über das Thema informieren kann und wieso das wichtig ist, es wird KEINE Resolution oder ein Positionspapier geschrieben.
      \item \textbf{Folge-AK}: nein
      \item \textbf{Zielgruppe}:  alle die Interesse an dem Thema haben
      \item \textbf{Materialien}: Neues Polizeigesetz aus Bayern und Pläne aus NRW (+Sachsen?) (\url{http://www.polizeiaufgabengesetz.bayern.de/assets/stmi/med/aktuell/180524pag-gesamt.pdf})
      \item \textbf{Voraussetzungen}: Ideensammlung und darauf Formulierung einer Handreichung
    \end{itemize}

  \subsection*{Einleitung}
    Das neue Polizeiaufgabengesetz in Bayern ist ein großer Eingriff in die Freiheit der Menschen. Dagegen etwas zu unternehmen ist sehr wichtig, deswegen müssen Leute über die Gefahren dieses Gesetz aufgeklärt werden, um dagegen etwas unternehmen zu können. Was man als Fachschaft dafür tun kann, soll in diesem AK besprochen werden und eine Sammlung dafür erstellt werden. Da dieses Gesetz als Vorbild für weitere solcher Gesetze genommen werden kann, ist es enorm wichtig sich damit zu beschäftigen, auch für Leute aus anderen Bundesländern.

  \subsection*{Protokoll}
    \paragraph{Zielsetzung}
      Handreichung für Fachschaften wie man mit dem neuen Polizeiaufgabengesetz umgehen soll.(Information,Meinungsbild, kein offzielles ZaPF Paper).

    \paragraph{Anstoß für den AK}
      Informationsvermittlung an die Allgemeinheit, aufgrund des Anstoßes der KIF (Reso siehe Wiki).

    \paragraph{Erläuterungen zum Gesetz}
      Wurde diskutiert

    \paragraph{Worüber wollen wir informieren?}
      Wie kann man sich gegen die Einsicht der eigenen Daten wehren (Verschlüsselung), Workshops dazu besuchen. \\
      Was speichere ich wo? \\ \\

      Aufbau einer Handreichung, Form eines Flyers(Vorschlag):
      \begin{outline}
        \1 Das ist das Polizeigesetz(Strichliste)
        \1 Was macht man dagegen? (Liste)
        \1 Was wir Studis raten wollen:
          \2 Speichert nichts unverschlüsselt im Internet (Traue keiner Cloud).
          \2 Eigene Rechte gegenüber der Polizei.
          \2 Wie gehe ich mit fremden Daten um?
          \2 Was können Firmen mit meinen Daten tun?
      \end{outline}

    \paragraph{Wie kann man Infos weitergeben?}
      \begin{itemize}
        \item Flyer mit Infos über das PAG
        \item Kryptoworkshops $\rightarrow$ allgemein: ``Wie geht man mit dem Internet um?''  (z.B. im Rahmen der O-Woche)
        \item Online streuen
        \item Für Demos werben
        \item Werbung für Veranstaltungen machen
      \end{itemize}
