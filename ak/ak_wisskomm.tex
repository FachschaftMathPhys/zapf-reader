% !TEX TS-program = pdflatex
% !TEX encoding = UTF-8 Unicode
% !TEX ROOT = main.tex

\section{AK Wissenschaftskommunikation}

	\textbf{Protokoll vom:} 31.05.2018, % ???
	Beginn: 14:10 Uhr,
	Ende: 15:50 Uhr \\
	\textbf{Redeleitung:} Marcus (Alumnus) \\
	\textbf{Protokoll:} Marcus (Alumnus)) \\
	\textbf{anwesende Fachschaften:} Potsdam, Gießen, Kiel, Ilmenau, Düsseldorf, WWU, Dresden, Bonn, Würzburg, Halle, Hamburg (alter Sack), Münster, Greifswald, Frankfurt, Augsburg, Siegen, Erlangen, Uni Wien, Saarbrücken

	\subsection*{Informationen zum AK}
		\begin{itemize}
			\item \textbf{Ziel des AKs}: vielleicht eine Resolution
			\item \textbf{Folge-AK}: ja, (WiSe `17 Siegen \url{https://zapf.wiki/WiSe17_AK_Wissenschaftskommunikation})
			\item \textbf{Vorwissen}: Positionspapiere (\url{https://zapf.wiki/Sammlung_aller_Resolutionen_und_Positionspapiere#Positionspapier_zur_F.C3.B6rderung_der_Wissenschaftskommunikation_in_der_akademischen_Ausbildung}, \url{https://zapf.wiki/Sammlung_aller_Resolutionen_und_Positionspapiere#Positionspapier_zur_Rolle_der_Wissenschaftskommunikation})
      \item \textbf{Materialien}: siehe altes Protokoll
			\item \textbf{Zielgruppe}: alle, die die Ergebnisse der ZaPF nicht für völlig egal erachten und eh vergessen
			% \item \textbf{Ablauf}: Thematik auffrischen, holistischen Lösungsansatz suchen, Resolution schreiben, Vorlage für Vertrag schreiben
			% \item \textbf{Voraussetzungen}: keine
		\end{itemize}

  \subsection*{Protokoll}
    \paragraph{Resonanz Positionspapier aus Siegen}
      In der Reakkreditierung des Bachelorstudiengangs Physik der Justus-Liebig Universität Gießen wird wegen der Positionspapiere
      der Siegener ZaPF ein 6 CP geltendes Modul ``wissenschaftliches Präsentieren'' eingeführt (Ortsbegehung der Reakkreditierung steht aus).
      Popsitionspapiere der letzten ZaPF wurden aufgenommen $\rightarrow$ Beleg wird nachgereicht.

    \paragraph{ThinkTank ,Siggener Kreis’:}
      Der Siggener Kreis stellt eine Plattform für Wissenschaftskommunikation, welche unabhängig von Instituten ist.
      Sie entwerfen Empfehlungen für gute wissenschaftliche Praxis. Allerdings decken sie nicht ganz unseren Themenbereich ab, nämlich der Wissenschaftskommunikation im Studium.
      Prinzipiell ist die Plattform offen für jeden Personenkreis.

    \paragraph{Kommunikation nach Außen}
      Der Siggener Kreis hat eine Resolution veröffentlicht (Wissenschaft braucht Courage), an die man sich gut anschließen könnte.
      Man könnte sich dann auch konkrete Adressaten ausdenken, da hat Marcus sich schon einige Gedanken gemacht. \\

      Es wird ein Auftrag an den StaPF formuliert: Einladen eines Referenten zur nächsten ZaPF zu dieser Thematik. \\

      Marcus schlägt vor, sich mit den Papieren des Siggener Kreises auseinanderzusetzen, um Argumente für eine eigene
      Resolution zu sammeln und außerdem, um einen Überblick über den Siggener Kreis und seine Arbeit zu bekommen.
      DPG ist da bereits aktiv. \\

      Vorschlag: Aufteilung in drei Arbeitskreise
      \begin{itemize}
        \item Resolution lesen und über Adressaten Gedanken machen
        \item Handout erstellen, wie man Wissenschaftskommunikation betreiben kann
      \end{itemize}
      Um 15:20 kommen die Teilgruppen wieder zusammen und berichten:
      \begin{outline}
        \1 2014 Papier:
          \2 Wissenschaftskommunikation ist wichtig.
          \2 Mit Merkmal, dass bei neuer Professur darauf geachtet wird, dass sie ausgebildet sind für breite Kommunikation. Soll als Teil ihrer Aufgabe betrachtet werden. Es wird auf die Wichtigkeit, die dies für die Bevölkerung haben könnte, hingewiesen.
          \2 Außerdem benötigt man Menschen, die sich um eine Plattform kümmern, auf der man Wissenschaft kommunizieren kann.
          \2 Beinhaltet auch einen Leitfaden, was gute Wissenschaftskommunikation ist.
        \1 2017 Papier:
          \2 Baut auf 2014 Papier auf.
          \2 Schwerpunkt lag auf Kommunikation zwischen Wissenschaftlern und Normalbürgern.
          \2 Problematik der Fehlinformation und daurch entstehenden Ängsten.
          \2 Es werden Wege aufgezeigt, dem entgegenzuwirken:
            \3 Ausbilden in Schulen und Universitäten: kritisches Lesen.
            \3 Appell an Wissenschaftler \textit{und} Interessierte, Wikipedia aufmerksam zu lesen und bei Bedarf zu korrigieren.
            \3 Appell an die öffentlich-rechtlichen Medien wissenschaftliche, gut recherchierte Themen zu behandeln.
          \2 Intransparenz wird als Problematik genannt: Dies schürt Zweifel. Woher kommt das Geld zur Forschung?
          \2 Es sollen mehr Anlaufstellen für Fragen ins Leben gerufen werden. Es soll den “Normalbürgern” Möglichkeiten schaffen, sich einzubringen. \\
            $\Rightarrow$ Für uns interessant: Schon im Studium hier einen Fokus drauf legen und auf Kommunikationsplattformen hinweisen.
        \1 2015 Papier:
          \2 Die Wissenschaft positioniert sich nicht mehr als neutraler Beobachter, sondern fungiert als Berater (Politik, Religion, Kommunikator). $\rightarrow$ Problematik: z.B. andere Regeln der Bereiche
          \2 Citizen Science: Es ist wichtig den interessierten Bürger mit einzubeziehen.
          \2 Wssenschaft muss sich internationalisieren,
          \2 Wissenschaft findet nicht mehr im Elfenbeinturm statt; wieder Einbezug des/der interessierten BürgerIn. Leute mit Ideen und Erfinder zusammen bringen. Muss nicht im Wissenschaftsort stattfinden.
        \1 Leitlinien: Wie geht man damit um, wenn das Ziel die Wissenschaftskommunikation ist?
          \2 Ehrlichkeit und Offenheit unter anderem bezüglich Finanzierung, Misserfolgen.
          \2 Dieser Leitfaden richtet sich nicht wirklich an uns, sondern mehr an WissenschaftlerInnen.
          \2 Schnittstellenbeschreibung zwischen Presse und Wissenschaft.
        \1 Resolution:
        Es wurde eine Resolution verfasst, deren Kernaussage ist, Wissenschaftskommunikation auch im Studium zu behandeln. Insbesondere sollen, wenn ein Studiengang unter Akkreditierung steht, neue Module diesbezüglich den Studiengängen hinzugefügt werden. \\
        Hier der Link zu der verfassten Resolution: \url{https://protokolle.zapf.in/37Wy_2oZREmwoRnOgX-yAA} \\
        Als weiterer Addressat wird das österreichische Bundesministerium für Bildung, Wissenschaft und Forschung genannt. Hier ist die Addresse der Webseite: \url{https://www.bmbwf.gv.at/}
      \end{outline}

    Der Schwerpunkt wird mehr auf die Kommunikation mit dem Siggener Kreis gelegt, als diese ZaPF die Resolution zu verabschieden. \\

    Es wird ein Back-Up AK verlangt, um den Arbeitsauftrag an den StaPF zu formulieren, Kontakt zum Siggener Kreis aufzunehmen. \\

    Ein Folge-AK ist erwünscht (in der Postersession). Fortbildung von Lehrenden in Wissenschaftskommunikation und Wissenschaftskommunikation als Kriterium bei Berufungen.
