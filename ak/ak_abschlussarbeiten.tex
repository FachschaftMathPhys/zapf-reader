% !TEX TS-program = pdflatex
% !TEX encoding = UTF-8 Unicode
% !TEX ROOT = main.tex

\section{AK Vorläufige Verträge für Abschlussarbeiten}

	\textbf{Protokoll vom:} 31.05.2018,
	Beginn: 10:30 Uhr,
	Ende: 12:20 Uhr \\
	\textbf{Redeleitung:} Gabriel (TU Chemnitz) \\
	\textbf{Protokoll:} Peter Steinmüller (KIT) \\
	\textbf{anwesende Fachschaften:} Freie Universität Berlin, Universität Bochum, Rheinische Friedrich-Wilhelms-Universität Bonn, Technische Universität Chemnitz, Heinrich Heine Universität Düsseldorf, Technische Universität Dortmund, Technische Universität Dresden Technische Universität Bergakademie Freiberg, Friedrich-Schiller-Universität Jena, Universität Potsdam, Karlsruher Institut für Technologie, Julius-Maximilians-Universität Würzburg, Bergische Universität Wuppertal

	\subsection*{Informationen zum AK}
		\begin{itemize}
			\item \textbf{Ziel des AKs}: Resolution, eventuell auch Vorlage für einen solchen Vertrag
			\item \textbf{Folge-AK}: ja
			\item \textbf{Vorwissen}: altes Protokoll lesen (\url{https://zapf.wiki/WiSe17_AK_Vorläufige_Verträge_für_Abschlussarbeiten})
      \item \textbf{Materialien}: siehe altes Protokoll
			\item \textbf{Zielgruppe}: alle, die sich für das Thema interessieren und natürlich besonders diejenigen, die Erfahrungen in Sachen Abschlussarbeiten oder in der Verschriftlichung von mündlichen Abmachungen zu belastbaren Verträgen haben
			\item \textbf{Ablauf}: Thematik auffrischen, holistischen Lösungsansatz suchen, Resolution schreiben, Vorlage für Vertrag schreiben
			\item \textbf{Voraussetzungen}: keine
		\end{itemize}

  \subsection{Einleitung}
  	Ziel des Folge-AKs ist es eine Resolution (an die Fachbereiche, KFP etc.) zu verabschieden, die Physik-Fachbereiche dazu auffordert die Rahmenbedingungen (Prüfungsordnung, Modulbeschreibung) so anzupassen, dass es Studierenden vor Ausgabe des Themas ihrer Abschlussarbeit möglich ist sich in abgesichertem Rahmen in das Projekt an dem die Abschlussarbeit angefertigt werden soll in der Arbeitsgruppe ein zu arbeiten. Die Einarbeitungsphase vor Anmeldung ist an vielen Universitäten Gang und Gäbe, wird jedoch nur mündlich vereinbart.

  \subsection*{Protokoll}
    \paragraph{Vorstellung des Programms}
      Folge-AK aus Siegen. Langfristig soll eine Resolution geschrieben werden, wie vermieden werden kann, dass Studierende während
      ihrer Einarbeitungszeit ausgenutzt werden. Ziel ist es, einen konstruktiven Vorschlag zu machen. In Siegen war die Tendenz,
      dass es einen Bearbeitungsvertrag bzw. Projektplan geben soll, der offen genug ist, um die Arbeit flexibel zu gestalten.
      Nach der letzten ZaPF wurden Rückmeldungen gesammelt.

    \paragraph{Situation an den einzelnen Universitäten}
      \begin{itemize}
        \item Karlsruhe: Im Master 2 Zusatzmodule, um für die Masterarbeit 1 Jahr Zeit zu haben, anstelle eines halben Jahres.\\ Im Bachelor soll es 3 Monate dauern, wird aber locker gehandhabt (bis zu 6 Monate), damit nebenher studiert werden kann.
        \item Kaiserslautern: Bachelorarbeit 2-3 Monate, Masterarbeit 6 Monate, Bearbeitung des Themas erst nach Anmeldung der Arbeit. Einarbeitung durch sogenanntes Laborpraktikum bzw. SHK-Stelle möglich.
        \item insgesamt: 4 der anwesenden Universitäten haben Vorbereitungsmodule zur Bachelorarbeit (mit expliziter Bindung zur Bachelorarbeit z.B. über Prüfungsleistung), ca. 8 (teilweise bedingt) für eine Masterarbeit
      \end{itemize}
      siehe auch Austausch AK WiSe17 in Siegen

    \paragraph{Stichpunktsammlung}
      Es sollen Punkte gesammelt werden, wie eine gute Abschlussarbeit aussehen soll. \\

      \textbf{Allgemein}: [Wichtigkeit]
      \begin{outline}
        \1 Klares Zeitfenster (unabhängig von der Anzahl der Module) [10]
        \1 Klar definierte Prüfungsleistung [4]
        \1 Verweis auf Qualifikationsplan [6]
        \1 Schriftlicher Projektplan [8]
        \1 Es sollte präzise genug sein, um ein Ziel zu haben, aber vage genug, um kleine Probleme ab zu fangen.
        \1 Änderungen/Entscheidungen einvernehmlich
        \1 Back-Up Plan
        \1 Keine Einarbeitungsphasen, die nicht Bestandteil des Studiengangs sind [10] (Gleichbehandlung, Vergleichbarkeit)
        \1 Zusammenhang SHK/HiWi mit Abschlussarbeit? [5]
          \2 thematische Trennung
        \1 Einbettung in Studienalltag [7]
        \1 Sofortiges Anmelden [10]
      \end{outline}

      \textbf{Fragen}:
      \begin{itemize}
        \item Wie soll verfahren werden, wenn sich Universitäten nicht an so einen Rahmen halten?
      \end{itemize}
