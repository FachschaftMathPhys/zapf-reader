% !TEX TS-program = pdflatex
% !TEX encoding = UTF-8 Unicode
% !TEX ROOT = main.tex

\section{AK Fortgeschrittenenpraktikum}

\textbf{Protokoll vom:} 31.05.2018,
Beginn: 16:30 Uhr,
Ende: 18:30 Uhr \\
\textbf{Redeleitung:} Lisa Dietrich (Uni Erlangen-Nürnberg) \\
\textbf{Protokoll:} Marius Anger (TU München) \\
\textbf{anwesende Fachschaften:} Uni Erlangen-Nürnberg, Uni Düsseldorf, Uni Bonn, Uni Frankfurt, Uni Augsburg,
TU München, Uni Jena, Uni Freiburg, Uni Osnabrück, Uni Wuppertal, Uni Tübingen, Uni Chemnitz, Uni Münster, Uni Cottbus, Uni Saarland,
TU Kaiserslautern, Uni Würzburg, Uni Gießen, Uni Darmstadt, Uni Wien, Uni Halle-Wittenberg, Uni Konstanz, Uni Bochum

\subsection*{Informationen zum AK}
	\begin{itemize}
		\item \textbf{Ziel des AKs}: Ziel des AKs ist es ein Positionspapier zu formulieren, wie es schon beim AK Praktika gemacht wurde, nur diesmal mit dem Fortgeschrittenenpraktikum
		\item \textbf{Folge-AK}: ja
	  \item \textbf{Materialien}: Am Besten schon mal ein Fortgeschrittenenpraktikum gemacht oder Erfahrung damit
		\item \textbf{Zielgruppe}: alle ZaPFika, aber vor allem die die schon mal Fortgeschrittenenpraktikum gemacht haben
		\item \textbf{Ablauf}: Ideensammlung, Diskussion und anschließen Positionspapier
		\item \textbf{Voraussetzungen}: Protokoll aus Siegen gelesen
	\end{itemize}

\subsection*{Protokoll}
  \subsubsection*{Einleitung}
    Es soll sich wie beim Grundpraktikum überlegt werden, welche Anforderungen wir an das Fortgeschrittenenpraktikum haben und welche Qualifikationen man nachdem Fortgeschrittenenpraktikum haben sollte. Im AK in Siegen haben wir bereits eine Ideensammlung gemacht, in Heidelberg soll an jener Stelle weiter gemacht werden in dem die vorher gesammelten Informationen und Ideen nach Wichtigkeit sortiert und diskutiert werden, um die Punkte, die man später im Positionspapier haben will, heraus zu arbeiten.

  \subsubsection{Sachen, die noch nicht einstimmig sind}

    \begin{itemize}
      \item Vor- \& Nachbesprechung (einstimmig angenommen)
        \begin{itemize}
          \item Vorbesprechung: Sicherstellung, dass der Versuch ohne Schäden durchgeführt werden kann und man den Versuch verstanden hat, Fragen stellen
          \item Nachbesprechung: Fehler besprechen (Protokoll) und aber auch was nehme ich mit aus dem Versuch, Fragen stellen
        \end{itemize}
      \item AUCH reale Versuchsaufbauten (einstimmig angenommen)
        \begin{itemize}
          \item nicht nur ein Mausklick um den Versuch zu machen, ein ding reinschieben, messen, Nächstes ist kein solcher Aufbau
        \end{itemize}
      \item Möglichkeiten als Blockpraktikum (BP) (einstimmig angenommen)
        \begin{itemize}
          \item Zweifel an der Umsetzbarkeit bezüglich Länge der einzelnen Versuche
          \item Außerdem gibt es Versuche die an Umweltphenomänen hängen (zb Teleskop bei Nebel)
          \item es soll die Qualität in keinem Fall einschränken
          \item ist eine Empfehlung
          \item Änderung des Titels: WENN möglich ein Angebot auf ein Blockpraktikum
        \end{itemize}
      \item Angmessene Arbeitszeit bei BP (ersatzlos gestrichen)
        \begin{itemize}
          \item wurde in Siegen schon angenommen, bzw falsch formuliert
          \item Änderung des Titels: Angemessene Arbeitszeit für das Praktikum unter dem Semester
          \item Ist durch ECTS gesetzlich geregelt dewegen wird der Punkt ersatzlos geschtrichen
        \end{itemize}
      \item Freie Versuchswahl (einstimmig angenommen)
        \begin{itemize}
          \item Möglichkeit aus einem Versuchspool auszuwählen
          \item Änderung des Titels: Freie Versuchswahl, wenn möglich
          \item bei kleinen Universitäten evtl nicht machbar, da nicht soviele Mittel für viele Versuche da
          \item Punkt meint aber auch freie Wahl für die Studenten, das inkludiert auch die Wahl aus verschiedenen Versuchsgruppen
          \item Ludi solls schöner formulieren
        \end{itemize}
      \item Vertiefte Statistik \& Plotkenntnisse (einstimmig angenommen)
        \begin{itemize}
          \item als Lernziel
          \item ist hier drin, da manchmal nicht im Grundpraktium
        \end{itemize}
      \item Freie Terminwahl (einstimmig angenommen)
        \begin{itemize}
          \item ist bedingt durch freie Versuchswahl
          \item das inkludiert aus einem Terminangebot (an dem der Betreuer da ist)
        \end{itemize}
      \item Laborbuch (einstimmig angenommen)
        \begin{itemize}
          \item als Mitschrift (wie auch immer die dann aussieht)
          \item Notizen unter dem Versuch (wie, was gemessen, evtl Beobachtungen)
          \item Über die Form wird diskutiert \\
              + Geräte gebunden oder Personenbezogen \\
              + meist von der Uni geregelt
          \item Änderung des Titels: Führung eines Messprotokolls
          \item Laborbuch (beides) ist gute wissenschafltiche Praxis (deswegen eigentlich Lehrinhalt im FoPra)
          \item Lernziel: gutes Messprotokoll führen zusätzlich zu einer Ausarbeitung/Gesamtprotokolls
          \item Der Begriff Laborbuch/Messprotokoll bedarf genauer Klärung
          \item keine losen Blätter aber in einer Form zusammengehalten (zb Hefter)
        \end{itemize}
      \item Plagiatsprüfung (einstimmig angenommen)
        \begin{itemize}
          \item zu kleiner Lösungsraum in den meisten Praktika
          \item Mögliche Software mit alt Berichten und Internet Referenzen \\
              + Mit Prozent Anzeige \\
              + Schlägt nur an bei ganzen Absätzen \\
              + Außerdem werden die Stellen angezeigt \\
              + Es MUSS ein Mensch darüberlesen
          \item Muss aber keine Software involvieren
        \end{itemize}
    \end{itemize}
