
% !TEX TS-program = pdflatex
% !TEX encoding = UTF-8 Unicode
% !TEX ROOT = main.tex

\section{AK Akkreditierung I+II}

  \textbf{Protokoll vom:} 31.05.2018, %???
  Beginn: 16:30 Uhr, %???
  Ende: 18:30 Uhr \\ %???
  \textbf{Redeleitung:} \\ %???
  \textbf{Protokoll:} \\ %???
  \textbf{anwesende Fachschaften:} Uni Wuppertal, Uni Dresden, Uni Gießen, Uni Frankfurt, Uni Jena

  \subsection*{Informationen zum AK}
    \begin{itemize}
    	\item \textbf{Ziel des AKs}: Interne Richtlinien der ZaPF mit Blick auf die geänderte Rechtslage anschauen und überarbeiten
    	\item \textbf{Folge-AK}: ja
      \item \textbf{Materialien}: Akkreditierungsrichtlinien (WS 2002), Akkreditierungsrichtlinien (WS 2008), Protokoll WS 2015, European Standards and Guidelines for Quality Assurance in the European Higher Education Area
    	\item \textbf{Zielgruppe}: alle ZaPFika, die Interesse an der Akkreditierung von Studiengängen haben
    	\item \textbf{Ablauf}: Zusammenfassung zu einer kommentierten Version, eventuell mit Lehramtsvariante
    	\item \textbf{Voraussetzungen}: Protokolle gelesen
    \end{itemize}

  \subsection*{Einleitung}
    Der AK Akkreditierung I hat das Ziel, die internen Richtlinien der ZaPF mit Blick auf die geaenderte Rechtslage noch einmal anzuschauen und zu ueberarbeiten. Zielgruppe hierfuer sind Menschen, die sich gut mit Akkreditierung auskennen. \\

Infos fuer weniger erfahrene Menschen gibt es im Workshop Akkreditierung. \\

Die Richtlinien sollen als Positionspapier gefasst werden

  \subsection*{Protokoll}
    \paragraph{Was nun passieren soll:}
      \begin{itemize}
        \item diesen Entwurf hübsch machen, damit man ihn vorstellen kann
        \item in Würzburg über die inhaltliche Sinnhaftigkeit der Punkte diskutieren und neue Richtlinen verabschieden (die alles bisherige ersetzen)
        \item die Wiki-Kategorie in einem Arbeits-AK in Würzburg aktualisieren und hübsch machen (Rücksprache mit dem StaPF)
        \item eine Variante für Lehramt entwerfen
        \item eventuell ESG, EQR, DQR, Lissabon und andere europäische Dokumente an entsprechenden Stellen zitieren
      \end{itemize}

    \subsubsection*{Entwurf des kommentierten Rasters für Akkreditierungsberichte des Akkreditierungsrates (Drs. AR 33/2018)}
      Dieses Dokument ersetzt fürs Erste die vormaligen Akkreditierungsrichtlinien des Akkreditierungsrates

    \textbf{Formale Kriterien}
      \paragraph{Studienstruktur und Studiendauer (§3 MRVO)}
        \begin{itemize}
          \item \sout{2002: Punkt 2 Studiendauer für den Bachelor sind 6 Semester inklusive Bachelorarbeit (H)} % Befehl benötigt Paket "ulem"
          \item \sout{2002: Masterstudium sollen 4 Semester inklusive Masterarbeit sein.}
          \item Vernetzung zum AK Vorläufige Verträge für Abschlussarbeiten
          \item \sout{2002: Punkt 15 Bachelor soll nicht nur Zugang zum Master sein, sondern wirklich ein berufsqualifizierender Abschluss (H)}
        \end{itemize}

      \paragraph{Studiengangsprofile (§ 4 MRVO)}
        \begin{itemize}
          \item \sout{2002: Es gibt eine Bachelorthesis (H), Umfang 2-6 Monate (W)}
          \item \sout{2002: Es gibt eine Masterthesis, mit mind. 6 (H) bzw. mind. 9 (W)}
          \item \sout{Verweis auf EQR}
        \end{itemize}

      \paragraph{Zugansvoraussetzungen und Übergänge zwischen Studienangeboten (§ 5MRVO)}
        \begin{itemize}
          \item \sout{2008: Wenn es viele Vorlesungen in Fremdsprachen gibt, muss das in den Zugangsvoraussetzung wenigstens als Hinweiß drin stehen.}
          \item \sout{2015 Kap. 2.3: bspw: Der Mathe-Vorkurs soll keinen Inhalt vermitteln. Die Zulassung zum Studiengang soll nicht restriktiv gahandhabt werden.}
          \item \sout{2015 Kap. 2.3: Im Master sollen Quereinsteiger nicht benachteiligt werden.}
          \item \sout{2017: Bei den bisherigen Zugangsvoraussetzungen für Masterstudiengänge \flqq Zugangsvoraussetzung für einen Masterstudiengang ist in der Regel ein berufsqualifizierender Hochschulabschluss\frqq entfällt das \flqq in der Regel\frqq, was beruflich qualifizierten Bewerbern ohne Hochschulabschluss den Zugang erschwert.}
        \end{itemize}

      \paragraph{Abschlüsse und Abschlussbezeichnungen (§ 5 MRVO)}
        NEU: MRVO schließt nicht aus, dass weiterhin Diplomstudiengänge in einer Systemakkreditierung akkreditiert werden.
        Wegen einer konsequenten Umsetzung des Bologna-Gedankens und der Mobilität, sollen unsere Gutachter das nicht unterstützen.


      \paragraph{Modularisierung (§ 7 MRVO)}
        \underline{(1) Modul}
          \begin{itemize}
            \item \sout{2002 Bachelor Punkt 6 / Master Punkt 5: Modularisierung soll sinnvoll sein}
            \item \sout{2008: Sinnvolle Modularisierung}
          \end{itemize}

        \underline{(2) Moduldauer}
          \begin{itemize}
            \item Nichts zur Moduldauer
          \end{itemize}

        \underline{(3) Modulbeschreibungen}
          \begin{itemize}
            \item \sout{2002 Punkt 7 (Bachelor): Studienbegleitende Prüfungen}
            \item \sout{2015 Kap. 2.5: Prüfungsform soll dem Inhalt des Moduls angemessen sein.}
            \item \sout{2015 Kap. 2.5: Zulassungsvoraussetzungen sollen der Persönlichkeitsentwicklung des Studenten nicht entgegenlaufen.}
            \item \sout{2015 Kap 2.5:  \flqq Sitzscheine\frqq sollen vermieden werden, Anwesenheitspflicht nur in Ausnahmefällen.}
            \item \sout{2008 Notwendige Sprachkenntnisse müssen klar definiert wird.}
            \item \sout{2015 Kap. 2.4: Interne Vorraussetzungen müssen möglichst vorsichtig eingesetzt werden $\rightarrow$ Flexibilität des Studienablaufs.}
          \end{itemize}




    \paragraph*{Leistungspunktesystem (§ 8 MRVO)}
      \begin{itemize}
        \item 2002 Punkt 4: Creditierung nach ECTS soll statt finden
        \item 2008 Creditierung nach ECTS soll statt finden
        \item 2002 Punkt 16 (Bachelor)/ Punkt 8 (Master) Realistische Bemessung der ECTS
        \item 2008 Workload-Erhebung mit Konsequenzen
        \item 2015 2.4: ECTS sollen möglichst dem Arbeitsaufwand entsprechen.
        \item 2008: Gewichtung der ECTS in frühen Semestern weniger in Abschlussarbeit mehr
        \item 2002: Es gibt eine Bachelorthesis (H), Umfang 2-6 Monate (W)
        \item 2002: Es gibt eine Masterthesis, mit mind. 6 (H) bzw. mind. 9 (W)
        \item 2008: Möglichst umfangreiche eigenständige Bachelorarbeit (Da sollte man über Änderungen nachdenken)
      \end{itemize}

    \paragraph*{Besondere Kriterien}
      NEU: Bei externe Abschlussarbeiten muss wissenschaftlichkeit durch Betreuung an der Hochschule gewährleistet werden.

    \paragraph*{Joint-Degree}
      keine Veränderungen

  \textbf{fachlich-inhaltlich Kriterien} \\
    \paragraph*{Qualifikationsziele und Abschlussniveau (§ 11 MRVO)}
      \begin{itemize}
        \item Positionspapier Wissenschaftskommunikation WS17/18 [...]Wissenschaftskommunikation ein elementarer Bestandteil im Studium sein sollte. [...] Diese sollte mindestens als Wahlpflichtmodul vorkommen. Sinnvoll für die Umsetzung erachten wir ein Seminar und/oder eine Ringvorlesung[...]
        \item Positionspapier zu Ethikinhalten im Physikstudium: Die ZaPF spricht sich dafür aus, Ethikinhalte in einem angemessen Umfang in das Physikstudium einzubinden, sodass die Möglichkeit geboten wird, sich auch im Rahmen des Studiums mit ethischen Fragenstellungen auseinanderzusetzen.
        \item 2002 Punkt 10 (Bachelor) Schlüsselqualifikation werden angerechnet
        \item 2015 Kapitel 2.1 Punkt 2: Nicht nur forschungsausrichtung im Studium, Übergang in Wirtschaft soll möglich sein
        \item 2002: Punkte 14 Der Bachelor soll eine solide physikalische Grundausbildung sein. (H)
        \item 2008 solide Physikausbildung und eine möglicher Übergang in die Wirtschaft
      \end{itemize}

    \paragraph*{Schlüssiges Studiengangskonzept und adäquate Umsetzung (§12 MRVO)}
      \begin{itemize}
        \item 2015 Kap. 2.4: Interne Vorraussetzungen müssen möglichst vorsichtig eingesetzt werden -> Flexibilität des Studienablaufs.
        \item 2015 Kap. 2.4: für mündliche Prüfungen kein Prüfungszeitraum.
        \item 2002 Punkt 12 (Bachelor): Spezialisierung ist auch möglich
        \item 2002 Punkt 13 (Bachelor): Ein nicht-physikalisches Nebenfach ist obligatorisch
        \item 2002 Wahlmöglichkeiten müssen exsistieren
        \item 2002 Master: Spezialisierung (30-70 % W)
        \item 2008: es kann eine Veranstaltung mit ECTS mit nicht-physikalischem Inhalt geben, Vorschläge für Nebenfach, Wahlpflichtbereich
        \item 2008: es soll eine Auswahlmöglichkeit an physikalischen Vertiefungen geben
        \item 2015 Kap. 2.1: Wahlfreiheit, nicht verschultes Curriculum
        \item SS10 (Empfehlungen zur Ausgestaltung der Bachelor- und Master-Studiengänge im Fach Physik Der Bachelor sollte Versuche im Grundpraktikum von mindestens 12 CP und im Fortgeschrittenenpraktikum im Umfang von 6-8 CP enthalten.
        \item 2002 Punkt 5 (Bachelor): Auslandsaufenthalt im Bachelor wird unterstützt
        \item 2002 Punkt 18 (Bachelor) / Punkt 10 (Master) Faires Konzept zur Anrechnung (auch in 2008)
        \item 2008: Auslandsaufenthalte sollen gefördert werden durch Anrechnung
        \item ESG: Hochschullehrer Qualifikation
        \item ZaPF-Beschluss zur Fortbildung??? War da was?
        \item Übungskonzepte WiSe 2010
        \item 2015 Kap. 2.7: Mechanismen zur Überholung/Wartung von Praktikumsversuchen und Qualifizierung von Tutoren, Weiterbildungsmöglichkeiten für Professoren
        \item 2015 Kap. 2.3: Anerkennung außerhalb der Hochschule erbrachter Leistungen
        \item 2002 (Punkt 11 Master): Defizite aus dem Vorstudium werden im Master ausgeglichen.
        \item 2008 Zeitnahe Prüfungswiederholungen
        \item 2008 (und 2002): Regelungen zur Notenverbesserung (Freiversuch) sind wünschenswert
        \item 2002 Punkt 1: Studierbarkeit
        \item 2015 Kap. 2.9: Einbindung von Studierenden in die Studiengangsentwicklung
        \item 2016 Positionspapier zu Programmierfähigkeiten im Physikstudium
        \item 2008: Bachelorarbeit soll so konzipiert sein, dass man auf jeden Fall zum Master fristgerecht die Hochschule wechseln kann.
      \end{itemize}


    \textbf{Fachlich-Inhaltliche Gestaltung der Studiengänge (§ 13 MRVO)}
      \begin{itemize}
        \item 2008: Lehrevaluationen muss Konsequenzen haben, es muss sinnvolle Mechanism zur Reaktion geben
        \item 2015 Kap. 2.9: Evaluation von Lehrveranstaltungen, Rückkopplung an die Lehrenden?
      \end{itemize}

    \textbf{Studienerfolg(§ 14 MRVO)}
      \begin{itemize}
        \item 2015 Kap. 2.9: Absolventenverbleib?
      \end{itemize}

    \textbf{Geschlechtergerechtigkeit und Nachteilsausgleich}
      \begin{itemize}
        \item 2015 Kap. 2.3: Benennung von Studierendenberatenden
        \item 2015 Kap. 2.3: Praktikumslabore sollen möglichst barrierefrei sein, ggf. müssen Ersatzversuche angegeben werden
      \end{itemize}

    \subsection{Über das Raster hinaus wuenscht sich die ZaPF:}
      \begin{itemize}
        \item Tutor*innen sollen bei Begehung im Gespräch mit den Lehrenden dabei sein (Protokoll 2015 2.7)
        \item Transparenz und Eindeutigkeit der Studiendokumente (war früher mal Kriterium 2.8 des Akkreditierungsrats)
        \item Lehramt: SS10 $\rightarrow$ \url{https://zapf.wiki/images/3/35/Lehramtstellungnahme.pdf}
        \item SS10 - Empfehlungen zur Ausgestaltung der Bachelor- und Master-Studiengänge im Fach Physik (\url{https://zapf.wiki/SoSe10_Beschl%C3%BCsse})
      \end{itemize}

Der Bachelorstudiengang soll 180 CP und der Master 120 CP umfassen.
Um Auslandsaufenthalte zu unterstützen und Hochschulwechsel zu ermöglichen, sollen extern erbrachte Studienleistungen im Pflichtbereich des Bachelorstudiums im vollen Leistungspunktumfang auf inhaltlich ähnliche Module der eigenen Hochschule angerechnet und als Qualifikation für Folgemodule anerkannt werden. Bei einer Differenz in der Anzahl der Leistungspunkte wird ein kulantes Vorgehen befürwortet. Gibt es an der eigenen Hochschule kein äquivalentes Modul, so sollen die Leistungen in einem entsprechenden Wahlbereich angerechnet werden.

        Es sollen wirksame Mechanismen zur Qualitätssicherung der Studiengänge und eine Instanz zur sinnvollen Zuordnung und zur Überprüfung des tatsächlichen Arbeitsaufwandes vorhanden sein.
        Die Prüfungs- und Studienordnungen müssen transparent und eindeutig sein.
        In der Experimentalphysik sollen im Bachelor mindestens folgende Inhalte vermittelt werden:
            Klassische Mechanik,
            Thermodynamik,
            Elektrodynamik,
            Optik,
            Quanten- / Atomphysik.\\
        In der theoretischen Physik sollen im Bachelor mindestens die folgenden Inhalte vermittelt werden:
            Klassische Mechanik,
            Analytische Mechanik,
            Elektrodynamik,
            Spezielle Relativitätstheorie,
            Einführung in die Quantenmechanik,
            Thermodynamik.\\
        Eine für die Bewältigung der Studieninhalte der Punkte 5 und 6 notwendige Vermittlung der entsprechenden Rechenmethoden soll rechtzeitig erfolgen und ggf. durch ein ergänzendes Modul gewährleistet werden.
        Der Umfang der Punkte 5 und 6 sollte insgesamt etwa 50-60 CP betragen, mit einer Gewichtung von 1:1 von Experiment und Theorie. Universitäten können selbst Schwerpunkte auf Theorie oder Experiment legen, wobei die Gewichtung nicht stärker als 2:1 sein sollte.
        In der mathematischen Ausbildung sollten folgende Inhalte vermittelt werden:
            Analysis einer Veränderlichen,
            Analysis mehrerer Veränderlicher,
            zugehörige Integrationstheorie,
            Lineare Algebra (elementare Matrixberechnungen bis Eigenwertprobleme),
            gewöhnliche Differentialgleichungen,
            Funktionentheorie,
            Operatorentheorie auf Hilberträumen.\\

            Diese Inhalte sollten etwa 30 CP umfassen.\\

        Weiterhin sollen grundlegende Kenntnisse im Experimentieren vermittelt werden. Der Bachelor sollte Versuche im Grundpraktikum von mindestens 12 CP und im Fortgeschrittenenpraktikum im Umfang von 6-8 CP enthalten. Ein Ziel der Praktika sollte das Erlernen eigenständigen Arbeitens sein. Dies kann z.B. realisiert werden durch die Integration eines Projektpraktikums, welches das Grundpraktikum zum Teil ersetzen könnte.
        Die Inhalte von Festkörperphysik, Kern- und Elementarteilchenphysik, Atom- und Molekülphysik, Höhere Quantenmechanik und Statistische Physik sind wichtige Themen des Physikstudiums und es soll sichergestellt werden, dass diese Inhalte bis zum Masterabschluss gehört und eingebracht werden können.
        Im Bachelor sollte es möglich sein, Qualifikationen im Umfang von etwa 10 CP wie z.B. Programmiersprachen, Elektronik oder wissenschaftliches Präsentieren zu erlernen und einzubringen. Außerdem sollte es Raum von 33-45 CP für einen physikalischen Wahlbereich geben, der ein breites Angebot an Seminaren und ersten Vertiefungsvorlesungen im Bachelor beinhaltet.\\
        Weiterhin sollte Raum für ein verpflichtendes nichtphysikalisches Nebenfach geschaffen werden, welches einen Umfang von höchstens 12 CP haben sollte. Für physiknahe Fächer können zusätzlich CP aus dem physikalischen Wahlbereich hinzugezogen werden.
        Die Bachelorarbeit sollte einen Umfang von etwa 15 CP haben. Für diese dürfen jedoch keine weiteren Zusatzkenntnisse verlangt werden, die über die entsprechende Ordnung hinausgehen.
        Schon frühzeitig im Bachelorstudium sollen abweichend von der Klausur als Prüfungsform auch andere Prüfungsformen angeboten werden. Insbesondere werden mündliche, möglicherweise modulübergreifende Prüfungen befürwortet, um vernetztes Lernen der Studierenden zu fördern.\\
        Im Master sollte es einen Bereich von 60 CP geben, der sowohl vertiefende Spezialisierungsveranstaltungen als auch Veranstaltungen über bisher nicht behandelte physikalische Themen beinhaltet. Ein verpflichtender Anteil sollte ingesamt einen Umfang von 20 CP nicht übersteigen.
        Das Masterstudium sollte mit einer einjährigen Forschungsphase abgeschlossen werden, die mit einem Umfang von 60 CP bemessen ist.
