% !TEX TS-program = pdflatex
% !TEX encoding = UTF-8 Unicode
% !TEX ROOT = main.tex

\section{AK Alumni}

  \textbf{Protokoll vom:} 31.05.2018,
  Beginn: 08:10 Uhr,
  Ende: 09:00 Uhr \\
  \textbf{Redeleitung:} Merten (Göttingen) \\
  \textbf{Protokoll:} Johannes (Tübingen) \\
  \textbf{anwesende Fachschaften:} Uni Würzburg , Uni Gießen , Uni Bonn , Uni Tübingen , Uni Oldenburg , TU Berlin , Uni Münster , TU Chemnitz , Uni Darmstadt , Uni Wien , TU Berlin , Uni Augsburg , Uni Göttingen / jDPG (Merten) , Uni Siegen , Uni Jena , Uni Bielefeld



  \subsection*{Informationen zum AK}
    \begin{itemize}
      \item \textbf{Ziel des AKs}: Austausch darüber, wie Subsidiarität an den Universitäten gelebt wird, gegebenenfalls Resolution dafür oder dagegen, dass Entscheidungen in der Regel auf der niedrigstmöglichen Ebene getroffen werden.
      \item \textbf{Folge-AK}: nein
      \item \textbf{Zielgruppe}: Hauptsächlich Gremienvertreter
      \item \textbf{Ablauf}: Zunächst Austausch, wie Dinge an den Unis gelebt werden, anschließend Diskussion, wie sinnvoll/schädlich Subsidiarität ist. Gegebenenfalls verfassen eines Positionspapiers.
    \end{itemize}

  \subsection*{Einleitung}
    In Göttingen wird derzeit die Systemakkreditierung eingeführt. Hierfür werden zunächst auf zentraler Ebene der Universität ein Leitbild ``Lehre'' und ein Kriterienkatalog beschlossen. In unserer Fakultät gibt es gegen beides massiven Gegenwind, da man das Gefühl hat, bestimmte Dinge werden den Fakultäten (den "dezentralen Einrichtungen") einfach vorgesetzt ohne ihnen wesentliches Mitspracherecht zu geben. \\
    Ich würde gerne wissen, wie das an anderen Unis aussieht, wie das Mitspracherecht der Fakultäten aussieht, was man gegen ein sich allzu diktatorisch verhaltendes Präsidium unternehmen kann etc. \\
    Als Ergebnis könnte ein Positionspapier oder eine Resolution beschlossen werden, das müsste sich aber aus der Diskussion ergeben.

  \subsection*{Protokoll}
      \paragraph{Begriffsklärung}

    \textbf{Subsidiarität}: Regelungen werden auf der höchst-nötigen und niedrigst-möglichen Ebene getroffen. \\
    \underline{Beispiel}: Staaten. Manche Regelungen werden auf Bundesebene abgesprochen, aber die Regelungen selbst werden auf Landesebene getroffen. \\

    Universitäten sind üblicherweiße ähnlich aufgebaut, sodass Regelungen oft auf Universitätsebene getroffen werden. \\

    \underline{Motivation für den AK}: Gefühl an der Uni Göttingen: Zentrale Stellen versuchen verstärkt, Regelungen zentral zu treffen und auf die Fakultäten und Departments überzustölpen (auf Fakultätsebene die Durchsetzung zu erzwingen). Genauer: Im Zuge der Systemakkreditierung in Göttingen gibt es Leistungskataloge, was die Lehre leisten soll (z.B. Abfrage interkultureller Kompetenzen). Unklar, was dies in der Physik bedeutet und wie es in diesem Fach abgefragt werden soll. Derartige Anforderungen im Studiengang unterzubringen ist abhängig vom Fach leichter oder schwerer. Anders formuliert: Die Uni formuliert zentral Qualitätsziele für ihre Qualitätssicherungssysteme, die für die Akkreditierung benötigt werden. Diese werden dann auf die Fakultäten verpflichtend verteilt.

    \paragraph{Aktuelle Beispiele}
      \begin{itemize}
        \item Hörsaalsponsoring: Hörsaal war plötzlich einfach da, ohne dass die dafür zuständigen Gremien befragt oder involviert waren.
        \item (Bonn) Musterstudienordnung: Gibt es für alle Studiengänge, soll eventuell für alle Studiengänge verwendet werden.
        \item Internationalisierung: Vorgaben, sollen alle Masterstudiengänge auf Englisch sein. Ein Masterstudiengang wurde erfolgreich akkreditiert, aber nicht angenommen, weil nicht auf Englisch.
        \item Frauenquote: Stand an Universität: Soll Landesquote entsprechen, würde bedeuten, dass zukünftige Berufungen alle durch Frauen erfüllt werden müssen.
        \item Rahmenprüfungsordnung: Wurde an einer Universität entwickelt, aufgrund von starker Kritik wurde diese teilweise noch überarbeitet, aber nicht vollständig. Wurde anschließend dann beschlossen.
        \item Rahmenprüfungsordnung/-studienordnung: Teilweise universitätsweit formuliert, allerdings passend für Geisteswissenschaften, aber nicht für Naturwissenschaften. Anpassung war bisher allerdings möglich, durch Anhänge/Besondere Regelungen.
        \item Berufungsverfahren: Listenempfehlung der Kommissionen wurden durch den Senat zweifach nicht angenommen. Anschließende neue Reihung wurde nochmals vom Ministerium umgereiht ("Liste gedreht"), sodass weibliche Bewerberin an erster Stelle war, statt dem präferiertem anderen Bewerber.
        \item Berufungsverfahren: "Liste drehen" passiert scheinbar verstärkt auch an anderen Universitäten.
      \end{itemize}
    \underline{Anmerkungen}: Unterschied zwischen: Wieviel hat sich wirklich in den letzten Jahren geändert? Schon lange der Stand, dass von zentraler Stelle Dinge erarbeitet werden, die dann auf unterer Ebene diskutiert werden. Die Eingaben der unteren Ebenen werden dann oft einfach nicht berücksichtigt oder sind in späteren Diskussionen nicht mehr vorhanden. \\

    \underline{Vorteile durch Subsidiarität}
      \begin{outline}
        \1 TU Berlin Haushalt: Alle wissenschaftlichen Angestellten auf Haushaltsstellen müssen für fünf Jahre mit einer vollen Stelle angestellt werden; wird aktuell noch teilweise untergraben. Prinzipielle Möglichkeit zur Reduktion von prekären Arbeitsbedingungen.
        \1 Vorlesungsumfrage: Vorlesungsumfragen von zentraler Stelle.
          \2 Gegenargument: Wäre ein zugeschnittener Fragebogen für die Evaluationen nicht besser? (Fakultäts-/Fachbereichsebene)
          \2 Einwand: Gegebenenfalls sinnvoll (am Beispiel ``Evaluationen''), Regelung: "Es muss Evaluationen mit folgender Qualität geben...", die Ausgestaltung dann aber auf Fachebene belassen.
          \2 Einheitliche Prüfungs- und Studienordnungen können eventuell Bürokratie reduzieren und Ordnungen weniger kompliziert machen (spezifische Regelungen in Anhängen oder besonderen Teilen regeln). Dadurch dann auch Vereinfachung des Studiengangswechsel, der Mobilität innerhalb der Uni und Verbesserung der Vergleichbarkeit von Studiengängen.
      \end{outline}

    \underline{Allgemein}
      \begin{outline}
        \1 Es gibt Themen, die auf zentraler Ebene behandelt werden können und manche, die auf dezentraler Ebene besser durchgeführt werden. Dies sollte irgendwo klar geregelt werden.
        \1 Besser: Bewusstes Durchführen und Umsetzen auf verschiedenen Ebenen, als unbewusste Umsetzung ("unbewusstes Schiefgehen")
        \1 Partizipative Prozesse sind sehr komplex und oft wird nicht klar kommuniziert, welche Faktoren in die Entscheidung letztendlich eingeflossen sind.
        \1 Eventuell manchmal Kommunikationsprobleme, Meinungen werden berücksichtigt, aber das Endergebnis nicht richtig kommuniziert, sodass Eindruck entsteht, dass die Meinungen nicht berücksichtigt wurden.
        \1 Manchmal führt Beharrlichkeit dazu, dass doch Pläne in Zusammenarbeit mit den unteren Instanzen ausgearbeitet werden.
        \1 Es kann sehr viel von den richtigen Ansprechpartnern an zentralen Stellen abhängen.
        \1 In manchen Gremien können solche Entscheidungen und Prozesse durch ein Veto verhindert werden.
        \1 Erfahrung: Vorteile durch Entscheidungen auf zentraler Ebene, allerdings wenn diese nicht richtig kommuniziert oder umgesetzt werden, dann kann das zu Problemen führen. \\ $\rightarrow$ Mit allen Betroffenen diskutieren und Änderungen und Auswirkungen besprechen, bevor Änderungen durchgeführt werden.
        \1 In anderen Ländern deutlich mehr zentrale Vorgaben (und damit auch Einschränkugen der Gestaltung der Fachbereiche / Studiengänge) als in Deuschland (z.B. Griechenland)
        \1 Es kann durchaus vermieden werden, dass etwas schief geht, indem in den richtigen Gremien auch Vertreter der zukünftig Betroffenen sitzen.
      \end{outline}

    \underline{Diskussion}
    \begin{outline}
      \1 Macht eine Sammlung Sinn?
        \2 Wenn Sammlung zu spezifisch: Nur schwer übertragbar.
        \2 Wenn Sammlung zu allgemin: Eventuell nicht wertvoll.
      \1 Macht es Sinn, dass sich die ZaPF dazu allgemein positioniert?
      \1 Was könnten Inhalte für eine mögliche Resolution werden?
        \2 Sehr allgemeine Resolution: Wann könnte diese verwendet werden? Wert für Fachschaften? Vermutlich nicht sehr hilfreich. Besser: Auf einzelne, spezifische Fälle eingehen und zu diesen positionieren (als Beispiel für spätere Fälle), z.B. zum Hörsaalbranding.
      \1 Gibt es Möglichkeiten, die Abnahme/Zunahme von Subsidiarität objektiver zu beobachten und die Entwicklung zu beurteilen?
      eventuell informelle Datenbank als Sammlung von Beobachtungen und Vorfällen. Mit dieser Datenbank versuchen einen Trend abzulesen.
        \2 Entsprechendes Festhalten in Form einer Selbstverpflichtung (teilw. Zustimmung und Ablehnung im AK)
        \2 Datenbank könnte z.B. im Wiki sein und enthalten:
          \3 Vorfälle aus Gremien (wenn mal wieder etwas schiefläuft)
        \2 Eventuell Überlappung mit Beschlussdatenbank-AK?
        \2 Problem einer solchen Dokumentation: Nicht alles, was im AK gesagt werden würde, würde auch im Wiki oder der Datenbank eingetragen werden.
        \2 Eine solche Sammlung wird als keine praktikable Idee angesehen
    \end{outline}
    \underline{Meinungsbild:} Wer wäre dafür, ein Positionspapier zur Befürwortung der Autarkie der Fachbereiche in DACH zu ersetllen? \\
      $\rightarrow$ 2 dafür - 6 dagegen - einige Enthaltungen (dafür / dagegen / Enthaltung)

    \underline{Anmerkungen zum Meinungsbild}
    \begin{outline}
      \1 Benötigen wir so ein allgemeines Positionspapier wirklich?
      \1 Gibt es ein bestimmtes Problem, bei welchem ein solches Positionspapier helfen könnte?
      \1 Nicht jede Resolutions-Idee wäre zustimmungswürdig, wenn die Resolution oder das Positionspapier klar und stark formuliert wird, dann wäre das gut.
      \1 Eine Resolution/Positionspapier für die Stärkung der Eigenständigkeit von Fachbereichen müsste mit der KfP/den Fachbereichen gemeinsam gemacht werden. Insofern müsste eine solche Resolution an die KfP gehen, und nach der Zustimmung der KfP ("Für mehr Autarkie der Fachbereiche") fragen.
      \1 Es gab vor 1-2 Jahren eine Resolution zum Thema "Bauvorhaben und Studierendenbeteiligung".
    \end{outline}

    \underline{Zusammenfassung zum Meinungsbild} \\
    Man könnte an die KfP und mit der KfP zusammen etwas machen. Aber dafür besteht im AK aktuell wenig Begeisterung. \\

    Problem: Thema wird üblicherweise nur angesprochen, wenn es ein konkretes Problem gibt und etwas schief geht oder schief zu gehen droht. \\

    \underline{Weiterführung der Diskussion}
    \begin{outline}
      \1 Novellierung des Hochschulgesetzes in Nordrhein-Westfalen: Durch die Novellierung soll, grob gesagt, "den Hochschulen die Freiheit gegeben werden, die Freiheit der Hochschulen selbst einzuschränken".
      \1 Einschränkungen auch auf höherer Ebene? etwa Besetzung der Akkreditierungspools, Einschränkungen und Anforderungen durch Förderstellen wie BMBF, DFG,..., welche ebenso die Autarktie der Fachbereiche einschränken.
        \2 So eine Diskussion könnte schnell bei dem Thema "Forschungsfinanzierung" enden.
        \2 Für eine Befassung mit diesem Thema fehlt leider ausreichendes Hintergrundwissen.
      \1 Überlappungen mit AK "Bauvorhaben" $\rightarrow$ Verweis der AK-Teilnehmer an diesen.
    \end{outline}

  \subsection*{Zusammenfassung}
    In dem AK wurden verschiedene Möglichkeiten für die Einführung von Alumni in die ZaPF e.V. Satzung diskutiert. Alumni sollen mit $0 \euro$ Beitrag aufgenommen werden und motiviert werden als Fördermitglieder den ZaPF e.V. ebenfalls finanziell zu unterstützen. \\
    Im zweiten Teil wurden verschiedene Varianten der Vernetzung und Informationsaustausch der Alumni diskutiert und Trendabstimmungen dazu durchgeführt.
