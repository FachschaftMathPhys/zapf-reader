% !TEX TS-program = pdflatex
% !TEX encoding = UTF-8 Unicode
% !TEX ROOT = main.tex

\section{AK Aktualisierung Studienführer}

	\textbf{Protokoll vom:} 31.05.2018,
	Beginn: 09:10 Uhr,
	Ende: 11:00 Uhr \\
	\textbf{Redeleitung:} Michel Vielmetter (Uni Köln) \\
	\textbf{Protokoll:} Michel Vielmetter (Uni Köln) \\
	\textbf{anwesende Fachschaften:} Uni Augsburg, FU Berlin, Uni Wuppertal, HU Berlin, Uni Gießen, Uni Würzburg, TU Darmstadt, TU Dresden, TU Chemnitz, Uni Konstanz

	\subsection*{Informationen zum AK}
		\begin{itemize}
			% \item \textbf{Ziel des AKs}: Austausch und Sortierung von Leuten, die an der Weiterentwicklung ihrer Studiengänge arbeiten (wollen)
			% \item \textbf{Folge-AK}: ja (WiSe `17 Siegen)
			% \item \textbf{Vorwissen}: alte Protokolle lesen (\url{https://zapf.wiki/WiSe17_AK_Rote_Fäden_der_Studienreform}, \url{https://zapf.wiki/SoSe17_AK_Rote_Faeden_der_Studienreform})
      % \item \textbf{Materialien}:
			% \item \textbf{Zielgruppe}: alle
			% \item \textbf{Ablauf}: Inputs zu einzelnen "Fäden", möglichst aus der Studienreformdebatte verschiedener Universitäten; Diskussion
			% \item \textbf{Voraussetzungen}:
		\end{itemize}

  \subsection{Einleitung}
	Alle Fachschaften sollen ihre Informationen im https://studienführer-physik.de/Kategorie:Hochschule aktualisieren. Außerdem sollen alle Fachschaften einen tabellarischer Studienverlaufsplan auf ihrer Seite erstellen (s. u.) Damit das funktioniert und unsre Studienführer dauerhaft sinnvol nutzbar ist, muss aus jeder Fachschaft eine Person in den AK oder in den Back Up AK - die Erfahrung zeigt, dass ein "Arbeitsauftrag für nach der ZaPF" nicht funktioniert!
	Tabellarischer Studienverlaufsplan: Es soll eine Übersicht über den jeweiligen Bachelor Studiengang auf den Seiten der Hochschulen integriert werden. Diese wird sich an den Tabellen im Projekt kommentierte Studienordnung orientieren: https://zapf.wiki/images/b/bd/KommentierteStudienordnungen.pdf (Bis zum AK wird dafür noch eine intuitiv zu bearbeitende Wiki-Vorlage erstellt)
	Hintergrund ist, dass eine so etwas seit längerem für die Weiterentwicklung des Studienführers geplant war(ca. ZaPF in Aachen 2015). Diese verzögert sich aber immer weiter, und da der Studienführer auch von der KFP mittlerweile ernst genommen wird und das Thema im Rahmen des CHE aufkam, wollen wir das jetzt einfach machen.

  \subsection*{Protokoll}
  	\textbf{KEIN Protokoll}
