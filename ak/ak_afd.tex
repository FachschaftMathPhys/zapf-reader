% !TEX TS-program = pdflatex
% !TEX encoding = UTF-8 Unicode
% !TEX ROOT = ../main.tex

\section{AK AFD und deren parlamentarische Arbeit}

  \textbf{Protokoll vom:} 31.05.2018
  Beginn: 16:30 Uhr,
  Ende: 18:30 Uhr \\
  \textbf{Redeleitung:} Jörg (FU Berlin) \\
  \textbf{Protokoll:} Jan (FU Berlin) \\
  \textbf{anwesende Fachschaften:} RWTH Aachen, Freie Universität Berlin, Hum\-boldt-Universität zu Berlin, Rheinische Friedrich-Wilhelms-Universität Bonn, Technische Universität Braunschweig, Technische Universität Chemnitz, Brandenburgische Technische Universität Cottbus, Technische Universität Darmstadt, Technische Universität Dresden, Friedrich-Alexander-Universität Erlangen-Nürnberg, Goethe-Universität Frankfurt am Main, Georg-August-Universität Göttingen, Ernst Moritz Arndt Universität Greifswald, Martin-Luther-Universität Halle-Wittenberg, Technische Universität Ilmenau, jDPG - Junge Deutsche Physikalische Gesellschaft, Universität zu Köln, Universität Konstanz, Fachhochschule Lübeck, Universität Mainz, Ludwig-Maximilians-Universität München, Philipps-Universität Marburg, Carl von Ossietzky Universität Oldenburg, Universität Potsdam, Hochschule Rhein-Main Wiesbaden, Universität Rostock, Eberhard Karls Universität Tübingen, Julius-Maximilians-Universität Würzburg, Universität Wien, Bergische Universität Wuppertal
  \subsection*{Informationen zum AK}
    \begin{itemize}
    	\item \textbf{Ziel des AKs}: Ziel des AKs ist der Hinweis auf die Aktivitäten der AfD in Parlamenten
    	\item \textbf{Folge-AK}: nein
      \item \textbf{Materialien}: keine
    	\item \textbf{Zielgruppe}: Interessenten an der parlamentarischen Arbeit der AfD
    	\item \textbf{Ablauf}: Kurzer Input, dann Diskussion
    	\item \textbf{Voraussetzungen}: keine
    \end{itemize}

  \subsection*{Protokoll}
Der Hintergrund des AKs ist eine kleine Anfrage im Berliner Abgeordnetenhaus, in der sie nach den Namen und Anschriften der ASten-Mitglieder der letzten Jahre gefragt haben. Die Antwort des Senats hat die meisten Fragen mit Verweis auf den Datenschutz abgewiesen.

Es wird auf die Webseite kleineanfragen.de hingewiesen, auf der man kleine Anfragen sehr gut suchen kann. Dort kann man einige Anfragen zu ASten und anderen stud. Belangen finden von der AfD finden, aber auch von der CDU und FDP. Es soll viele Beispiele aus Hamburg geben. 

Es wird nachgefragt, warum die Anfragen der AfD eine besondere Dimension haben, wenn auch CDU und FDP teilweise ähnliche Anfrage stellen. Es ist davon auszugehen, dass die AfD-Anfragen eine Art Einschüchterungen sein sollen und bei CDU/FDP von einer Nähe zum Grundgesetz ausgegangen werden kann. 

Im Folgenden werden im AK Probleme mit rechten Gruppierungen vorgestellt. 
In Halle (Saale) gibt es Probleme mit der identitären Bewegung. 

Auch auf die Vorbereitung der Berliner Anfrage durch die FDP wird eingegangen. 

Auf die Nachfrage, wie auf die Anfrage in Berlin reagiert wurde, wird erklärt, dass der Senat diese entsprechend bearbeiten und z.B. bei Datenschutzbedenken keine weiteren Informationen einholen. Die Antwort war recht nüchtern formuliert. Die AfD hat in Lübeck versucht über die ASten Werbung für eine Demonstration zu verbreiten. 

In Baden-Württemberg gab es eine relativ umfangreiche Anfrage, die innerhalb einer Woche beantworten werden sollte. Ca. 30 der 42 Hochschulen haben diese beantwortet, aber zusammen mit Öffentlichkeitsarbeit. 

In Österreich gibt es das Problem, dass durch die Regierungsbeiteilung der FPÖ schon Auswirkungen gibt, wie die Verteuerung des Studiums durch z.B. Studiengebühren. 

In Köln sind an einer Baustelle Personen mit typisch rechtsextremer Kleidung aufgefallen, der AStA hat sich dagegen positioniert, darauf gab es eine große Kommentierungsaktion durch verschiedene Verbände, auch z.B. aus Sachsen.

Es wird in einem Beitrag nochmal klargestellt, dass die Anfragen von CDU/FDP ein ganz anderes Kaliber waren als die von der AfD. Es wird auf die deutlich höhere Eskalationsstufe der AfD-Rhetorik verwiesen, wo z.B. Arbeitseinsätze von \flqq links-grün-versifften\frqq Studis gefordert wurden. 

In Halle gibt es bereits ein Mitglied der identitären Bewegung im Studierendenparlament, was Probleme in der parlamentarischen Arbeit bereiten wird. In Marburg gibt es auch viele Probleme mit der identitären Bewegung, die dort auch historische Hintergründe haben. 

In Greifswald wurden Namen und Bilder von Mitgliedern der identitären Bewegung veröffentlicht, dort wird auf den Datenschutz und die Rechte der Betroffenen verwiesen, sodass die Aktion nicht als besonders geeignet erschien. Dort soll es auch Probleme bei Wahlen gegeben haben, wo Mitgliedschaften in der identitären Bewegung verschwiegen wurden. 

In Hamburg soll eine Plattform erstellt werden, auf der Eltern und Schüler LehrerInnen und Dozierende melden sollen, die sehr liberale Werte vertreten.  

Es wird auf die Methode von Bernd Höcke verwiesen, welcher rechtsnationale Gedanken vertritt, von dem sich die AfD zwar immer wieder distanziert, Folgen gab es für ihn jedoch keine. Es wird von einer Aktion erzählt, bei der man TeilnehmerInnen einer Höcke-Veranstaltung überzeugen konnte, dass die AfD rechte Positionen vertritt und diese dann nicht mehr an der Veranstaltung teilgenommen haben. 

Es wird noch aus Österreich berichtet, wie die Arbeit der FPÖ läuft und wie mittlerweile eine Diskurs mit AnhängerInnen der FPÖ immer schwieriger wird. In dem Kontext wird aus einer Veröffentlichung der FPÖ zitiert. 

In einem Beitrag wird als eine mögliche Ursache für das Erstarken der AfD die geänderte Diskussionskultur genannt. Als Beispiel werden in der Bundespolitik die Antworten in der Bundespressekonferenz genannt. 

Es wird aufgefordert, weniger über die Ursache zu diskutieren, sondern mehr über konkrete Arbeit, die wir als Fachschaften machen können. 

In Köln gibt es in letzter Zeit auch vermehrt Probleme mit Stickeraktionen auf dem Campus. 

Es werden in der Diskussion auch Burschenschaften erwähnt. Die historische Entwicklung der identitären Bewegung in Marburg hat ihre Wurzel in der Burschenschaftsszene. Dort gibt es auch übergreifende Probleme mit anderen Gruppen, wie dem RCDS. 

Die FPÖ-AnhängerInnen rekrutieren sich auch stark aus den in Österreich an den einzelnen Hochschulen aktiven Burschenschaften. 

In Mainz gibt eher weniger Probleme auf dem Campus, aber es gab eine Hörsaal-Vergabe an eine Veranstaltung mit TeilnehmerInnen aus dem rechten Spektrum. 

Im folgenden wurden folgende Meinungsbilder erfragt:
\begin{itemize}
\item Wer hat Probleme an der Hochschulen mit rechten Gruppierungen? Bei 7 anwesenden Fachschaften.
\item Insbesondere mit der identitären Bewegung? Bei 5 anwesenden Fachschaften
\item Wo gibt es AfD-Hochschulgruppen? Bei 2 anwesenden Fachschaften
\item Bei welchen Hochschulen gibt es etwas allgemeiner gefasst qualifizierbare Probleme (Werbung, Stickerverteilung, ...) mit rechten Gruppen? Bei 15 anwesenden Fachschaften
\item Wo gibt es rechte Burschenschaften? Bei 17 anwesenden Fachschaften
\item Wo gehören die Burschenschaften zu den problematischen rechten Gruppen? Bei 3 anwesenden Fachschaften
\end{itemize}

Im weiteren Verlauf des AKs wurden Ideen gesammelt, wie Fachschaften gegen rechte Gruppierungen vorgehen können.

\begin{itemize}
\item Wenn man gegen rechte Gruppe argumentiert, sollte man nicht vergessen seine eigenen Positionen auch zu vertreten.
\item Gegen Vergabe von Hörsälen soll man sich mit den entsprechenden Zuständigen vernetzen, um dort eingreifen zu können.
\item Bei Aktionen der entsprechenden Gruppen kann man die Aktionen durch friedlichen Protest begleiten, um die Gegenmeinung zu vertreten. Dabei soll darauf geachtet werden, dass man keine durch die Gruppen ausnutzbare Opferrolle verursacht.
\item Natürlich steht am Anfang der inhaltlichen Arbeit gegen rechte Gruppen die Aufklärung von Mitmenschen, um die Inhalte der Gruppen offen zu kommunizieren und dagegen zu argumentieren.
\item Teilweise entstehen Beziehungen zu den Gruppen durch die Wohnungsnot bei Studies. Dort muss man auch eingreifen, um dort den Boden wegzunehmen.
\item Es gibt Kurse zum Thema "Verhalten gegen rechts" von verschiedenen Gruppen wie Parteien und Gewerkschaften, an welchen Einzepersonen oder ganze Fachschaften teilnehmen können.
\end{itemize}
Aus Hessen kommt die Bitte für ein Meinungsbild, ob die AfD zu Diskussionsrunden eingeladen werden sollte, besonders im Vorfeld der Landtagswahl.

Zu dieser Frage wird ein Beispiel aus der Lokalpolitik erzählt, bei der man vor Ort mit den Menschen argumentiert habe und in Diskussionen mit dem Kandidaten dort die Position widerlegt hat. 

Es wird davor gewarnt, kopflos Regeln für Hochschulgruppen zu ändern, da dort auch andere Gruppen schnell betroffen werden können. 

In einem Beitrag wird die Position vertreten, dass Fachschaften vor allem Informationen bereitstellen sollten, um die Mitstudis über die Arbeit von rechten Gruppen aufzuklären. 

Wenn man rechte VertreterInnen einlädt, muss man das ordentlich vorbereiten, um am Ende nicht selbst als "Pöbler" da zu stehen. 

Bezugnehmend auf das Meinungsbild wird empfohlen, dass man seinen Modus operandi nicht wegen der AfD ändert, sondern erstmal zu Grundsätzen Stellung bezieht. Dazu wird unterstützend auch dargelegt, dass man den rechten Bewegungen erstmal keinen Sonderstatus geben soll. 

Meinungsbild: Ist es sinnvoll, die AfD zu politischen Podiumsdiskussion an einer Hochschule einzuladen, wenn bisher Bundestagsparteien eingeladen wurden? 

Die überwiegende Mehrheit der Anwesenden unterstützt, dass man die AfD auch dazu einladen soll.

  \subsection*{Zusammenfassung}
\vspace{-3mm}

 In dem AK besteht Konsens darüber, dass man rechtsnationalen Gruppen durch Diskussionen und Gegenargumente entgegen treten sollte. Hierfür kann man sich vorher auch rhetorisch und verbal schulen lassen.
\newpage