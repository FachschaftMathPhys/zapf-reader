% !TEX TS-program = pdflatex
% !TEX encoding = UTF-8 Unicode
% !TEX ROOT = ../main.tex

\section{AK Alumni}

  \textbf{Protokoll vom:} 02.06.2018,
  Beginn: 09:30 Uhr,
  Ende: 11:15 Uhr \\
  \textbf{Redeleitung:} Elli Schlottmann (TU Berlin) \\
  \textbf{Protokoll:} Elli Schlottmann (TU Berlin) \\
  \textbf{anwesende Fachschaften:} diverse Alumni, Uni Konstanz, Uni Würzburg

  \subsection*{Informationen zum AK}
    \begin{itemize}
      \item \textbf{Ziel des AKs}: Verfassen eines Vorschlags zur Satzungsänderung des e.V. und Strategieentwicklung zum Aufbau des Netzwerks
      \item \textbf{Folge-AK}: ja, Protokoll aus Siegen\footnote{\url{https://zapf.wiki/WiSe17_AK_Alumni}}
      \item \textbf{Zielgruppe}: Satzungs-erfahrene, e.V.-erfahrene und/oder ZaPFika mit Interesse an Alumni-Einbindung
      \item \textbf{Materialien}: Schreiben des ZaPF e.V.\footnote{\url{https://zapfev.de/verein/satzung/\%7CSatzung})}
      \item \textbf{Ablauf}: Diskussion
    \end{itemize}

  \subsection*{Einleitung}
Seit vier Jahren gibt es Vertrauenspersonen auf der ZaPF. Es werden sechs Vertrauenspersonen gewählt und zwei von der ausrichtenden Fachschaft ernannt. Die Wahl der Vertrauenspersonen ist, sofern mehr als sechs Personen kandidieren, etwas kompliziert. \\ 

    Ist die Zahl der Vertrauenspersonen sechs und zwei gut? Ist das zuviel (beliebig) oder zu wenig (nicht immer verfügbar)? Brauchen oder wollen wir überhaupt eine Beschränkung der Anzahl? \\
    Sind wir mit dem Wahlprozedere glücklich? Funktioniert es in der Praxis gut? \\
    Von den Vertrauenspersonen darf es keine Besprechung oder Rückmeldung geben, auch nicht anonymisiert. Das heißt wir wissen nicht, wie häufig und aus welchen Gründen Vertrauenspersonen angesprochen werden. \\ 

    Hätten die Vertrauenspersonen Daten, ließe sich daraus eventuell ableiten, welche Fortbildungen/Vorbereitungen/Ansprüche für Vertrauenspersonen hilfreich sind. Oder es ließe sich herausfinden, ob es wiederkehrende "Probleme" gibt, auf die, die Orga schon im Vorfeld eine Lösung finden könnte.
    Wäre es denkbar, Daten hinreichend anynomisiert zu erheben? Beispielsweise in dem sich die Vertrauenspersonen (anonymisiert) während/nach der ZaPF austauschen und gegebenenfalls direkte Schlüsse für die nächste ZaPF ziehen. Das könnte beispielsweise ausschließlich die Anzahl der Anfragen betreffen und gegebenenfalls an dem StaPF genannt werden.
    Die wichtigste Frage ist natürlich: Wollen wir das?

  \subsection*{Protokoll}
\paragraph{Satzungsänderungsvorschlag}
Auf der letzten ZaPF wurde im AK Alumni besprochen, dass Alumni als Mitglieder im ZaPF e.V. aufgenommen werden. Im ersten Teil des AKs werden notwendige Satzungsänderungen diskutiert:
\begin{itemize}
\item Alumni sind wie Fördermitglieder außerordentliche Mitglieder.
\item Die meisten Regelungen für außerordentliche Mitglieder müssen nicht angepasst werden, um die Alumni, wie gewünscht, mit aufzunehmen.
        \item § 4 2. Außerordentliches Mitglied (*zum Beispiel* Förder- oder Alumnimitglied) kann jede natürliche und juristische Person werden, die die Satzung des Vereins anerkennt. Über Ausnahmen entscheidet der Vorstand.
        \item \flqq Ein\frqq streichen: § 4 3. Jedes ordentliche Mitglied kann gleichzeitig auch *ein* <- streichen außerordentliches Mitglied sein.
        \item Rechtschreibfehler: § 4 4. Die Mitgliedschaft wird durch Antrag in Textform erworben. Es sollen die *be*reitgestellten Vorlagen verwendet werden*.*
        \item § 9 8. 2. Akquise und Betreuung von Förder- und Alumnimitgliedern und Spendern.
      \end{itemize}
      Frage: Außerordentliche Mitglieder können auch juristische Personen sein. Ist das sinnvoll? $\rightarrow$ Es ist auf jeden Fall nicht schlimm, wenn es so ist. Wenn beispielsweise Fachschaften die Alumni-Infos bekommen wollen oder zur Zeit nicht aktiv sind, dürften sie auch Alumni Mitglieder werden.
      \begin{outline}
        \1 Der schwierigste Punkt ist das Mitgliedsschaftsende:
        \2 Was soll passieren, wenn Alumni einen Mitgliedsbeitrag zahlen wollen, dies aber nicht tun?
        \3 Wie bei Fördermitgliedern: Die Mitgliedschaft endet.
        \3 Garnichts. Sie bekommen im Folgejahr wieder eine Zahlungsaufforderung.
        \3 Der Mitgliedsbeitrag wird auf 0 € gesenkt.
        \2 Für alle drei Varianten müsste die Satzung geändert werden.
        \2 Es sollte eine Regelung geben, die für alle Arten von außerordentlichen Mitgliedern gleich ist. Evtl. Beisatz einfügen, dass der Vorstand über den Ausschluss berät und einstimmig entscheiden muss. nach diesem Vorschlag könnten auch "Karteileichen" aufgespürt werden.
      \end{outline}
      Nach einiger Diskussion haben sich 2 Varianten herauskristallisiert:
      \begin{itemize}
        \item Variante A: Alumni zahlen selbt gewählten Beitrag $\leq 0 \euro$.
        \item Variante B: Alumni zahlen alle $0 \euro$ Beitrag und können zusätzlich Fördermitglieder werden.
      \end{itemize}
      \textbf{Pro A}: Alumni Mitglieder werden leichter motiviert den ZaPF e.V. finanziell zu unterstützen, wobei dies durch persönliche Kontaktaufnahme auch mit Variante B gut funktionieren kann. \\
      \textbf{Pro B}: Ist für Datenschutz-Grundverordnung einfacher, da man für beide Mitgliedsarten klar definierte Daten erheben kann und nicht nochmal Beitragzahlende und Nicht-Beitragszahlende unterscheiden muss. Ist auch finanziell sinnvoll, da Kosten durch Überweisungen entstehen und sehr geringe Mitgliedsbeiträge uns unter Umständen mehr Kosten als Nutzen. \\

      \textbf{Trendabstimmung}: A: 0 -  B: 9 
      Mit Variante B ist auch keine Anpassung der Satzung zur Beendigung der Mitgliedschaft 
notwendig. 

      \paragraph{Strategie und Netzwerkaufbau}
        Ideen, welcher Informationsaustausch mit Alumni denkbar ist und mit dem Ergebnis der Trendabstimmung:
        \begin{outline}
          \1 ZaPF-Reader soll elektronisch an alle Alumni geschickt werden (dafür: alle/dagegen: 0) oder evtl. postalisch für Fördermitglieder (Konzept erstellen).
          \1 Stellenangebote von Alumni an ZaPFika verteilen:
            \2 Über analoges schwarzes Brett auf der ZaPF (dafür: 0/dagegen: viele)
            \2 Digitales schwarzes Brett auf Alumni domain (dafür: viele/dagegen: 0)
            \2 Über ZaPF-List (dafür: 0/dagegen: alle)
            \2 Über extra eingerichtete Mailingliste (dafür: 3/dagegen: 4)
            \2 Emails von ZaPFika an Alumni (dafür: 3/dagegen: 3)
          \1 Vermittlung von Exkursionen und Fachvorträgen (dafür: viele/dagegen: 0)
          \1 Analoges schwarzes Brett mit alten Geschichten (dafür: viele/dagegen: 0)
          \1 Spendenaufrufe (falls die Finanzierung einer ZaPF stark gefährdet ist) (Diskussionswürdig)
          \1 Einladung zur ZaPF (dafür: alle/dagegen: 0)
          \1 Parallelprogramm von Alumni für Alumni auf der ZaPF (zum Beispiel Exkursion) (dafür: alle/dagegen: 0)
          \1 Unterkunftsplanung für die ZaPF (dafür: viele/dagegen: 0)
          \1 Newletter alle 2-3 Monate
          \1 Kommentare: Man kann Probleme bekommen, wenn für Firmen Werbung gemacht wird. Stellenangebote auf der Homepage problematisch: Das soll keine Stellenbörse werden. Über den allgemeinen ZaPF-Verteiler würden sich manche ZaPFika über die Stellenanzeigen nicht freuen.
        \end{outline}


  \subsection*{Zusammenfassung}
    In dem AK wurden verschiedene Möglichkeiten für die Einführung von Alumni in die ZaPF e.V. Satzung diskutiert. Alumni sollen mit $0$ \euro\ Beitrag aufgenommen werden und motiviert werden als Fördermitglieder den ZaPF e.V. ebenfalls finanziell zu unterstützen. \\
    Im zweiten Teil wurden verschiedene Varianten der Vernetzung und Informationsaustausch der Alumni diskutiert und Trendabstimmungen dazu durchgeführt.
