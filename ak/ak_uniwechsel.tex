% !TEX TS-program = pdflatex
% !TEX encoding = UTF-8 Unicode
% !TEX ROOT = ../main.tex

\section{AK Uniwechsel}

  \textbf{Protokoll vom:} 02.06.2018, %???
  Beginn: 15:00 Uhr,
  Ende: 16:00 Uhr \\
  \textbf{Redeleitung:} Fabs (TU Berlin) \\
  \textbf{Protokoll:} Marius Anger (TU München) \\
  \textbf{anwesende Fachschaften:} TU München, Würzburg, Marburg, Bochum, Augsburg, Ilmenau, Frankfurt am Main, Insbruck, Dresden, Halle, Karlsruhe, FUB, Potsdam, LMU, Köln
\vspace{-5mm}
  \subsection*{Informationen zum AK}
    \begin{itemize}
      \item \textbf{Ziel des AKs}: Resolution, \textbf{Folge-AK}: ja\footnote{\url{https://zapf.wiki/WiSe17_AK_Uni-Wechsel}}
      % \item \textbf{Zielgruppe}:
      \item \textbf{Voraussetzungen}: Protokoll aus Siegen
    \end{itemize}

  \subsection*{Protokoll}
      \paragraph{Zusammenfassung Siegen}
        Folge-AK aus Siegen: Einige Bundesländer haben das Problem, dass wenn man sich im selben Studiengang bewirbt, die Einstufung in ein Semester, in dem man schon war, nicht möglich ist.
        Dies wurde in Siegen ausführlich diskutiert. Man sollte sich zwischen den ZaPFen mit den Gesetztestexten befassen. Da jetzt viele neue Gesichter im AK sind, lässt sich der AK nicht wie angedacht weiterführen.

      \paragraph{Diskussion}
        \begin{itemize}
          \item Lucy (LMU): Manche Fächer werden an verschiedenen Unis mit einer anderen ECTS Punktzahl angerechnet. Dadurch kann ein Creditverlust stattfinden, der sich auf die Einstufung auswirkt.
          \item Paul (Köln): Sind Fachsemester überhaupt notwendig?
          \item Fabs: Krankenkassen sowie Versicherungen beruhen darauf. Dies soll hier nicht behandelt werden.
          \item Lucy (LMU): Rückstufung sollte möglich sein, aber Einstufung anhand der ECTS des Moduls.
          \item Paul (Köln): Eigentlich sollten die ECTS die modulunabhängige Einstufung sein.
          \item Ilmenau: Es soll nicht nur die Punktzahl stimmen, sondern auch die Fachinhalte. Deswegen ist eine reine ECTS Einstufung nicht sinnvoll.
          \item Lucy (LMU): Die Arbeitszeit wird leider nicht angerechnet.
          \item Fabs: Beim Umschreiben in eine neue Fachprüfungs- und Studienordnung (FPSO) können Punkte verloren gehen.
          \item Lucy (LMU): Das Problem ist, dass sich aufgrund von weniger ECTS die Studienzeit verlängert.
          \item Fabs: Ein weiterer Punkt ist, dass manche Universitäten fordern, dass 50\% der Leistungen, die zum Abschluss führen, an der eigenen Uni gemacht werden müssen. Die Fachsemesterzahl muss streng monoton steigend sein. Dies ist nachvollziehbar.
          \item Lucy (LMU): \textit{Vorschlag:} Nur die Abschlussarbeit muss an der Uni abgelegt werden.
          \item Ilmenau: Eine Einteilung in Master und Bachelor in diesem Punkt sollte man machen, da die Bachelorarbeit nicht so viel im Bachelor beiträgt, wie im Master.
          \item Paul (Köln): Bei einem Wechsel kurz vor Abschluss hängt es sehr stark vom Einzelfall ab.
          \item Fabs: Einzelfallentscheidungen sind kritisch. Darauf kann man sich nicht verlassen.
          \item Frankfurt: Es sollte eine freie Entscheidung beim Studierenden in Bezug auf die Uni liegen.
          \item Marius (TUM): Der Übergang von einer FPSO in die nächste muss geregelt sein.
          \item Lucy (LMU): Keine Universität nimmt einen Studenten mit vollen ECTS und Abschlussarbeit. Bitten wir sie doch darum als Vorgabe zu setzen, dass die Abschlussarbeit an der neuen Universität geschrieben werden muss.
        \end{itemize}
        \textbf{Zusammenfassung:}
        \begin{itemize}
          \item Einzelfallentcheidung wurde diskutiert als ein Ansatz für das Problem.
          \item Die Prozenthürde soll nur auf die Abschlussarbeit reduziert werden.
        \end{itemize}

      \paragraph{Einstufungsproblematik}
        Eine Immatrikulation kann nur in ein höheres Fachsemester durchgeführt werden. Es wird ein Toleranzsemester diskutiert.
        Eine Benachteiligung derjenigen, die nicht wechseln, muss vermieden werden. Nach der Höchstsemetserzahl an der Uni noch zu wechseln, sollte nicht möglich sein.
        Die Entscheidung über ein Toleranzsemester (als eine Art von EInzelfallentscheidung), kann hier, nach einer Immatrikulation, verwendet werden.
        Eine Einzelfallentscheidung wird immer nötig sein, da man nie alle ohne Benachteiligung abdenken kann. Die Entscheidungen ``Immatrikulation'' und ``Wie geht es in der Uni weiter?'' werden so getrennt.
        Es gibt bereits Bewerbungen mit Vorstellung. Bewerbung mit Begründung wollen wir aus Datenschutz- und Gleichstellungsgründen nicht.
        Ein Motivationschreiben, das häufig verlangt wird, fällt in die Kategorie einer Begründung. Wenn wir eine Bewerbung ohne Begründung fordern,
        müssen wir diskutieren, bis wann das geschehen soll. Weiterer Vorschlag: Bei einem Wechsel unter dem geforderten Soll, muss Stoff nachgehört werden.

      \paragraph{Forderungen}
        \begin{itemize}
          \item Der Student soll eine freie Wahl bei einer Anrechung haben; welche Module werden angerechnet? Wir sprechen uns explizt gegen eine ``Alles oder Nichts'' Regelung aus.
          \item Einzelfallentscheidungen sollen getrennt von der Zulassung gehandhabt werden.
          \item In Fällen, wo eine Rückstufung nicht möglich ist, soll in das nächsthöhere eingestuft werden, ohne eine Regelung durch ECTS oder ähnliches.
          \item In Fällen, wo eine Rückstufung möglich ist, soll nach ECTS eingestuft werden.
        \end{itemize}

    \subsubsection*{Resolution}
      \paragraph{Ziel}
        Resolution für eine freie Möglichkeit des Uniwechsels

      \paragraph{Problematik}
        \begin{itemize}
          \item Uniwechsel schwierig, da ECTS unterschiedlich bewertet werden. \\
             $\rightarrow$ Regelstudienzeit ist dadurch gefährdet.
          \item Immatrikulation in abgeschlossens Semester ist nicht möglich.
        \end{itemize}

      \paragraph{Adressaten}
        \begin{itemize}
          \item alle Hochschulen (Fachschaften, Prüfungsausschüsse)
          \item AStA Fulda (Quelle des Zitats)
          \item MeTaFa
          \item KMK
          \item Akkreditierungsagenturen
          \item Studentischer Akkreditierungspool
          \item BMBF
          \item Bildungspolitische Sprecher der Parteien
        \end{itemize}

      \paragraph{Resolutionsentwurf}
        Der Leitgedanke der Bologna-Reform ist es, die inter- und intranationale Mobilität der Studierenden zu fördern.
        Besonders im Vordergrund steht die "Förderung der Mobilität durch Überwindung der Hindernisse, die der Freizügigkeit in der Praxis im Wege stehen"
        \footnote{Der Europäische Hochschulraum – Gemeinsame Erklärung der Europäischen Bildungsminister, 19. Juni 1999, Bologna}. Dieses Ziel wird in Deutschland aus diversen Gründen nicht erreicht. \\

        Unter Anderem stellt die Thüringer Staatskanzlei schon 2011 fest \footnote{Claire Weiß, Tim Wiewiorra: Reform des Bologna-Prozesses als Voraussetzung für innovative und kreative Ausbildung in Europa.
        In: Europäisches Informations-Zentrum in der Thüringer Staatskanzlei: Reform des Bologna-Prozesses an deutschen Hochschulen als Voraussetzung für innovative und kreative Ausbildung in Europa. Erfurt 2011, S. 105}:
        \flqq Selbst ein einfacher Standortwechsel in Deutschland wird, auch auf Grund des Bildungsföderalismus, oft durch die engen Modulpläne der einzelnen Universitäten oder Hochschulen verhindert. \frqq \\

        Weiterhin bestehen an einigen Hochschulen formale Gründe (u.A.: Zugangs- und Zulassungssatzungen bzw. -ordnungen, Landeshochschulgesetze), die einen Hochschulwechsel, insbesondere innerhalb eines Studiums, verhindern.
        Es entsteht etwa ein Konflikt, wenn eine Rückstufung unmöglich ist \footnote{Es ist nicht möglich, mehrfach das selbe Fachsemester zu studieren.} und eine leistungsbasierte Einstufung \footnote{Die erbrachten Leistungspunkte nach ECTS bestimmen das Fachsemester.} erfolgen soll.
        Eine Einstufung in ein zu niedriges Fachsemester verhindert hier eine Immatrikulation und damit einen Hochschulwechsel. Dies steht in direktem Widerspruch zu den Leitgedanken des Bologna-Prozesses. \\

        Ein Hochschulwechsel innerhalb eines Studienganges verlängert nahezu zwingend die Studiendauer. Grund hierfür ist vor allem die unterschiedliche Bewertung der einzelnen Module sowie die zu begrüßende unterschiedliche Schwerpunktsetzung der Hochschulen. Dies stellt in Verbindung mit Studienhöchstdauern eine erhebliche Hürde in der Studierendenmobilität im Sinne der Bologna-Reform dar. \\

        Hinzu kommen oft Bedingungen zu Mindestleistungen an der Zieluniversität, etwa,  dass die Hälfte der Leistungen an der abschlussgebenden Hochschule erbracht werden muss.
        Dies verhindert bei einem Hochschulwechsel, bei der oft erforderlichen Anerkennung aller vorherigen Leistungen, einen Abschluss. \\

        Für eine vollständige Umsetzung der Bologna-Reform ist die Gewährleistung der Mobilität unabdingbar. Konkret bedeutet dies: 
        \begin{itemize}
          \item In Fällen, in denen eine Immatrikulation nicht möglich ist, da der Studierende nach bestehenden Regelungen in ein zu niedriges Fachsemester einzustufen wäre, ist eine Einstufung in das nächsthöhere Fachsemester vorzunehmen. Ist eine Rückstufung formal möglich, ist eine Einstufung nach ECTS vorzunehmen.
          \item Bestehen unglücklicherweise Höchststudiendauern oder andere Zwangsbedingungen, ist ein Hochschulwechsel als Begründung für Toleranzsemester oder andere Härtefallregelungen anzusehen.
          \item Es muss der Entscheidung des Studierenden obliegen, welche Leistungen zur Anerkennung der Zieluniversität zur Verfügung stehen. Ist dies formal nicht möglich und steht eine Regelung zur Mindestleistung an der Zieluniversität einem Abschluss im Weg, so ist eine Regelung zu finden, die den erfolgreichen Studienabschluss ermöglicht.
          \item Die Akkreditierungsagenturen sowie der Studentische Akkreditierungspool werden gebeten, bei der Akkreditierung darauf zu achten, Mobilität dadurch zu fördern \footnote{§12 (1) Musterrechtsverordnung gemäß §4 (1-4) Studienakkreditierungsstaatsvertrag, Beschluss der KMK vom 7.12.2017}, dass diese Mobilitätshürden abgebaut werden.
        \end{itemize}
