% !TEX TS-program = pdflatex
% !TEX encoding = UTF-8 Unicode
% !TEX root = ../main.tex
\newpage

\section{AK MeTaFa}

	\textbf{Protokoll vom:} 31.05.2018,
	Beginn: 10:30 Uhr,
	Ende: 11:00 Uhr \\
	\textbf{Redeleitung:} Björn (RWTH Aachen) \\
	\textbf{Protokoll:} Patrick (Uni Konstanz) \\
	\textbf{anwesende Fachschaften:} RWTH Aachen, Rheinische Friedrich-Wilhelms-Universität Bonn, Friedrich-Alexander-Universität Erlangen-Nürnberg, Universität Konstanz, Universität Rostock, Universität Siegen, Julius-Maximilians-Universität Würzburg

	\subsection*{Zusammenfassung}
		Die MeTaFa ist die Meta-Tagung der Fachschaften. Hier tauschen sich Vertreter verschiedener Bundesfachschaftentagungen über ihre Arbeit aus und besprechen Fachbereichsübergreifende Themen, wie beispielsweise Studiengebühren, Zulassungsbeschränkungen und ähnliches. Auch tauscht man sich über die Arbeitsweise der Tagungen aus (Konsenssystem, Aufbau AKs, etc.).
