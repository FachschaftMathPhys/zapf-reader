% !TEX TS-program = pdflatex
% !TEX encoding = UTF-8 Unicode
% !TEX ROOT = main.tex

\section{AK Weiterentwicklung des Studienführers}

  \textbf{Protokoll vom:} 02.06.2018, %???
  Beginn: 15:00 Uhr,
  Ende: 17:00 Uhr \\
  \textbf{Redeleitung:} Peter Steinmüller (KIT) \\
  \textbf{Protokoll:} Peter Steinmüller (KIT) \\
  \textbf{anwesende Fachschaften:} Universität Bielefeld, Brandenburgische Technische Universität Cottbus, Technische Universität Darmstadt, Technische Universität Dortmund, Westfälische Wilhelms-Universität Münster, Karlsruher Institut für Technologie, Julius-Maximilians-Universität Würzburg

  \subsection*{Informationen zum AK}
    \begin{itemize}
      \item \textbf{Ziel des AKs}: Positionspapier
      \item \textbf{Folge-AK}: nein
      \item \textbf{Zielgruppe}: Leute, die an der menschenfreundlichen und kommunikativen Weiterentwicklung dezentraler Raumstrukturen interessiert sind
      \item \textbf{Voraussetzungen}: keine
    \end{itemize}

  \subsection*{Protokoll}
    \paragraph{Bisheriger Stand}
      Der bisherige Stand und die letzten AKs werden besprochen. \\

      Die wichtigen Punkte:
      \begin{enumerate}
        \item Anforderungs-Dokument (siehe \url{https://zapf.wiki/WiSe17_AK_Weiterentwicklung_Studienführer})
        \item Drittfirma für Programmierung und Setup
        \item Sponsoring
        \item Verteilung an andere BuFaTas
      \end{enumerate}

    \paragraph{Überlegungen zum weiteren Vorgehen}
      Anforderungs-Dokument TODO:
      \begin{itemize}
        \item Abstract/Übersicht
        \item Layout
        \item Weiterführende Ziele
      \end{itemize}

      Idee ist, an eine Drittfirma mit einem Photoshop-Entwurf für einen (kostenfreien) Kostenvoranschlag heranzutreten.
      Daraus ergibt sich die Geldmenge, die man anschließend an Sponsoren braucht. \\

      Es wird festgestellt, dass im \textit{zapf.wiki} noch mehr Infos zu den LEUTEN zu WAS (wer macht was, wann?) sein könnten. Das soll in ``Kategorie:Weiterentwicklung\_Studienführer'' passieren. \\

      Vorschläge guter Webseiten (allgemein, um Unternehmen zu finden) und mehr allgemeine Übersicht wird ebenfalls in die Kategorie-Seite geschrieben.

  \subsection*{Zusammenfassung}
    Leute, die bis Würzburg was machen wollen, können das Dokument mit Anforderungen an die neue Webseite fertigstellen und in lesbare Form bringen.
    Zeitgleich soll eine Liste möglicher Sponsoren erstellt werden. In Würzburg soll dann überlegt werden, wie man an Drittfirmen herantreten kann.
