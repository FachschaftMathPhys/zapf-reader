% !TEX TS-program = pdflatex
% !TEX encoding = UTF-8 Unicode
% !TEX ROOT = ../main.tex

\section{AK Fachschaftsfreundschaften}

\textbf{Protokoll vom:} 02.06.2018,
Beginn: 19:30 Uhr,
Ende: 20:20 Uhr \\
\textbf{Redeleitung:} Tobias Löffler (Uni Düsseldorf) \\
\textbf{Protokoll:} Rebekka Baum (Uni Konstanz) \\
\textbf{anwesende Fachschaften:} Uni Erlangen-Nürnberg, Uni Düsseldorf, Uni Bonn, Uni Frankfurt, Uni Augsburg,
TU München, Uni Jena, Uni Freiburg, Uni Osnabrück, Uni Wuppertal, Uni Tübingen, Uni Chemnitz, Uni Münster, Uni Cottbus, Uni Saarland,
TU Kaiserslautern, Uni Würzburg, Uni Gießen, Uni Darmstadt, Uni Wien, Uni Halle-Wittenberg, Uni Konstanz, Uni Bochum

\subsection*{Informationen zum AK}
  \begin{itemize}
  	\item \textbf{Ziel des AKs}: Ziel des AKs ist es, die überregionale Vernetzung von ZaPFika untereinander zu fördern
  	\item \textbf{Folge-AK}: ja
    \item \textbf{Materialien}: Bilder mitbringen, falls vorhanden
  	\item \textbf{Zielgruppe}: alle ZaPFika
  	\item \textbf{Ablauf}: Vorstellung der geplanten Aktionen
  	\item \textbf{Voraussetzungen}: keine
  \end{itemize}

\subsection*{Protokoll}
  \subsubsection*{Einleitung}
    Grundziel des AK Fachschaftsfreundschaften ist es, die überregionale Vernetzung von ZaPFika untereinander zu fördern und zu Dokumentieren. In Stichpunkten heißt das:
    \begin{itemize}
      \item Finden eines neuen Verantwortlichen für die ZaPF-Couchsurfingliste
      \item Erneuerung der ZaPF-Couchsurfingliste
      \item Diskussion über ein ZaPF-SommerZelten
      \item Lustige Bilderstrecken, komische Vernetzungsgeschichten, Viele Bilder
    \end{itemize}

    Traditionell liegt dieser AK so, dass keine anderen Inhaltlichen AKs gleichzeitig oder danach sind. So hat jedes ZaPFikon die Möglichkeit sich zu vernetzen. Oder er liegt zumindest irgendwo am Ende des Tages, da es oft Klug ist, wenn nach diesem AK kein weiter AK liegt.

  \subsubsection{FB-Gruppe}
    Es wird Werbung für selbige gemacht. Diese wird im Anschluss der ZaPF in der Telegram-Gruppe wiederholt.

  \subsubsection{Telegram-Gruppen}
    Es wird Werbung für die Telegram-Gruppen gemacht und festgestellt, dass die QR-Codes in der Präsentation nicht stimmen.

  \subsubsection{ZaPF Couchingliste}
    Torsten Umlauf (Würzburg) stellt sich freundlicherweise zur Verfügung die Couch-Surfing-Liste weiter zuführen.

  \subsubsection{ZaPF Kartenspiel}
    Vicky berichtet über die bewegte Geschichte des Bestellprozesses:
    \begin{itemize}
      \item Anfanglicher Optimismus, dass es bis Weihnachten klappen kann
      \item Bestellzahlen die bei einigen Wenigen anfangen, dann aber schnell auf über 100 bei manchen Fachschaften steigen
      \item Danmit sind wir jetzt bei über 1200 Kartenspielen die schon vorbestellt sind
      \item Probleme zu bestellen beginnen damit, dass nicht klar ist, wohin geliefert werden soll.
      \item Dann hat der Hersteller probleme die Dateien zu Lesen. Sie werden überarbeitet.
      \item Ein neuer Bestelltermin wird gesucht und gefunden
      \item Der Hersteller hat immer noch Probleme, es wird nochmal Überarbeitet aber nun klappt es
      \item Außer das der Hersteller nun die Bestellung einfach mal Vergessen hat.
      \item Man stellt erschreckt fest, dass 20-25 Tage mehr als ein Monat entsprechen, wenn damit Werktage gemeint sind.
      \item Es sind Werktage gemeint
      \item Neuer Liefertermin ist daher nicht ende Mai sondern Mitte/Ende Juni
      \item Vicky freut sich auf eine Rundreise um die Kartenspiele auszuliefern und auf Besuche von Selbstabholern
    \end{itemize}

  \subsubsection{ZaPF-Sommer-Zelten}
    Es gibt einiges Hin und Her beim Thema. Es wird von verschiedenen Orten und möglichen Organisatoren geredet. Nach einigem hin und her erklärt sich Karola (Potzdam) bereit die Organisation zu übernehmen. Allein schon weil sie ja in bälde am Südpol ist.

  \subsubsection{Bierquellenwanderweg}
    Der Termin für die zapfige Bierquellenwanderung in Franken (Nähe Trockau) steht fest: Sie findet vom 13.-15. Juli 2018 (Freitag anreisen, Samstag wandern, Sonntag abreisen) statt!
    Es gibt schon 16 Anmeldungen und wird bestimmt wieder sehr toll.

  \subsubsection{Fachschaftsveranstaltungen (Methode ändern?)}
    Es wird über Fachschaftenveranstaltungen geredet. Es wird auf den Platz im Wiki hingewiesen. Und darauf, dass dort meist nur kurz nach der ZaPF etwas Passiert.
    Es wird darauf hingewiesen, dass sowohl die ZaPF-List als auch die Facebook und die Telegramgruppe ein guter Ort zum Bewerben von Fachschaftsveranstaltungen sind.
    Ein gemeinsammer Kalender wird angesprochen, aber es gibt keinen der sich darum kümmert (?)

  \subsubsection{ZaPF FS-Freundschaften-Treffen (Bilder schauen)}
    Es werden lustige Bilder gezeigt. Nicht gerade Haufenweise, aber immerhin. Das ist... sehr lustig
    Und dann wird noch das Video für Margret gezeigt. Auch kommt während des AKs eine Rückmeldung, dass das Video gut angekommen ist.
