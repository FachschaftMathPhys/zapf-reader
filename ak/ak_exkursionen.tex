% !TEX TS-program = pdflatex
% !TEX encoding = UTF-8 Unicode
% !TEX root = ../main.tex

\section{AK Exkursionen}

  \textbf{Protokoll vom:} 31.05.2018
  Beginn: 08:00 Uhr,
  Ende: 10:00 Uhr \\
  \textbf{Redeleitung:} Marie-Rachel (RWTH Aachen) \\
  \textbf{Protokoll:} Marius Anger (TU München) \\
  \textbf{anwesende Fachschaften:} Uni Rostock, Uni Greifswald, Uni Wien, Uni Innsbruck, Uni Potsdam, TU Freiberg, Uni Osnabrück, Uni Konstanz, LMU München, TU München, Uni Halle-Wittenberg, Uni Siegen, TU Darmstadt, Uni Wuppertal, BTU Cottbus, Uni Göttingen, Uni Bonn, Uni Münster, Uni Frankfurt am Main, Uni Graz, TU Ilmenau, KIT, Uni d. Saarlandes Uni Mainz, Uni Bochum, Uni Dresden, Uni Würzburg, Uni Freiburg, RWTH Aachen

  \subsection*{Informationen zum AK}
    \begin{itemize}
    	\item \textbf{Ziel des AKs}: Erfahrungsaustausch und Impulsanstoss
    	\item \textbf{Folge-AK}: nein
      \item \textbf{Materialien}\footnote{\url{https://zapf.wiki/images/5/56/Exkursion_2008.pdf} \& \url{https://zapf.wiki/images/5/5a/Programmheft_Mai_2018.pdf}}
    	\item \textbf{Zielgruppe}: alle ZaPFika, die Erfahrung oder Interesse an der Organisation solcher Veranstaltungen haben.
    	\item \textbf{Ablauf}: Erfahrungsaustausch, anschließend Erstellen eines Leitfadens für solche Events
    	\item \textbf{Voraussetzungen}: keine
    \end{itemize}

  \subsection*{Protokoll}
    \paragraph{Erfahrungsaustausch}
      \begin{itemize}
        \item Rostock: ein Tag Kanu Exkursion
        \item Greifswald/Rostock: ein Tag DESY
        \item Uni Innsbruck:
        2x 3 Tage am CERN je 36 Personen (Sponsoring durch Industriellen Vereinigung, Universität und Österreichische Hochschülerschaft)
        \item Uni Wien:
        Erstitutorenschulung 4 Tage
        \item Uni Potsdam: CERN, Semesteranfangsfahrten ausschließlich Züge benutzen. Wer nicht da ist, kommt nicht mit. Zugbindung.
        \item LMU München:
        Erstifahrt und fachschaftsinternes Seminar, keine Physikexkursionen, Geophysik und Meteorologie, verleihbare Busse; meist 1 tägige Exkursionen
        \item TU München:
        Lehrstühle veranstalten Exkursionen, Bachelor/Masterübergreifend 4-5 Tage (CERN/KATRIN/Grand Sasso)
        \item KIT:
        Gar keine Fachschaftsexkursionen; nur Fachbezogene Exkursionen in der Geophysik
        \item TU Illmenau:
        Uniintern zu den Laboren; Fachschaftziel: eine Exkursion pro Jahr Doodle an Studenten mit der Ideen sammlung.
        \item Uni Saarland:
        Ersti Exkursion Wochenende. Gerne mehr Exkursionen. Hin und wieder Laborführungen
        \item Uni Münster:
        Nur eine ESWE gerne mehr; Generell: Über Professoren und Institutsnahe Kontakte kann mehr erreicht werden bei den Besuchszielen.
        \item Uni Graz:
        Jährlich Seminar (Arbeitsgruppen der Studierendeschaft) externe Exkursion für alle Physiker in Kaderasch und am CERN, ansonsten auch noch Osteuropa Tschernobyl
        \item TU Berlin:
        Keine Exkursionen, Praktika können statt Versuchen Exkursionen Durchführen
        \item Uni Wuppertal:
        Vor mehreren Jahren ans DLR und letztes Jahr ans DESY. Problem: Studenten erreichen da Vorlesungszeit und Übungen. Kosten werden vom ESR getragen. Dekanat organisierte einen Bus.
        \item Cottbus: Problem: Studenten tragen großen Kostenanteil, da kleine Uni (40 Physiker), eintägige Ausflüge klappen gut
        \item Uni Göttingen:
        Vorlesungsgebundene Exkursionen, CERN 3-4 Tage, Physiker im Freien: Stausee und GRILLEN
        \item Uni Osnabrück:
        Gerne zweitägige Exkursionen, eintägige Exkursionen mit 50 Teilnehmern kommen gut an
        \item Uni Freiburg:
        gute Lage: nicht weit vom CERN, 2 Tage CERN, DESY alle 2 Jahre, DESY zahlt Geld (weil Professur von Uni DESY)
        \item TU Dresden:
        mehrere Exkursionen pro Jahr. in der vorlesungsfreien Zeit: 1 Woche im März ans CERN, Härtefallregelung, um sozial Benachteiligte mitzunehmen; Mit einem Bus kann man nach Annecy fahren. HZDR. Schacht Konrad (von Institut). Laborführungen. Manchmal DESY, Global Foundries. Viele kleine eintägige Exkursionen (Wandern, Fahrrad, Erstie-Veranstaltungen), Studierende werden ermuntert, selbst Exkursionen zu organisieren, Fachschaftsrat bietet Kontakte/Expertise/finanzielle Unterstützung
        (bisher sehr selten angenommen)
        \item Uni Würzburg:
        kleinere Uni, innerhalb des Semesters schwer Studenten zu motivieren. Mehr Werbung. Sommerfest: eintägige Veranstaltung, fällt riesengroß aus. Exkursionen vom CERN vom Lehrstuhl aus
        \item Uni Bonn:
        Exkursion für Vorlesungsteilnehmer, Efelsberg-Teleskop, 20 Studis Erstis, eintäger Ausflug
        \item Uni Siegen:
        ESA Fahrt, für Professoren/Doktoranden innerhalb des Semesters schwierig Teilnehmer zu finden. Es besteht die Möglichkeit, am anderen Campus ein Teleskop zu besuchen; Für eine handvoll von Studenten, Ausflüge zum CERN möglich. Es gab in NRW Studiengebühren: vom Erstatzgeld wird Geld von der Uni gegeben. In Planung: Wendelsteinreaktor 4 Tage von Fachschaftlern organisiert
        \item Darmstadt:
        keine Exkursionen, ESWE mit einer Übernachtung
        \item Freiberg:
        Blockveranstaltung zum DESY an credit points (3) gebunden, 10 Studenten, Kosten werden vom Lehrstuhl getragen und Wissen via Klausur überprüft
        \item Frankfurt am Main:
        keine Exkursionen von der FS aber Professoren sind da gut dabei
        \item Halle Wittenberg:
        diverse Exkursionen DESY Berlin, Busfahrer zahlen der Bus wurde gestellt. Bachelor/Masterübergreifend. Exkursionen zu diversen Institutionen; Pflichtveranstaltung für Medizinphysiker Strahlenphysik Medizin-physikalische Anlage, zB Siemens MRT, PET, CT Idee auch extra anbieten. Idee sowohl angewandte als auch physikalische?
      \end{itemize}

    \paragraph{Tipps}
      \begin{itemize}
        \item Durchführung:
        Erfahrungsgemäß Treffzeit 15 Minuten vor eigentlichem Exkursionsbeginn in Infomail schreiben.
        \item Sponsoren:
        Innsbruck: Sponsoring von Wirtschaftsverbänden
        Sponsoren, die die ZaPF unterstützen, sind gute Ansprechpartner
        \item Motivieren:
        Motivieren der Studenten in der Vorlesungszeit an Exkursionen teilzunehmen durch cooles Programm (ein langes Wochendende bietet sich an)
        \item Bei der Suche nach Kontaktpersonen:
        Über Professoren und institutsnahe Kontakte kann mehr erreicht werden bei den Besuchszielen.
        \item Bei zu wenigen Teilnehmern:
        Wenn die Busse nicht voll werden: Das CERN ist auch für andere Naturwissenschaftler und Ingenieure gut
        \item Bei Geldsorgen:
        Es gibt manchmal die Möglichkeit von der Uni Geld für Exkursionen zu bekommen (bspw. über Studiengebühren)
        \item Bei Sponsoring/Telefonaquisen:
        Den Leuten immer weiß machen, dass du sie toll findest! Trettet überzeugt auf; Glaubt an euch!
        \item Bei der Abrechnung:
        Klärt vorher ab, was ihr auf die Rechungen schreiben könnt, zB schreibt keinen Alkohol auf Sponsorenrechnungen
        \item Bei der Planung/Vorbereitung:
        Sucht euch Mitstreiter bei der Durchführung (eventuell auch nicht in der Fachschaft)
      \end{itemize}

    \paragraph{Ansätze zu einem Leitfaden für Exkursionen}
      \begin{outline}
        \1 Wohin?
          \2 Bei Exkursion mit mehreren Etappen genau festlegen, wo man sich wie lange aufhält
          \2 Größere Ausgaben für die Planung müssen evtl. vom AStA genehmigt werden
          \2 Befragung der Studenten zu Vorschlägen
          \2 Mystery-Tour: nicht bekanntgeben, wohin die Fahrt gehen wird (kann gut und schlecht ankommen, eher für große Unis geeignet)
        \1 Zeitpunkt:
          \2 Einige (8/20) Unis haben eine dedizierte Exkursionswoche, aber viele andere wiederum nicht und auch keine Brückentage
          \2 Januar und Februar eignen sich besonders, da dann der LHC im Wintershutdown ist und man auch die Cavernen mit den Detektoren besuchen kann. (2019 durchgehend im Shutdown für Major Maintenance, daher immer voll besichtigbar)
          \2 Am Anfang des Semester oder in den Ferien oder eventuell auch langes Wochenende
          \2 Profs anschreiben und im Vorfeld darum bitten, Klausuren in einer Woche zu vermeiden (klappt teilweise)
          \2 Mit Profs reden, um Befreiung von Pflichtveranstaltungen (Tutoriumsanwesenheit nötig für Prüfungsteilnahme etc.) zu erreichen
        \1 Zielgruppe:
          \2 Andere Unis (national und international) oder auch weitere Studiengänge
          \2 Es muss nicht immer riesig sein
          \2 Anzahl sollte relativ genau feststehen
        \1 Unterkünfte:
          \2 Hostels, Jugendherbergen, Turnhallen, Selbstversorgerhäuser
          \2 Beachtet die Kosten!
          \2 Tipps bei dortigen Fachschaften holen
        \1 Transport:
          \2 Bus, Bahn
          \2 Frühbucher und Gruppenrabatte beachten (6 Monate vorher bei der Deutschen Bahn; direkt nach Fahrplanwechsel sind Billigkontingente frei)
        \1 Programm:
          \2 Hefte sind immer gut, aber nicht notwendig
          \2 Physikalisches Programm: Institut (Hauptziel), andere nahe Institute, aber auch zusätzliche Vortäge (naheliegende Unis?)
          \2 Außerpyhsikalisches Programm: Abendessen, Ausflüge etc.
          \2 Etwas freie Zeit lassen / nicht durchtakten
          \2 Orientiert euch eventuell an einem ZaPF-Programm oder offiziellen Besucherangeboten
          \2 Kooperiert mit Fachschaften vor Ort (bspw. für Ortskunde)
        \1 Werbung:
          \2 Profs sind die beste Werbung (ECTS \dots)
          \2 auf Übungsblätter drucken
          \2 Studiverteiler, Monatsmail/Newsletter
          \2 Vorlesungswerbung
          \2 Standbildschirme von CIP-Pools
          \2 Klopapier, Innenseite von Toilettentüren
          \2 Social Media! und Chats (Telegram Channel)!
          \2 Website
          \2 Plakate und Flyer (auf die Rückseite kariertes Papier drucken, damit Studis den Flyer als Schmierzettel verwenden und herumtragen)
          \2 Semestersprecher informieren
          \2 Mund-zu-Mund Propaganda
        \1 Finanzierung:
          \2 Fakultät
          \2 persönliche Kontakte zum Institut
          \2 Lehrstühle
          \2 Nahestehende Intitute (Max-Planck etc.)
          \2 AStA/StuVe (es wird eventuell nur Mitgliedsbeitrag wieder ausgezahlt)
          \2 Studienzuschüsse
          \2 "Qualitätsverbesserungsmittel"
          \2 Externe Sponsoren (physiknahe Firmen, orientiert euch evtl an der ZaPF)
          \2 Alumni-/Fördervereine
          \2 KEINE Gratis-Ausflüge! Beiträge zwischen 10€-250€
          \2 evtl. Stiftungen (wenn Organisator auch Stipendiat)
          \2 Sponsorenverträge beachten, Bedingungen
        \1 Durchführung:
          \2 Kühlen Kopf bewahren
          \2 Dokumentation für künftige Orga
          \2 Nachher Abrechung
          \2 Feedbackbögen an die Teilnehmer
      \end{outline}
