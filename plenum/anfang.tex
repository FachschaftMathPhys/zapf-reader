
Begrüßungsworte durch den Dekan Prof. Dr. Schulz-Coulon, der unser ewiges Frühstück mag. Er lobt die Fachschaften als Ansprechpartner für's Dekanat, aber insbesondere auch für die Erstis und die Partys. Er wünscht uns viel Freude, guten Austausch der Erfahrung, viel Spaß und gutes Gelingen.

\section{Begrüßung der Teilnehmer durch die Orga}
  \subsection{Einführung in die Regeln}
    \begin{itemize}
      \item kein Strom und Internet im Plenum
      \item Steckplätze vor dem Hörsaal (nur 75 leider)
      \item im Erdgeschoss bitte NICHTS anschließen!
      \item Internet über Eduroam (als einziges gesichert)
      \item sonst Heidelbergforyou
      \item W-Lan im Hörsaal sparsam nutzen, sonst Rückkopplung mit Mikros
    \end{itemize}

  \subsection{Schlafplätze}
    \begin{itemize}
      \item Jugendherberge:
      \item Da schlafen auch noch andere Menschen, bitte Rücksicht nehmen!
      \item Rezeption nur zwischen
      \item Das  8 und 23 Uhr besetzt, sonst Schlüssel mitnehmen
      \item bitte Frühstückst dort! 7-10 Uhr
      \item 21-9 Uhr Ansprechpartner in Notunterkünften
      \item da gibt es keine Duschen, deshalb bitte im Tagungsbüro Schlüssel für Duschen abholen
    \end{itemize}

  \subsection{Essen}
    \begin{itemize}
      \item Ewiges Frühstück im goldenen Käfig
      \item Da gibt es dann auch warmes Essen
      \item Coupon für Essen in der Mensa im Tagungsbüro
    \end{itemize}

  \subsection{Exkursion}
    \begin{itemize}
    \item Freitagmorgen muss das Unigebäude für normalen Unibetrieb zur Verfügung stehen, deshalb bitte zu Exkursionen gehen
    \item Friedrich-Ebert Gedenkstätte fällt aus
    \item EMBL ist 10 min später
    \item Volume Graphics und Wanderung hat noch Plätze frei
    \end{itemize}

  \subsection{Mörderspiel}
    \begin{itemize}
      \item Punkte außen auf die Plastikbadge kleben (auf jede Seite der Badge kleben)
      \item Mord durch Gegenstand übergeben ohne dass es jemand sieht
      \item Online eintragen
      \item Mordfreie Zonen: Unterkünfte und Toiletten Bitte respektieren
      \item Auf Tagungsgelände und Wegen sind Morde möglich
      \item Zwei unabhängige Runden
      \item Nach Mord erhält man Punkte von Ermordetem
    \end{itemize}

  \subsection{Samstag}
    \begin{itemize}
      \item Samstag Party
      \item 21 Uhr Treffen auf dem Exerzierplatz für Heidelberger Schloßbeleuchtung
    \end{itemize}

  Alles weitere Wichtige kommt über Mailinglisten. \\
  Freitagabend ist Kollogium im Hörsaalzentrum \textit{INF 308 HS1}. \\ \\

  Offizielles Anzapfen der \textit{\#ZaPFinHD}

  \begin{success}{Annahme der Plenumsleitung per Akklamation}
    Bekka, Benni, Maik, Thomi, Kathii
  \end{success}

  \begin{success}{Annahme der Protokollführenden per Akklamation}
    Daniela und Ludi
  \end{success}

  \section{Tagesordnung}
    \begin{enumerate}
      \item Feststellen der Beschlussfähigkeit
      \item Wahl der Vertrauenspersonen
      \item Vorstellung der Arbeitskreise
      \item Berichte
    \end{enumerate}

\section{Anwesenheit aller Fachschaften}
  Die anwesenden Fachschaften sind in \autoref{chap:anwesend} aufgelistet. \\
  Zu Beginn des Plenums sind 43 Fachschaften anwesend und somit ist das Plenum beschlussfähig.

\section{Wahl der Vertrauenspersonen}

  Als Vertrauenspersonen der ausrichtenden Fachschaft stehen Irina Meister und Thomas Rudzki zur Verfügung. \\

  Zur Wahl als Vertrauensperson lassen sich zusätzlich aufstellen:
  \begin{itemize}
    \item Christian Birk (Marburg)
    \item Christian Hoffmann (Oldenburg)
    \item Elli (TU Berlin)
    \item Karola Schulz (Potsdam)
    \item Lina Vandre (Innsbruck, ehemals Siegen)
    \item Lisa Dietrich (Erlangen)
    \item Maik Rodenbeck (Bielefeld)
    \item Mandy Hannemann (Potsdam)
    \item Peter Steinmüller (KIT)
  \end{itemize}

  \begin{info}{Vertrauenspersonen}
    \textbf{Wahlleitung}: Andy (Würzburg), Jenny (Berlin)
    \tcblower
    \textbf{gewählte Vertrauenspersonen (alphabetisch)}:
    \begin{itemize}
      \item Christian Birk (Marburg)
      \item Christian Hoffmann (Kruemel, Oldenburg)
      \item Elli (TU Berlin)
      \item Karola Schulz (Potsdam)
      \item Lina Vandre (Innsbruck, ehemals Siegen)
      \item Mandy Hannemann (Potsdam)
    \end{itemize}
  \end{info}
  Die gewählten Vertrauenspersonen sollen sich bitte bei der Orga bezüglich Mailaddressen melden.

\section{Selbstberichte}
  Selbstberichte bitte an Tobi schicken. \\
  Wer Interesse am Bewerten hat, bei Tobi melden. \\
  Fachschafts-Freundschaften AK bittet um Bilder und Anekdoten.

\section{Vorstellung der Arbeitskreise}
  Ausgedruckte Arbeitskreise hängen an den Seiten. Es gibt neun Leseblöcke von oben nach unten. Jeweils fünf Minuten pro Lesenblock und dazu jeweils fünf Minuten für Fragen. \\
  Zur Planung der AKs wird das Interesse einzeln abgefragt.

  \subsection{Ergänzungen}
    \begin{description}
      \item[AK CHE-Ranking] Neues Ranking anschauen; Wie soll das überhaupt weiter gehen? $\rightarrow$ keine Vorkenntnisse benötigt
      \item[AK Fortgeschrittenenpraktikum] Fortsetzung zum letzen Mal, Anforderungen vom letzten Mal sollen wieder diskutiert werden, Anforderungen an Betreuer sollen formuliert werden.
      \item[AK Polizei in Bayern] Nur eine Handreichung soll entstehen.
    \end{description}

  \subsection{Fragen zu den Arbeitskreisen}
    \begin{itemize}
      \item -204- Was für Vorwissen ist für Wissenschaftsrecht nötig? $\rightarrow$ Eigentlich keines.
      \item -168- Parlamentarische Arbeit: Worum geht es? Steht nichts im Wiki? $\rightarrow$ -310-: Hauptsächlich ein Berichts-AK; Was ist da in Berlin passiert? Und was passiert da sonst so? Evtl. mit TV-Stud zusammen.
      \item -294- Geht es nur um AfD oder auch um Identitäre Bewegungen? Vorbereitung für AK war kleine Anfragen anschauen, da waren auch andere Parteien dabei.
      \item -183- Was war die Anfrage der AfD? Was ist der Brückenschlag zur Hochschulpolitik? $\rightarrow$ -310- Anfrage der AfD war alle Leute, die was mit dem AStA zu tun hatten rausfinden (letzten 10 Jahre mit Name und Anschrift).
      \item -261- Ist da auch Platz für Erfahrungen aus Österreich? Und kann da vielleicht auch eine Aktion draus werden? $\rightarrow$ -310- Sehr gerne! Es können auch gerne noch andere Punkte zu diesem Thema eingebracht werden.
    \end{itemize}

    \begin{info}{}
      Die Fachschaft der Uni Gießen kommt um \textbf{19:16} verspätet dazu.
    \end{info}

    \begin{itemize}
      \item -311- Gibt es in NRW wohl auch. Da könnte etwas entstehen. $\rightarrow$ -215- Dieser AK soll wirklich keinen Brückenschlag versuchen.
      \item Vertrauenspersonen: Warum ist eigentlich die Anzahl beschränkt? Warum dürfen sie sich nicht untereinander austauschen? Es soll diskutiert werden, ob da vielleicht Änderungen wünschenswert sind.
      \item Ak Lehramt: Dieses Mal soll über Quereinsteiger gesprochen werden. Was versteht man darunter? Wie finden wir das? Was sollte dabei berücksichtigt werden?
      \item AK Lehramt: Pöbeln als Bier-AK soll bitte getrennt davon sein und dazu dienen mal über alles Hässliche im Lehramt und darum herum zu sprechen.
      \item -107- Worum geht es bei der BaMa-Umfrage? $\rightarrow$ -279- Es soll eine weitere Durchführung der BaMa Umfrage (Wie sehen Physikstudiengänge in Deutschland aus?) gemacht werden. Im AK geht es um die Details der neuen Umfrage.
      \item -163- Was soll eigentlich herausgefunden werden? $\rightarrow$ -279- Diese Umfrage soll alle 4 Jahre wiederholt werden, um die Entwicklung der Studiengänge und der Zurfriedenheit der Studierenden in diesen Studiengängen zu beobachten. Dazu gab es bereits einen Beschluss im Sommersemester 2017\footnote{\url{https://zapf.wiki/images/7/72/BaMa_SoSe17.pdf}}.
      \item -214- Soll das nur auf deutsche Universitäten eingeschränkt sein? $\rightarrow$ -279- Nein, alle Fachschaften, die hier anwesend sind oder sein können, sollen angeschrieben werden.
    \end{itemize}

    \begin{info}{}
      \textbf{Orga-Ankündigung}: Bier, Wein und Mate auf Strichliste. Bitte eintragen!
    \end{info}

    \begin{info}{}
      Die Fachschaft der Uni Duisburg-Essen kommt um \textbf{19:41} verspätet dazu.
    \end{info}

    \begin{itemize}
      \item -183- Fragen zu Friedenangelegenheiten: Köln ist zum Beantworten noch nicht anwesend. Mutmaßungen: Nachfolge zum ICAN AK vom letzten Mal.
        $\rightarrow$ -266- Nachtrag: Es gab einen Vortrag in Siegen, bei dem ein Mensch von ICAN (Internationale Vereinigung zur Abschaffung von Atomwaffen) da war. Dazu sollte es eine Resolution geben, die kontrovers diskutiert wurde. Nun geht es um die Fortsetzung.
      \item -286- Was ist das Ziel des AK Rückläufige Studierendenzahlen? $\rightarrow$ -265- Ideen austauschen, wie man kleine Studiengänge retten kann.
      \item -103- Studierendenzahlen wurden schon oft im Austausch thematisiert, warum AK? $\rightarrow$ Bisher wurde gesagt, dass es den Rahmen des Austausch AKs sprengt.
      \item -112- Was ist das Ziel des AK Frauen in akademischen Karrieren? $\rightarrow$ -232- Diesen AK gab es in Berlin, wo viel erarbeitet wurde. Damit soll weiter gearbeitet werden. Es wird aber noch eine AK-Leitung gesucht.
      \item -302- hat in Berlin das Protokoll geschrieben. Sie sieht nicht direkt, was weiter getan werden sollte. Ist in Würzburg vielleicht sinnvoller. \\
        Für Würzburg gibt es eine AK-Leitung. Es wird nach Interesse für einen Bier-AK hier gefragt und der AK aus der offiziellen Liste rausgenommen.
      \item -294- Wird für Bier-AKs / Mindestgrößen-AKs ein Foto von den Verantwortlichen gemacht? Sinnvoll, um sie wiederzufinden. $\rightarrow$ Orga: Alle Zuständigen für Bier-AKs sollen der Orga ein Bild und Kontaktdaten schicken.
      \item -265- Wird AK Aktualisierung des Studienführers leiten. Bitte je eine Person pro Fachschaft am AK oder im Backup-AK teilnehmen. Modulhandbuch mitbringen, Inhalt in Vorlage einfügen. \\
        -226- Auch für Österreicher interessant? $\rightarrow$ -265- Ja.
      \item -140- Wissenschaftskommunikation. Worum gehts es im AK? $\rightarrow$ -232- Auch als Outreach bekannt, Wissenschaft wird nach außen getragen, verständlich erklärt. Im AK geht es darum, welche Rolle das im Studium spielen soll.
      \item -248- Beschlussdatenbank: Was ist das? $\rightarrow$ -265- Aktuell steht alles irgendwo verstreut im Wiki \footnote{\url{https://zapf.wiki/Hauptseite}}, eine Strukturierung wäre hilfreich.
      \item -133- Ak Interdisziplinarität: Was soll genau besprochen werden? Die 6 CP Regel? $\rightarrow$ -128- In der ursprünglichen Form nicht umgesetzt worden, aber die Idee war, dass alles auf 6 CP gekürzt werden sollte und es geht um die Probleme damit.
    \end{itemize}

    \begin{info}{Verwirrung um Exkursionswahl}
      Eine Infomail zur Erklärung wird von der Orga verschickt.
    \end{info}

  \subsection{weitere Arbeitskreise}
    \begin{description}
      \item[AK organizing an international welcome (Lina, Innsbruck)]
        Organisation von Willkommensveranstaltungen für international Studierende; Handreichung soll vorgestellt und weiter erarbeitet werden; AK wird auf Englisch gehalten werden.
      \item[AK Akkreditierung I (Phillipp, Wuppertal)]
        Interne Richtlinien der ZaPF für studentische Gutachter werden aktualisiert. Ziel ist die überarbeitete Fassung der ZaPF-internen Richtlinien als Positionspapier.
      \item[AK Akkreditierung II (Pillipp, Wuppertal)]
        Ziel ist die Neuordnung des Akkreditierungswesens, die zum 1.1.2018 angelaufen ist. Die aktuellen Verfahren laufen noch nach der alten Ordnung, jetzt geht es darum, dass wir uns zur neuen Ordnung für neue Verfahren positionieren können.
        Auch für studentische Gutachter, die bereits mit dem alten Verfahren vertraut sind und die aktuellen Änderungen wissen bzw. zusammengefasst als Handreichung haben möchten.
      \item[WS Akkreditierung (Daniela)]
        Einführung in das Thema Akkreditierung (nach neuem Recht, Änderungen werden erklärt), dieses mal für Deutschland, Österreich und die Schweiz!
        Für alle, die in den Akkreditierungspool möchten, eine Akkreditierung an eurer Hochschule ansteht, ... Empfehlung für den Workshop. Es wird eine gute Einführung.
        \begin{info}{}
          Ein Posten als studentischer Vertreter in der ASIIN (Verein zur Akkreditierung) wird frei, wer Interesse hat soll sich bitte melden.
        \end{info}
      \item[AK Open-Science (Merten, Göttingen \& jDPG)]
        Allgemeines Thema, bei dem es um das Zugänglichmachen von wissenschaftlichen Ergebnissen und Aspekten der Allgemeinheit geht. Zum Thema Open-Access gibt es eine Stellungnahme der ZaPF\footnote{\url{https://zapf.wiki/images/8/8c/Reso_WiSe12_OpenAccess.pdf}}, Thema Open-Data und weitere wurden auf der ZaPF bisher nicht diskutiert.
        Ein Positionspapier ist nicht angedacht.
    \end{description}

    \begin{info}{}
      Die Fachschaft der Uni Augsburg kommt um \textbf{20:16} verspätet dazu.
    \end{info}

    \begin{info}{}
      Die neuen Protokollanten (-184-, ) wurden bestätigt und übernehmen das Schreiben des Protokolls.
    \end{info}

  \subsection{weitere Fragen}
    \begin{itemize}
      \item -254- Werden alle Thema, also Open-Access, Open-Data, Open-Science, ... behandelt? $\rightarrow$ Merten: Ja
      \item -279- Anmerkungen zu AK "FHs auf der ZaPF": FolgeAK war gewünscht in Siegen. Frage: Ist es sinnvoll, mehr Fachschaften von FHs einzuladen?
        Dies wurde getan und soll nun diskutiert werden. Umfrage der Redeleitung ergab: Es gibt kein Interesse an diesem AK, der AK wird daher gestrichen.
      \item -107- Was ist die MeTaFa? $\rightarrow$ -232- Meta-Tagungen der Fachschaften, Austausch der BuFaTas, in den letzten MeTaFas
        waren jeweils Vertreter der ZaPF (StaPF) anwesend. Im AK werden Themen der MeTaFa vorgestellt und besprochen.
      \item -147- zum AK Uniwechsel: Gibt es Probleme beim Uniwechsel? Sind (neue) Problemfälle bekannt? Gibt es Bedarf für den AK oder über das Thema zu sprechen?
        Wunsch der AK-Leitung (falls der AK stattfindet), dass die zukünftig Anwesenden Rücksprache mit ihren Fachschaften halten, um ggf. Vorfälle gesammelt einbringen zu können. \\
        -265- Es sind in Karlsruhe viele Fälle von Leuten bekannt, die dorthin wechseln, vorallem Anrechnung von Mathemodulen ist ein Problem.
    \end{itemize}

    \textbf{Leseblock VIII} \\

    \begin{itemize}
      \item -112- Soll im AK Raum- und Bibliotheksgestaltung nochmals offen disktutiert werden oder soll nur über das Ergebnis gesprochen werden?
      \item -254- Frage zum AK Hochschuldidaktik: Vertretung der Fachschaft Köln ist auf der DPG Frühjahrstagung eingeladen, was woll dort passieren? Bleibt unklar.
      \item -261- Frage zum AK Raum und Bibliiotheksgestaltung: TU Wien bekommt einen Zubau zur Bibliothek, zu welchem evtl. Mitspracherecht besteht. Gibt es im AK Hinweise/Erfahrungen dazu, wie derartige Mitsprache gestaltet werden kann?
      \item -189- Vorschlag: Fragen sammeln, dann kann Köln diese noch rechtzeitig beantworten.
      \item -267- Frage zum AK Prüfungsanmeldungen: Was soll dort geschehen? $\rightarrow$ -277- Es gibt einen vorläufigen Beschluss\footnote{\url{https://fachschaften.rwth-aachen.de/etherpad/p/}}, dem sich möglichst viele BuFaTas anschließen sollten.
        In dem Pad wird das Thema bereits ausführlich mit Hintergrundinformationen beschrieben. Das Thema ist von der KaWuM aus verwaist, d.h. die ZaPF würde das Thema wiederbeleben. Wird das Thema aufgegriffen, könnte/sollte daraus eine Resolution entstehen.
      \item -320- Vorschlag zur Reihenfolge der AKs, aber in Abwesenheit von Köln schwierig.
    \end{itemize}

    \begin{info}{}
      Die Fachschaft der Uni Bochum kommt um \textbf{20:52} verspätet dazu.
    \end{info}

    \textbf{Leseblock IX} \\

    \begin{description}
      \item[AK Subsidarität]
      \begin{itemize}
        \item -119- Was ist Subsidiarität? $\rightarrow$ -248- Dinge sollen an der niedrigstmöglichen Ebene geregelt werden. ("Was Wo Wie So!")
      \end{itemize}
    \end{description}

    \begin{info}{}
      Mailinglisten für Vertrauenspersonen wurden eingerichtet.
    \end{info}

    \textbf{Information zum AK MeTaFa}: Die AK Leitung hat beschlossen, den AK in Form eines Bier-AKs durchzuführen. \\
      Umfrage der Redeleitung: Wer den AK ebenfalls lieber als Bier-AK durchführen möchte, möge sich bitte melden.

    \subsection{Informationen zu den einzelnen Workshops}
      \begin{itemize}
        \item WS ZaPF-IT: Eigener Laptop mitbringen!
        \item WS GIT: Eigener Laptop mitbringen! Software davor installieren! Keine Installations und IT-Support Workshop! Wer Fragen oder Probleme hat, kann dich vorher dennoch bei Jörg melden.
      \end{itemize}

    \subsection{weitere Informationen zur Tagung}
      \paragraph{Wetterbericht}
        Das Wetter heute ist noch einigermaßen gut. \\
        Wie es in den nächsten Tagen wird, ist noch nicht klar.

      \paragraph{Türschließung}
        Abends/Nachts müssen die Türen geschlossen werden.
        Wenn die Türen von innen geöffnet werden, kann es sein, dass diese \textbf{offen} bleiben. Das darf nicht sein! \\
        Stattdessen: \textbf{Nutzt bitte nur die Türen mit Schild "Ausgang"}. Die Tür am Orga-Büro ist immer offen. Im Notfall einfach dort anrufen, Nummer steht in der Tagungsbadge.

% ## Bericht des StAPFs
% 1. Jenny erklärt kurz, was der StaPF ist (nicht etwas mit "Organisation", da die Präsentation noch nicht verfügbar ist - OTon [Anm. der Protokolls])
% 2. Der StaPF besteht aus Svenja, Niklas, Jennifer, Ann-Kathrin und Marcus, von diesen laufen drei Amtszeiten aus, die neu besetzt werden müssen in dieser ZaPF.
% 3. Alle Resolutionen wurden verschickt, ebenso die Positionspapiere
% 4. ZaPF-Bericht wurde veröffentlicht und verschickt.
% 5. Mehrere Online-Treffen (7) seit Siegen und zwei Klausurtagungen (2) zur Vorbereitung der nächsten Themen und ZaPF
% 6. Es gabe eine Anfrage und zwei Nachfragen an das BMBF wegen der ausgebliebenen Förderung für die ZaPF in Siegen. Prinzipiell steht nichts einer Föderung im Wege, aber es waren lt. Aussage des BMBF nicht genug Mittel in der Föderrunde 2017/18 zur Verfügung um alle BuFaTas zu fördern. Die Anfrage wurde besprochen und weiteres Vorgehen. Heidelberg wurde zum Beispiel auch nicht gefördert (allerdings, da überhaupt gar kein Antrag gestellt wurde). Diese Entscheidung des BMBFs für Siegen war einmalig.
% -120- Wie sieht es mit der Förderung für die nächsten ZaPFen aus?
% -StaPF- Antrag geplant, allerdings noch nicht bekannt, ab wann der Antrag gestellt werden kann. Wegen später Regierungsbildung gibt es aktuell gar keine Fristen beim BMBF. Antrag wird bereits vorbereitet, sodass er direkt verschickt werden kann.
% 7. Es laufen die Mandate von einigen Akkreditierungspoolmitgliedern aus. Wer weiter im Pool sein will muss bestätigt werden.
% 8. Kommende ZaPFen sind wie folgt vergeben: WiSe Würzburg, SoSe 19 Bonn, **WiSe 19 noch nicht vergeben**, SoSe 2020 Rostock+Greifswald. **Bitte bewerben! Optimalerweise bewerben sich Fachschaften selbst.**
% 9. Für die Organistion zukünftiger ZaPFen gibt es den AK von Jenny, in welchem auch eventuelle Fragen zur Organisation besprochen werden können.
%
% ## Aktueller AK Plan
% Die Redeleitung und Orga stellen den aktuellen Zeitplan für die AKs vor. Wer **große** Probleme mit dem Plan hat, soll an [plenum@zapf.in](plenum@zapf.in) schreiben.
%
% ## Bericht des KomGrem
% -KomGrem- stellt sich vor. Das KomGrem dient der Vernetzung zwischen ZaPF und jDPG. Aktuelle Themen sind vorallem das CHE Rankung und die BaMa Umfrage. Mitglieder: Sonja(Bonn), Niklas (Oldenburg) + jDPG Leute. Sonjas Platz muss neu gewählt werden.
% 1. Waren auf der jDPG Mitgliederversammlung, um die ZaPF vorzustellen. Diskutiert wurde, wie man Probleme zwischen jDPG Ortgruppen und lokalen Fachschaften schlichten kann. Einen äquivalenten AK gibt es auf der ZaPF.
% 2. CHE Fachbeirat (Fredrica und Thomi): Zwei Studenten wirken im Fachbeirat mit, um Input für die kommenden Umfragen zu geben. Auch der Indikator "Studierendensituration allgemein" wurde diskutiert. Details werden im CHE AK diskutiert werden.
% Konkret wurde auch daraufhin gearbeitet, was in dem Studienführer abgedruckt werden soll, und was nicht abgedruckt werden sollen. Beispiele dazu im AK. Es ist nicht klar, wieviel Input und Aufwand dieser AK noch wert ist. Auf jedenfall sollte weiterhin hierzu ein Mitglied im Fachbeirat von der ZaPF gestellt werden, um Schlimmeres zu verhindern.
% Frederika und Thomi sind aus diesem Beirat jetzt raus, d.h. Ersatz ist notwendig.
%
% -111- Wer organisiert das CHE?
% -Thomi- Die Bertelsmann-Stiftung.
%
% 3. Vorbereitung BaMa-Umfrage: Viele Informationen zu der Umfrage stehen im ZaPF-Wiki dokumentiert, auch zur Entwicklung. Es wurde gerade ein Zettel herumgegeben zum aktuellen Stand. Die Befragung soll in ein paar Wochen starten (im kompletten deutschsprachigen Raum). Dafür wird die Mitarbeit der Fachschaften benötigt, es wäre toll wenn der Rücklauf diesesmal auch wieder höher als der des CHE wäre.
% Auf dem verteilten Zettel sind zwei Fragebögen verlinkt (1x für Studentika, 1x für Studiengänge). Bitte: Füllt die Fragebögen testweise aus, damit der Fragebogen überprüft werden kann und evtl. angepasst werden kann.
% Die Besprechung wird im Mumble in ca. 1 Woche gemacht (Termin siehe Zettel). Nachwuchs für den AK ist gewünscht, da alle aktuell Involvierten demnächst ihr Studium abschließen.
% 4. Konferenz der Fachbereiche Physik (KFP) wurde besucht. Findet halbjährlich statt, je ein Vertreter pro Universität. Viele der Themen wiederholen sich jedes (halbe) Jahr. Es gibt auch Bericht von Fachhochschul-Vertretern, welche u.A. die Anregung für den entsprechenden FH-AK auf der letzten ZaPF in Siegen geliefert hat.
% Studierendenstatistik war auch ein Thema, ebenso die Frage, wie man diese sinnvoll erfasst, um Parkstudenten zu umgehen.
% Ars Legendi Fakultätenpreis ist einer der größten Preise für Didaktik, die in Deutschland vergeben werden. Die Gewinner werden auf die KfP eingeladen und berichten darüber, warum sie den Preis bekommen haben.
% Die KFP versucht einen Studienatlas als komplementäres Projekt zum CHE aufzubauen. In Berlin gab es einen Vortrag zum Thema "Mathematische Lernvoraussetzung für MINt-Studiengäng" -  was für Mathematik sollte in der Schule gelernt werden als Voraussetzung für alle MINT-Studiengänge (Stichwort: MALE-MINT).
% Es wurde Physik für Medizinerinnen und Mediziner besprochen, vorallem Praktika. Vernetzung und Informationsaustausch findet allerdings nicht statt.
% Die DPG hat außerdem eine Promotionsstudie durchgeführt, in welcher versucht wurde, alle Promovierenden in Deutschland zu erreichen. Der Rücklauf war ca. 30%, die Studie wurde in Bad Honnef letzte vorgestellt.
% Letzte Woche in Bad Honnef wurde außerdem der neue DPG Präsident Dieter Meschede vorgestellt, der OpenScience zu einem Hauptprojekt machen will.
% Es gab weiterhin einen Bericht von der DFG, was sich an den Programmen wie Emmy-Noether oder Heisenberg geändert hat.
% Der Akkreditierungsrat muss sich an geänderte gesetzliche Grundlagen neu ausrichten, an Verfahren soll sich allerdings nicht viel ändern.
% Es wurde zuletzt eine OpenScience Policy Plattform vorgestellt, alle machen es irgendwie anders, dazu sollte man mal eine gemeinsame Linie finden (Verweis auf den AK).
%
% ## Bericht ZaPF e.V.
% Vorstellung der Aktivitäten des ZaPF e.V. in dem letzten halben Jahr.
% Aktuell 10 (zehn) Vorstände. Normalerweise finden Sitzungen während der ZaPFen statt, dieses Jahr gab es noch zusätzlich ein Treffen im Januar.
% Der aktuelle Kassenwart Patrick tritt von seiner Stelle zurück. Für ihn wird ein Nachfolger gesucht. Falls Interesse besteht, kann Paddy oder Ludi dazu mehr Informationen geben und mögliche Fragen beantworten. Anwesenheit wird dann im AK der Mitgliederversammlung zum ZaPF e.V. empfohlen.
% Es wurde ein neuer Vorstand für Alumni eingeführt. Vielleicht will man nach dem Ende des Studiums weiter mit der ZaPF vernetzt bleiben.
% Jede natürliche oder juristische Person (Menschen, Fachschaften...) kann Fördermitglieder werden. Um weitere Fördermitglieder wird gebeten.
%
% ## Aktueller AK Plan (Fortsetzung)
% Es gab sehr viele Rückmeldungen für Probleme.
% Leider ist es nicht möglich, dass jede Fachschaft an jedem AK teilnehmen kann. Es wird darum gebeten, dass wirklich nur Kollissionen genannt werden, wo beispielsweise AK Leiter gleichzeitig in zwei AK anwesend sein müssten. *Der aktuelle AK-Plan wird dafür nochmals gezeigt.*
%
% *Anm. der Orga: Bitte Wortmeldungen mit "Tausche AK Slot XX mit AK Slot YY"*
% -310- Überschneidung in AK Slot 4 (AK ZaPF-IT).
% Orga: Verschiebung in 70
% -180- Überschneidung.
% -125- Appell an alle Fachschaften: Teilt auch bitte in die AKs Fachschafts-intern auf.
% -294- Verschieben AK-Slot aus gesundheitlichen Gründen.
% -311- Verschieben eines AK-Slot aus breiter Interesse.
% -232- Verschieben eines AK-Slots aufgrund von Termin-Kollission.
% -169- AK im Zeitplan, der nicht stattfindet.
% -180- Überschneidung.
% -262- Frage zu Umbenennung eines AK.
% -265- Frage wegen Verschiebung aus persönlichen Gründen.
% -267- Frage zu AK Depressionen und AK Barrierefrei: Zeitliche und evtl. thematische Kollission, könnten diese zeitlich auseinander gezogen werden?
% -Orga- Vorschlag und Änderung wird vorgestellt.
% **Die AK Planung wurde abgeschlossen.**
% **Hinweis: Menschen in der Jugenherberge werden nicht geweckt und sind selbst für ihr Erscheinen in AKs verantwortlich**
% ++Außerdem Hinweise:++
% * Nur die beiden Türen am Exerzierplatz sollen benutzt werden, sowie die eine Tür am Orgabüro.
% * Materialtisch
% * Ausgänge
% * Tagungsbüro
% * Aushang-Resolutionen
% * Frühstück und Wecken in der Jugendherberge
% * Duschen sind über Nachfrage im Tagungsbüro erreichbar.
%
% -221- Wecken in der Notunterkunft durch Musik möglich?
% -Orga- Nein, da dort auch andere Menschen (nicht ZaPFika wohnen)
%
% -183- Wird es einen ZäPFchen AK geben?
% -Orga- Ja, gleich gibt es dazu mehr Informationen.
%
% -284- Anmeldung Spät-Kommer?
% -Orga- Im Tagungsbüro möglich.
%
% -167- Details zu den Notunterkünften?
% -Orga- Zwischen den Gebäuden gibt es einen Tisch mit zwei vertrauenswürdigen Personen. Diese kümmern sich darum, dass ihr dort rein kommt. Sprecht diese an, diese können Euch weiterhelfen.
%
% -267- Bier-AKs: Vorstellung der Bier-AKs und wie diese funktionieren noch erwünscht?
% -Orga- Alle Leiter von diesen Bier-AKs ("Mindestgrößen-AKs") haben im Wiki auch Leiter. Diese "Mindestgrößen-AK-Leiter" sollen sich bis morgen (Donnerstag) im Tagungsbüro melden, welche Ideen und notwendigen Location vom AK benötigt werden. Das Tagungsbüro wird sich um die Umsetzung kümmern und die Informationen zum AK streuen.
%
% -215- Wie findet man heraus, wer in welchen Zimmern ist?
% -Orga- Tagungsbüro sollte Bescheid wissen
%
% -215- Kneipentour und alternative Veranstaltungen
% -Orga- Es gibt wie immer
% * Kneipentouren am Freitag, die Möglichkeit zum Eintragen gibt es ab morgen vor dem Tagungsbüro.
% * (Alternativprogramm) Für alle die mehr Lust auf Pen-and-Paper (nicht die anderen Rollenspiele) haben, dafür gab es bereits eine Infomail mit einer Kontaktperson. Kontaktadresse wird nochmals bekannt gegeben.
%
% -287- BMBF-Listen?
% -Orga- Nein, gibt es nicht. Wir sind anderweitig sehr gut finanziert.
%
% -200- Gibt es eine Art Raumplan für die AKs und wo z.B. das Mathematikon ist?
% -Orga- Ein Raumplan ist im Tagungsheft. Alle AKs befinden sich in den Stockwerken über den Plenumsraum.
%
% -109- Wie finde ich meine Exkursion heraus?
% -Orga- Siehe Mail, ging etwa um 21.05 Uhr rum. Wer nicht weiß, welche Exkursion er hat, kann auch im Tagungsbüro nachprüfen.
%
% -262- Wann und wo treffen sich die Exkursionen?
% -Orga- Siehe Tagungsheft, E-Mail.
%
% -265- Wohin muss das Bier für den Bier-Austausch-AK?
% -Orga- Am ewigen "goldenen" Frühstück gibt es den Verantwortlichen für die Lagerung von Lebensmitteln. Bitte entsprechend mit der Uni markieren, damit dies nicht frühzeitig "verwendet" wird.
%
% **ZäPFchen-AK:** Am Anschluss an das Anfangsplenum hier im Hörsaal.
% **Jeopardy**: Nach dem ZäPFchen AK findet im Hörsaal des Anfangsplenums wie immer Jeopardy statt! Mit Gewinnen!
%
% -168- Zeitplan, wann Jeopardy beginnt?
% -StaPF- ZäPFchen AK geht etwa bis 23:30.
%
%
%
% **Ende Anfangsplenum um 22:51 **
% **Bitte an die Fachschaften, ihre Stimmkarten bei der Redeleitung ab zu geben**
%
%
%
% <!--
% Ich lass mal hier hinten nen Kommentarblock stehen, damit man da kurz was reinschreiben kann, also einfach hier drüber protokollieren
% -->
