% !TEX TS-program = pdflatex
% !TEX encoding = UTF-8 Unicode
% !TEX ROOT = main.tex

\label{chap:ende}
\textbf{Beginn des Plenums}: 09:24 Uhr

\section{Formalia}
  \subsection{Redeleitung}
    Es werden vorgeschlagen:
    \begin{itemize}
      \item Rebekka - Konstanz
      \item Benni - Siegen
      \item Kathi - Frankfurt
      \item Maik - Bielefeld
      \item Thomi - Heidelberg
    \end{itemize}

    \begin{success}{}
      \begin{center}
        \textbf{Bestätigung der Redeleitung per Akklamation}
      \end{center}
    \end{success}

  \subsection{Wahl der Protokollanten}
    Es werden vorgeschlagen:
    \begin{itemize}
      \item Vicky
      \item Johannes
    \end{itemize}

    \begin{success}{}
      \begin{center}
        \textbf{Bestätigung per Akklamation}
      \end{center}
    \end{success}
    Außerdem stehen als Ersatz zur Verfügung: Christian, Anja, Benedikt Bieringer - U Münster:

  \subsection{Audio-Aufnahme}
    Zur Nachbearbeitung des Protokolls bei Unklarheiten oder anderen Fehler wird um eine Audio-Aufnahme des Plenums gebeten.
    Die dort enthaltenen Daten stehen ausschließlich den Heidelberger Organisatoren für die interne Nachbereitung zur Verfügung. Die Aufnahme wird während der zu führenden Personaldebatten unterbrochen.

    \begin{success}{}
      Zustimmung des Plenums
    \end{success}

  \subsection{Beschlussfähigkeit}
    Zu Beginn des Plenums sind Vertreter von 47 aller 53 teilnehmenden Fachschaften anwesend. Damit ist die
    Beschlussfähigkeit festgestellt. Eine Auflistung aller anwesenden Fachschaften ist in \ref{chap:anwesend}
    zu finden. \\

    \begin{success}{Feststellung der Beschlussfähigkeit}
      Mit 47 anwesenden Fachschaften ist das Plenum der ZaPF satzungsgemäß beschlussfähig.
    \end{success}

    Folgende Fachschaften sind bereits abgereist:
    \begin{itemize}
      \item RTWH Aachen
      \item HU Berlin
      \item Uni Hamburg
      \item TU Kaiserslautern (nimmt verspätet teil)
      \item Uni Mainz
      \item TU Wien
    \end{itemize}

  \subsection{Tagesordnung}
    \begin{success}{Tagesordnung}
      Die vorliegende Tagesordnung wird \textbf{per Akklamation angenommen}
    \end{success}

  \subsection{Ehrung}
    Tobias Löffler wird für die Teilnahme an seiner 20. ZaPF geehrt.

\section{Wahlen}
    vorgeschlagene Kandidaten für den Wahlausschuss:
    \begin{itemize}
     \item Marcel - Uni Bonn
     \item Christian - Uni Oldenburg
     \item Ludi - Uni Erlangen
     \item Elli - Uni irgend Berlin
    \end{itemize}

    \begin{success}{Wahlausschuss}
      \textbf{Bestätigung der Liste per Akklamation}
    \end{success}

  \subsection{StaPF}
    Folgende Personen stehen zur Wahl bereit:
    \begin{itemize}
      \item Niklas Donocik - TU Braunschweig
      \item Colin Heckmeyer - U Tübingen
      \item Luise Siegl - TU Dresden
      \item Marius Anger - TU München
    \end{itemize}

    \textbf{Die Kandidaten stellen sich vor.} \\

    Es besteht Möglichkeit für Fragen an die Kandidaten:
    \begin{outline}
      \1 Lisa Dietrich - U Erlangen-Nürnberg:  In welchem Stadium im Studium seid ihr aktuell? Was macht ihr so?
        \2 Marius: Master Student
        \2 Luise: Noch nicht so viel in der Fachschaft, weil frisch nach Dresden und in die Fachschaft gewechselt.
        \2 Colin: Im zweiten Bachelor-Semester, in der Fachschaft alles was so ansteht. Faires Aufteilen der Aufgaben ist wichtig.
        \2 Niklas: Hat noch etwas Zeit im Studium vor sich, keine gewählte Position in der Fachschaft, Kandidat des StudierendenParlaments (StuPa) und verschiedene Projekte.
    \end{outline}

    \begin{info}{}
      Die Fachschaft der Uni Osnabrück reist vorzeitig um \textbf{09:50} ab und gibt ihre Stimmkarte ab. Gute Heimfahrt!
    \end{info}

    \begin{outline}
      \1 Thomas Rudzki - U Heidelberg:  Wisst ihr, was auf Euch zukommt? Könnt ihr die Arbeit abschätzen? Ward ihr im StaPF AK?
        \2 Niklas: Ist aktuell Mitglied des StaPF.
        \2 Colin: War nicht im AK, hat sich im Wiki informiert.
        \2 Luise: War oft im StaPF AK - auch in Dresden, in Heidelberg.
        \2 Marius: War nicht im StaPF AK, bekommt das aber bestimmt hin.
      \1 Thomi: Seid ihr politisch aktiv oder Mitglied in politischen Gruppierungen?
        \2 Niklas: Mitglied einer nicht politischen Liste des StuPa, war Mitglied der jungen Liberalen, ist aber ausgetreten.
        \2 Alle Anderen: Politisch nicht aktiv
      \1 Jakob Brenner - LMU München:  Mitglieder der (j)DPG?
        \2 Marius: Möchte Mitglied werden.
        \2 Alle Anderen: keine Mitglieder
    \end{outline}
    Lisa Dietrich - U Erlangen-Nürnberg:  Antrag auf eine Personaldebatte unter Ausschluss der Kandidaten.

    \begin{info}{Personaldebatte}
      Die Kandidaten verlassen den Raum. Protokoll und Audiomitschnitt werden für die Dauer der Personaldebatte angehalten.
    \end{info}

    Der Wahlausschuss erklärt das Wahlprozedere. \\
    Es folgt eine Beratungszeit für alle Fachschaften. Anschließend wird die Wahl, wie beschrieben durchgeführt. \\

    Folgende Stimmen wurden für die Kandidaten abgegeben: \textit{(Zustimmung/Ablehnung/Enthaltung)}
    \begin{itemize}
      \item Niklas Donocik - TU Braunschweig  (44/1/2)
      \item Colin Heckmeyer - U Tübingen  (26/6/15)
      \item Luise Siegl - TU Dresden  (34/2/11)
      \item Marius Anger - TU München  (23/9/5)
    \end{itemize}

    \begin{success}{Wahlergebnis}
      Folgende Personen wurden in den StaPF gewählt:
        \begin{itemize}
          \item Niklas Donocik - TU Braunschweig
          \item Colin Heckmeyer - U Tübingen
          \item Luise Siegl - TU Dresden
        \end{itemize}
        \tcblower
        Die Gewählten nehmen ihre Wahl an.
    \end{success}

  \subsection{KommGrem}
    Folgende Person steht zur Wahl:
    \begin{itemize}
      \item Sonja Gehring - KommGrem
    \end{itemize}
    Sonja möchte ihre Arbeit weiterführen und stellt sich vor.
    Fragen an die Kandidatin:
    \begin{outline}
      \1 Jakob Brenner - LMU München: Bist du (partei)politisch aktiv und Mitglied in der DPG oder jDPG?
        \2 Sonja: Nicht parteipolitisch engagiert und auch nicht in anderen ähnlichen Organisationen. Sie ist Mitglied in der DPG, aber nicht aktiv in der jDPG.
      \1 Jan Luca Naumann - TOPF: Warst du bereits auf der KFP im Rahmen deiner Arbeit und wie hast du dort mit Ihnen zusammengearbeitet?
        \2 Sonja: War zweimal bei der KFP, als sie in Bad Honnef getagt hat und hat unsere Arbeit und Beschlüsse erklärt. Außerdem hat sie sich dort aktiv in Diskussionen eingebracht und nach bestem Gewissen versucht unsere Positionen zu vertreten.
    \end{outline}

    \begin{info}{}
      Die Fachschaft der Uni Jena reist vorzeitig um \textbf{10:20} ab und gibt ihre Stimmkarte ab. Gute Heimfahrt!
    \end{info}

    Der Wahlausschuss erklärt das Wahlprozedere.
    Es folgt eine Beratungszeit für alle Fachschaften. Anschließend wird die Wahl durchgeführt. \\ \\

    Folgende Stimmen wurden (bei einer ungültigen Stimme) für die Kandidatin abgegeben: \textit{(Zustimmung/Ablehnung/Enthaltung)}
    \begin{itemize}
      \item Sonja Gehring - KommGrem  (43/0/1)
    \end{itemize}

    \begin{success}{Wahlergebnis}
      Damit ist in das KommGremm gewählt:
      \begin{itemize}
        \item Sonja Gehring - KommGrem
      \end{itemize}
      \tcblower
      Die Gewählte nimmt die Wahl an.
    \end{success}

  \subsection{TOPF}
    Folgende Personen stehen zur Wahl bereit:
    \begin{itemize}
      \item Philipp Jäger - Alte Säcke
      \item Lydia Naumann - TU Dresden
      \item Lennart Stipulkowski - U Heidelberg
    \end{itemize}
    Jan Luca Naumann - TOPF:  Als aktuelles Mitglied stellt er Philipp vor, der beim Poolvernetzungstreffen ist. Die anderen stellen sich selbst vor. \\ Fragen zu den Kandidaten:

    \begin{outline}
      \1 Patrick Haiber - ZaPF e.V.:  Frage an Jan: Inwieweit ist der TOPF mit der BaMa Umfrage involviert, da dies ein Hauptgrund für die Bewerbung von Philip ist.
        \2 Jan: Die Leute von der BaMa-Umfrage hatten sich nachgefragt, ob man eine Online-Umfrage nutzen könnte und ob die Daten eventuell auf dem Server gespeichert werden könnten. Er findet es sinnvoll, dass es Kontakt gibt.
        \2 Sonja Gehring - KommGrem:  Als Ergänzung zur Notwendigkeit der TOPF Funktionen für die BaMa Umfrage. Aktuell läuft die Umfrage über Server der Fachschaft in Göttingen. Außerdem muss die Zugänglichkeit der Daten in Zukunft organisiert werden, sodass zukünftig auch die Daten der alten Umfragen zugänglich sind.
      \1 Manuel Lohoff - U Marburg:  Was steht im nächsten Jahr alles an Arbeit im TOPF an?
        \2 Jan: Alter Server bei Strato muss umgezogen werden. Sonst bisher keine großen Projekte in Planung. Primär Grundarbeit/Haushalt und Erhalten der Funktionen.
        \2 Fabian Freyer - TU Berlin:  Mitarbeit im TOPF geht auch als Henkel. Als Deckel macht man eher korrdinative Dinge. Man kann sich auch zusätzlich engagieren.
      \1 Peter Steinmüller - KIT Karlsruhe:  Mit Bezug auf den Umzug des Servers: Habt ihr in eurer Fachschaft Erfahrung mit dem Aufsetzen und Umziehen von Servern?
        \2 Lennard: Hat ein bisschen Erfahrung. Hat die IT für die aktuelle ZaPF aufgestellt. Möchte sich aber einarbeiten und dazu lernen.
        \2 Lydia: Keine Erfahrungen, in der Fachschaft gibt es Leute, die es gut können und die können dabei helfen.
        \2 Lina für Philip: Philip hat sich um genau solche Dinge für die ZaPF in Siegen gekümmert.
      \1 Niklas Brandt - U Oldenburg:  Thema DSGVO: Habt ihr euch damit schon auseinander gesetzt?
        \2 Lennard: Hat sich erst LastMinute damit auseinander gesetzt, aber schon Texte für seine Webseiten aufgesetzt. Er muss sich noch mehr einlernen.
        \2 Lydia: Genauso.
        \2 Für Philip sind dazu keine Informationen bekannt.
      \1 Jan: Freut sich, dass so reges Interesse besteht. Was sind die Zeitvorstellungen? Wie lange werdet ihr noch auf ZaPFen gehen? Wie lange studiert ihr in etwa noch? Er ist bald fertig und möchte Nachfolger einarbeiten.
        \2 Lydia: Bachelor im vierten Semester. Wenn Nichts dazwischen kommt, wird sie noch einige ZaPFen mitmachen.
        \2 Lennard: Ist in etwa in der Mitte seines Studiums. Steht vermutlich noch relativ lange zur Verfügung.
        \2 Lina für Philip: Philip hat gerade eine Promotion angefangen und ist also noch ein paar Jahre für die ZaPF erhalten und verfügbar.
    \end{outline}
    Es wird eine Personaldebatte unter Ausschluss der Kanditaden gewünscht.

    \begin{info}{Personaldebatte}
      Die Kandidaten verlassen den Raum. Protokoll und Audiomitschnitt werden für die Dauer der Personaldebatte angehalten.
    \end{info}

    Der Wahlausschuss erklärt das Wahlprozedere.
    Es folgt etwas Beratungszeit für alle Fachschaften. Anschließend wird die Wahl durchgeführt. \\ \\

    Folgende Stimmen wurden (bei einer ungültigen Stimme) für die Kandidaten abgegeben: \textit{(Zustimmung/Ablehnung/Enthaltung)}
    \begin{itemize}
      \item Philipp Jäger - Alte Säcke  (26/5/15/1)
      \item Lydia Naumann - TU Dresden  (18/19/10/0)
      \item Lennart Stipulkowski - U Heidelberg  (31/6/9/1)
    \end{itemize}

    \begin{success}{Wahlergebnis}
      Damit ist in den TOPF gewählt:
      \begin{itemize}
        \item Lennart Stipulkowski - U Heidelberg
      \end{itemize}
      \tcblower
      Die Gewählten nehmen die Wahl an.
    \end{success}
%%% mehrere Gewählte? Wer noch?

    \begin{info}{}
      Die Fachschaft der TU Kaiserslautern trifft um \textbf{11:05} ein und nimmt ihre Stimmkarte entgegen.
    \end{info}

  \subsection{Entsendung in den Akkreditierungspool}
    Folgende Fachschaftler wollen in Zukunft Studiengänge akkreditieren und wollen daher vom Plenum in den studentischen Akkreditierungspool entsandt werden.
    \begin{itemize}
      \item Marco Nüchel - RWTH Aachen
      \item Maurice Jensen - U Heidelberg
      \item Moritz Brinkmann - U Heidelberg
      \item Christoph Blattgerste - U Heidelberg
    \end{itemize}
    Die Kandidaten sind nicht anwesend, wurden aber schon entsendet. Sie müssen also nur nochmal bestätigt werden. Maurice und Moritz sind sehr aktiv in der Akkreditierung, Marco hat einen Workshop besucht. \\

    \begin{info}{}
      Die Fachschaft der TU Wien trifft um \textbf{11:11} ein und nimmt ihre Stimmkarte entgegen.
    \end{info}

    Der Wahlausschuss erklärt das Wahlprozedere.
    Es folgt etwas Beratungszeit für alle Fachschaften. Anschließend wird die Wahl durchgeführt. \\ \\

    Folgende Stimmen wurden für die Kandidaten abgegeben: \textit{(Zustimmung/Ablehnung/Enthaltung)}
    \begin{itemize}
      \item Marco Nüchel - RWTH Aachen
      \item Maurice Jensen - U Heidelberg
      \item Moritz Brinkmann - U Heidelberg
      \item Christoph Blattgerste - U Heidelberg
    \end{itemize}

    \begin{success}{Wahlergebnis}
      Damit sind folgende Personen für die Entsendung in den Akkreditierungspool bestätigt:
      \begin{itemize}
        \item Marco Nüchel - RWTH Aachen
        \item Maurice Jensen - U Heidelberg
        \item Moritz Brinkmann - U Heidelberg
        \item Christoph Blattgerste - U Heidelberg
      \end{itemize}
    \end{success}

  \subsection{LEUTE}
      Bestätigt werden sollen folgende aktuelle Mitglieder:
      \begin{itemize}
        \item Miria Granfors - TU Dresden
        \item Jacob Brunner - U Augsburg
        \item Stephan Hagel - U Gießen
      \end{itemize}
% <unvollständig>
% ...
    \paragraph{LEUTE zur SACHE}
      \textit{Aufgabenbereich}: Sacharbeit zum CHE-Ranking
      \begin{success}{}
        Bestätigung dieser LEUTE die Zwischen den ZaPFen WAS machen.
      \end{success}

    \paragraph{LEUTE für WAS}
      \textit{Aufgabenbereich}: Weiterarbeit am Studienführer
      \begin{success}{}
        Bestätigung dieser LEUTE die Zwischen den ZaPFen WAS machen.
      \end{success}

    \paragraph{LEUTE für HUMBUG}
      \textit{Aufgabenbereich}: BaMa-Umfrage
      \begin{success}{}
        Bestätigung dieser LEUTE zum Weitermachen von HUBUG zwischen den ZaPFen.
      \end{success}

    \begin{success}{Wahlergebnis}
      Damit sind folgende Personen für die Entsendung in den Arbeitskreis LEUTE bestätigt:
      \begin{itemize}
        \item Miria Granfors - TU Dresden
        \item Jacob Brunner - U Augsburg
        \item Stephan Hagel - U Gießen
      \end{itemize}
    \end{success}

    Vielen Dank an den Wahlausschuss.

\section{Ausrichtung zukünftiger ZaPFen}
  \subsection{ZaPF in Würzburg}
    Wie bereits beschlossen wurde, organisiert Würzburg die ZaPF mithilfe von Zeitreisen. Dazu werden uns jetzt schon mithilfe eines Videos Einblicke in die Winter-ZaPF in Würzburg gezeigt. Es wird toll gewesen sein.


  \subsection{Ausrichtung der Winter-ZaPF 2019}
    Das Plenum schlägt folgende Fachschaften als Ausrichter vor:
    \begin{itemize}
      \item Heidelberg
      \item Die TU München gibt eine Absichtsklärung ab. \\ $\rightarrow$ Marius möchte es aber noch mit seiner Fachschaft abklären.
    \end{itemize}

    \begin{itemize}
      \item Thomas Rudzki - U Heidelberg: Findet es kein gutes Konzept, dass wir das allgemein so machen. Haben wir einen Plan B, falls die Fachschaft der TU München 'Nein' sagt und keine ZaPF vergeben wird?
      \item Lisa Dietrich - U Erlangen-Nürnberg: Bei der KOMA hat das schon gut funktioniert.
      \item Maik Rodenbeck - U Bielefeld: Gutes Konzept, wenn wir durch Abstimmung der TU München die ZaPF zusprechen. Dann sollte innerhalb gewisser Zeit die Rückmeldung an den StaPF ergehen, notfalls wäre es möglich, die Winter-ZaPF 2019 auf der Winter-ZaPF in Würzburg noch zu vergeben.
      \item  Nils - U Frankfurt am Main: Gebt ihr ein Zeitfenster vor, bis wann die TU München bestätigen soll, dass die ZaPF an München geht?
      \item Mario Jakobs - TU Darmstadt: Die TU München bittet selbst darum.
    \end{itemize}
    Vorschlag zur Abstimmung

    \begin{success}{Ausrichtung der Winter-ZaPF durch die TU München}
      Welche Fachschaften befürworten die Vergabe der Winter-ZaPF 2019 an die TU München, unter Voraussetzung der positiven Rückmeldung von der Fachschaft der TU München an den StaPF innerhalb des nächsten Monats? \textit{(Zustimmung/Ablehnung/Enthaltung)}
      \tcblower
      Ergebnis der Abstimmung: 46/0/2 \\
      Dem Verfahren wird demnach zugestimmt.
    \end{success}

    \begin{itemize}
      \item Anna Summers - U Kiel: Der Ergebnis soll zeitnah vom StAPF weiter kommuniziert werden.
    \end{itemize}

\section{Anträge}
  \subsection{Resolutionen, GO Änderungen, Arbeitsaufträge, Positionspapiere}
    Es gibt insgesamt 26 Anträge. \\
    Über die Reihung der folgenden Anträge wird nach 10 Minuten Pause diskutiert werden.

    \begin{info}{}
      Die Fachschaft der Uni Rostock verlässt das Plenum vorzeitig um \textbf{11:36} und gibt ihre Stimmkarte ab. Gute Heimfahrt!
    \end{info}

    Ansage von Köln: In NRW gibt es Diskussion um Studiengebühren. Mehrere Hochschulsenate und die Hochschulrektorenkonferenz (HRK) haben sich dagegen positioniert. Es gibt eine Unterstützer-Unterschriftenliste, die gleich verteilt und unterschrieben werden kann, auch elektronisch ist die möglich, sowie für Menschen aus anderen Bundesländern. \\

    Änderungen an der Reihenfolge sollen bitte mit der Redeleitung abgesprochen werden. Änderungen bitte an \url{resos@zapf.in} schicken.

    \begin{info}{}
      Die Fachschaft der Uni des Saarlandes verlässt das Plenum vorzeitig um \textbf{11:51} und gibt ihre Stimmkarte ab. Gute Heimfahrt!
    \end{info}

  \subsection{Handreichung : Organizing an international welcome (Antrag 1)}
    Antragsteller (AS) stellen den AK vor. Die Handreichung kann hochgeladen werden oder für Präsentationen verwendet werden. Es soll der StAPF beauftragt werden, dies weiter an die MeTaFa zu verschicken.

    \begin{itemize}
      \item Johannes Fleck - U Bonn: Soll dies in Zukunft aktualisiert werden? $\rightarrow$ AS: Die meisten Sachen müssen nicht unbedingt geändert werden, da sie länger gültig sind. Vielmehr müssen sie an die örtlichen Gegebenheiten angepasst werden.
      \item Jennifer Hartfiel - FU Berlin: Anmerkung: Ein Satz spiegelt Vorurteile über Deutsche, Schweizer und Österreicher wieder, würde den gerne gestrichen sehen (z. 319). Formulierung ist ungünstig. $\rightarrow$ AS: Soll eine kulturelle Anmerkung sein. Da Pünktlichkeit doch international unterschiedlich gewichtet wird.
      \item Tobias Guttenberger - U Bonn: Gegenvorschlag: Satz umbauen, sodass deutlich ist, bei welchen Umständen Pünktlichkeit erwartet wird. $\rightarrow$ AS: Vorschlag "Bei einigen offiziellen Terminen ist es nötig pünktlich zu sein."
      \item Fabian Freyer - TU Berlin: Anmerkung an einen Satz über erlaubte Arbeitszeiten, die sich mittlerweile geändert haben. Etwa, dass man mittlerweile 40 Stunden pro Woche arbeiten darf. $\rightarrow$ AS: überprüfen das nochmal und ändern das nachträglich.
      \item Marvin Bende - U Tübingen: Papiere von der ZaPF sollen keine Werbung für bestimmte Firmen machen. $\rightarrow$ Laura Weber - U Würzburg:  Spezielle Firmen sollen explizit erwähnt werden, weil diese im Ausland nicht bekannt sind.
      \item Jan Luca Naumann - TOPF: Es gibt bei einigen erwähnten Gebieten deutlich mehr Auswahl und die getroffene wirkt beliebig. $\rightarrow$ AS: Gibt es Gegenvorschläge?
      \item Tobias Guttenberger - U Bonn: Statt einer Fluglinie könnte man auch Flugportale nennen. $\rightarrow$ AS: Nehmen den Vorschlag an. Merkt aber an, dass es gerne von den Fachschaften angepasst werden soll.
      \item Rebekka Garreis - U Konstanz: Es ist keine Resolution, sondern eine interne Handreichung für Fachschaften, die Vorschläge macht, aber bearbeitet werden kann und soll.
      \item  Jakob - TU Wien: redaktionelle Änderungen nötig
    \end{itemize}

    \begin{info}{}
      Die Fachschaft der Uni Halle-Wittemberg verlässt das Plenum vorzeitig um \textbf{12:06} und gibt ihre Stimmkarte ab. Gute Heimfahrt!
    \end{info}

    \begin{itemize}
      \item Thomas Rudzki - U Heidelberg: Es sollen nur Fortbewegungsmöglichkeiten erwähnt werden, keine Anbieter. Anbieter herauszufinden sollte machbar sein. $\rightarrow$ AS: Dem ist nicht immer direkt so.
      \item Tobias Guttenberger - U Bonn: Stellen, die von den Fachschaften geändert werden sollen, sollen markiert werden.
        $\rightarrow$ AS: Dies wird nur gerade nicht angezeigt.
      \item Köppel Quirin - TU München: Wegen Nachhaltigkeit sollte nicht Ryanair erwähnt werden, Carsharing hingegen wäre erwähnenswert.
        $\rightarrow$ AS: Das wurde schon geklärt. Es werden Suchmaschinen für Flüge genannt. Flüge können aber nicht generell abgelehnt werden.
    \end{itemize}

    \begin{danger}{GO-Antrag auf Abstimmung}
      Abstimmung der Handreichung. Keine Gegenrede
    \end{danger}

    \begin{success}{Abstimmung}
      Ergebnis der Abstimmung: 40/0/3 (\textit{(Zustimmung/Ablehnung/Enthaltung)}) \\
      Dem Antrag wird demnach zugestimmt.
    \end{success}

  \subsection{GO Änderung Passives Wahlrecht (Antrag 3)}
    Antragsteller (AS) stellt den Antrag vor.

    \begin{success}{Abstimmung}
      Ergebnis der Abstimmung: 41/1/1 (\textit{(Zustimmung/Ablehnung/Enthaltung)}) \\
      Dem Antrag wird demnach zugestimmt.
    \end{success}

  \subsection{GO Änderung Aktives Wahlrecht (Antrag 2)}
    Antragsteller (AS) stellt den Antrag vor.

    \begin{success}{Abstimmung}
      Ergebnis der Abstimmung: 42/1/0 (\textit{(Zustimmung/Ablehnung/Enthaltung)}) \\
      Dem Antrag wird demnach zugestimmt.
    \end{success}

  \subsection{GO Änderung Wahlgleichstand Vertrauenspersonen (Antrag 5)}
    Antragsteller (AS) stellt den Antrag vor. \\
    \begin{danger}{Änderungsantrag per Mail}
      Die Nummerierung soll angepasst werden. Da bisher aber keine Nummerierung besteht, ist dieser nichtig. \\
      $\rightarrow$ Der Antrag wird zurück gezogen.
    \end{danger}

    \begin{success}{Abstimmung}
      Ergebnis der Abstimmung: 42/0/1 (\textit{(Zustimmung/Ablehnung/Enthaltung)}) \\
      Dem Antrag wird demnach zugestimmt.
    \end{success}

  \subsection{Depressionen im Studium (Antrag 6)}
    Antragsteller (AS) stellt den Antrag vor. Es soll für das nächste mal einE ReferentIn eingeladen werden.\\

    \begin{success}{Abstimmung}
      Ergebnis der Abstimmung: 42/0/0 (\textit{(Zustimmung/Ablehnung/Enthaltung)}) \\
      Dem Antrag wird demnach zugestimmt.
    \end{success}

    \begin{info}{}
      Die Fachschaften der Uni Wuppertal und der TU Dresden verlassen das Plenum vorzeitig um \textbf{12:22} und geben ihre Stimmkarten ab. Gute Heimfahrt!
    \end{info}

    \textbf{Mit frenetischem Jubel wird der ZaPF-Orga in Heidelberg gedankt und eine Zugabe gewünscht.} \\
    Die Heidelberg Hauptverantwortlichen bedankt sich auch für die schöne Zeit und die viele Hilfe von der Fachschaft vor Ort und den Teilnehmern. \\ \\
    Kathi bittet um einen Applaus für alle fleißigen Leute, die aus ihren gewählten Organen ausscheiden.

  \subsection{Resolution zu Depressionen im Studium (Antrag 7)}
    Antragsteller (AS) stellen den Antrag vor. Resolution soll Fachschaften zu dem Problem informieren.

    \begin{itemize}
      \item Peter Steinmüller - KIT Karlsruhe: Leitfaden wird erwähnt. Soll dieser mitverschickt werden?
        $\rightarrow$ AS: Soll mit der Resolution mitgeschickt werden als Link zum wiki, das aktualisiert werden soll.
      \item Patrick Walkowiak - U Bochum: Warum als Resolution und nicht Arbeitsauftrag? Das ist doch nichts offizielles.
        $\rightarrow$ AS: Soll eine offizielle Äußerung sein und damit mehr Gewicht bekommen.
      \item Elisabeth Schlottmann - TU Berlin: Obwohl die Handreichung nicht zur Resultion gehört, stimmt der Satz zur Schweigepflicht nicht. Außerdem soll ein Hinweis zu Anfrage externer Hilfe hinein.
        $\rightarrow$ AS: Nimmt die Änderung an.
      \item Fabian Freyer - TU Berlin: Ist es nicht potentiell gefährlich, eine veränderbare Version im Wiki zu verschicken.
        $\rightarrow$ AS: Soll nur Hilfe zur Selbsthilfe sein. Außerdem werden Wiki-Accounts moderiert. Daher sollte Vertrauen zu ZAPFika herrschen.
      \item Marius Anger - TU München: Stimmen wir nur über die Resolution ab, oder auch über die Handreichung?
        $\rightarrow$ AS: Nur über die Resolution wird abgestimmt.
    \end{itemize}

    \begin{success}{Abstimmung}
      Ergebnis der Abstimmung: 37/0/4 (\textit{(Zustimmung/Ablehnung/Enthaltung)}) \\
      Dem Antrag wird demnach zugestimmt.
    \end{success}

    Der Ritterschlag an Wolle (Tobias) wird um \textbf{12:40} vollzogen.

  \subsubsection{Resolution: Wissenschaftskommunikation (Antrag 8)}
    Die Antragssteller (AS) stellen ihren Antrag vor.
    Der Antrag enthält die Beschlüsse aus Siegen und fasst diese zusammen.
    Nur die kursiv hervorgehobenen Sätze sind neu.

    \begin{info}{}
      Die Fachschaften der Uni Ilmenau und der Uni Innsbruck verlassen das Plenum vorzeitig um \textbf{12:44} und geben ihre Stimmkarten ab. Gute Heimfahrt!
    \end{info}

    \begin{itemize}
      \item Daniela Kern-Michler - U Frankfurt am Main: Thema ist wichtig und unterstützenswert, aber der Gedankengang ist recht unübersichtlich. Gerade bei einem Papier zum Thema Kommunikation sollte hier besonders darauf Wert gelegt werden. Thema ist vielleicht nicht so zeitkritisch, daher der  Vorschlag zur erneuten Überarbeitung auf der nächsten ZaPf. \\
      Peter Steinmüller - KIT Karlsruhe: Findet die Idee einer Vertagung nicht schlecht. \\
      $\rightarrow$ AS: Sieht die Resolution zwecks Überarbeitung zurück. \\
      \underline{Änderung:} Entsprechende Passage aus dem Brief im folgenden Punkt streichen.
      \item Peter Steinmüller - KIT Karlsruhe: "wir freuen uns ..." ist die falsche Formulierung. $\rightarrow$ AS: Ist so formal korrekt, wird nur selten verwendet.
      \item Martin Rößner - FU Berlin: dritter Absatz: Will eine freundlichere Formulierung für eine Einladung. $\rightarrow$ AS: Mehr als direktes Einladen wird als unangebracht angesehen, ob die Eingeladenen die Einladung annehmen ist ja nicht vorhersehbar.
      \item Patrick Haiber - ZaPF e.V.: Datenschutztechnisch: Hat der Hr. XXX zugestimmt, dass seine Telefonnummer dort stehen darf und diese in elektronischer Form weiterverarbeitet wird durch euren Arbeitsauftrag? $\rightarrow$ AS: Hat sie offen auf einem Kontaktformular, wird aber trotzdem entfernt.
    \end{itemize}

    \begin{info}{Arbeitsauftrag}
      Arbeitsauftrag: Brief an den Siggener Kreis (Antrag 9)
      AS stellt den Siggener Kreis, sowie den Antrag vor.
    \end{info}

    \begin{success}{Abstimmung}
      Ergebnis der Abstimmung: 34/0/2 (\textit{(Zustimmung/Ablehnung/Enthaltung)}) \\
      Dem Antrag wird demnach zugestimmt.
    \end{success}

  \subsection{Flexibler Umgang mit Prüfungsan- und -abmeldungen (Antrag 10)}
    Die Antragstellika (AS) stellen den Antrag vor.
    Die Prüfungsan- und -abmeldung wird an manchen Hochschulen zu unflexibel gehandhabt. Die Hochschulen sollen Systeme einrichten, sodass kurzfristigere An- und Abmeldungen möglich werden.
    Nach Ansicht des AKs ist der Hauptgrund für die üblicherweiße langen Fristen rein technischer Natur und damit nicht zumutbar.

    \begin{info}{}
      Die Fachschaft der Uni Greifswald verlässt das Plenum vorzeitig um \textbf{12:56} und gibt ihre Stimmkarte ab. Gute Heimfahrt!
    \end{info}

    \begin{outline}
      \1 Marius Anger - TU München:  Eine Erklärung ist nicht deutlich genug.
      \1 Niklas Donocik - TU Braunschweig:  Vierter Absatz: "... ist eine absolute Zumutung" Kann die Formulierung gestrichen werden, um weniger wertende Adjektive zu haben. \\ \underline{Änderungsantrag}: Bedauerlich und absolute streichen. \\
      $\rightarrow$ AS: Will die Sätze genau so haben.
      \1 \underline{Änderungsantrag zur Resolution}. $\rightarrow$ Inhaltliche Gegenrede:
        Stefan Brackertz - U Köln:  "Absolut" ist nicht entscheidend, aber "bedauerlicherweiße" schon. Vor der Grundsatzposition die eingenommen werden soll, wird hier ein Zustand, der nicht richtig empfunden wird, dargestellt.
    \end{outline}
    Es folgt eine kurze Bedenkzeit für alle Fachschaften.

    \begin{danger}{Abstimmung}
      Abstimmung über den Änderungsantrag: 19/9/10 (\textit{(Zustimmung/Ablehnung/Enthaltung)}) \\
      Der Antrag ist damit \textbf{abgelehnt}.
    \end{danger}

    \begin{outline}
      \1 Daniela Kern-Michler - U Frankfurt am Main:  Satz "Gerade hinsichtlich mehrerer Prüfungsversuche..." soll entfernt werden und stattdessen nur auf die Resolution aus Siegen verwiesen werden, welche sich auführlich mit dem Thema "Zwangsexmatrikulation" beschäftigt. \\ Änderungsantrag folgt. \\
      $\rightarrow$ AS möchten den Änderungsantrag nur übernehmen, wenn er hinzugefügt wird (zusätzlich), anstatt einen Teil des Antrags zu ersetzen.
        \2 Daniela Kern-Michler - U Frankfurt am Main:  Stellt den Änderungsantrag so, dass der Satz gestrichen werden soll.
        \2 Daniela Kern-Michler - U Frankfurt am Main:  Ändert den Änderungsantrag. AS und Daniela Kern-Michler - U Frankfurt am Main:  sind sich einig, der Antrag wird angepasst.
      \1 Ilija Funk - U Konstanz:  Änderungsantrag per Mail wurde nicht berücksichtigt.
        \2 Die Redeleitung informiert darüber, dass Änderungsanträge zwar per Mail zugesandt werden können, aber weiterhin noch explizit im Plenum zu stellen sind.
      \1 Ilija Funk - U Konstanz:  stellt den Änderungsantrag. $\rightarrow$ AS übernehmen den Änderungsantrag.
      \1 Jan Luca Naumann - FU Berlin:  Findet es wichtig, dass der Teil zur "Zwangsanmeldung" als essentieller Teil auch in der einführenden Zusammenfassung beinhaltet ist.
        $\rightarrow$ AS: Umbau des Textes ist derzeit zu kompliziert.
        \2 Jan Luca Naumann - FU Berlin:  formuliert einen konkreten Textvorschlag.
      \1 Patrick Walkowiak - U Bochum:  "Kurzfristige Anmeldungen" ist nicht definiert, dabei wird viel darüber gesprochen. Was ist mit "kurzfristig" gemeint? Immer oder eine Woche vor der Prüfung? $\rightarrow$ AS: ``Kurzfristig'' vermeidet eine Nennung einer konkreten Zeitspanne. Gemeint sind ca. 0-48 Stunden. Angesprochen werden Fachbereiche mit mehreren Wochen Anmeldefrist.
        \2 Patrick Walkowiak - U Bochum:  Findet das nicht sehr hilfreich.
          $\rightarrow$ AS: Schön wäre natürlich 0 oder 12 Stunden vorher. Mit einer Nennung könnten aber Hochschulen, die bereits kürzere Fristen haben, dazu verleitet sein, die Fristen zu verlängern.
      \1 Matthias Grinewitschus - U Duisburg-Essen:  Soll Prüfungsanmeldung oder Revidierung kurzfristig möglich sein? \underline{Vorschlag}: Revision \\
        $\rightarrow$ AS: Änderungsantrag wird von den Antragsstellern übernommen.
      \1 Marius Anger - TU München:  Vorschlag zu "selbstsicher": ergänze danach "vorbereitet". $\rightarrow$ AS: Übernehmen den Vorschlag.
      \1 Colin Heckmeyer - U Tübingen:  Mit Bezug auf die Änderung, welche den Absatz in Bezug auf Siegen eingefügt hat: Der Satz vor dem Absatz könnte kontraproduktiv wirken. \underline{Änderungsantrag auf Streichung dieses Satzes}. \\
      $\rightarrow$ AS: Würde gerne die Reso aus Siegen kurz dem Plenum zeigen, um Klarheit zu schaffen.
      \1 Die Resolution aus Siegen zur Zwangsexmatrikulation wird am Beamer gezeigt. \\
      $\rightarrow$ AS: Übernehmen den gewünschten Satz aus der Resolution nicht.
    \end{outline}

    \begin{danger}{Abstimmung}
      Abstimmung über den Änderungsantrag: 8/9/21 (\textit{(Zustimmung/Ablehnung/Enthaltung)}) \\
      Der Antrag ist damit \textbf{abgelehnt}.
    \end{danger}

    \begin{outline}
      \1 Jonas Lautenschläger - HU Berlin:  Teilweise inhaltliche Wiederholung in den Absätzen. \underline{Änderungsvorschlag zur Umstrukturierung}. \\
      $\rightarrow$ AS: übernehmen den Änderungsantrag.
      \1 \underline{GO-Antrag auf sofortige Abstimmung}. $\rightarrow$ Inhaltliche Gegenrede: \\
        Jan Luca Naumann - TOPF:  Ungeeigneter GO-Antrag, da eine sichere Ablehnung des Antrags aus dem GO-Antrag folgen würde. Vermutlich ist das nicht vom GO-Antragsteller intendiert ist.
    \end{outline}

    \begin{danger}{Abstimmung}
      Abstimmung über den GO-Antrag: 23/7/9 (\textit{(Zustimmung/Ablehnung/Enthaltung)}) \\
      Der GO-Antrag ist damit \textbf{abgelehnt} (keine 2/3 Mehrheit).
    \end{danger}

    \begin{outline}
      \1 Patrick Walkowiak - U Bochum:  \underline{GO-Antrag auf Vertagung in einen anderen TOP}. $\rightarrow$ Formale Gegenrede.
    \end{outline}

    \begin{success}{Abstimmung}
      Abstimmung über den GO-Antrag: 25/13/0 (\textit{(Zustimmung/Ablehnung/Enthaltung)}) \\
      Der GO-Antrag ist damit \textbf{angenommen}. Der ürsprüngliche Antrag wird damit auf später vertagt.
    \end{success}

    Bitte der Redeleitung um das Gespräch mit den Antragstellern im Foyer. \\
    Es folgen fünf Minuten Pause zur Technik-Umstellung.

    \begin{info}{}
      Die Fachschaft der TU Freiberg verlässt das Plenum vorzeitig um \textbf{13:45} und gibt ihre Stimmkarte ab. Gute Heimfahrt!
    \end{info}

    \begin{success}{Feststellung der Beschlussfähigkeit}
      Es sind 36 Fachschaften anwesend. \\
      Das Plenum ist damit \textbf{beschlussfähig}.
    \end{success}


\section{ENTENTANZ}
  Tobias Löffler - U Düsseldorf:  prämiert die Selbstberichte. \\
  Es gibt Preise!
  \begin{itemize}
    \item Bielefeld bekommt einen blauen Kugelschreiber für den unauffälligsten Selbstbericht.
    \item Bonn bekommt einen \textit{Negativ}-Preis, weil sie das schlechteste Latex-File abgeben haben.
    \item Chemnitz hat den besten Selbstbericht einer Fachschaft, die demnächst ausrichtet geschrieben, deshalb ein Reader aus Düsseldorf.
    \item Darmstadt hat die tollsten Bilder abgegeben und bekommt deshalb ein Teil des alten Düsseldorfer Schranks mit allen Versionen des Entenaufklebers.
    \item Dresden bekommt den \textit{Karma}-Preis: Warmmach-Ente und ein Kondom.
    \item Düsseldorf bekommt den \textit{Handwerker}-Preis: Handtattoos.
    \item Freiburg bekommt den \textit{konsequentester-Säufer-beim-Schreiben}-Preis: Knallfolie.
    \item Gießen bekommt den \textit{tl;dr} - Preis: Einen "Frozen" Prinzessinenstempel.
    \item Göttingen \textit{Kratulation-konkreter-KOmpetenzen}-Preis: Kawumm-Bändchen mit "K" und Enten
    \item Innsbruck: Preis für puristischten Selbstbericht.
    \item Karlsruhe bekommt den \textit{PCB-Geschädigte}-Preis: Gesunde Zahnpasta mit Salz.
    \item Kiel bekommt ein \textit{Gordon-Freeman}-Preis: Ein T-Shirt.
    \item Konstanz bekommt den \textit{Visionen}-Preis: USB-Lüfter für iPhones
    \item Marburg den \textit{6ECTS}-Preis: Einen Schlüsselring mit Enten ... mehr konnte nicht gegeben werden.
    \item TUM: Enten-Kerze
    \item LMU bekommt den \textit{Echo}-Preis
    \item Osnabrück bekommt ein Kondom mit Augen
    \item Potsdam:
    \item Würzburg bekommmt den \textit{Michelin}-Preis.
    \item Wuppertal für ...
    \item Tübingen den \textit{Kontinuitäts}-Preis.
  \end{itemize}

\section{Fortsetzung Anträge}
  \subsection{Arbeitsauftrag: Abiturwissen (Antrag 24)}
    Das Antragstellikon (AS) stellt den Auftrag vor.
    Mit einem Fachdidaktiker sollen verschiedene Informationen erarbeitet werden.
    Es geht um den Wissensstand vor jeglichen universitären Kursen, etwa vor einem Brückenkurs.
    Die Fachschaften sollen die Fachdidaktiker vor Ort ansprechen und mit dem Brief um Unterstützung fragen.
    Insbesondere sollen die Daten dann für alle, im Sinne von Open-Data, zur Verfügung stehen.

    \begin{info}{}
      Die Fachschaft der TU Wien verlässt das Plenum vorzeitig um \textbf{14:05} und gibt ihre Stimmkarte ab. Gute Heimfahrt!
    \end{info}

    \begin{outline}
      \1 Tabea Lang - TU Chemnitz:  Soll ein Test/Fragebogen erstellt werden? Gute Fragebögen zu erstellen, erfordert psychologisches Wissen und Zusammenarbeit mit dem Fachbereich Psychologie. $\rightarrow$ AS: Will keinen Fragebogen über die Wahrnehmung der eigenen Kentnisse bauen, sondern rein objektiv einen Fragebogen mit Übungsfragen und mit Metainformationen (Studienberechtigung, ...)

      \1 Thomas Rudzki - U Heidelberg:  \underline{Technische Nachfrage mit Änderungsantrag}.
        $\rightarrow$ AS: Nimmt den Änderungsantrag an.

      \1 Lars Vosteen - U Gießen:  Auch das Erstellen von mathematischen Fragebögen erfordert psychologische Kenntnisse. \\
        $\rightarrow$ AS: Findet das korrekt und würde bei Bedarf einen entsprechenden Experten dazu ziehen. Findet allerdings nach aktuellem Plan Mathematik-Didaktik dafür ausreichend, da dies das Kerngeschäft der Mathematik-Fachdidaktik ist.

      \1 Martin Rößner - FU Berlin:  Frage zu möglichen Ergebnissen, Konsequenzen und Nutzen für die ZaPF? $\rightarrow$ AS:
        \begin{enumerate}
          \item 1. Open-Data: Man kann später anhand der Daten immer noch auswerten.
          \item 2. Für die einzelnen Fachschaften könnten sich die Brücken- und Vorkurse verbessern.
          \item 3. Erstellen von Forderungen durch die ZaPF, etwa wenn bestimmte Metadaten mit Problemen korreliert sind.
        \end{enumerate}

      \1 Peter Steinmüller - KIT Karlsruhe:  Um Fachschaften Vergleichbarkeit zu ermöglichen: Fragebogen vor und nach dem Vorkurs? \\
        $\rightarrow$ AS: Bearbeitet nicht Vorkurse, sondern Abiwissen. Ziel ist die Verbesserung von Vorkursen.

      \1 Johannes Hampp - U Tübingen:  Wie weit überlappt sich das Thema mit anderen Fragebögen (etwa aus BaMA-Umfrage)? Einladung an Verantwortlichen, um übwe weitere Fragebögen zu reden. \\
        $\rightarrow$ AS: Wird im Hinterkopf gehalten.
    \end{outline}

    \begin{success}{Abstimmung}
      Abstimmung über den Arbeitsauftrag: 33/0/3 (\textit{(Zustimmung/Ablehnung/Enthaltung)}) \\
      Der Arbeitsauftrag ist damit \textbf{angenommen}.
    \end{success}


  \subparagraph{Handreichung: Bachelorbörse und Bacheloranden Recruiting}
    Die Antragstellika (AS) stellen den Auftrag vor.

    \begin{itemize}
      \item Hauke Schäfer - TU Kaiserslautern:  Wie genau kann eine Handreichung in der GO eingeordnet werden? Die ist dort nicht definiert. \\
        $\rightarrow$ AS: Nicht genügend Inhalt für ein Positionspapier. \\
        $\rightarrow$ Redeleitung: Handreichungen existieren formal gesehen nicht. Sie sind für den internen Gebrauch gedacht.
      \item Peter Steinmüller - KIT Karlsruhe:  Handreichungen sind ok, aber die Abstimmung hat keine Auswirkungen. Etwa wird es nicht an die Fachschaften geschickt. \\
        $\rightarrow$ AS: Stellt hiermit einen \underline{Arbeitsauftrag an den StaPF}, dass die Handreichung verschickt werden soll.
      \item Elisabeth Schlottmann - TU Berlin:  Eigenständiges Verschicken an die FSen ist möglich, Text ist auch im Wiki verfügbar. \\
        $\rightarrow$ Maik: Ist halt schöner, wenn das durchs Plenum eine Berechtigung hat.
      \item Christian Birk - U Marburg:  Es gibt schlechte Erfahrungen in Marburg mit Teilen der Maßnahmen in der Handreichung. Soll diese noch überarbeitet werden? \\
        $\rightarrow$ Maik: Handreichungen sind nur für den internen Bereich und können an die eigenen Gegebenheiten angeglichen werden. Sollten zu starke negative Erfahrungen gemacht worden sein, kann man das auch gerne in einem weiteren AK bearbeiten.
    \end{itemize}

    \begin{success}{Abstimmung}
      Abstimmung über den Arbeitsauftrag: 34/0/2 (\textit{(Zustimmung/Ablehnung/Enthaltung)}) \\
      Der Arbeitsauftrag ist damit \textbf{angenommen}.
    \end{success}

  \subsection{Resolution: Streik der Studentischen Beschäftigten in Berlin (Antrag 11)}
    Die Antragstellika (AS) stellen den Auftrag vor. \\
    Auf der nächsten ZaPF wollen wir uns noch einmal allgemein zu Arbeitsbedingungen Studierender unterhalten.

    \begin{itemize}
      \item Patrick Haiber - ZaPF e.V.:  Was heißen die Abkürzungen (ASH und UdK)? In Antragsskizzen sollten keine Abkürzungen stehen.
        $\rightarrow$ AS: Alice-Salomon-Hochschule und Universität der Künste.
      \item Jonas Broleen - U Oldenburg:  Frage zum letzten Satz: Weshalb muss hier auf die Solidaritätserklärung verwiesen werden?
        $\rightarrow$ AS: Deswegen zuerst Resulotion, dann Solidaritätserklärung.
      \item Patrick Haiber - ZaPF e.V.:  Zu den Änderungsanträgen: "wir sind der Meinung" vermeiden, ersetze "Adressaten" durch "Verhandlungsführenden"
        $\rightarrow$ AS: Besser "die Verhandlungsführenden der Arbeitgeber", weil nicht die Studierenden zu Verzögerungen führen.
      \item \underline{AS nehmen die Änderungsanträge an}.
    \end{itemize}

    \begin{success}{Abstimmung}
      Abstimmung über die Resolution: 30/1/5 (\textit{(Zustimmung/Ablehnung/Enthaltung)}) \\
      Die Resolution ist damit \textbf{angenommen}.
    \end{success}

  \subsection{Resolution: Solidaritätserklärung zum Streik der Studentischen Beschäftigten in Berlin (Antrag 12)}
    Die Antragstellika (AS) stellen die Solidaritätserklärung vor. \\

    \begin{itemize}
      \item Jan Luca Naumann - TOPF:  Bitte um Ergänzung der Adressaten durch die Gewerkschaften. \\
        $\rightarrow$ AS: Wird übernommen.
      \item Weiterer Änderungsantrag (per Mail) wird ebenfalls übernommen.
    \end{itemize}

    \begin{success}{Abstimmung}
      Abstimmung über die Solidaritätserklärung: 33/1/3 (\textit{(Zustimmung/Ablehnung/Enthaltung)}) \\
      Die Solidaritätserklärung ist damit \textbf{angenommen}.
    \end{success}
    AS: Jeder kann sich auch einzeln solidarisch erklären, bitte tut das!
    auf \url{www.tvstud.de} ist dies möglich.

  \subsection{Positionspapier zum gültigen Studienakkreditierungsstaatsvertrag (SV) und der dazugehörigen Musterrechtsverordnung (MRVO) (Antrag 19)}
    Die Antragstellika (AS) stellen das Thema und die folgenden Anträge mit Hilfe eines Tafelbilds vor.
    Aktuelle Position der ZaPF insbes. für studentische Gutachter wird erläutert.

    \begin{itemize}
      \item Stefan Brackertz - U Köln:  ``ZaPF möchte weiter studentische Gutachter bei Begehungen'' - weshalb fordern wir das? \\
        $\rightarrow$ AS: Wir unterscheiden zwischen Begutachtung ung Begehungen. Begutachtungen können ohne Anwesenheit gemacht werden. Studierende können trotzdem entscheiden eine Begehung zu machen. Die Punkte sollen fetgehalten werden, dass sie bei der Überarbeung nicht verloren gehen.

      \item Stefan Brackertz - U Köln:  Geht es um System- vs. dezentrale Akkreditierung?
        $\rightarrow$ AS:
      \todo{Antowrt AS raussuchen}

      \item Stefan Brackertz - U Köln:  Welche Kritik an Rat und Agentur ist gemeint? \\
        $\rightarrow$ AS : Das bezieht sich auf das Positionspapier aus Siegen, weil die Formulierungen noch nicht klar genug definiert wurden.

      \item Peter Steinmüller - KIT Karlsruhe:  \underline{Änderungsantrag}: Wenn wir auf das alte Positionspapier verweisen, sollte das alte nicht aufgehoben, sondern ergänzt werden. (Erster Satz) \\
        $\rightarrow$ AS: Das aktuelle Papier beinhaltet auch widersprüchliche Punkte zum Papier aus Siegen. Kann deshalb nicht (nur) eine Ergänzung sein. Es gibt bereits viele Positionspapiere zur Akkreditierung, die wir nicht immer "mitschleppen" möchten. Offene Punkte aus Siegen sind hier enthalten. Die Position zur Akkreditierung wird in Würzburg noch erweitert werden. \\
        \underline{Einwurf}: Ändern zu "Aktualisiert"
          $\rightarrow$ AS: nehmen den Einwurf an.
      \item Jan Luca Naumann - FU Berlin:  \underline{Änderungsanträge}: "betroffene Fachschaft" zu "betroffene Studierendenschaft", sowie "die in Siegen [..]" zu "die im Positionspapier aus Siegen" ändern. \\
        $\rightarrow$ AS: Nimmt diese Änderungsanträge an.
      \item Patrick Walkowiak - U Bochum: Zweiter Punkt, kritisiert die Grenze von 8 Jahren. Wir fordern nun 5 Jahre. Früher haben die doch auch 5 Jahre als sinnvoll angesehen. Warum wurden denn nun 8 Jahre festgesetzt?
        $\rightarrow$ AS: Vermutung für die 8 Jahre:
        \begin{enumerate}
          \item Einfacher, wenn immer die gleiche Frist.
          \item Akkreditierungsrat entscheidet jetzt bundesweit $\rightarrow$ großer Aufwand.
        \end{enumerate}
        Für Studierende ist das schlecht, weil acht Jahre zwei Kohorten umfassen.
        Wir wollen unsere Kritikpunkte insgesamt im Bewusstsein behalten.
        Wenn wir das hier festschreiben, können unsere studentischen Begutachter fünf Jahre fordern.
        In der Resolution an die Länder können wir eine Konkretisierung fordern.
      \item Patrick Haiber - ZaPF e.V.: Imperative sind ungünstig. Deswegen \underline{Änderungsantrag} Z.170 - "soll" streichen. Dadurch wird es auch härter gefasst.
        $\rightarrow$ AS: Änderungsantrag wird angenommen.
    \end{itemize}

    \begin{success}{Abstimmung}
      Abstimmung über das Positionspapier: 36/0/2 (\textit{(Zustimmung/Ablehnung/Enthaltung)}) \\
      Das Positionspapier ist damit \textbf{angenommen}.
    \end{success}

    \todo{Antragsnummer raussuchen}
  \subsection{Resolution zum Ablauf des Akkreditierungsverfahren (Antrag )}
    Das Akkreditierungsverfahren ist noch nicht fest definiert. Um die Stakeholder zu stärken, wurden Vorschläge in der Resolution erarbeitet. Diese wird nun näher erläutert. \\
    Das Antragstellikon (AS) stellt bereits bekannte Änderungsanträge vor, die sie annimmt.

    \begin{itemize}
      \item Stefan Brackertz - U Köln: Frage zur Intransparenz der Besetzung der Gutachter. Diese sind ungewählt. Transparenz ist ein Schritt in die richtige Richtung, wie soll Legitimation sichergestellt werden? Wir machen das über die ZaPF, aber das ist nicht allgemein geregelt.
      AS: Nach der Gesetzlage soll die HRK eine Richtlinie zur Bestellung der Gutachter erstellen. Für Hochschullehrende gibt es ein Vorschlagsrecht der Fachbereiche (nach allgemein geltenden Befangenheitsregeln)
      Für die Berufspraxis gibt es Berufspraxisverbände, die dafür angesprochen werden sollen.
      Alle möglichen Studierendenvertretungen können Studierende in den Pool entsenden. Dadurch wächst dieser Pool immer weiter. Dies scheint sinnvoll.
      Wir wollen diese Richtlinie mitgestalten und bekommen über den Akkreditierungspool auch Rohfassungen.
    \end{itemize}

    \begin{success}{Abstimmung}
      Abstimmung über die Resolution: 33/0/3 (\textit{(Zustimmung/Ablehnung/Enthaltung)}) \\
      Die Resolution ist damit \textbf{angenommen}.
    \end{success}

  \subsection{Resolution: Länderspezifische Rechtsverordnungen als Spezifizierung der Musterrechtsverordnung (Antrag 18)}
    Musterrechtsverordnung ist nur ein Muster nach dem die Länder Rechtsverordnungen erstellen bzw. erstellt haben. Einige Punkte können hier geändert werden. Wir wollen hierfür Vorschläge machen.
    Rechtverordnung wird in den Landtagen erarbeitet, deshalb Landtagsfraktionen als Ziel der Resolution. \\

    Änderungsanträge werden \textbf{angenommen}.

    \begin{success}{Abstimmung}
      Abstimmung über die Resolution: 34/0/2 (\textit{(Zustimmung/Ablehnung/Enthaltung)}) \\
      Die Resolution ist damit \textbf{angenommen}.
    \end{success}

    \begin{info}{}
      Die Fachschaft der Uni Erlangen/Nürnberg verlässt das Plenum vorzeitig um \textbf{15:05} und gibt ihre Stimmkarte ab. Gute Heimfahrt!
    \end{info}

    AS: Musterrechtsverordnung im Wiki kommentiert mit den Beschlüssen der ZaPF, Diskussion auf der Winter-ZaPF in Würzburg.

    \paragraph{Einschub Uni Bonn}
      Die Uni Bonn halten einen Werbevortrag für die ZaPF in Bonn im Sommer 2019. Bonn liegt ziemlich weit links.

  \subsection{Resolution: Novellierung von Hochschulgesetzen (Antrag 14)}
    Die Antragstellika (AS) stellen ihre Resolution vor. Es gibt die Überlegung, verschiedene Resolutionen aus dem Papier zu machen. Allerdings wären sowohl Anlass als auch Adressaten identisch. \\
    Er geht über bisherige Positionspapiere hinaus, da teilweise die demokratischen Beteiligten in Gesetzesvorschlägen optionalisiert werden.

    \begin{itemize}
      \item Hauke Schäfer - TU Kaiserslautern:  Was heißt ``Wissenschaftadäquat''?
        $\rightarrow$ AS: soll eine Art der Zusammenarbeit sein, die für die Wissenschaft aus Sicht der Logik kompatibel ist. Beispielsweise werden Theorien nicht über Mehrheitsentscheide widerlegt.
      \item Hauke Schäfer - TU Kaiserslautern:  Könnte man das einfacher formulieren? Änderungsantrag kommt.
      \item Hauke Schäfer - TU Kaiserslautern:  Bedeutet Abgabe der Aufgabe ... nicht Einschränkung der Freiheit nicht weniger Druck? \\
\todo{Was genau wird gefragt?}
        $\rightarrow$ AS: Im AK wurde besprochen, dass Hochschulen von allen Seiten unter Druck stehen (Stichwort: Wahrheitsfindung). Etwa Druck der Rüstungsforschung, davor kann wohl eine Zivilklausel schützen. Außerdem schränkt das Bereiche ein, die staatlich finanziert werden müssen.
      \item Hauke Schäfer - TU Kaiserslautern:  Frage zur besonderen Verantwortung. Warum müssen Hochschulen ``in der Position sein''? \\
        $\rightarrow$ AS: Wörtliche Kopie aus Positionspapier aus Berlin. \\
        Hauke Schäfer - TU Kaiserslautern:  Das wird nicht klar, weil nicht da steht, dass das die Meinung der ZaPF ist.
        $\rightarrow$ AS: Vorschlag, einen einleitenden Satz aufzunehmen.
      \item Colin Heckmeyer - U Tübingen:  Gesellschaftliche Verantwortung im zweiten Absatz. Sind das Beispiele oder alle Punkte? \\
        $\rightarrow$ AS: Das sind Beispiele. "etc." kann ergänzt werden.
      \item Colin Heckmeyer - U Tübingen:  Nächster Absatz: Beginn mit RWTH Aachen. Ist das eine Begründung für die folgenden Ausführungen oder wo kommt das her? \underline{Änderungsantrag} wird erstellt.
        $\rightarrow$ AS: Stellen einen eigenen Änderungsantrag und nehmen diesen an.
      \item Hauke Schäfer - TU Kaiserslautern:  \underline{Änderungsantrag} \\
        $\rightarrow$ AS: Nehmen den Änderungsantrag an.
      \item Jan Luca Naumann - TOPF:  Adressaten: Was sind Landesasten, warum nicht Landesastenkonferenzen? \\
        $\rightarrow$ AS: Meinung des Plenums zu den Adressaten erwünscht. Aufgrund der Dringlichkeit: Verschicken an die Asten aller Hochschulen in NRW möglich? \\
        Niklas Donocik - TU Braunschweig:  Es ist zumutbar. \\
        Patrick Haiber - ZaPF e.V.:  Mit fünf Freiwilligen ist es jetzt während des Plenums machbar.
      \item Jan Luca Naumann - TOPF:  Welche Landesfraktionen sind gemeint? \\
        $\rightarrow$ Vanessa Fahrenschon - Alte Säcke:  Es gibt eine Liste der bildungspolitischen Sprecher aller Fraktionen.
      \item Jan Luca Naumann - TOPF:  \underline{Änderungsantrag}: Begriff von studentischen Hilfskräften spezifizieren.
        $\rightarrow$ AS: Änderungsantrag wird übernommen.
      \item Jan Luca Naumann - TOPF:  Statusgruppenvetorecht, Verweis auf frühere Position. Was verstehen wir darunter? \\
        $\rightarrow$ AS: Fußnote einfügen, bei der man auf Reso von 2013 verweist.
      \item Jakob Schneider - U Göttingen:  Beim Satz zur gesellschaftlichen Verantwortung, setzt ihr Geld und Freiheit gleich?
        $\rightarrow$ AS: Wie kommst du darauf?
      \item Jakob Schneider - U Göttingen:  Die Zivilklausel hat nichts direkt mit der Finanzierung des Lehrbetriebs zu tun. Es besteht kein Zwang zur Forschung an kriegsrelevanten Themen. \\
        $\rightarrow$ AS gibt als Beispiel NRW Landeskoordinierungsplan zur Hochschulentwicklung: höherer Studierendenzahlen und Bildungsauftrag, aber keinen Frieden $\rightarrow$ Verschiebung des Schwerpunktes (?)
      \item Jakob Schneider - U Göttingen:  \underline{Änderungsantrag}: Einfügen der Begründung, warum ein Streichen der Zivilklausel die genannten Folgen hat. \\
        $\rightarrow$ AS: Eigener Vorschlag, wird angenommen.
      \item Jakob Schneider - U Göttingen:  \underline{GO-Antrag} auf blockweise Abstimmung. \\
        inhaltliche Gegenrede: Abstimmungen funktionieren so.
    \end{itemize}

    \begin{danger}{Abstimmung}
      Abstimmung über den GO-Antrag: 4/27/5 (\textit{(Zustimmung/Ablehnung/Enthaltung)}) \\
      Der GO-Antrag ist damit \textbf{abgelehnt}.
    \end{danger}

    \begin{success}{Abstimmung}
      Abstimmung über die Resolution: 28/2/6 (\textit{(Zustimmung/Ablehnung/Enthaltung)}) \\
      Die Resolution ist damit \textbf{angenommen}.
    \end{success}

  \subsection{Resolution: Position der SHK Räte im neuen NRW Hochschulgesetz (Antrag 15)}
    Die Antragstellika (AS) stellen ihre Resolution vor.
    \begin{itemize}
      \item Jan Luca Naumann - FU Berlin:  Wirklich an alle Mitglieder des Landtages? Gibt es eine Mailliste?
        $\rightarrow$ AS: Es gibt eine Webseite mit allen Mitgliedern und deren E-Mailadresen. Aufwand hält sich also in Grenzen.

      \item Christian Birk - U Marburg:  Änderungsantrag dazu: deutlichere Formulierung: "verurteilen" statt "widersprechen". Eigene Formulierung des letzten Satzes.
        $\rightarrow$ AS: Unklar, wie das deutlicher ist.
      \item Christian Birk - U Marburg:  expliziter Aufruf gewünscht.
        $\rightarrow$ AS: "verurteilen" wird als Änderung angenommen.
      \item \underline{Zweiter Antrag wird zurückgezogen.}
      \item Andre Jakubowski - U Bonn:  Redeleitung sollte nicht den eigenen Antrag vorstellen. Sollte in Zukunft jemand anders übernehmen.
    \end{itemize}

    \begin{success}{Abstimmung}
      Abstimmung über die Resolution: 33/1/1 (\textit{(Zustimmung/Ablehnung/Enthaltung)}) \\
      Die Resolution ist damit \textbf{angenommen}.
    \end{success}

    \begin{info}{}
      Die Fachschaft der TU Doofmund verlässt das Plenum vorzeitig um \textbf{15:50} und gibt ihre Stimmkarte ab. Gute Heimfahrt!
    \end{info}

  \subsection{Resolution: Flexibler Umgang mit Prüfungsan- und -abmeldungen (Antrag 10)}
    Die Antragstellika (AS) stellen ihre Resolution vor.
    \begin{itemize}
      \item Martin Rößner - FU Berlin:  Frage zu Zeile 87.
      \item Patrick Walkowiak - U Bochum:  ``kurzfristig'' zu unspezifisch. \underline{Antrag} auf Festsetzung auf eine Woche. Grund dafür ist Stress von Studierenden, der zu Absagen führen kann.
      \item Fabian Freyer - TU Berlin:  Bedeutet vielerorts eine Verschlechterung der Zustände.
    \end{itemize}

    \begin{danger}{Abstimmung}
      Abstimmung über den Änderungsantrag: 2/26/2 (\textit{(Zustimmung/Ablehnung/Enthaltung)}) \\
      Der Änderungsantrag ist damit \textbf{abgelehnt}.
    \end{danger}

    \begin{success}{Abstimmung}
      Abstimmung über die Resolution: 26/2/4 (\textit{(Zustimmung/Ablehnung/Enthaltung)}) \\
      Die Resolution ist damit \textbf{angenommen}.
    \end{success}

    \begin{info}{}
      Die Fachschaft der Uni Bielefeld verlässt das Plenum vorzeitig um \textbf{16:03} und gibt ihre Stimmkarte ab. Gute Heimfahrt!
    \end{info}

  \subsection{Selbstverpflichtung: IT Konzept der ZaPF (Antrag 16)}
    Das Antragstellikon (AS) Jan Luca Naumann - FU Berlin:  stellt die Selbstverpflichtung und Änderungsanträge vor.
    \begin{itemize}
      \item Patrick Haiber - ZaPF e.V.:  Punkt 5. - Was heißt "in der Regel"?
        $\rightarrow$ AS: erfolgt in Absprache mit den Nutzern der Mailingliste
      \item Patrick Haiber - ZaPF e.V.:  Es geht also nur um die Teilnehmer der Mailingliste?
        $\rightarrow$ AS: ja, gilt für alle Mails einer Mailingliste
      \item Jonas Broleen - U Oldenburg:  1.punkt zweiter Satz. Daher setzt die ZaPF... zu setzen. Gedoppelt. Zweites setzen  wegenehmen. -- redaktionell
        $\rightarrow$ AS: wird nicht angenommen, "soll" ist schwächer als einfach Präsens.
      \item Jakob Brenner - LMU München:  redaktionelle Änderung
        $\rightarrow$ AS:
    \todo{Was wurde hier geantwortet?}
      \item Peter Steinmüller - KIT Karlsruhe:  was bedeutet öffentlich bei Mailinglisten?
        $\rightarrow$ AS öffentlich öffentlich - ohne Passwort
    \end{itemize}

    \begin{info}{}
      Die Fachschaften der Uni Freiburg und der FH Lübeck verlassen das Plenum vorzeitig um \textbf{16:10} und geben ihre Stimmkarten ab. Gute Heimfahrt!
    \end{info}

    \begin{success}{Abstimmung}
      Abstimmung über die Selbstverpflichtung: 29/0/3 (\textit{(Zustimmung/Ablehnung/Enthaltung)}) \\
      Die Selbstverpflichtung ist damit \textbf{angenommen}.
    \end{success}

  \subsection{Resolution: Konsolidierung der Aussagen der ZaPF zur Besetzung und Ausgestaltung von Professuren in der Physikdidaktik (Antrag 20)}
    Das Antragstellikon (AS) erklärt die Resolution und die Geschichte dazu. Er stellt Änderungsanträge vor, die er direkt übernimmt.

    \begin{info}{}
      Die Fachschaften der TU München und der Uni Augsburg verlassen das Plenum vorzeitig um \textbf{16:15} und geben ihre Stimmkarten ab. Gute Heimfahrt!
    \end{info}

    \begin{itemize}
      \item Peter Steinmüller - KIT Karlsruhe:  \underline{Änderungsantrag} für den ersten Satz, Konjunktiv entfernen.
        $\rightarrow$ AS übernimmt die Änderung.
      \item Matthias Sommer - U Duisburg-Essen:  ProfessorInnen sollen auf jeden Fall Praxiserfahrung haben. Praxiserfahrung in der Uni ist hier als Alternative widersinnig.
        $\rightarrow$ AS: Wir wünchen uns explizit Praxiserfahrung aus der Realität. Wenn er das nicht hat, soll er mindestens andere Praxiserfahrungen haben.
      \item Daniela Kern-Michler - U Frankfurt am Main:  Deutlicher: Schulpraxis bevorzugen, Satz als Änderungsvorschlag.
        $\rightarrow$ AS: wird angenommen.
    \end{itemize}

    \begin{success}{Abstimmung}
      Abstimmung über die Resolution: 23/0/3 (\textit{(Zustimmung/Ablehnung/Enthaltung)}) \\
      Die Resolution ist damit \textbf{angenommen}.
    \end{success}

  \subsection{Medienwechsel: Tobi stellt den Bier-Austausch-AK vor}
    Das Ergebnis der Verkostung und Bewertung der verschiedenen regionalen Bierspezialitäten hat ergeben: Bier ist geil!
    \begin{enumerate}
      \item Erlangen-Nürnberg: Schatzenbräu Rotbier
      \item Mühlen-Kölsch
      \item Bochum: Moritz
    \end{enumerate}

  \subsection{Resolution: Studierendenmobilität (Antrag 27)}
    Das Antragstellikon (AS) stellt die Resolution vor.
    \begin{itemize}
      \item Jan Luca Naumann - FU Berlin:  Sind Vertreter der Landesparteien gemeint?
        $\rightarrow$ AS: Können es gerne ändern auf ``Bildungspolitische Sprecher im Landtag''.
      \item Jan Luca Naumann - FU Berlin:  Bildungspolitische Sprecher der Bundesministerien?
        $\rightarrow$ AS: Sind mit Bezug auf das BMBF zu verstehen.
      \item Jan Luca Naumann - FU Berlin:  Sind alle Hochschulen gemeint oder alle auf der Liste des StaPF?
        $\rightarrow$ AS: Alle Hochschulen für die die ZaPF eine Liste hat. Es wird nicht verlangt, dass \textit{alle} Hochschulen gemeint sind; nur die Hochschulen, für welche die Adressen bekannt sind.
      \item AS: \underline{Redaktionelle Änderung}: Ersetze "Bologna Prozess" durch "Bologna Reform".
      \item Marcel Nitsch - U Bonn:  Nachfrage zu Punkt 3: Was ist die Aussage?
        $\rightarrow$ AS: Nach einem Hochschulwechsel müssen nicht alle Module der vorigen Hochschule eingebracht werden. Wir fordern, dass der Studierende sich hierzu heraussuchen kann, welche Leistungen überhaupt berücksichtigt werden für die Überprüfung zur Anrechenbarkeit.
      \item Marcel Nitsch - U Bonn:  Das heißt also das eine Prüfung, die ich anrechnen könnte beim Uniwechsel nochmal machen könnte und die bessere Note nehmen könnte?
        $\rightarrow$ AS: Das würde es bedeuten. Das wäre eine Lücke in dieser Forderung, in der der AS kein großes Problem sieht.
    \end{itemize}

    \begin{success}{Abstimmung}
      Abstimmung über die Resolution: 27/0/0 (\textit{(Zustimmung/Ablehnung/Enthaltung)}) \\
      Die Resolution ist damit \textbf{angenommen}.
    \end{success}

    Anmerkung des AS: Das Zitat in der Resolution ist eine Sekundärquelle (Wikipedia), das Orginal wird demnächst aus der Thüringer Landesbibliothek geholt. Dann ist redaktionell gegebenenfalls eine Anpassung notwendig.

    \paragraph{Es folge eine fünfminütige Pause.}

    \begin{success}{Feststellung der Beschlussfähigkeit}
      Es sind 27 Fachschaften anwesend und haben ihre Stimmkarte noch.
      Das Plenum ist damit \textbf{beschlussfähig}.
    \end{success}

  \subsection{Resolution zum Umgang mit Nullergebnissen}
    Das Antragstellikon (AS) stellt den Antrag vor. Es handelt sich um die englische Übersetzung der Resolution aus Siegen.
    \begin{itemize}
      \item \underline{Änderungsantrag}, da Teile beispielsweise zu Datenspeicherung nicht für Journals relevant sind.
      \item Merten Dahlkemper - KommGrem:  Journals werden große Player im Bereich Science-Tools, deswegen sollte der Abchnitt nicht gelöscht werden.
        $\rightarrow$ AS: Wenn es Argumente dafür gibt, diese darin stehen zu lassen, dann ist das für den AS in Ordnung.
      \item Hauke Schäfer - TU Kaiserslautern:  Vor allem der Begriff "Daten" hat bei den Journals gar nichts zu suchen. Die "Daten" sollten daher bei den Forschungsinstituten gelagert werden und nicht bei den Journals.
      \item Merten Dahlkemper - KommGrem:  Eigentlich sollten Daten immer beim Institut bleiben. Aber Journals werden in Zukunft (auch wegen Open-Science) immer mehr Einfluss dabei haben.
      Es hat mehr Nutzen, die Journals an dieser Stelle in der Resolution zu haben.
      \item Stefan Brackertz - U Köln:  Es sollte drin bleiben, damit Empfänger auch Einblick in das Gesamtkonzept erhalten.
    \end{itemize}

    \begin{danger}{Abstimmung}
      Abstimmung über den Änderungsantrag: 1/17/9 (\textit{(Zustimmung/Ablehnung/Enthaltung)}) \\
      Der Änderungsantrag ist damit \textbf{abgelehnt} und der zweite Aufzählungspunkt bleibt damit erhalten.
    \end{danger}

    \begin{itemize}
      \item Hauke Schäfer - TU Kaiserslautern:  Wird Grammatik und Rechtschreibung nochmals überprüft?
        $\rightarrow$ Niklas Donocik - TU Braunschweig:  Macht der StaPF allgemein für jede Veröffentlichung, auch bei englischen Veröffentlichungen. Die Idee des Konsolitieren eines Übersetzers wird aufgenommen.
      \item Einwurf: Wird erklärt, was die ZaPF ist?
        $\rightarrow$ Niklas Donocik - TU Braunschweig:  Ja, immer.
    \end{itemize}

    \begin{success}{Abstimmung}
      Abstimmung über das Positionspapier: 25/0/2 (\textit{(Zustimmung/Ablehnung/Enthaltung)}) \\
      Das Positionspapier ist damit \textbf{angenommen}.
    \end{success}

  \subsection{Positionspapier: Gegen Werbung auf dem Gelände und in Gebäuden von Hochschulen (Antrag 22)}
    Die Antragstellika (AS) stellen den Antrag vor. \\
    Möchte sich die ZaPF mit diesem Thema nochmals auseinandersetzen, ist ein tiefergehender AK auf folgenden ZaPFen notwendig.

    \begin{itemize}
      \item Christian Birk - U Marburg:  "Lehr- und Lehrräume" sollten vermutlich "Lehr- und Lernräume" sein?
        $\rightarrow$ AS: Angenommen.
      \item Jonas Lautenschläger - HU Berlin:  Ist das nicht sehr restriktiv? Eventuell könnten kleine Unternehmen davon abhängen (etwa Nachhilfeinstitute), wenn sie gar keine Werbung mehr machen dürfen.
        $\rightarrow$ AS: Gilt nur in Lehr- und Lernräumen. Orte wie schwarze Bretter sind in dem Positionspapier ausgenommen, genau wie auch Mensen. Das Positionspapier fokussiert sich rein auf ``innerhalb dieser Räume''.
      \item Jonas Lautenschläger - HU Berlin:  Was ist beispielsweise mit Schwarzen Brettern in Leseräumen von Bibliotheken?
        $\rightarrow$ AS: Es sind rein die Räume in welchen direkt gelernt und gelehrt wird.
      \item Daniela Kern-Michler - U Frankfurt am Main:  Immer oder nur während den Zeiten des Lehrbetriebs? Lässt sich dies konkretisieren, dass es nur während der Funktion des Raums zutrifft?
        $\rightarrow$ AS: Es steht schon explizit "Bei Lehrbetrieb" da.
      \item Daniela Kern-Michler - U Frankfurt am Main:  Verstehe den Absatz nicht so, dass bei anderen Veranstaltungen (Nicht-Lehr-Veranstaltungen) in den gleichen Räumen eine Ausnahme gilt.
        $\rightarrow$ AS: Vorschlag für eine passende Formulierung?
      \item Daniela Kern-Michler - U Frankfurt am Main:  Schlaf, dann dürfte die Formulierung in aktueller Form doch passen.
      \item Florian Tönnies - U Freiburg:  Bedenkt, Werbung heißt viel mehr als Plakate oder ähnliches. Es könnte auch subtiler sein als die Beispiele im Text. Sind die Folgen und die mögliche Bedeutung des kompletten Verbots von "Werbung" (auch nur ansatzweiße) bewusst?
    \end{itemize}

    \begin{success}{Abstimmung}
      Abstimmung über das Positionspapier: 21/3/4 (\textit{(Zustimmung/Ablehnung/Enthaltung)}) \\
      Das Positionspapier ist damit \textbf{angenommen}.
    \end{success}

    \subsection{Arbeitsauftrag: Beschlussdatenbank der ZaPF (Antrag 25)}
    Die Antragstellika (AS) stellen den Antrag vor.
    Ziel ist das Verhindern von redundanten Arbeiten zu Themen auf konsektuiven ZaPFen durch Einführen einer aktuellen und zum Arbeiten verwendbaren Beschlussdatenbank, damit Beschlüsse nicht wiederholt getroffen werden (müssen).
    Dazu soll nach erster Überarbeitung auch ein Dauer-AK, etwa im Backup-AK-Slot, eingeführt werden, um die Pflege und den aktuellen Stand zu gewährleisten.

    \begin{itemize}
      \item Niklas Donocik - TU Braunschweig:  Was heißt ``Arbeitskreis einrichten''? Heißt das, er muss das machen, wenn er niemanden findet, der ihn leitet?
        $\rightarrow$ AS: Arbeitsauftrag ist, dass in Zukunft ein AK im Backup-AK-Slot stattfindet. Der soll nachbereiten, überarbeiten und vorarbeiten in Hinsicht auf zu erwartende Resolutionen.
      \item Niklas Donocik - TU Braunschweig:  Soll der StaPF aktiv jemanden suchen, übernehmen oder nur in die Tabelle schreiben?
      \item Tobias Guttenberger - U Bonn:  Soll also praktisch eine eigene Wiki-Kategorie werden? Wie notwendig ist das wirklich? Es scheint nur deutlich mehr Arbeit zu machen und wiederholt Angesprochenes, verlangsamt die ZaPF, aber bringt scheinbar keine Vorteile.
        $\rightarrow$ AS: Nein, das sollte uns viel Arbeit abnehmen. Teilnehmer können in Kategorien schon mal recherchieren.
      Insbesondere hinsichtlich Zäpfchen-AKs wäre das wertvoll. Die einzige Arbeit, die in Zukunft dann anfällt, ist die Arbeit des StaPFes initial und der fortwährenden Aktualisierung. \\
        Thomas Rudzki - U Heidelberg:  Die Argumente scheinen klar. Es handelt sich um zwei verschiedene Meinungen.
      \item Colin Heckmeyer - U Tübingen:  Vielleicht spart man keine Zeit, aber vereinfacht (insbesondere bei langkettigen Nachfolge-AKs) und  die Einarbeitungszeit, besonders wenn eine anfängliche Einarbeitung zur Regel wird.
      \item Tobias Guttenberger - U Bonn:  Antrag wird zurück gezogen.
      \item Peter Steinmüller - KIT Karlsruhe:  Zwei Dinge: Pflege des Wiki ist Aufgabe des StaPFes. Deswegen ist die Forderung des AKs klar.
      Versteht den Antrag derart, dass es klares Format über das Aussehen eines Wiki-Eintrages gibt. Sollten Seiten nicht diesem Format entsprechen, dann übernimmt der ZaPF das Anpassen an das Format (selbst oder durch den AK-Leiter).
      \item Niklas Donocik - TU Braunschweig:  Möchte einfach nur klare Anweisungen vom Plenum.
        $\rightarrow$ AS: Änderung zu einem freiwilligen Angebot, dass empfohlen ist, dass dieses angeboten wird, aber nicht zwangsweiße auf jeder ZaPF.
    \end{itemize}

    \begin{success}{Abstimmung}
      Abstimmung über das Positionspapier: 26/0/3 (\textit{(Zustimmung/Ablehnung/Enthaltung)}) \\
      Das Positionspapier ist damit \textbf{angenommen}.
    \end{success}

  \subsection{Arbeitsauftrag: Frauen in Wissenschaftlicher Karriere (Antrag 26)}
    Die Antragstellika (AS) stellen den Antrag vor.
    Ziel soll das Einladen eines Referenten für die nächste ZaPF sein, Einladung vom StaPF in Rücksprache mit der AK-Leitung.

    \begin{success}{Abstimmung}
      Abstimmung über das Positionspapier: 27/0/2 (\textit{(Zustimmung/Ablehnung/Enthaltung)}) \\
      Das Positionspapier ist damit \textbf{angenommen}.
    \end{success}

\section{AK-Ergebnisse}

  \subsection{Gremienworkshop}
    Daniela Kern-Michler - U Frankfurt am Main:  Unterlagen liegen nicht im Wiki, sind aber auf Anfrage bei Tobi und Daniela anfragbar.

  \subsection{Opa erzählt vom Krieg}
    Manuel Lohoff - U Marburg:  AK wird wieder in Würzburg statt finden.

  \subsection{Wissenschaftsrecht}
    Manuel Lohoff - U Marburg:  Der AK ist ganz gut gelaufen, wird in Würzburg auch wiederholt.
    Es ging um die Rechtssprechung des Bundesverfassungsgerichts bezüglich der Forschungsfreiheit, als Beispiel speziell um Ordinarien- vs. Gruppenuniversität.

  \subsection{E-Learning}
    Jakob Brenner - LMU München:  Fand das erste Mal statt und war für reinen Austausch.
    Für Würzburg ist eine Materialsammlung geplant und Fortführung des AK.

  \subsection{Studienführer}
    Peter Steinmüller - KIT Karlsruhe:  Es waren sehr viele ZaPFika anwesend, das war super. \\
    Fahrplan bis nach Würzburg existiert.
    Ergebnisse werden in saubere Form gebracht, nach Würzburg soll eine Drittfirma mit der Entwicklung beauftragt werden.
    Zeitgleich werden Sponsoren für die Finanzierung gesucht.
    In Würzburg kommt eventuell eine fachschaftsnahe Fachperson vorbei.

  \subsection{Open-Science}
    Merten Dahlkemper - KommGrem:  Austausch über mögliche Themen und Roadmap für zukünftige ZaPFen aufgestellt. Hierzu werden unter anderem auch Fachvertreter eingeladen und Dokumente erstellen.

  \subsection{Was ist eigentlich die jDPG und was will die von mir?}
    Merten Dahlkemper - KommGrem:  Ein Entwurf für eine Handreichung wurde erstellt. In Würzburg soll der AK mit der jDPG gleichzeitig stattfinden.

  \subsection{BaMa-Umfrage}
    Merten Dahlkemper - KommGrem:  Danke für alle, die schon teilgenommen und Feedback gegeben haben.
    Nächstes Arbeitstreffen findet am 11.06. im StaPF-Mumble statt.
    Im Anschluss geht die Umfrage los, Fachschaften werden darüber nochmals informiert.

  \begin{info}{}
    Die Fachschaft der Uni Duisburg/Essen verlässt das Plenum vorzeitig um \textbf{17:25} und gibt ihre Stimmkarte ab. Gute Heimfahrt!
  \end{info}

  \subsection{Alumni}
    Elisabeth Schlottmann - TU Berlin:  Eine Satzungsänderung wurde besprochen, das Thema kommt voran. Zwischen den ZaPFen wird weitergearbeit.

  \subsection{Wahlprozedere Vertrauenspersonen}
    Elisabeth Schlottmann - TU Berlin:  Es soll auf der nächsten ZaPF nochmal einen AK geben.
    Handreichung etwa zur besseren Sichtbarkeit soll erstellt werden.
    Vorschläge für die nächste ZaPF-Orga werden ebenfalls gegeben und die Orga so unterstützt.
    In Würzburg gibt es vielleicht auch eine Fortbildung zum Thema "Opfer und Täterschutz".

  \subsection{AK Bachelor-Börse}
    Benedikt Bieringer - U Münster:  Viele Dinge auch außerhalb der Handreichung wurden diskutiert.
    Das Protokoll enthält mehr als die Handreichung, es ist alles sehr interessant.

  \subsection{Hochschuldidaktik}
    Stefan Brackertz - U Köln:  Ideensammlung im AK. Leute, die sich an Arbeit beteiligen wollen, melden sich bitte bei Fachschaft der Uni Köln.

  \subsection{Bibliotheks- und Raumentwicklung}
    Stefan Brackertz - U Köln:  Digitalisierung hat nicht nur online Folgen, sondern auch an Bibliotheken. Eine Erkentniss: Bib

  \begin{info}{}
    Die Fachschaft der Uni Oldenburg verlässt das Plenum vorzeitig um \textbf{17:30} und gibt ihre Stimmkarte ab. Gute Heimfahrt!
  \end{info}

\section{Sonstiges}
    Tobias Löffler - U Düsseldorf:  Es gibt eine Rückmeldung von Margreths Hochzeit - Sie hat sich sehr gefreut und war sehr gerührt. \\ \\

    Torsten Umlauf - U Würzburg:  ZaPF-Couchsurfing Liste wird gemacht und in der kommenden Woche verschickt.

  \subsection{Kleinigkeiten von der Orga}
    Das Mörderspiel hat Bekka gewonnen. \\

    War wirklich, wirklich schöne ZaPF. Es ist komisch, dass es schon vorbei ist.
    Vielen Dank an Teilnehmika. \\

    \begin{center}
      \textit{"Ich weiß nicht, ob ich nochmals zu ZaPFen gehe."} \\
      \textit{"Applaudiert euch lieber nochmal selber!"} \\
      $\Rightarrow$ Standing Ovations
    \end{center}

    \begin{success}{Feststellung der Beschlussfähigkeit}
      Es sind noch 27 Fachschaften anwesend.
      Das Plenum ist damit immer noch **beschlussfähig**.
      \begin{center}
        \textbf{Was für ein Stehvermögen diese geilen ZaPFika haben!}
      \end{center}
    \end{success}

    Stimmkarten können mitgenommen werden. Bitte nehmt alles mit, damit Heidelberg weniger Arbeit beim Aufräumen hat. \\

    \textbf{Vielen Dank an Redeleitung und Protokollführer.} \\ \\

    \textbf{Ende des Plenums}: 17:35 Uhr
