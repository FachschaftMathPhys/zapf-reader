% !TEX TS-program = pdflatex
% !TEX encoding = UTF-8 Unicode
% !TEX ROOT = ../main.tex
\newpage
\subsection{Molibität zwischen Hochschulen}

Der Leitgedanke der Bologna-Reform ist es, die inter- und intranationale
Mobilität der Studierenden zu fördern. Besonders im Vordergrund steht die
"Förderung der Mobilität durch Überwindung der Hindernisse, die der
Freizügigkeit in der Praxis im Wege stehen"\footnote{Zitat: Der Europäische Hochschulraum – Gemeinsame Erklärung der Europäischen Bildungsminister, 19. Juni 1999, Bologna}.  Dieses Ziel wird in
Deutschland aus diversen Gründen nicht erreicht.

Unter Anderem wird bereits am 15. November 2010 auf einer
Podiumsdiskussion zum Thema "Reform des Bologna-Prozesses als Vorraussetzung
für innovative und kreative Ausbildung in Europa"\footnote{Zitat: Claire Weiß, Tim Wiewiorra: \textit{Reform des Bologna-Prozesses als Voraussetzung
  für innovative und kreative Ausbildung in Europa}. In: Europäisches
  Informations-Zentrum in der Thüringer Staatskanzlei:
  \textit{Reform des Bologna-Prozesses an deutschen Hochschulen als Voraussetzung für
  innovative und kreative Ausbildung in Europa}. Erfurt 2011, S. 105} im Europäischen
Informations-Zentrum in der Thüringer Staatskanzlei festgestellt:
\begin{displayquote}
 Selbst ein einfacher Standortwechsel in Deutschland wird, auch auf Grund des
 Bildungsföderalismus, oft durch die engen Modulpläne der einzelnen
 Universitäten oder Hochschulen verhindert.
\end{displayquote}
Weiterhin bestehen an einigen Hochschulen formale Gründe (u.A.: Zugangs- und
Zulassungssatzungen bzw. -ordnungen, Landeshochschulgesetze), die einen
Hochschulwechsel, insbesondere innerhalb eines Studiums, verhindern. Es entsteht
z.B. ein Konflikt wenn eine Rückstufung unmöglich ist\footnote{es ist nicht
möglich, mehrfach das selbe Fachsemester zu studieren} und eine
Leistungsbasierte Einstufung\footnote{die erbrachten Leistungspunkte nach ECTS
bestimmen das Fachsemester} erfolgen soll. Eine Einstufung in ein zu niedriges
Fachsemester verhindert hier eine Immatrikulation und damit einen
Hochschulwechsel. Dies steht in direktem Widerspruch zu den Leitgedanken des
Bologna-Prozesses.
Ein Hochschulwechsel innerhalb eines Studienganges verlängert nahezu zwingend
die Studiendauer. Grund hierfür ist vor allem die unterschiedliche Bewertung der
einzelnen Module sowie die zu begrüßende unterschiedliche Schwerpunktsetzung der
Hochschulen. Dies stellt in Verbindung mit Studienhöchstdauern eine erhebliche
Hürde in der Studierendenmobilität im Sinne der Bologna-Reform dar.
\newpage
Hinzu kommen oft Bedingungen zu Mindestleistungen an der Zieluniversität, z.B.
dass die Hälfte der Leistungen an der abschlussgebenden Hochschule erbracht
werden muss. Dies verhindert bei einem Hochschulwechsel bei der oft
erforderlichen Anerkennung aller vorherigen Leistungen einen Abschluss.

Für eine vollständige Umsetzung der Bologna-Reform ist die Gewährleistung der
Mobilität unabdingbar. Konkret bedeutet dies:
\begin{itemize}

\item In Fällen, in denen eine Immatrikulation nicht möglich ist, da der Studierende
  nach bestehenden Regelungen in ein zu niedriges Fachsemester einzustufen wäre,
  ist eine Einstufung in das nächsthöhere Fachsemester vorzunehmen. Ist eine
  Rückstufung formal möglich, ist eine Einstufung nach ECTS vorzunehmen.
\item Bestehen unglücklicherweise Höchststudiendauern oder andere Zwangsbedingungen,
  ist ein Hochschulwechsel als Begründung für Toleranzsemester oder andere
  Härtefallregelungen anzusehen.
\item Es muss der Entscheidung der oder des Studierenden obliegen, welche Leistungen
  zur Anerkennung der Zieluniversität zur Verfügung stehen. Ist dies formal
  nicht möglich und steht eine Regelung zur Mindestleistung an der
  Zieluniversität einem Abschluss im Weg, so ist eine Regelung zu finden, die
  den erfolgreichen Studienabschluss ermöglicht.
\item Die Akkreditierungsagenturen sowie der Studentische Akkreditierungspool sowie die Qualitätsmanagementsysteme der Hochschulen werden
  gebeten, bei der Akkreditierung darauf zu achten, Mobilität dadurch zu
  fördern\footnote{Zitat: §12 (1) Musterrechtsverordnung gemäß §4 (1-4)
    Studienakkreditierungsstaatsvertrag, Beschluss der KMK vom 7.12.2017}, dass diese Mobilitätshürden abgebaut werden.
 \end{itemize}
