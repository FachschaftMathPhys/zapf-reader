% !TEX TS-program = pdflatex
% !TEX encoding = UTF-8 Unicode
% !TEX ROOT = main.tex

\subsection{Zur Entwicklung des Ablaufs für Akkreditierungsverfahren}
Im Studienakkrediterungsstaatsvertrag und in der Musterrechtsverordnung (MRVO) in ihrer aktuellen Fassung (Staatsvertrag vom 7.2017, MRVO vom 7.12.2017) werden die Richtlinien für die formalen Kriterien des Ablaufs von Akkreditierungsverfahren geregelt. Darunter fällt auch die Benennung der externen Gutachter*innen (§3 des Staatsvertrags) und die Aufgabenverteilung verschiedener Akteure während des Verfahrens (MRVO §§ 24, 27, 28).

Die ZaPF fordert hierfür, dass das zu entwickelnde Verfahren für die Benennung der externen Gutachter*innen die Benennung für alle Statusgruppen regelt (Staatsvertrag §§ 3 (2) Punkt 5 und 3 (3)). Von entscheidender Bedeutung für die Qualität der Begutachtung ist hierbei eine in Akkreditierung durch Erfahrung oder entsprechende Fortbildung geschulte Gutachtergruppe. Die ZaPF fordert daher über MRVO § 25 (3) hinaus, dass alle Gutachter*innen über eine solche Befähigung verfügen sollen. Für die studentischen Gutachter*innen empfiehlt die ZaPF, das Angebot des studentischen Akkreditierungspools zu nutzen.

Weiterhin fordert die ZaPF mehr Transparenz in den Abläufen der Verfahren. Dabei sollen insbesondere Rückkopplungsmechanismen zwischen Agenturen, Gutachtergremien und dem Akkreditierungsrat formalisiert und veröffentlicht werden (vor allem bezüglich der Aufgaben in den §§ 22, 27, 28 MRVO).

Die weitere
Entwicklung und Evaluation der Verfahrensabläufe soll unter studentischer Beteiligung stattfinden.
