% !TEX TS-program = pdflatex
% !TEX encoding = UTF-8 Unicode
% !TEX ROOT = main.tex

\subsection{About the handling of null results}
The ZaPF views null results [1] as natural byproduct of proper scientific research. As such, they are not waste, but have scientific value worth protecting and preserving. Even though they are not the final conclusion to a topic, they may be of valuable help to future projects. We want to further their recognition as a product of thorough scientific research.

Of particular importance to this goal is the scientific community's proper access to null results. Thereby, scientists can profit from experiences made by their colleagues and avoid following the same inconclusive paths. This saves resources and is therefore in the interest of every participant in the research process.

The handling of null results should be discussed during the planning and preparation of scientific projects and the development of suitable procedures included into the regulations of funding associations. In this way the publication of null results can be established as a part of everyday research practice in the long term.

To accomplish these goals, the ZaPF proposes the following measures:
\begin{itemize}
  \item Inclusion of information about null results obtained during a project in the appendix of related publications. This would allow to simultaneously research the current scientific state of the art and the problems regarding its realization.
  \item Establishment of infrastructure providing services to store and share data that may be of value to the scientific community after the termination of a project regardless of whether it is raw data or processed in any way to multiple institutions.
\end{itemize}
\newpage
[1] The ZaPF defines null results as follows: a result of scientific research, fulfilling one of the following criteria:
\begin{itemize}
  \item falsification of the original working hypothesis,
  \item ambiguous or inconclusive result,
  \item or a result of small relevance not necessarily pertaining to the current work obtained during the creation of a publication ("trial and error"),
    the only prerequisite being that the results are obtained maintaining proper scientific standards.
\end{itemize}
