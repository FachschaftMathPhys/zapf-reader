% !TEX TS-program = pdflatex
% !TEX encoding = UTF-8 Unicode
% !TEX ROOT = main.tex

\subsection*{Novellierung der Hochschulgesetze}

Im Rahmen der laufenden Hochschulgesetz-Novellierungen in mehreren Bundesländern hält die ZaPF die folgenden Punkten für besonders relevant und nimmt wie folgt Stellung:

\textbf{Gremien}

Grundsätzlich ist es falsch, wenn eine Statusgruppe in einem demokratischen Gremium automatisch die Mehrheit besitzt. Vielmehr ist es notwendig, dass keine Position übergangen werden kann. Dies kann etwa durch eine paritätische Zusammensetzung oder ein Statusgruppen-Vetorecht\footnote{Zur Spezifizierung siehe Resolution aus Jena: } %\url{https://zapf.wiki/Sammlung_aller_Resolutionen_und_Positionspapiere#Positionspapier_zur_demokratischen_Mitgestaltung_in_Hochschulgremien}
sicher gestellt werden. Voraussetzung dafür ist, dass die Teilhaberechte Aller gesetzlich sichergestellt sind und nicht nur optional gewährt werden.

Dass allen Bedenken ernsthaft Rechnung getragen wird, ist Voraussetzung für qualitätsvolle und langfristige Lösungen, die von allen Beteiligten getragen und nicht nur aus Pflicht ausgeführt werden. So zeigt etwa die Erfahrung, dass in Studiengängen, die von Anfang an unter Einbeziehung der Studierenden geplant wurden, Zwangsmaßnahmen wie Anwesenheitspflichten nicht nötig sind.

Es entspricht guter wissenschaftlicher Praxis, argumentativ darüber zu streiten, was richtig und sinnvoll ist. Entscheidungen über reine Mehrheitsabstimmungen übergehen die Möglichkeit, einen Konsens zu finden oder produktiv mit einem unüberbrückbaren Dissens umzugehen.
\newpage
\textbf{Personalvertretung}

\textbf{Alle} Arbeitnehmer*innen müssen durch eine vollwertige, gesetzlich verankerte Personalvertretung repräsentiert werden, welche derzeit oft für Hilfskräfte/ studentische Beschäftigte nicht existiert. Gerade wenn Hochschulen wissentlich oder unwissentlich gegen geltendes Arbeitsrecht verstoßen, brauchen die oftmals prekär angestellten Hilfskräfte/studentische Beschäftigte eine Vertretung, die effektiv gegen Missstände vorgehen kann.

\textbf{Studienverlaufsvereinbarungen}

Zur Entwicklung persönlicher und fachlicher Kompetenzen muss ein selbstverantwortliches und interessengeleitetes Studium ermöglicht sein. Um Willkür bei der Aushandlung der Studienverlaufsvereinbarung und Verschulung im Studium zu verhindern, spricht sich die ZaPF gegen die Ermöglichung von verbindlichen Studienverlaufsvereinbarungen aus.

Die damit indirekt angedrohte Zwangsexmatrikulation legt ein absicherungs- statt entwicklungsorientiertes Studium nahe. Die Drohung mit dem Ausschluss vom Studiums untergräbt zudem eine vertrauensvolle Studienberatung, da so keine ehrlich Kommunikation über die wirklichen Probleme von Studierenden möglich ist. Studienverlaufsvereinbarungen sind daher nicht geeignet, Abbrecherquoten positiv zu beeinflussen, und verhindern ein freies Studium, welches auch Blicke über den Tellerand und Interdisziplinariät ermöglicht.

Anstatt Studierende, die unerfreuliche Studienverläufe haben, individuell unter Druck zu setzen, sollten strukturell die Bedingungen verbessert werden.

\textbf{Gesellschaftliche Verantwortung}\footnote{\url{https://zapfev.de/resolutionen/sose17/gesellschaftlich_verantwortung/PosPapier_gesellschaftliche_verwantwortung.pdf}}

Es ist nicht optional, sondern notwendig, dass die Hochschulen einen Beitrag zu einer gerechten, nachhaltigen, friedlichen und demokratischen Welt leisten und ihrer besonderen Verantwortung für eine nachhaltige Entwicklung nach innen und außen nachkommen.

Hochschulen müssen in der Position sein, zu Aufklärung über Falschdarstellungen, Kriegsursachen und -profiteure, etc. beizutragen, sowie an – nicht ergriffenen und noch zu entwickelnden – zivilen Möglichkeiten zum Beispiel zur Lösung von Ressourcenkonflikten zu forschen. Dieser Funktion können Hochschulen nur nachkommen, wenn ihre Unabhängigkeit gewahrt ist und ihnen ausreichende Mittel zur Erfüllung dieser Aufgaben zur Verfügung stehen.
Insbesonder ist eine Verankerung dieser Aufgaben in den Hochschulgesetzen dafür unabdingbar. Nur so ist sicher gestellt, dass die Landesregierungen verbindlich die Verantwortung dafür übernehmen, den Hochschulen die nötigen Rahmenbedingungen zur Verfügung zu stellen.
Die Bedeutung wird z.B. daran deutlich, dass die RWTH Aachen vor kurzem ein Drittmitellprojekt abbrach, bei dem es um eine Machbarkeitsstudie für ein Werk für Militärfahrzeuge in der Türkei ging. Sie betonte dabei explizit, dass sie in dieser Entscheidung durch die Friedensklausel im NRW-Hochschulgesetz bestärkt wurde \footnote{\url{http://www.rwth-aachen.de/cms/root/Die-RWTH/Aktuell/Pressemitteilungen/September-2017/~oktv/Statement-der-RWTH-Aachen-zur-Machbarkei/}}. Eine Streichung dieser Klausel, wie sie momentan geplant ist, bedeutet nicht mehr Freiheit für die Hochschulen, sondern einen erhöhten Druck auch inhumanen Vorhaben zuzuarbeiten.\\
Auch angesichts der strukturellen Unterfinanzierung der Hochschulen bedeutet die Streichung dieser Klausel nicht mehr Freiheit für die Hochschulen, sondern einen erhöhten Druck auch inhumanen Vorhaben zuzuarbeiten.
