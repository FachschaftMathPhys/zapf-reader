% !TEX TS-program = pdflatex
% !TEX encoding = UTF-8 Unicode
% !TEX ROOT = main.tex

\subsection*{Zur Besetzung und Ausgestaltung von Professuren in der Physikdidaktik}
Die ZaPF zieht folgende Stellungnahmen zur Besetzung von Fachdidaktikprofessuren zurück:
\begin{itemize}
\item die Stellungnahme zum Thema "`Fachdidaktikprofessuren"' vom 17.11.2013, verabschiedet in Wien,
\item die "`Ergänzung zur Stellungnahme der Zusammenkunft aller Physik-Fachschaften zu Fachdidaktikprofessuren"' vom 01.06.2014, verabschiedet in Düsseldorf und
\item die Resolution zum selbigen Thema vom 13.11.2016, verabschiedet in Dresden
\end{itemize}
und ersetzt sie durch folgende, konsolidierte Fassung:
\vspace{0.5\baselineskip}\\
Die ZaPF bekräftigt ihre bereits in der \textit{Empfehlung der ZaPF und der jDPG zur Ausgestaltung der Lehramtstudiengänge im Fach Physik}
(verabschiedet am 16.05.2010 in Frankfurt)\footnote{https://zapfev.de/resolutionen/sose10/Lehramtstellungnahme.pdf} zum Ausdruck gebrachte Position, dass an jeder Universität, die Lehrer*innen für das Fach Physik ausbildet, eine Professur für die Fachdidaktik der Physik existieren soll.
\vspace{0.5\baselineskip}\\
\textbf{Zuständigkeiten und Verantwortungen der Fachdidaktik}\\
Der/ Die Inhaber*in dieser Professur soll sich für die Betreuung, Begleitung und Qualitäts\-sicherung der Unterrichts- sowie Experimentierpraktika\footnote{das bezieht sich nicht auf die Fachpraktika der Physik} und der fachdidaktischen Veranstaltungen sowie die  Betreuung von Abschlussarbeiten im Rahmen der Prüfungsordnung verantwortlich zeichnen.\\
Allgemein soll die Fachdidaktik sowohl mit der allgemeinen Erziehungswissenschaft, als auch mit der Fachwissenschaft (Physik) vernetzen und bei der Modul-/ Inhaltsplanung der Fachphysik für Studierende des Lehramts mitwirken.\\\newpage
Im Sinne der \textit{Einheit von Forschung und Lehre} müssen diese Aufgaben zeitlich mit der fachdidaktischen Forschung abgestimmt werden; einige dieser Aufgaben müssen daher sicherlich aus zeitlichen Gründen von Lehrbeauftragten übernommen werden.
Für diese Stellen sind Lehrende mit eigener Praxiserfahrung im Schulbereich (z. B. abgestellte Lehrer) erforderlich.
\vspace{0.5\baselineskip}\\
\textbf{Praxiserfahrung der Bewerber*innen auf eine Professur}\\
Die ZaPF fordert, dass in den Berufungskommissionen für Stellen in der Fachdidaktik ausdrücklich auf die bisherige Praxiserfahrung der Bewerber eingegangen wird. Bewerber mit Praxiserfahrungen in der Schule sind zu bevorzugen.\\
Solche Praxiserfahrung kann neben der Lehre in der Schule (bspw. Referendariat) z. B. in Schülerlaboren, bei museumspädagogischen Tätigkeiten (mit Bezug zur Physik), in Planetarien, oder auch im universitären Kontext, wie z. B. bei der Betreuung von Nebenfachpraktika erfolgt sein.\\
Der fortwährende Praxisbezug soll in der Lehrtätigkeit sichergestellt sein.
\vspace{0.5\baselineskip}\\
\textbf{Akademische Voraussetzung}\\
Für die Berufung muss eine Promotion in einem physikalischem Fach oder in der Physikdidaktik vorliegen.\\
Außerdem halten wir Erfahrung in der didaktischen Forschung und Lehre, sofern sie nicht schon in der Promotion/ Praxistätigkeit erfolgt ist, für unabdingbar.
