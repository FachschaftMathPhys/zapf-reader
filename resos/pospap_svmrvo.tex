% !TEX TS-program = pdflatex
% !TEX encoding = UTF-8 Unicode
% !TEX ROOT = main.tex

\subsection*{Zum gültigen Studienakkreditierungsstaatsvertrag und der dazugehörigen Musterrrechtsverordnung}
Dieses Positionspapier aktualisiert das Positionspapier \glqq Zu Änderungen im Akkreditierungssystem\grqq\footnote{\url{https://zapfev.de/resolutionen/wise17/Akkreditierung\_PosPap/Pospap\_Akkreditierung.pdf}}, welches im Wintersemester 2017 auf der ZaPF in Siegen erarbeitet wurde. Hierbei wird nun auf die aktuelle Fassung (Staatsvertrag vom Juni 2017, Musterrrechtsverordnung vom 7.12.2017) der betreffenden Dokumente eingegangen, welche auf der ZaPF in Siegen noch nicht vollständig beschlossen waren. Neben neuen Punkten, insbesondere zum Staatsvertrag, werden Punkte des letzten Positionspapiers erneut aufgegriffen.

\paragraph{Punkte zum Staatsvertrag}
\begin{itemize}
  \item Die ZaPF begrüßt, dass alle auf Grundlage des Staatsvertrags akkreditierten Studiengänge in Deutschland bundesweit als gleichwertig qualitätsgesichert anerkannt werden (vgl. § 1 (3))
  \item In § 3 (2) wird der Ablauf von System- und Programmakkreditierungsverfahren geregelt, wobei alternative Verfahren (nach § 3 (1) Punkt 1) ausgenommen sind. Allerdings hält die ZaPF es für notwendig, dass folgende Punkte auch für alternative Verfahren explizit gelten: %***steichen->sollen<:
  \begin{itemize}
      \item die studentische Beteiligung am Verfahren muss sichergestellt sein
      \item eine Begutachtung mit Vor-Ort Begehung durch externe Gutachter*innen soll in jedem Fall stattfinden
      \item bei der Konzeption von Studiengängen muss die Rücksprache mit den Uni-internen Gremien, insbesonderen unter studentischer Beteiligung, sicher gestellt sein
   \end{itemize}
      Die Vorschrift für Begehungen in allen Verfahren ist im Moment in §24 der Musterrechtsverordnung geregelt. Die Aufnahme einer verpflichtenden Begehung in den Staatsvertrag in §3 (2) finden wir für alle Verfahren wünschenswert.
  \item Zur Erarbeitung von Richtlinien zur Gutachter*innenbestellung durch die Hochschulrektorenkonferenz (§ 3 (3)) fordert die ZaPF, dass die Richtlinien alle Statusgruppen berücksichtigen.
  \item Die ZaPF begrüßt, dass Gutachten und Entscheidungen veröffentlicht werden müssen (§ 3 (6)).
  \item Die ZaPF begrüßt eine Evaluation und gegebenenfalls die entsprechende Korrektur des Akkreditierungssystems (§ 15).
\end{itemize}

\paragraph{Punkte zur Musterrechtsverordnung (MRVO)}
\begin{itemize}
  \item Die ZaPF beharrt weiterhin auf der Unverzichtbarkeit von Begehungen bei allen Akkreditierungsverfahren (siehe Positionspapier Wintersemester 2017 Siegen), da dies die einzig direkte Austauschmöglichkeit zwischen Gutachtern und der betroffenen Studierendenschaft ist.
      Neben dem Wunsch nach einer entsprechenden Vorschrift in § 3 (2) des Staatsvertrags, muss die Begehung in § 24 (5) MRVO auf jeden Fall festgeschrieben werden.

  \item Eine Akkreditierungsfrist von 8 Jahren (§ 26 (1) MRVO) für eine Erstakkreditierung ist zu lang. Für neueingerichtete Studiengänge fordert die ZaPF eine erstmalige Reakkreditierung ein Jahr nach Ablauf der Regelstudienzeit, spätestens nach 5 Jahren.

  \item Wir freuen uns, dass die Befähigung zum gesellschaftlichen Engagement und die Berücksichtigung der Vielfalt von Studierenden Eingang in die MRVO erhalten haben.

  \item Die im Positionspapier aus Siegen aufgeführten Punkte "`Zugangsvoraussetzungen Master"', "`Gebündelte Akkreditierung"', "`Vertreter der Berufspraxis im Lehramt"' und "`Kombinationsstudiengänge"' sollen in den zu erarbeitenden neuen Akkreditierungsrichtlinien der ZaPF für studentische Gutachter*innen Berücksichtigung finden.
\end{itemize}
Zur neuen Aufgabenverteilung zwischen Akkreditierungsrat und Agenturen:
\begin{itemize}
  \item Als Folge unserer Kritik\footnote{Siehe Positionspapier aus Siegen} an den Unklarheiten der neuen Aufgabenverteilung (Staatsvertrag § 3, MRVO § 22, 27, 28) zwischen Rat und Agentur, fordert die ZaPF Transparenz bezüglich Rückkopplungsmechanismen zwischen Agenturen, Gutachtergremium und Akkreditierungsrat.
\end{itemize}
