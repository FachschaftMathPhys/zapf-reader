% !TEX ROOT = main.tex

\usepackage[ngerman]{babel}
\usepackage[utf8]{inputenc}
\usepackage{libertine}
\usepackage{libertinust1math}
\usepackage[T1]{fontenc}
\usepackage{microtype}
\usepackage{amsmath}
\usepackage{eurosym}
\usepackage[pdftex]{graphicx}
\usepackage{fancyhdr}
\usepackage{todonotes}
\usepackage{geometry}
\usepackage{url}
\usepackage{fontawesome}
\usepackage{hyperref}
\usepackage{wrapfig}
\usepackage{pdfpages}
\usepackage{tocloft}
\usepackage{subcaption}
\usepackage{paralist}
\usepackage{float}
\usepackage{afterpage}
\usepackage{ulem}
\usepackage{outlines}
\usepackage{xcolor}
\usepackage{blindtext}
\usepackage{mdwlist}     % Änderung der Zeilenabstände bei itemize und enumerate
\usepackage{lmodern}     % Ersatz fuer Computer Modern-Schriften
\usepackage[scaled=0.81]{helvet}     % Schriftart fuer Resos
% \usepackage[draft, markup=underlined]{changes}
\usepackage{csquotes}

  % braucht man das?
%\usepackage{titlesec}    % Abstand nach Überschriften neu definieren
%\titlespacing{\subsection}{0ex}{3ex}{-1ex}
%\titlespacing{\subsubsection}{0ex}{3ex}{-1ex}

\definecolor{urlred}{HTML}{660000}
\hypersetup{ % Design der Hyperrefs
    colorlinks=true,
    linkcolor=black,    % Farbe der internen Links (u.a. Table of Contents)
    urlcolor=black,    % Farbe der url-links
    citecolor=black} % Farbe der Literaturverzeichnis-Links

\parindent 0pt                 % Absatzeinrücken verhindern (pck:blindtext)
\parskip 12pt                 % Absätze durch Lücke trennen

% Suche nach Grafiken in ./media und .:
\graphicspath{{./media/}{./}}

% Satzspiegel
% \geometry{papersize={154mm,216mm}, layout=a4paper, layouthoffset=3mm,     layoutvoffset=3mm, inner=20mm, outer=15mm, top=15mm, bottom=25mm, heightrounded, marginparwidth=37mm, marginparsep=5mm}
%\setlength{\parindent}{0pt}

% Schriftarten
\renewcommand{\sfdefault}{phv}
\newenvironment{resofont}{\fontfamily{phv}\selectfont}{\par}

%Headlines
\pagestyle{fancy}
\fancyhf{}
\renewcommand{\headrulewidth}{0pt}
%\fancyhead[LE]{\leftmark}
%\fancyhead[RO]{\rightmark}
\fancyfoot[RO,LE]{\thepage}

% used for resos and pospaps
% \pagestyle{empty}
\cfoot{}
\lfoot{Zusammenkunft aller Physik-Fachschaften}
\rfoot{www.zapfev.de\\stapf@zapf.in}
\renewcommand{\headrulewidth}{0pt}
\renewcommand{\footrulewidth}{0.1pt}
\newcommand{\gen}{*innen}

%Unterdrückt Section und Subsection Nummern, also werden nur Chapter Nummer angezeigt
  \setcounter{secnumdepth}{0}
